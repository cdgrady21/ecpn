%\documentclass[]{article}
\documentclass[11pt]{article}
\usepackage[usenames,dvipsnames]{xcolor}

\usepackage[T1]{fontenc}
%\usepackage{lmodern}
\usepackage{tgtermes}
\usepackage{amssymb,amsmath}
%\usepackage[margin=1in]{geometry}
\usepackage[letterpaper,bottom=1in,top=1in,right=1.25in,left=1.25in,includemp=FALSE]{geometry}
\usepackage{pdfpages}
\usepackage[small]{caption}

\usepackage{ifxetex,ifluatex}
\usepackage{fixltx2e} % provides \textsubscript
% use microtype if available
\IfFileExists{microtype.sty}{\usepackage{microtype}}{}
\ifnum 0\ifxetex 1\fi\ifluatex 1\fi=0 % if pdftex
\usepackage[utf8]{inputenc}
\else % if luatex or xelatex
\usepackage{fontspec}
\ifxetex
\usepackage{xltxtra,xunicode}
\fi
\defaultfontfeatures{Mapping=tex-text,Scale=MatchLowercase}
\newcommand{\euro}{€}
\fi
%

\usepackage{fancyvrb}

\usepackage{ctable,longtable}

\usepackage[section]{placeins}
\usepackage{float} % provides the H option for float placement
\restylefloat{figure}
\usepackage{dcolumn} % allows for different column alignments
\newcolumntype{.}{D{.}{.}{1.2}}

\usepackage{booktabs} % nicer horizontal rules in tables

%Assume we want graphics always
\usepackage{graphicx}
% We will generate all images so they have a width \maxwidth. This means
% that they will get their normal width if they fit onto the page, but
% are scaled down if they would overflow the margins.
%% \makeatletter
%% \def\maxwidth{\ifdim\Gin@nat@width>\linewidth\linewidth
%%   \else\Gin@nat@width\fi}
%% \makeatother
%% \let\Oldincludegraphics\includegraphics
%% \renewcommand{\includegraphics}[1]{\Oldincludegraphics[width=\maxwidth]{#1}}
\graphicspath{{.}{../Soccom_Code/socom_2013/}}


%% \ifxetex
%% \usepackage[pagebackref=true, setpagesize=false, % page size defined by xetex
%% unicode=false, % unicode breaks when used with xetex
%% xetex]{hyperref}
%% \else
\usepackage[pagebackref=true, unicode=true, bookmarks=true, pdftex]{hyperref}
% \fi


\hypersetup{breaklinks=true,
  bookmarks=true,
  pdfauthor={Christopher Grady, Rebecca Wolfe, Danjuma Dawop, and Lisa Inks},
  pdftitle={Promoting Peace Amidst Group Conflict: An Intergroup Contact Field Experiment in Nigeria - Theory},
  colorlinks=true,
  linkcolor=BrickRed,
  citecolor=blue, %MidnightBlue,
  urlcolor=BrickRed,
  % urlcolor=blue,
  % linkcolor=magenta,
  pdfborder={0 0 0}}

%\setlength{\parindent}{0pt}
%\setlength{\parskip}{6pt plus 2pt minus 1pt}
\usepackage{parskip}
\setlength{\emergencystretch}{3em}  % prevent overfull lines
\providecommand{\tightlist}{%
  \setlength{\itemsep}{0pt}\setlength{\parskip}{0pt}}

%% Insist on this.
\setcounter{secnumdepth}{2}

\VerbatimFootnotes % allows verbatim text in footnotes

\title{Promoting Peace Amidst Group Conflict: An Intergroup Contact Field
Experiment in Nigeria - Theory}

\author{
Christopher Grady, Rebecca Wolfe, Danjuma Dawop, and Lisa Inks
}


\date{August 09, 2019}


\usepackage{versions}
\makeatletter
\renewcommand*\versionmessage[2]{\typeout{*** `#1' #2. ***}}
\renewcommand*\beginmarkversion{\sffamily}
  \renewcommand*\endmarkversion{}
\makeatother

\excludeversion{comment}

%\usepackage[margins=1in]{geometry}

\usepackage[compact,bottomtitles]{titlesec}
%\titleformat{ ⟨command⟩}[⟨shape⟩]{⟨format⟩}{⟨label⟩}{⟨sep⟩}{⟨before⟩}[⟨after⟩]
\titleformat{\section}[hang]{\Large\bfseries}{\thesection}{.5em}{\hspace{0in}}[\vspace{-.2\baselineskip}]
\titleformat{\subsection}[hang]{\large\bfseries}{\thesubsection}{.5em}{\hspace{0in}}[\vspace{-.2\baselineskip}]
%\titleformat{\subsubsection}[hang]{\bfseries}{\thesubsubsection}{.5em}{\hspace{0in}}[\vspace{-.2\baselineskip}]
\titleformat{\subsubsection}[hang]{\bfseries}{\thesubsubsection}{1ex}{\hspace{0in}}[\vspace{-.2\baselineskip}]
\titleformat{\paragraph}[runin]{\bfseries\itshape}{\theparagraph}{1ex}{}{\vspace{-.2\baselineskip}}
%\titleformat{\paragraph}[runin]{\itshape}{\theparagraph}{1ex}{}{\vspace{-.2\baselineskip}}

%%\titleformat{\subsection}[hang]{\bfseries}{\thesubsection}{.5em}{\hspace{0in}}[\vspace{-.2\baselineskip}]
%%%\titleformat*{\subsection}{\bfseries\scshape}
%%%\titleformat{\subsubsection}[leftmargin]{\footnotesize\filleft}{\thesubsubsection}{.5em}{}{}
%%\titleformat{\subsubsection}[hang]{\small\bfseries}{\thesubsubsection}{.5em}{\hspace{0in}}[\vspace{-.2\baselineskip}]
%%\titleformat{\paragraph}[runin]{\itshape}{\theparagraph}{1ex}{}{\vspace{-.5\baselineskip}}

%\titlespacing*{ ⟨command⟩}{⟨left⟩}{⟨beforesep⟩}{⟨aftersep⟩}[⟨right⟩]
\titlespacing{\section}{0pc}{1.5ex plus .1ex minus .2ex}{.5ex plus .1ex minus .1ex}
\titlespacing{\subsection}{0pc}{1.5ex plus .1ex minus .2ex}{.5ex plus .1ex minus .1ex}
\titlespacing{\subsubsection}{0pc}{1.5ex plus .1ex minus .2ex}{.5ex plus .1ex minus .1ex}



%% These next lines tell latex that it is ok to have a single graphic
%% taking up most of a page, and they also decrease the space around
%% figures and tables.
\renewcommand\floatpagefraction{.9}
\renewcommand\topfraction{.9}
\renewcommand\bottomfraction{.9}
\renewcommand\textfraction{.1}
\setcounter{totalnumber}{50}
\setcounter{topnumber}{50}
\setcounter{bottomnumber}{50}
\setlength{\intextsep}{2ex}
\setlength{\floatsep}{2ex}
\setlength{\textfloatsep}{2ex}



\begin{document}
\VerbatimFootnotes

%\begin{titlepage}
%  \maketitle
%\vspace{2in}
%
%\begin{center}
%  \begin{large}
%    PROPOSAL WHITE PAPER
%
%BAA 14-013
%
%Can a Hausa Language Television Station Change Norms about Violence in Northern Nigeria? A Randomized Study of Media Effects on Violent Extremism
%
%Jake Bowers
%
%University of Illinois @ Urbana-Champaign (jwbowers@illinois.edu)
%
%\url{http://jakebowers.org}
%
%Phone: +12179792179
%
%Topic Number: 1
%
%Topic Title: Identity, Influence and Mobilization
%
%\end{large}
%\end{center}
%\end{titlepage}

\maketitle

\hypertarget{theory}{%
\section{Theory}\label{theory}}

\hypertarget{intergroup-conflict-as-a-bargaining-problem}{%
\subsection{Intergroup Conflict as a Bargaining
Problem}\label{intergroup-conflict-as-a-bargaining-problem}}

Intergroup conflict is most often conceptualized as a bargaining problem
(Fearon 1994b; Powell 2006), and most solutions to reducing intergroup
conflict strive to help the groups overcome barriers to bargaining {[}Di
Salvatore and Ruggeri (2017); chris: this cite is just for
peacekeeping/intervention{]}. Intergroup conflict is a bargaining
problem because both groups want some resource -- land, power, etc --
but cannot reach an agreement about how to distribute that resource
peacefully. Because fighting is costly, the groups are better off
reaching a bargained compromise than fighting. However, two problems
prevent successful bargaining: information problems and commitment
problems (Fearon 1995). Groups are less likely to successfully bargain
without (1) accurate information about each other's strengths and
preferences, and/or (2) assurance that each side will abide by its
agreements.

An \emph{information problem} arises when neither group possesses
accurate information about the other. In this situation, both groups
have an incentive to deceive the other to achieve an advantageous
bargaining outcome. Groups have an incentive to portray themselves as
stronger, more willing to fight, and less willing to make concessions
than they truly are (Fearon 1995). This causes bargaining failures
because neither group knows what agreements the other side is willing to
accept or what their side should receive from bargaining. A
\emph{commitment problem} arises when neither group can credibly commit
to honor bargained agreements. For example, if bargaining power shifts
after the agreement, one side will have an incentive to renege on the
status quo agreement to achieve a better agreement. This causes
bargaining failures because neither group will commit to agreements
today that they believe will be broken tomorrow. Without the ability to
commit to agreements, bargaining will not be successful.{[}chris: maybe
revise this paragraph so the second part (commitment problems) is
structurally parallel to the first part (information problems).{]}

Groups in conflict overcome information and commitment problems in two
main ways. The first way is through third-parties. Third parties can
solve information problems by mediating disputes, providing accurate
information to both sides (Kydd 2006). Third parties can also solve
commitment problems by intervening to punish defection from agreements.
Though each group may have an incentive to defect on an agreement after
it is made, the groups have less incentive to defect if a strong third
party is capable of and willing to punish defection from bargained
agreements (Fearon 1995).

The second way groups solve information and commitment problems is
through reputations for trustworthiness (Kydd 2000). Reputations for
trustworthiness are how groups bargain in the absence of formal
enforcement mechanisms like third parties (Ostrom and Walker 2003).
Reputations encourage bargaining because future bargaining depends on
groups trusting each other to abide by agreements, and groups with
reputations for defection are unlikely to attract other trustworthy
partners or elicit trusting behavior from partners (Kydd 2000; Ostrom
and Walker 2003). Cultivating a reputation as a trustworthy partner in
previous interactions gives a bargaining partner confidence that you are
trustworthy in the present interaction.\footnote{The reputation argument
  is generally mobilized for contexts in which many groups observe the
  behavior of many other groups. In those contexts a good reputation
  assists in obtaining \emph{other} cooperative partners. In our case,
  there are two main sides forming perceptions about the reputation of
  each other. In this context, a good reputation assists in obtaining
  cooperative behavior in the future from the same partner. This two
  player context is similar to Kydd (2000).}

\hypertarget{the-persistence-of-intergroup-conflict}{%
\subsection{The Persistence of Intergroup
Conflict}\label{the-persistence-of-intergroup-conflict}}

If we know how to resolve intergroup conflict, why does conflict
persist? First, it is not possible for third parties to mediate or
intervene into every conflict. Sometimes no third party is willing or
able to become involved (Fey and Ramsay 2010, 530), and sometimes the
conflict is too decentralized for a third party to effectively mediate
or intervene. This situation is common for internal conflicts in weak
states, where conflicts are diffuse and the state lacks the capacity to
mediate or intervene into the conflict. Second, third party mediation
and intervention are not effective in all contexts (Autesserre 2017;
Beardsley 2008; Beber 2012; Weinstein 2005).

Where third parties cannot effectively intervene, conflicting groups
could negotiate based on reputation, but groups in conflict are unlikely
to have reputations that assist with bargaining. Groups in conflict
dehumanize the outgroup (Bandura 1999; Haslam and Loughnan 2014; Leyens
et al. 2007; Opotow 1990), view the outgroup as innately immoral (Brewer
1999, 435; Parker and Janoff-Bulman 2013; Weisel and Böhm 2015), do not
naturally feel empathy for outgroup members (Gutsell and Inzlicht 2010),
are unlikely to forgive outgroup transgression; Tam et al. (2007){]},
and believe outgroup members to be untrustworthy and dishonest (Eidelson
and Eidelson 2003; LeVine and Campbell 1972). These psychological
mechanisms introduce significant friction in the ability of groups to
build trusting relationships and significantly restrain opportunities
through which trust could be built.

These psychological mechanisms prevents peaceful resolution of conflict
in two main ways. First, they bias each group's perception of the other
side's behavior and preferences. Second, they increase the costs of
cooperation and the benefits of defection. These two psychological
mechanisms decrease the likelihood that the two sides bargain
successfully. These mechanisms prevent each side from accurately
interpreting the other side's signals, create and reinforce negative
outgroup stereotypes, and limit the number of agreements that each side
prefers to fighting.

\begin{center}\rule{0.5\linewidth}{\linethickness}\end{center}

Group members interpret the behavior of ingroup and outgroup members
differently. More belligerent and threatening behavior. Groups
over-generalize negative behaviors of outgroup members as representative
of the entire outgroup and under-generalize positive behaviors as
exceptional to the outgroup (Hewstone 1990) These biases produces low
trust, belief that other side owes your side something/other side has
committed transgressions whereas your side is defensive.

biases interpretations of ingroup and outgroup behavior and prevents
accurate perceptions about the attitudes and preferences of the
outgroup. Ingroups will perceive their own belligerent actions as
defensive and justified, and are more likely to perceive outgroup
actions as aggressive, negatively motivated, and unjustified (Amir 1969;
Hewstone 1990; Hunter, Stringer, and Watson 1991 chris: also cite
confirmation bias, anchoring bias). The perceived negative behavior may
be seen as \emph{defining} the group, whereas any perceived positive
behavior may be seen as the \emph{exception} to the group (Allison and
Messick 1985; Pettigrew 1979). Even positive intergroup interactions may
be re-interpreted as negative to avoid cognitive dissonance (Festinger
1962; Gubler 2013; Paolini, Harwood, and Rubin 2010). Interpreting
interactions negatively saps the power of each group to reassure the
other with costly signals of willingness to cooperate in future
interactions (Kydd 2000, @rohner2013war). It also adds to information
problems as groups will hold inaccurate views about each other's
willingness to cooperate and likelihood of upholding agreements. As
Axelrod (1980) shows, perceived untrustworthy behavior by one side often
begets a cycle of mistrust.

Groups tend to ascribe negative traits to the outgroup, and also tend to
remember negative events that corroborate their negative beliefs (Brewer
1991; Klein and Kunda 1992; Tajfel 1981, @brewer1999ingroupOutgroup).

Groups in conflict are given many events to justify their negative
perceptions. Initial negative perceptions, and the biased
interpretations they beget, make it very difficult for a group to
develop a positive reputation with a group they are in conflict with,
even when both groups are motivated to end the conflict. This bias
likely pushes each group's perception of the other side's willingness to
make peace further from their true willingness to make peace and so
reputations hinder, rather than aid, intergroup bargaining processes.

Compounding these problems, reputations for trustworthiness are also
hampered by a lack of opportunities for groups to observe each others
behavior and update perceptions of the outgroup's trustworthiness.
Compounding that problem, few of the outgroup's interactions will be
with groups that are relevant for predicting the outgroup's behavior
towards the ingroup, providing fewer opportunities for updating. This
means that the main opportunity to observe outgroup behavior and learn
their reputation is the ingroup's own interactions with the outgroup.
For groups in conflict, these opportunities are likely rare and almost
always adversarial.

\begin{center}\rule{0.5\linewidth}{\linethickness}\end{center}

For groups in conflict, the costs of cooperation and the benefits of
defection are high. When forming preferences for fighting or bargaining,
groups in conflict are not strictly weighing the material costs and
benefits of conflict versus peace. Their preferences are affected by
psychological and social punishments and rewards/evaluations. The social
and psychological consequences add costs to cooperating with the
outgroup and add benefits to defecting from agreements with the
outgroup.

Individual group members receive psychological benefits from social
differentiation with the outgroup and from beating the outgroup (Turner,
Brown, and Tajfel 1979; Wood 2000). Many groups define ``us'' by
positive differences with a ``them'', and an individual can derive
self-esteem from positively comparing their group identity to a rival
group (Brewer 1999; Tajfel 1981). When group members derive self-esteem
from feeling superior to an outgroup, group members may reject actions
that recognize the outgroup as equals and rhetoric about group
similarity. Groups that see the other side as immoral may even receive
some internal benefit from \emph{harming} the outgroup (Weisel and Böhm
2015).

Along with psychological benefits, group members receive social benefits
for strong anti-outgroup stances and social sanctioning for cooperative
behavior. The utility a group members gets for attitudes and behaviors
depends largely on how those attitudes and behaviors are received by
their ingroup (Wood 2000). If group members perceive outgroup animosity
as socially desirable, they may profess attitudes and engage in
behaviors that signal outgroup animosity. These social \emph{benefits}
also entail reciprocal social \emph{costs} that constrain the actions of
group members and group leaders. Individuals who cooperate with the
outgroup, as opposed to taking a hard stance against the other side's
perceived transgressions, might be accused of betraying the outgroup for
cooperating (Dreu 2010). Individuals in these groups might not engage in
ingroup policing, a strong, costly signal to the other side that your
group will uphold its peace agreements (Fearon and Laitin 1996). While
cooperation and ingroup policing might be punished, aggressive actions
may be seen as righteous self-defense of the ingroup and rewarded.

Leaders are also constrained by animosity among their group. Groups are
known to punish leaders for cooperating or compromising with the
outgroup (Fearon 1994a), so the set of peace agreements available to the
leader of one group is likely unacceptable to the other.. Leaders of
hostile groups also cannot credibly commit to keep their group members
in check, as some subgroups may feel confident enough to disobey
agreements made by group leaders. Due to increased (1) likelihood of
information and commitment problems, (2) internal psychological
evaluations that favor competition over cooperation, and (3) social
sanctioning for group members and leaders perceived as betraying the
ingroup, animosity reduces the likelihood of successful bargaining and
makes violent conflict more likely.

\begin{center}\rule{0.5\linewidth}{\linethickness}\end{center}

Intergroup animosity prevents peace by directly exacerbating informtion
and commitment problems. First, information and commitment problems are
more likely to occur because groups are less likely to trust information
they receive from the other side or any peace commitment they get from
the other side.

\hypertarget{peacebuilding-interventions-to-improve-intergroup-attitudes}{%
\subsection{Peacebuilding Interventions to Improve Intergroup
Attitudes}\label{peacebuilding-interventions-to-improve-intergroup-attitudes}}

The problems of negative intergroup attitudes suggests that improving
those attitudes could lead to peace-promoting behaviors and reduce
conflict. One approach to improving intergroup attitudes comes from
intergroup contact theory (Allport 1954). Intergroup contact theory
hypothesizes that intergroup relations can be improved through
interactions in which group members (1) cooperate (2) with equal status
(3) to achieve shared goals (4) with the support of elites.\footnote{Under
  other conditions -- incidental contact, intergroup competition,
  interactions in which one group has power over the other, elite
  disapproval of intergroup contact -- the structure of the contact may
  \emph{not} improve relations (Enos 2014; Forbes 1997).{]}}. Improving
relations -- especially improving trust -- can help groups overcome
bargaining problems and reduce the likelihood of violence.

The effectiveness of intergroup contact has been demonstrated in a
variety of contexts and using a variety of methodological approaches
(Paluck, Green, and Green 2017; Pettigrew and Tropp 2006). Notably,
intergroup contacted programs improved relations between white people
and black people in the U.S. South Africa, and Norway (Burns, Corno, and
La Ferrara 2015; Carrell, Hoekstra, and West 2015; Finseraas and
Kotsadam 2017; Marmaros and Sacerdote 2006), Jews and Arabs (Ditlmann
and Samii 2016; Yablon 2012), and Hindus and Muslims in India (Barnhardt
2009). In Nigeria, a recent study found that intergroup contact between
Muslims and Christians decreased discrimination relative to a group that
experienced \emph{intragroup} contact, suggesting that intergroup
contact can work by countering the adverse effects of ingroup-only
experiences (Scacco and Warren 2018).

Intergroup contact is proposed to affect a myriad of intergroup
attitudes. Here we focus on six: (1) increased trust, (2) reduced
anxiety and threat, (3) reduced social distance (4) expansion of ingroup
to include the former outgroup, and (5) perceptions of material benefit
from cooperation.

Intergroup contact gives groups an opportunity to learn about each other
and update opinions based on personal experience. This updating can
increase trust and decrease feelings of threat and anxiety (Hewstone et
al. 2006; Page-Gould, Mendoza-Denton, and Tropp 2008). Intergroup trust
increases because contact gives groups the opportunity to signal
trustworthiness and preferences for cooperation to the other group (Kydd
2000, @rohner2013war). Threat and anxiety reduce as familiarity with
outgroup members increases. Feelings of threat and anxiety often arise
from fear of the unknown, and through intergroup contact the two groups
to get to know each other.

Intergroup contact can show the groups how their values and interests
align, making the groups feel closer. The increased closeness can
manifest as reduced social distance and even a new collective
identification that includes both groups. Groups in conflict often use
the outgroup to define the ingroup -- the outgroup's \_bad\_ness helps
define the ingroup's \_good\_ness (Allport 1954; Brewer 1999; Tajfel
1981). Contact can make salient many similarities between the groups,
reducing feelings of social distance and even helping to craft a joint
identity that encompasses both groups (Gaertner and Dovidio 2014).

Intergroup contact can also show the groups how their material status
benefits from cooperation. Group animosity often arises due to the
competition for resources that both groups claim or desire (Sherif
1958). Intergroup contact to achieve a goal that benefits both groups
(1) alleviates material deprivation and (2) associates intergroup
cooperation with positive material outcomes. By cooperating for joint
benefit in the present groups can see how cooperative behavior in the
future will benefit both groups.

Contact-based interventions can also assist third-party mediation or
intervention. Helps mediation because the conflicting groups are more
likely to update perceptions based on mediated information. Helps
intervention because the groups are still able to commit to agreements
without the threat of punishment.

\hypertarget{references}{%
\section{References}\label{references}}

\begin{center}\rule{0.5\linewidth}{\linethickness}\end{center}

\textbf{Thoughts about trust}

Ostrom and Walker (2003) ch.2 discusses how reciprocity, reputation, and
trust help overcome selfish behavior that leads to pareto-inferior
outcomes; players achieve self-enforcing equilibria by comitting
themselves to punishing noncooperators to deter noncooperation; repeated
games \& uncertainty about player types increases cooperation; face to
face interaction increases cooperation bcuz increases trust, also could
add subjective value to payoff structure, gives group ID, reinforces
normes, not because players realize optimal strategies and not from
non-face-to-face promises. People seek to improve values they find
important. People use heuristics; people use internal evaluations that
add or subtract from objective payoffs (+ for feeling good, - for being
a sucker); ch.12 discusses how trust is minimal in situations where
reputations for trust cannot be established. ({\textbf{???}}) ch.3
gaming trust -- concept of mutual trust: trust based on long-term
interaction between two parties. Thick trust: diverse layers of
interaction; interactions in one layer influence interactions in
another; repeated relationships and 3rd party relationships.
({\textbf{???}}) talks about being trustworthy. ({\textbf{???}}) and
costly signals of trustworthiness. Trust and Reciprocity 2003 ch.15 -
knowledge of the other, repeated interactions, and strong possibility of
future interactions predict trustworthy and trusting
relationships.--\textgreater{}

\hypertarget{refs}{}
\leavevmode\hypertarget{ref-allison1985group}{}%
Allison, Scott T, and David M Messick. 1985. ``The Group Attribution
Error.'' \emph{Journal of Experimental Social Psychology} 21(6):
563--79.

\leavevmode\hypertarget{ref-allport1954prejudice}{}%
Allport, Gordon. 1954. ``The Nature of Prejudice.'' \emph{Garden City,
NJ Anchor}.

\leavevmode\hypertarget{ref-amir1969contact}{}%
Amir, Yehuda. 1969. ``Contact Hypothesis in Ethnic Relations.''
\emph{Psychological bulletin} 71(5): 319.

\leavevmode\hypertarget{ref-autesserre2017international}{}%
Autesserre, Severine. 2017. ``International Peacebuilding and Local
Success: Assumptions and Effectiveness.'' \emph{International Studies
Review} 19(1): 114--32.

\leavevmode\hypertarget{ref-axelrod1980effective}{}%
Axelrod, Robert. 1980. ``Effective Choice in the Prisoner's Dilemma.''
\emph{Journal of conflict resolution} 24(1): 3--25.

\leavevmode\hypertarget{ref-bandura1999moral}{}%
Bandura, Albert. 1999. ``Moral Disengagement in the Perpetration of
Inhumanities.'' \emph{Personality and social psychology review} 3(3):
193--209.

\leavevmode\hypertarget{ref-barnhardt2009near}{}%
Barnhardt, Sharon. 2009. ``Near and Dear? Evaluating the Impact of
Neighbor Diversity on Inter-Religious Attitudes.'' \emph{Unpublished
working paper}.

\leavevmode\hypertarget{ref-beardsley2008agreement}{}%
Beardsley, Kyle. 2008. ``Agreement Without Peace? International
Mediation and Time Inconsistency Problems.'' \emph{American journal of
political science} 52(4): 723--40.

\leavevmode\hypertarget{ref-beber2012international}{}%
Beber, Bernd. 2012. ``International Mediation, Selection Effects, and
the Question of Bias.'' \emph{Conflict Management and Peace Science}
29(4): 397--424.

\leavevmode\hypertarget{ref-brewer1991social}{}%
Brewer, Marilynn B. 1991. ``The Social Self: On Being the Same and
Different at the Same Time.'' \emph{Personality and social psychology
bulletin} 17(5): 475--82.

\leavevmode\hypertarget{ref-brewer1999ingroupOutgroup}{}%
---------. 1999. ``The Psychology of Prejudice: Ingroup Love and
Outgroup Hate?'' \emph{Journal of social issues} 55(3): 429--44.

\leavevmode\hypertarget{ref-burns2015interaction}{}%
Burns, Justine, Lucia Corno, and Eliana La Ferrara. 2015.
\emph{Interaction, Prejudice and Performance. Evidence from South
Africa}. Working paper.

\leavevmode\hypertarget{ref-carrell2015impact}{}%
Carrell, Scott E, Mark Hoekstra, and James E West. 2015. \emph{The
Impact of Intergroup Contact on Racial Attitudes and Revealed
Preferences}. National Bureau of Economic Research.

\leavevmode\hypertarget{ref-di2017effectiveness}{}%
Di Salvatore, Jessica, and Andrea Ruggeri. 2017. ``Effectiveness of
Peacekeeping Operations.'' \emph{Oxford Research Encyclopedia of
Politics}.

\leavevmode\hypertarget{ref-ditlmann2016can}{}%
Ditlmann, Ruth K, and Cyrus Samii. 2016. ``Can Intergroup Contact Affect
Ingroup Dynamics? Insights from a Field Study with Jewish and
Arab-Palestinian Youth in Israel.'' \emph{Peace and Conflict: Journal of
Peace Psychology} 22(4): 380.

\leavevmode\hypertarget{ref-dreu2010social}{}%
Dreu, Carsten KW de. 2010. ``Social Value Orientation Moderates Ingroup
Love but Not Outgroup Hate in Competitive Intergroup Conflict.''
\emph{Group Processes \& Intergroup Relations} 13(6): 701--13.

\leavevmode\hypertarget{ref-eidelson2003dangerous}{}%
Eidelson, Roy J, and Judy I Eidelson. 2003. ``Dangerous Ideas: Five
Beliefs That Propel Groups Toward Conflict.'' \emph{American
Psychologist} 58(3): 182.

\leavevmode\hypertarget{ref-enos2014causal}{}%
Enos, Ryan D. 2014. ``Causal Effect of Intergroup Contact on
Exclusionary Attitudes.'' \emph{Proceedings of the National Academy of
Sciences} 111(10): 3699--3704.

\leavevmode\hypertarget{ref-fearon1994domestic}{}%
Fearon, James D. 1994a. ``Domestic Political Audiences and the
Escalation of International Disputes.'' \emph{American political science
review} 88(3): 577--92.

\leavevmode\hypertarget{ref-fearon1994ethnic}{}%
---------. 1994b. ``Ethnic War as a Commitment Problem.'' In
\emph{Annual Meetings of the American Political Science Association},
2--5.

\leavevmode\hypertarget{ref-fearon1995rationalist}{}%
---------. 1995. ``Rationalist Explanations for War.''
\emph{International organization} 49(3): 379--414.

\leavevmode\hypertarget{ref-fearon1996explaining}{}%
Fearon, James D, and David D Laitin. 1996. ``Explaining Interethnic
Cooperation.'' \emph{American political science review} 90(4): 715--35.

\leavevmode\hypertarget{ref-festinger1962cognitiveDissonance}{}%
Festinger, Leon. 1962. 2 \emph{A Theory of Cognitive Dissonance}.
Stanford university press.

\leavevmode\hypertarget{ref-fey2010shuttle}{}%
Fey, Mark, and Kristopher W Ramsay. 2010. ``When Is Shuttle Diplomacy
Worth the Commute? Information Sharing Through Mediation.'' \emph{World
Politics} 62(4): 529--60.

\leavevmode\hypertarget{ref-finseraas2017does}{}%
Finseraas, Henning, and Andreas Kotsadam. 2017. ``Does Personal Contact
with Ethnic Minorities Affect Anti-Immigrant Sentiments? Evidence from a
Field Experiment.'' \emph{European Journal of Political Research} 56(3):
703--22.

\leavevmode\hypertarget{ref-forbes1997ethnic}{}%
Forbes, Hugh Donald. 1997. \emph{Ethnic Conflict: Commerce, Culture, and
the Contact Hypothesis}. Yale University Press.

\leavevmode\hypertarget{ref-gaertner2014reducing}{}%
Gaertner, Samuel L, and John F Dovidio. 2014. \emph{Reducing Intergroup
Bias: The Common Ingroup Identity Model}. Psychology Press.

\leavevmode\hypertarget{ref-gubler2013humanizing}{}%
Gubler, Joshua R. 2013. ``When Humanizing the Enemy Fails: The Role of
Dissonance and Justification in Intergroup Conflict.'' In \emph{Annual
Meeting of the American Political Science Association},

\leavevmode\hypertarget{ref-gutsell2010empathy}{}%
Gutsell, Jennifer N, and Michael Inzlicht. 2010. ``Empathy Constrained:
Prejudice Predicts Reduced Mental Simulation of Actions During
Observation of Outgroups.'' \emph{Journal of experimental social
psychology} 46(5): 841--45.

\leavevmode\hypertarget{ref-haslam2014dehumanization}{}%
Haslam, Nick, and Steve Loughnan. 2014. ``Dehumanization and
Infrahumanization.'' \emph{Annual review of psychology} 65: 399--423.

\leavevmode\hypertarget{ref-hewstone1990ultimate}{}%
Hewstone, Miles. 1990. ``The `Ultimate Attribution Error'? A Review of
the Literature on Intergroup Causal Attribution.'' \emph{European
Journal of Social Psychology} 20(4): 311--35.

\leavevmode\hypertarget{ref-hewstone2006intergroup}{}%
Hewstone, Miles et al. 2006. ``Intergroup Contact, Forgiveness, and
Experience of `the Troubles' in Northern Ireland.'' \emph{Journal of
Social Issues} 62(1): 99--120.

\leavevmode\hypertarget{ref-hunter1991intergroup}{}%
Hunter, John A, Maurice Stringer, and RP Watson. 1991. ``Intergroup
Violence and Intergroup Attributions.'' \emph{British Journal of Social
Psychology} 30(3): 261--66.

\leavevmode\hypertarget{ref-klein1992motivated}{}%
Klein, William M, and Ziva Kunda. 1992. ``Motivated Person Perception:
Constructing Justifications for Desired Beliefs.'' \emph{Journal of
experimental social psychology} 28(2): 145--68.

\leavevmode\hypertarget{ref-kydd2000trust}{}%
Kydd, Andrew. 2000. ``Trust, Reassurance, and Cooperation.''
\emph{International Organization} 54(2): 325--57.

\leavevmode\hypertarget{ref-kydd2006can}{}%
Kydd, Andrew H. 2006. ``When Can Mediators Build Trust?'' \emph{American
Political Science Review} 100(3): 449--62.

\leavevmode\hypertarget{ref-levine1972ethnocentrism}{}%
LeVine, Robert A, and Donald T Campbell. 1972. ``Ethnocentrism: Theories
of Conflict, Ethnic Attitudes, and Group Behavior.''

\leavevmode\hypertarget{ref-leyens2007infra}{}%
Leyens, Jacques-Philippe et al. 2007. ``Infra-Humanization: The Wall of
Group Differences.'' \emph{Social Issues and Policy Review} 1(1):
139--72.

\leavevmode\hypertarget{ref-marmaros2006friendships}{}%
Marmaros, David, and Bruce Sacerdote. 2006. ``How Do Friendships Form?''
\emph{The Quarterly Journal of Economics} 121(1): 79--119.

\leavevmode\hypertarget{ref-opotow1990moral}{}%
Opotow, Susan. 1990. ``Moral Exclusion and Injustice: An Introduction.''
\emph{Journal of social issues} 46(1): 1--20.

\leavevmode\hypertarget{ref-ostrom2003trust}{}%
Ostrom, Elinor, and James Walker. 2003. \emph{Trust and Reciprocity:
Interdisciplinary Lessons for Experimental Research}. Russell Sage
Foundation.

\leavevmode\hypertarget{ref-page2008little}{}%
Page-Gould, Elizabeth, Rodolfo Mendoza-Denton, and Linda R Tropp. 2008.
``With a Little Help from My Cross-Group Friend: Reducing Anxiety in
Intergroup Contexts Through Cross-Group Friendship.'' \emph{Journal of
personality and social psychology} 95(5): 1080.

\leavevmode\hypertarget{ref-paluck2017contact}{}%
Paluck, Elizabeth Levy, Seth Green, and Donald P Green. 2017. ``The
Contact Hypothesis Revisited.''

\leavevmode\hypertarget{ref-paolini2010negative}{}%
Paolini, Stefania, Jake Harwood, and Mark Rubin. 2010. ``Negative
Intergroup Contact Makes Group Memberships Salient: Explaining Why
Intergroup Conflict Endures.'' \emph{Personality and Social Psychology
Bulletin} 36(12): 1723--38.

\leavevmode\hypertarget{ref-parker2013lessons}{}%
Parker, Michael T, and Ronnie Janoff-Bulman. 2013. ``Lessons from
Morality-Based Social Identity: The Power of Outgroup `Hate,' Not Just
Ingroup `Love'.'' \emph{Social Justice Research} 26(1): 81--96.

\leavevmode\hypertarget{ref-pettigrew1979ultimate}{}%
Pettigrew, Thomas F. 1979. ``The Ultimate Attribution Error: Extending
Allport's Cognitive Analysis of Prejudice.'' \emph{Personality and
social psychology bulletin} 5(4): 461--76.

\leavevmode\hypertarget{ref-pettigrew2006meta}{}%
Pettigrew, Thomas F, and Linda R Tropp. 2006. ``A Meta-Analytic Test of
Intergroup Contact Theory.'' \emph{Journal of personality and social
psychology} 90(5): 751.

\leavevmode\hypertarget{ref-powell2006war}{}%
Powell, Robert. 2006. ``War as a Commitment Problem.''
\emph{International organization} 60(1): 169--203.

\leavevmode\hypertarget{ref-rohner2013war}{}%
Rohner, Dominic, Mathias Thoenig, and Fabrizio Zilibotti. 2013. ``War
Signals: A Theory of Trade, Trust, and Conflict.'' \emph{Review of
Economic Studies} 80(3): 1114--47.

\leavevmode\hypertarget{ref-scacco2018nigeria}{}%
Scacco, Alexandra, and Shana S Warren. 2018. ``Can Social Contact Reduce
Prejudice and Discrimination? Evidence from a Field Experiment in
Nigeria.'' \emph{American Political Science Review} 112(3): 654--77.

\leavevmode\hypertarget{ref-sherif1958superordinate}{}%
Sherif, Muzafer. 1958. ``Superordinate Goals in the Reduction of
Intergroup Conflict.'' \emph{American journal of Sociology} 63(4):
349--56.

\leavevmode\hypertarget{ref-tajfel1981groups}{}%
Tajfel, Henri. 1981. \emph{Human Groups and Social Categories: Studies
in Social Psychology}. CUP Archive.

\leavevmode\hypertarget{ref-tam2007impact}{}%
Tam, Tania et al. 2007. ``The Impact of Intergroup Emotions on
Forgiveness in Northern Ireland.'' \emph{Group Processes \& Intergroup
Relations} 10(1): 119--36.

\leavevmode\hypertarget{ref-turner1979social}{}%
Turner, John C, Rupert J Brown, and Henri Tajfel. 1979. ``Social
Comparison and Group Interest in Ingroup Favouritism.'' \emph{European
journal of social psychology} 9(2): 187--204.

\leavevmode\hypertarget{ref-weinstein2005autonomous}{}%
Weinstein, Jeremy M. 2005. ``Autonomous Recovery and International
Intervention in Comparative Perspective.'' \emph{Available at SSRN
1114117}.

\leavevmode\hypertarget{ref-weisel2015ingroup}{}%
Weisel, Ori, and Robert Böhm. 2015. ```Ingroup Love' and `Outgroup Hate'
in Intergroup Conflict Between Natural Groups.'' \emph{Journal of
experimental social psychology} 60: 110--20.

\leavevmode\hypertarget{ref-wood2000attitude}{}%
Wood, Wendy. 2000. ``Attitude Change: Persuasion and Social Influence.''
\emph{Annual review of psychology} 51(1): 539--70.

\leavevmode\hypertarget{ref-yablon2012we}{}%
Yablon, Yaacov B. 2012. ``Are We Preaching to the Converted? The Role of
Motivation in Understanding the Contribution of Intergroup Encounters.''
\emph{Journal of Peace Education} 9(3): 249--63.

\end{document}
