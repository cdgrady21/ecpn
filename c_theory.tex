%\documentclass[]{article}
\documentclass[11pt]{article}
\usepackage[usenames,dvipsnames]{xcolor}

\usepackage[T1]{fontenc}
%\usepackage{lmodern}
\usepackage{tgtermes}
\usepackage{amssymb,amsmath}
%\usepackage[margin=1in]{geometry}
\usepackage[letterpaper,bottom=1in,top=1in,right=1.25in,left=1.25in,includemp=FALSE]{geometry}
\usepackage{pdfpages}
\usepackage[small]{caption}

\usepackage{ifxetex,ifluatex}
\usepackage{fixltx2e} % provides \textsubscript
% use microtype if available
\IfFileExists{microtype.sty}{\usepackage{microtype}}{}
\ifnum 0\ifxetex 1\fi\ifluatex 1\fi=0 % if pdftex
\usepackage[utf8]{inputenc}
\else % if luatex or xelatex
\usepackage{fontspec}
\ifxetex
\usepackage{xltxtra,xunicode}
\fi
\defaultfontfeatures{Mapping=tex-text,Scale=MatchLowercase}
\newcommand{\euro}{€}
\fi
%

\usepackage{fancyvrb}

\usepackage{ctable,longtable}

\usepackage[section]{placeins}
\usepackage{float} % provides the H option for float placement
\restylefloat{figure}
\usepackage{dcolumn} % allows for different column alignments
\newcolumntype{.}{D{.}{.}{1.2}}

\usepackage{booktabs} % nicer horizontal rules in tables

%Assume we want graphics always
\usepackage{graphicx}
% We will generate all images so they have a width \maxwidth. This means
% that they will get their normal width if they fit onto the page, but
% are scaled down if they would overflow the margins.
%% \makeatletter
%% \def\maxwidth{\ifdim\Gin@nat@width>\linewidth\linewidth
%%   \else\Gin@nat@width\fi}
%% \makeatother
%% \let\Oldincludegraphics\includegraphics
%% \renewcommand{\includegraphics}[1]{\Oldincludegraphics[width=\maxwidth]{#1}}
\graphicspath{{.}{../Soccom_Code/socom_2013/}}


%% \ifxetex
%% \usepackage[pagebackref=true, setpagesize=false, % page size defined by xetex
%% unicode=false, % unicode breaks when used with xetex
%% xetex]{hyperref}
%% \else
\usepackage[pagebackref=true, unicode=true, bookmarks=true, pdftex]{hyperref}
% \fi


\hypersetup{breaklinks=true,
  bookmarks=true,
  pdfauthor={Christopher Grady, Rebecca Wolfe, Danjuma Dawop, and Lisa Inks},
  pdftitle={Promoting Peace Amidst Group Conflict: An Intergroup Contact Field Experiment in Nigeria - Theory},
  colorlinks=true,
  linkcolor=BrickRed,
  citecolor=blue, %MidnightBlue,
  urlcolor=BrickRed,
  % urlcolor=blue,
  % linkcolor=magenta,
  pdfborder={0 0 0}}

%\setlength{\parindent}{0pt}
%\setlength{\parskip}{6pt plus 2pt minus 1pt}
\usepackage{parskip}
\setlength{\emergencystretch}{3em}  % prevent overfull lines
\providecommand{\tightlist}{%
  \setlength{\itemsep}{0pt}\setlength{\parskip}{0pt}}

%% Insist on this.
\setcounter{secnumdepth}{2}

\VerbatimFootnotes % allows verbatim text in footnotes

\title{Promoting Peace Amidst Group Conflict: An Intergroup Contact Field
Experiment in Nigeria - Theory}

\author{
Christopher Grady, Rebecca Wolfe, Danjuma Dawop, and Lisa Inks
}


\date{May 27, 2019}


\usepackage{versions}
\makeatletter
\renewcommand*\versionmessage[2]{\typeout{*** `#1' #2. ***}}
\renewcommand*\beginmarkversion{\sffamily}
  \renewcommand*\endmarkversion{}
\makeatother

\excludeversion{comment}

%\usepackage[margins=1in]{geometry}

\usepackage[compact,bottomtitles]{titlesec}
%\titleformat{ ⟨command⟩}[⟨shape⟩]{⟨format⟩}{⟨label⟩}{⟨sep⟩}{⟨before⟩}[⟨after⟩]
\titleformat{\section}[hang]{\Large\bfseries}{\thesection}{.5em}{\hspace{0in}}[\vspace{-.2\baselineskip}]
\titleformat{\subsection}[hang]{\large\bfseries}{\thesubsection}{.5em}{\hspace{0in}}[\vspace{-.2\baselineskip}]
%\titleformat{\subsubsection}[hang]{\bfseries}{\thesubsubsection}{.5em}{\hspace{0in}}[\vspace{-.2\baselineskip}]
\titleformat{\subsubsection}[hang]{\bfseries}{\thesubsubsection}{1ex}{\hspace{0in}}[\vspace{-.2\baselineskip}]
\titleformat{\paragraph}[runin]{\bfseries\itshape}{\theparagraph}{1ex}{}{\vspace{-.2\baselineskip}}
%\titleformat{\paragraph}[runin]{\itshape}{\theparagraph}{1ex}{}{\vspace{-.2\baselineskip}}

%%\titleformat{\subsection}[hang]{\bfseries}{\thesubsection}{.5em}{\hspace{0in}}[\vspace{-.2\baselineskip}]
%%%\titleformat*{\subsection}{\bfseries\scshape}
%%%\titleformat{\subsubsection}[leftmargin]{\footnotesize\filleft}{\thesubsubsection}{.5em}{}{}
%%\titleformat{\subsubsection}[hang]{\small\bfseries}{\thesubsubsection}{.5em}{\hspace{0in}}[\vspace{-.2\baselineskip}]
%%\titleformat{\paragraph}[runin]{\itshape}{\theparagraph}{1ex}{}{\vspace{-.5\baselineskip}}

%\titlespacing*{ ⟨command⟩}{⟨left⟩}{⟨beforesep⟩}{⟨aftersep⟩}[⟨right⟩]
\titlespacing{\section}{0pc}{1.5ex plus .1ex minus .2ex}{.5ex plus .1ex minus .1ex}
\titlespacing{\subsection}{0pc}{1.5ex plus .1ex minus .2ex}{.5ex plus .1ex minus .1ex}
\titlespacing{\subsubsection}{0pc}{1.5ex plus .1ex minus .2ex}{.5ex plus .1ex minus .1ex}



%% These next lines tell latex that it is ok to have a single graphic
%% taking up most of a page, and they also decrease the space around
%% figures and tables.
\renewcommand\floatpagefraction{.9}
\renewcommand\topfraction{.9}
\renewcommand\bottomfraction{.9}
\renewcommand\textfraction{.1}
\setcounter{totalnumber}{50}
\setcounter{topnumber}{50}
\setcounter{bottomnumber}{50}
\setlength{\intextsep}{2ex}
\setlength{\floatsep}{2ex}
\setlength{\textfloatsep}{2ex}



\begin{document}
\VerbatimFootnotes

%\begin{titlepage}
%  \maketitle
%\vspace{2in}
%
%\begin{center}
%  \begin{large}
%    PROPOSAL WHITE PAPER
%
%BAA 14-013
%
%Can a Hausa Language Television Station Change Norms about Violence in Northern Nigeria? A Randomized Study of Media Effects on Violent Extremism
%
%Jake Bowers
%
%University of Illinois @ Urbana-Champaign (jwbowers@illinois.edu)
%
%\url{http://jakebowers.org}
%
%Phone: +12179792179
%
%Topic Number: 1
%
%Topic Title: Identity, Influence and Mobilization
%
%\end{large}
%\end{center}
%\end{titlepage}

\maketitle

\section{Theory}\label{theory}

\subsection{Intergroup Conflict as a Bargaining
Problem}\label{intergroup-conflict-as-a-bargaining-problem}

Intergroup conflict is most often conceptualized as a bargaining problem
(Fearon 1994b; Powell 2006), and most solutions to reducing intergroup
conflict strive to help the groups overcome those bargaining problems
{[}Di Salvatore and Ruggeri (2017); chris: this cite is just for
peacekeeping/intervention{]}. Intergroup conflict is a bargaining
problem because both groups want some resource -- land, power, etc --
but cannot reach an agreement about how to distribute that resource
peacefully. Because fighting is costly, the groups are better off
reaching a bargained compromise and not fighting. However, two
bargaining problems prevent this: information problems and commitment
problems. To successfully bargain, the groups need (1) accurate
information about each other's strengths and preferences, and/or (2) the
assurance that each side will abide by its agreements.

An \emph{information problem} arises because neither group possesses
accurate information about the other, and both groups have an incentive
to deceive the other in order to achieve an advantageous bargaining
outcome. Groups have an incentive to portray themselves as stronger,
more willing to fight, and less willing to make concessions than they
truly are (Fearon 1995). This causes bargaining failures because neither
group knows what agreements the other side is willing to accept or what
their side should receive from bargaining. A \emph{comittment problem}
arises because neither group can credibly commit to honor bargained
agreements if bargaining power shifts in the future. If bargaining power
shifts, one side will have an incentive to renege on the status quo
agreement to achieve a better agreement. Neither group can commit to
honor agreements made today when both groups know that bargaining power
may shift tomorrow.\footnote{Power between the groups can shift due to
  factors that affect each group's preferences and capabilities.
  Internally, one group may grow in power or size, one group may disarm
  before the other, the groups may have factions that reject the
  agreement, the groups may change leaders, or group members attitudes
  may change for other reasons. Externally, the groups may gain or lose
  de facto or de jure state support, other external actors may switch
  allegiances, or some other shock may change each group's power
  relative to the other {[}Fearon (2004); Reed et al. (2016); chris:
  need to cite and add more{]}. Anything that will change group power in
  the future can cause commitment problems in the present.} Without the
ability to commit to agreements, bargaining will not be successful.

Groups in conflict overcome these bargaining problems in several ways.
Groups can overcome information problems through fighting, as they learn
about each others capacity and willingness to fight (Smith and Stam
2003). Groups can also overcome information problems through mediation.
An interested third party mediator with no stake in the conflict can
provide accurate information to both sides, helping each side reach a
bargain (Beber 2012; Kydd 2006; Ott 1972) Even if groups overcome
information problems, commitment problems could prevent groups from
reaching an agreement. The main way that groups overcome commitment
problems is through strong third parties that add large costs to
reneging on agreements (Doyle and Sambanis 2000; Fearon 1998; Powell
2006). Though each group may have an incentive to defect on its
agreement if bargaining power changes in a vacuum, the groups have no
incentive to defect if a strong third party is capable of and willing to
punish defection from bargained agreements. With a third party punishing
defection, the groups can bargain in good faith knowing that the other
will abide by its word.

If we know how to resolve intergroup conflict, why does conflict
persist? International mediation and intervention are dogged by two
issues of motivation. First, mediators are often motivated for peace.
This motivation implies that mediator's provide information that is
biased towards encouraging the groups to negotiate a peace agreement.
This means that peace-biased mediators do not solve information
problems. For mediators to reduce information problems, groups engaged
in bargaining must believe that mediators provide accurate, unbiased
information. Since both groups know that mediators are biased towards
peace, mediators may not help groups achieve peace (Fey and Ramsay 2010;
Kydd 2006; Smith and Stam 2003). This same argument prevents mediators
who are biased for or against disputants from solving information
problems (Kydd 2006).\footnote{Whether mediation is benefited from
  biased or unbiased information is still a matter of debate. Some
  scholars believe bias improves mediation. The primary mechanisms
  proposed are that biased mediators are more likely to have relevant
  private information (Savun 2008) and that a biased mediator who tells
  his \emph{favored} group to compromise will be heeded (Kydd 2003;
  Svensson 2009). We tend to side with Beber (2012), Rauchhaus (2006),
  Crescenzi et al. (2011), and Kydd (2006) and believe that unbiased
  mediation improves mediation outcomes. Kydd (2006) and Beber (2012)
  provide compelling theoretical arguments: effective mediators must be
  both unbiased and motivated, but that motivation and bias often
  overlap. We are also amenable to the argument of Fey and Ramsay (2010)
  that mediator success actually has less to do with information
  provision and more with agenda setting, offering carrots/threatening
  sticks, and serving as a back channel.}

Second, international actors may lack the motivation to mediate or to
intervene into the conflict, even if one group reneges on its agreement.
Intervention is costly and many international actors either cannot
credibly commit to intervene into many conflicts or have no interest in
intervening into many conflicts (Beber 2012; Fey and Ramsay 2010; Kydd
2006). This lack of motivation is especially relevant for intergroup
conflict between groups within a state. Most international actors and
strong third parties lack an incentive to intervene into intrastate
intergroup conflicts, and these disputes tend to take place within
states that lack the capacity to compel both sides themselves. Since
intervention is costly, strong third parties have an incentive to
intervene only \emph{after} fighting escalates, so intervention will
generally not be used to prevent conflict from escalating or to reduce
the persistent, smaller-scale violence that plague many countries. Where
external actors are motivated to intervene, intervention is a short-term
peace solution and its effects do not endure with the departure of the
external actor (Beardsley 2008; Rohner, Thoenig, and Zilibotti 2013;
Weinstein 2005). Once the external actor leaves, the groups' commitment
problem returns: with no one to punish defection, the groups have no
reason to trust each other.

Rather than rely on third parties to mediate or punish defection, groups
in conflict can achieve peace by building mutual trust. Reputations for
trustworthiness are how groups overcome bargaining problems in the
absence of formal enforcement mechanisms (Kydd 2000, Rohner, Thoenig,
and Zilibotti (2013); Ostrom 2000; Ostrom and Walker 2003).\footnote{Reputations
  based on previous behavior work because each group wants to be known
  as a trustworthy partner. As a trustworthy partner, they are likely to
  (1) attract other trustworthy partners, and (2) elicit trusting
  behavior from partners. Though defecting may be beneficial in one
  specific instance, getting a reputation for defection harms a group's
  ability to achieve agreements in the future. By relying on reputation
  and the knowledge that groups desire good reputations, groups can
  coordinate in the absence of a strong third party. The reputation
  argument is generally mobilized for contexts in which many groups
  observe the behavior of each other group. Because there are many
  potential partners in the future, a good reputation has high payoff in
  the form of many potential cooperative partners in the future. In our
  case, there are two main sides forming perceptions about the
  reputation of each other. This closely mimics repeating prisoner's
  dilemma interactions as shown in Axelrod (1980a) and Axelrod (1980b).}
Mutual trust overcomes bargaining problems because information and
commitment problems are both, at their heart, problems of trust. For
information problems, groups do not trust the information they get from
the other side. For commitment problems, groups do not trust the other
side to abide by its agreements. Cultivating a reputation as a
trustworthy partner in previous interactions gives a bargaining partner
confidence that you are trustworthy in the present interaction.

However, groups in conflict are unlikely to build trusting
relationships. Intergroup conflict fuels and is fueled by intergroup
animosity, and animosity makes peace difficult to attain (Allport 1954;
Sherif 1958). Groups in conflict dehumanize the outgroup (Bandura 1999;
Haslam and Loughnan 2014; Leyens et al. 2007; Opotow 1990), view the
outgroup as innately immoral (Brewer 1999, 435; Parker and Janoff-Bulman
2013; Weisel and Böhm 2015), do not naturally feel empathy for outgroup
members (Gutsell and Inzlicht 2010), are unlikely to forgive outgroup
transgression (Tam et al. 2007), and believe outgroup members to be
untrustworthy and dishonest (Eidelson and Eidelson 2003; LeVine and
Campbell 1972). With this set of attitudes, outgroups are unlikely to
develop reputations as trustworthy partners, even if both groups prefer
mutual cooperation to war. The groups are trapped in a
mutually-defecting Prisoner's Dilemma.

\subsection{How Intergroup Animosity Perpetuates
Conflict}\label{how-intergroup-animosity-perpetuates-conflict}

Pre-existing intergroup animosity prevents peaceful resolution of
conflict in multiple ways. First, and most obviously, animosity directly
exacerbates bargaining problems . Mutual distrust creates information
and commitment problems. Animosity also changes the payoff that each
group receives from peace and war through non-material rewards and
costs, called internal evaluations (Ostrom and Walker 2003). Second,
animosity biases perceptions of ingroup and outgroup behavior and
preferences. These biases increase the likelihood that each group
misperceives the other side's intentions and prevents groups from
developing reputations for trustworthiness.

Intergroup animosity prevents peace by directly exacerbating bargaining
problems. First, information and commitment problems are more likely to
occur because groups are less likely to trust information they receive
from the other side or any peace commitment they get from the other
side. Second, animosity may change each groups preferences for peace and
war. Individual group members might receive psychological benefits from
``beating'' the other side and from social differentiation with the
outgroup (Wood 2000). Many groups define ``us'' by positive differences
with a ``them'', and an individual can derive self-esteem from
positively comparing their group identity to a rival group (Brewer 1999;
Tajfel 1981). When group members derive self-esteem from feeling
superior to an outgroup, group members may reject actions that recognize
the outgroup as equals and rhetoric about group similarity due to
cognitive dissonance. With these internal evaluations of peace and war,
any outcome in which the other side achieves \emph{any} utility could be
viewed as a loss. Groups that see the other side as immoral may even
receive some internal benefit from \emph{harming} the outgroup (Weisel
and Böhm 2015).

Along with psychological benefits, group members may receive social
benefits for strong anti-outgroup stances and social sanctioning for
cooperative behavior. The utility a group members gets for attitudes and
behaviors depends largely on how those attitudes and behaviors are
received by their ingroup (Wood 2000). If group members perceive
outgroup animosity as socially desirable, they may profess attitudes and
engage in behaviors that signal outgroup animosity. These social
\emph{benefits} also entail reciprocal social \emph{costs} that
constrain the actions of group members and group leaders. Individuals
who cooperate with the outgroup, as opposed to taking a hard stance
against the other side's perceived transgressions, might be accused of
betraying the outgroup for cooperating (Dreu 2010). Individuals in these
groups might not engage in ingroup policing, a strong, costly signal to
the other side that your group will uphold its peace agreements (Fearon
and Laitin 1996). While cooperation and ingroup policing might be
punished, aggressive actions may be seen as righteous self-defense of
the ingroup and rewarded.

Leaders are also constrained by animosity among their group. Groups are
known to punish leaders for cooperating or compromising with the
outgroup (Fearon 1994a), so the set of peace agreements available to the
leader of one group is likely unacceptable to the other.. Leaders of
hostile groups also cannot credibly commit to keep their group members
in check, as some subgroups may feel confident enough to disobey
agreements made by group leaders. Due to increased (1) likelihood of
information and commitment problems, (2) internal psychological
evaluations that favor competition over cooperation, and (3) social
sanctioning for group members and leaders perceived as betraying the
ingroup, animosity reduces the likelihood of successful bargaining and
makes violent conflict more likely.

Animosity also sabotages peace in indirect, pernicious ways. Indirectly,
intergroup animosity biases interpretations of ingroup and outgroup
behavior and prevents accurate perceptions about the attitudes and
preferences of the outgroup. Ingroups will perceive their own
belligerent actions as defensive and justified, and are more likely to
perceive outgroup actions as aggressive, negatively motivated, and
unjustified (Amir 1969; Hewstone 1990; Hunter, Stringer, and Watson 1991
chris: also cite confirmation bias, anchoring bias). The perceived
negative behavior may be seen as \emph{defining} the group, whereas any
perceived positive behavior may be seen as the \emph{exception} to the
group (Allison and Messick 1985; Pettigrew 1979). Even positive
intergroup interactions may be re-interpreted as negative to avoid
cognitive dissonance (Festinger 1962; Gubler 2013; Paolini, Harwood, and
Rubin 2010). Interpreting interactions negatively saps the power of each
group to reassure the other with costly signals of willingness to
cooperate in future interactions (Kydd 2000, Rohner, Thoenig, and
Zilibotti (2013)). It also adds to information problems as groups will
hold inaccurate views about each other's willingness to cooperate and
likelihood of upholding agreements. As Axelrod (1980a) shows, perceived
untrustworthy behavior by one side often begets a cycle of mistrust.

The indirect effect of animosity on intergroup bargaining poses a major
obstacle to groups overcoming bargaining problems through trustworthy
reputations. Groups tend to ascribe negative traits to the outgroup, and
also tend to remember negative events that corroborate their negative
beliefs (Brewer 1991; Klein and Kunda 1992; Tajfel 1981, Brewer (1999)).
Groups in conflict are given many events to justify their negative
perceptions. Initial negative perceptions, and the biased
interpretations they beget, make it very difficult for a group to
develop a positive reputation with a group they are in conflict with,
even when both groups are motivated to end the conflict. This bias
likely pushes each group's perception of the other side's willingness to
make peace further from their true willingness to make peace and so
reputations hinder, rather than aid, intergroup bargaining processes.

Reputations for trustworthiness are also hampered by a lack of
opportunities for groups to observe each others behavior and to learn
the outgroup's reputation. Compounding that problem, few of the
outgroup's interactions will be with groups that are relevant for
predicting the outgroup's behavior towards the ingroup. This means that
the main opportunity to observe outgroup behavior and learn their
reputation is your own interactions with the outgroup. For groups in
conflict, these opportunities are likely rare and almost always
adversarial.

\subsection{Intergroup Contact to Increase Trust and Reduce Conflict
(How to Increase Trust and Why it Can
Work)}\label{intergroup-contact-to-increase-trust-and-reduce-conflict-how-to-increase-trust-and-why-it-can-work}

The problems of negative intergroup attitudes suggests that improving
those attitudes could lead to peace-promoting behaviors and reduce
conflict. One approach to improving intergroup attitudes comes from
intergroup contact theory. Intergroup contact theory hypothesizes that
interactions in which group members cooperate to achieve shared goals
will improve intergroup relations. Improving relations -- especially
improving trust -- can help groups overcome bargaining problems and
reduce the likelihood of violence.

Intergroup contact theory posits several conditions necessary for
intergroup contact to improve relations. Groups must cooperate with
equal status to achieve shared goals with the support of elites.
Intergroup contact under these conditions gives the ability give strong
costly signals about a group's reputation and the ``group type'' as a
conditional complier: we will honor our agreements if you honor yours.
And about perceptual fairness/unbiasedness: we will interpret things you
do fairly and we will interpret things we do fairly (i.e.~we will not
think everything we do is justified and everything you do is
unjustified). When groups are in conflict, most incidental intergroup
contact will be highly adversarial.

Groups need opportunity to signal trustworthiness and preferences for
cooperation (Kydd 2000, Rohner, Thoenig, and Zilibotti (2013)).
Structured intergroup contact provides opportunities for these signals.
When the contact is group-to-group, trustworthy behavior by the outgroup
is immediately known by several ingroup members. The intergroup contact
to focus on superordinate goals so the groups can see how intergroup
cooperation benefits both sides materially (Sherif 1958). Sees how
\emph{both groups} have a self-interested incentive to cooperate with
peace agreements.

Ingroup identification does not necessitate outgroup hate (Allport 1954;
Brewer 1999; Halevy, Bornstein, and Sagiv 2008).

Could improve reputations/remove stereotypes. Increase perception of
trustworthiness/intergroup trust. Contact gives opportunity for costly
signals. Reputations will help groups overcome information and
commitment problems. Groups trust information they get from the other
group. Groups trust the other group to abide by agreements because they
believe the other group is also motivated by peace.

Can give group members interactions that reduce perceptual biases. Can
decrease prejudice and encourage ingroup policing. Decrease prejudice
also allows leaders to credibly commit entire group to peace. Encourage
ingroup policing. Create norms against prejudice and violence, or
connect existing norms to the outgroup by humanizing the outgroup.

Could improve dispute resolution and reduce conflict over resources.
Encourage sharing resources \emph{or} separation of resources. Mediation
training -- help community leaders resolve disputes.

Could decrease threat, though in cases of active conflict this might
exacerbate rather than solve conflict (information about reduced
outgroup threat == higher chance of my group winning).

Could increase empathy. Seeing the other side's argument, seeing their
motivations. Increase empathy/less prejudice helps interpret information
in non-biased way. {[}Chris: cite literature on empathy making me
interpret information about other people more accurately. Kertzer?{]}

Could expand ingroup/make group similarities salient. Seeing our two
groups as one, seeing our incentives as aligned.

Groups work together to gain more material resources, improves us now,
makes me think working together in the \emph{future} would be good.
Realistic Group Conflict Theory -- resource competition causes conflict,
and intergroup relations improved when the groups work together to
improve material status of both groups.

Can give group members the capacity to affect conflict.

As an alternative to top-down international intervention to reduce
conflict, bottom-up peacebuilding programs can reduce conflict by
focusing on both the immediate economic concerns contributing to
conflict (superordinate goals?) and the psychological attitudes
contributing to conflict. Bottom-up psychological reconciliation
programs should work for the types of conflicts for which international
intervention is rarely used and is unlikely to be successful. Persistent
intergroup conflict that plagues many countries. Low-level conflict
before it builds to large-scale conflict. There are XX conflicts of this
type accounting for XX deaths each year.

\subsection{Why Peacebuilding Could
Fail}\label{why-peacebuilding-could-fail}

Yet peacebuilding generally, and intergroup contact specifically, may
not work in contexts of ongoing violence. Many top-down attempts to
solve bargaining problems with a strong external actor (through
mediation and intervention) are ineffective while conflict is going on
and only effective at maintaining peace once the formal conflict has
ended. Likewise, psychological reconciliation may prevent resumption of
conflict but requires conflict to already be resolved (Bar-Tal 2000).
Active conflict produces many grievances and high prejudice; cognitive
dissonance causes backlash for the most prejudiced people (Festinger
1962; Gubler 2011). Negative contact experiences reinforce negative
stereotypes (Paolini, Harwood, and Rubin 2010). Motivated reasoning for
attitudes towards others (Klein and Kunda 1992). If you don't want to
like someone, you will find a reason not to. Many reasons contact may
not work in this context.

Intergroup conflict is commitment problem solved by strong third party
(Fearon 1994b; Powell 2006) and bottom-up programs do not provide a
strong third party that can enforce commitments. Psychological
reconciliation is not how we've thought of solving commitment problems
between conflicting groups or improving reputations. Can psych
reconciliation improve reputations in contexts of ongoing violence?

Underlying cause of conflict still present. Groups still engage in
competition for resources, which breeds conflict (Campbell 1965; Sherif
et al. 1988).

Few norms against prejudice, possibly group norms that support this
particular intergroup prejudice. If intergroup contact works by
activating ``norms against prejudice'', unlikely to work here.

Intergroup contact may only reduce prejudice from high-status group to
low-status group.

Relative deprivation theory: the groups still see the other as
responsible for ingroup deprivations.

\subsection{Hypotheses}\label{hypotheses}

\section{References}\label{references}

\hypertarget{refs}{}
\hypertarget{ref-allison1985group}{}
Allison, Scott T, and David M Messick. 1985. ``The Group Attribution
Error.'' \emph{Journal of Experimental Social Psychology} 21(6):
563--79.

\hypertarget{ref-allport1954prejudice}{}
Allport, Gordon. 1954. ``The Nature of Prejudice.'' \emph{Garden City,
NJ Anchor}.

\hypertarget{ref-amir1969contact}{}
Amir, Yehuda. 1969. ``Contact Hypothesis in Ethnic Relations.''
\emph{Psychological bulletin} 71(5): 319.

\hypertarget{ref-axelrod1980effective}{}
Axelrod, Robert. 1980a. ``Effective Choice in the Prisoner's Dilemma.''
\emph{Journal of conflict resolution} 24(1): 3--25.

\hypertarget{ref-axelrod1980more}{}
---------. 1980b. ``More Effective Choice in the Prisoner's Dilemma.''
\emph{Journal of Conflict Resolution} 24(3): 379--403.

\hypertarget{ref-bandura1999moral}{}
Bandura, Albert. 1999. ``Moral Disengagement in the Perpetration of
Inhumanities.'' \emph{Personality and social psychology review} 3(3):
193--209.

\hypertarget{ref-bar2000intractable}{}
Bar-Tal, Daniel. 2000. ``From Intractable Conflict Through Conflict
Resolution to Reconciliation: Psychological Analysis.'' \emph{Political
Psychology} 21(2): 351--65.

\hypertarget{ref-beardsley2008agreement}{}
Beardsley, Kyle. 2008. ``Agreement Without Peace? International
Mediation and Time Inconsistency Problems.'' \emph{American journal of
political science} 52(4): 723--40.

\hypertarget{ref-beber2012international}{}
Beber, Bernd. 2012. ``International Mediation, Selection Effects, and
the Question of Bias.'' \emph{Conflict Management and Peace Science}
29(4): 397--424.

\hypertarget{ref-brewer1991social}{}
Brewer, Marilynn B. 1991. ``The Social Self: On Being the Same and
Different at the Same Time.'' \emph{Personality and social psychology
bulletin} 17(5): 475--82.

\hypertarget{ref-brewer1999ingroupOutgroup}{}
---------. 1999. ``The Psychology of Prejudice: Ingroup Love and
Outgroup Hate?'' \emph{Journal of social issues} 55(3): 429--44.

\hypertarget{ref-campbell1965ethno}{}
Campbell, Donald T. 1965. ``Ethnocentric and Other Altruistic Motives.''
In \emph{Nebraska Symposium on Motivation}, 283--311.

\hypertarget{ref-crescenzi2011supply}{}
Crescenzi, Mark JC, Kelly M Kadera, Sara McLaughlin Mitchell, and
Clayton L Thyne. 2011. ``A Supply Side Theory of Mediation 1.''
\emph{International Studies Quarterly} 55(4): 1069--94.

\hypertarget{ref-di2017effectiveness}{}
Di Salvatore, Jessica, and Andrea Ruggeri. 2017. ``Effectiveness of
Peacekeeping Operations.'' \emph{Oxford Research Encyclopedia of
Politics}.

\hypertarget{ref-doyle2000international}{}
Doyle, Michael W, and Nicholas Sambanis. 2000. ``International
Peacebuilding: A Theoretical and Quantitative Analysis.'' \emph{American
political science review} 94(4): 779--801.

\hypertarget{ref-dreu2010social}{}
Dreu, Carsten KW de. 2010. ``Social Value Orientation Moderates Ingroup
Love but Not Outgroup Hate in Competitive Intergroup Conflict.''
\emph{Group Processes \& Intergroup Relations} 13(6): 701--13.

\hypertarget{ref-eidelson2003dangerous}{}
Eidelson, Roy J, and Judy I Eidelson. 2003. ``Dangerous Ideas: Five
Beliefs That Propel Groups Toward Conflict.'' \emph{American
Psychologist} 58(3): 182.

\hypertarget{ref-fearon1994domestic}{}
Fearon, James D. 1994a. ``Domestic Political Audiences and the
Escalation of International Disputes.'' \emph{American political science
review} 88(3): 577--92.

\hypertarget{ref-fearon1994ethnic}{}
---------. 1994b. ``Ethnic War as a Commitment Problem.'' In
\emph{Annual Meetings of the American Political Science Association},
2--5.

\hypertarget{ref-fearon1995rationalist}{}
---------. 1995. ``Rationalist Explanations for War.''
\emph{International organization} 49(3): 379--414.

\hypertarget{ref-fearon1998commitment}{}
---------. 1998. ``Commitment Problems and the Spread of Ethnic
Conflict.'' \emph{The international spread of ethnic conflict} 107.

\hypertarget{ref-fearon2004civil}{}
---------. 2004. ``Why Do Some Civil Wars Last so Much Longer Than
Others?'' \emph{Journal of peace research} 41(3): 275--301.

\hypertarget{ref-fearon1996explaining}{}
Fearon, James D, and David D Laitin. 1996. ``Explaining Interethnic
Cooperation.'' \emph{American political science review} 90(4): 715--35.

\hypertarget{ref-festinger1962cognitiveDissonance}{}
Festinger, Leon. 1962. 2 \emph{A Theory of Cognitive Dissonance}.
Stanford university press.

\hypertarget{ref-fey2010shuttle}{}
Fey, Mark, and Kristopher W Ramsay. 2010. ``When Is Shuttle Diplomacy
Worth the Commute? Information Sharing Through Mediation.'' \emph{World
Politics} 62(4): 529--60.

\hypertarget{ref-gubler2011diss}{}
Gubler, Joshua R. 2011. ``The Micro-Motives of Intergroup Aggression: A
Case Study in Israel.'' PhD thesis.

\hypertarget{ref-gubler2013humanizing}{}
---------. 2013. ``When Humanizing the Enemy Fails: The Role of
Dissonance and Justification in Intergroup Conflict.'' In \emph{Annual
Meeting of the American Political Science Association},

\hypertarget{ref-gutsell2010empathy}{}
Gutsell, Jennifer N, and Michael Inzlicht. 2010. ``Empathy Constrained:
Prejudice Predicts Reduced Mental Simulation of Actions During
Observation of Outgroups.'' \emph{Journal of experimental social
psychology} 46(5): 841--45.

\hypertarget{ref-halevy2008group}{}
Halevy, Nir, Gary Bornstein, and Lilach Sagiv. 2008. ```In-Group Love'
and `Out-Group Hate' as Motives for Individual Participation in
Intergroup Conflict: A New Game Paradigm.'' \emph{Psychological science}
19(4): 405--11.

\hypertarget{ref-haslam2014dehumanization}{}
Haslam, Nick, and Steve Loughnan. 2014. ``Dehumanization and
Infrahumanization.'' \emph{Annual review of psychology} 65: 399--423.

\hypertarget{ref-hewstone1990ultimate}{}
Hewstone, Miles. 1990. ``The `Ultimate Attribution Error'? A Review of
the Literature on Intergroup Causal Attribution.'' \emph{European
Journal of Social Psychology} 20(4): 311--35.

\hypertarget{ref-hunter1991intergroup}{}
Hunter, John A, Maurice Stringer, and RP Watson. 1991. ``Intergroup
Violence and Intergroup Attributions.'' \emph{British Journal of Social
Psychology} 30(3): 261--66.

\hypertarget{ref-klein1992motivated}{}
Klein, William M, and Ziva Kunda. 1992. ``Motivated Person Perception:
Constructing Justifications for Desired Beliefs.'' \emph{Journal of
experimental social psychology} 28(2): 145--68.

\hypertarget{ref-kydd2000trust}{}
Kydd, Andrew. 2000. ``Trust, Reassurance, and Cooperation.''
\emph{International Organization} 54(2): 325--57.

\hypertarget{ref-kydd2003side}{}
---------. 2003. ``Which Side Are You on? Bias, Credibility, and
Mediation.'' \emph{American Journal of Political Science} 47(4):
597--611.

\hypertarget{ref-kydd2006can}{}
Kydd, Andrew H. 2006. ``When Can Mediators Build Trust?'' \emph{American
Political Science Review} 100(3): 449--62.

\hypertarget{ref-levine1972ethnocentrism}{}
LeVine, Robert A, and Donald T Campbell. 1972. ``Ethnocentrism: Theories
of Conflict, Ethnic Attitudes, and Group Behavior.''

\hypertarget{ref-leyens2007infra}{}
Leyens, Jacques-Philippe et al. 2007. ``Infra-Humanization: The Wall of
Group Differences.'' \emph{Social Issues and Policy Review} 1(1):
139--72.

\hypertarget{ref-opotow1990moral}{}
Opotow, Susan. 1990. ``Moral Exclusion and Injustice: An Introduction.''
\emph{Journal of social issues} 46(1): 1--20.

\hypertarget{ref-ostrom2000collective}{}
Ostrom, Elinor. 2000. ``Collective Action and the Evolution of Social
Norms.'' \emph{Journal of economic perspectives} 14(3): 137--58.

\hypertarget{ref-ostrom2003trust}{}
Ostrom, Elinor, and James Walker. 2003. \emph{Trust and Reciprocity:
Interdisciplinary Lessons for Experimental Research}. Russell Sage
Foundation.

\hypertarget{ref-ott1972mediation}{}
Ott, Marvin C. 1972. ``Mediation as a Method of Conflict Resolution: Two
Cases.'' \emph{International Organization} 26(4): 595--618.

\hypertarget{ref-paolini2010negative}{}
Paolini, Stefania, Jake Harwood, and Mark Rubin. 2010. ``Negative
Intergroup Contact Makes Group Memberships Salient: Explaining Why
Intergroup Conflict Endures.'' \emph{Personality and Social Psychology
Bulletin} 36(12): 1723--38.

\hypertarget{ref-parker2013lessons}{}
Parker, Michael T, and Ronnie Janoff-Bulman. 2013. ``Lessons from
Morality-Based Social Identity: The Power of Outgroup `Hate,' Not Just
Ingroup `Love'.'' \emph{Social Justice Research} 26(1): 81--96.

\hypertarget{ref-pettigrew1979ultimate}{}
Pettigrew, Thomas F. 1979. ``The Ultimate Attribution Error: Extending
Allport's Cognitive Analysis of Prejudice.'' \emph{Personality and
social psychology bulletin} 5(4): 461--76.

\hypertarget{ref-powell2006war}{}
Powell, Robert. 2006. ``War as a Commitment Problem.''
\emph{International organization} 60(1): 169--203.

\hypertarget{ref-rauchhaus2006mediation}{}
Rauchhaus, Robert W. 2006. ``Asymmetric Information, Mediation, and
Conflict Management.'' \emph{World Politics} 58(2): 207--41.

\hypertarget{ref-reed2016bargaining}{}
Reed, William, David Clark, Timothy Nordstrom, and Daniel Siegel. 2016.
``Bargaining in the Shadow of a Commitment Problem.'' \emph{Research \&
Politics} 3(3): 2053168016666848.

\hypertarget{ref-rohner2013war}{}
Rohner, Dominic, Mathias Thoenig, and Fabrizio Zilibotti. 2013. ``War
Signals: A Theory of Trade, Trust, and Conflict.'' \emph{Review of
Economic Studies} 80(3): 1114--47.

\hypertarget{ref-savun2008information}{}
Savun, Burcu. 2008. ``Information, Bias, and Mediation Success.''
\emph{International studies quarterly} 52(1): 25--47.

\hypertarget{ref-sherif1958superordinate}{}
Sherif, Muzafer. 1958. ``Superordinate Goals in the Reduction of
Intergroup Conflict.'' \emph{American journal of Sociology} 63(4):
349--56.

\hypertarget{ref-Sherif1988robbersCave}{}
Sherif, Muzafer et al. 1988. ``Intergroup Conflict and Cooperation: The
Robbers Cave Experiment, Norman: Institute of Group Relations,
University of Oklahoma.''

\hypertarget{ref-smith2003mediation}{}
Smith, Alastair, and Allan Stam. 2003. ``Mediation and Peacekeeping in a
Random Walk Model of Civil and Interstate War.'' \emph{International
Studies Review} 5(4): 115--35.

\hypertarget{ref-svensson2009brings}{}
Svensson, Isak. 2009. ``Who Brings Which Peace? Neutral Versus Biased
Mediation and Institutional Peace Arrangements in Civil Wars.''
\emph{Journal of conflict resolution} 53(3): 446--69.

\hypertarget{ref-tajfel1981groups}{}
Tajfel, Henri. 1981. \emph{Human Groups and Social Categories: Studies
in Social Psychology}. CUP Archive.

\hypertarget{ref-tam2007impact}{}
Tam, Tania et al. 2007. ``The Impact of Intergroup Emotions on
Forgiveness in Northern Ireland.'' \emph{Group Processes \& Intergroup
Relations} 10(1): 119--36.

\hypertarget{ref-weinstein2005autonomous}{}
Weinstein, Jeremy M. 2005. ``Autonomous Recovery and International
Intervention in Comparative Perspective.'' \emph{Available at SSRN
1114117}.

\hypertarget{ref-weisel2015ingroup}{}
Weisel, Ori, and Robert Böhm. 2015. ```Ingroup Love' and `Outgroup Hate'
in Intergroup Conflict Between Natural Groups.'' \emph{Journal of
experimental social psychology} 60: 110--20.

\hypertarget{ref-wood2000attitude}{}
Wood, Wendy. 2000. ``Attitude Change: Persuasion and Social Influence.''
\emph{Annual review of psychology} 51(1): 539--70.

\end{document}
