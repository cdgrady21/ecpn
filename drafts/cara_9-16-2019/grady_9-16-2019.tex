%\documentclass[]{article}
\documentclass[11pt]{article}
\usepackage[usenames,dvipsnames]{xcolor}

\usepackage[T1]{fontenc}
%\usepackage{lmodern}
\usepackage{tgtermes}
\usepackage{amssymb,amsmath}
%\usepackage[margin=1in]{geometry}
\usepackage[letterpaper,bottom=1in,top=1in,right=1.25in,left=1.25in,includemp=FALSE]{geometry}
\usepackage{pdfpages}
\usepackage[small]{caption}

\usepackage{ifxetex,ifluatex}
\usepackage{fixltx2e} % provides \textsubscript
% use microtype if available
\IfFileExists{microtype.sty}{\usepackage{microtype}}{}
\ifnum 0\ifxetex 1\fi\ifluatex 1\fi=0 % if pdftex
\usepackage[utf8]{inputenc}
\else % if luatex or xelatex
\usepackage{fontspec}
\ifxetex
\usepackage{xltxtra,xunicode}
\fi
\defaultfontfeatures{Mapping=tex-text,Scale=MatchLowercase}
\newcommand{\euro}{€}
\fi
%

\usepackage{fancyvrb}

\usepackage{ctable,longtable}

\usepackage[section]{placeins}
\usepackage{float} % provides the H option for float placement
\restylefloat{figure}
\usepackage{dcolumn} % allows for different column alignments
\newcolumntype{.}{D{.}{.}{1.2}}

\usepackage{booktabs} % nicer horizontal rules in tables

%Assume we want graphics always
\usepackage{graphicx}
% We will generate all images so they have a width \maxwidth. This means
% that they will get their normal width if they fit onto the page, but
% are scaled down if they would overflow the margins.
%% \makeatletter
%% \def\maxwidth{\ifdim\Gin@nat@width>\linewidth\linewidth
%%   \else\Gin@nat@width\fi}
%% \makeatother
%% \let\Oldincludegraphics\includegraphics
%% \renewcommand{\includegraphics}[1]{\Oldincludegraphics[width=\maxwidth]{#1}}
\graphicspath{{.}{../Soccom_Code/socom_2013/}}


%% \ifxetex
%% \usepackage[pagebackref=true, setpagesize=false, % page size defined by xetex
%% unicode=false, % unicode breaks when used with xetex
%% xetex]{hyperref}
%% \else
\usepackage[pagebackref=true, unicode=true, bookmarks=true, pdftex]{hyperref}
% \fi


\hypersetup{breaklinks=true,
  bookmarks=true,
  pdfauthor={Christopher Grady, Rebecca Wolfe, Danjuma Dawop, and Lisa Inks},
  pdftitle={Promoting Peace Amidst Group Conflict: An Intergroup Contact Field Experiment in Nigeria},
  colorlinks=true,
  linkcolor=BrickRed,
  citecolor=blue, %MidnightBlue,
  urlcolor=BrickRed,
  % urlcolor=blue,
  % linkcolor=magenta,
  pdfborder={0 0 0}}

%\setlength{\parindent}{0pt}
%\setlength{\parskip}{6pt plus 2pt minus 1pt}
\usepackage{parskip}
\setlength{\emergencystretch}{3em}  % prevent overfull lines
\providecommand{\tightlist}{%
  \setlength{\itemsep}{0pt}\setlength{\parskip}{0pt}}

%% Insist on this.
\setcounter{secnumdepth}{2}

\VerbatimFootnotes % allows verbatim text in footnotes

\title{Promoting Peace Amidst Group Conflict: An Intergroup Contact Field
Experiment in Nigeria}

\author{
Christopher Grady, Rebecca Wolfe, Danjuma Dawop, and Lisa Inks
}


\date{September 16, 2019}


\usepackage{versions}
\makeatletter
\renewcommand*\versionmessage[2]{\typeout{*** `#1' #2. ***}}
\renewcommand*\beginmarkversion{\sffamily}
  \renewcommand*\endmarkversion{}
\makeatother

\excludeversion{comment}

%\usepackage[margins=1in]{geometry}

\usepackage[compact,bottomtitles]{titlesec}
%\titleformat{ ⟨command⟩}[⟨shape⟩]{⟨format⟩}{⟨label⟩}{⟨sep⟩}{⟨before⟩}[⟨after⟩]
\titleformat{\section}[hang]{\Large\bfseries}{\thesection}{.5em}{\hspace{0in}}[\vspace{-.2\baselineskip}]
\titleformat{\subsection}[hang]{\large\bfseries}{\thesubsection}{.5em}{\hspace{0in}}[\vspace{-.2\baselineskip}]
%\titleformat{\subsubsection}[hang]{\bfseries}{\thesubsubsection}{.5em}{\hspace{0in}}[\vspace{-.2\baselineskip}]
\titleformat{\subsubsection}[hang]{\bfseries}{\thesubsubsection}{1ex}{\hspace{0in}}[\vspace{-.2\baselineskip}]
\titleformat{\paragraph}[runin]{\bfseries\itshape}{\theparagraph}{1ex}{}{\vspace{-.2\baselineskip}}
%\titleformat{\paragraph}[runin]{\itshape}{\theparagraph}{1ex}{}{\vspace{-.2\baselineskip}}

%%\titleformat{\subsection}[hang]{\bfseries}{\thesubsection}{.5em}{\hspace{0in}}[\vspace{-.2\baselineskip}]
%%%\titleformat*{\subsection}{\bfseries\scshape}
%%%\titleformat{\subsubsection}[leftmargin]{\footnotesize\filleft}{\thesubsubsection}{.5em}{}{}
%%\titleformat{\subsubsection}[hang]{\small\bfseries}{\thesubsubsection}{.5em}{\hspace{0in}}[\vspace{-.2\baselineskip}]
%%\titleformat{\paragraph}[runin]{\itshape}{\theparagraph}{1ex}{}{\vspace{-.5\baselineskip}}

%\titlespacing*{ ⟨command⟩}{⟨left⟩}{⟨beforesep⟩}{⟨aftersep⟩}[⟨right⟩]
\titlespacing{\section}{0pc}{1.5ex plus .1ex minus .2ex}{.5ex plus .1ex minus .1ex}
\titlespacing{\subsection}{0pc}{1.5ex plus .1ex minus .2ex}{.5ex plus .1ex minus .1ex}
\titlespacing{\subsubsection}{0pc}{1.5ex plus .1ex minus .2ex}{.5ex plus .1ex minus .1ex}



%% These next lines tell latex that it is ok to have a single graphic
%% taking up most of a page, and they also decrease the space around
%% figures and tables.
\renewcommand\floatpagefraction{.9}
\renewcommand\topfraction{.9}
\renewcommand\bottomfraction{.9}
\renewcommand\textfraction{.1}
\setcounter{totalnumber}{50}
\setcounter{topnumber}{50}
\setcounter{bottomnumber}{50}
\setlength{\intextsep}{2ex}
\setlength{\floatsep}{2ex}
\setlength{\textfloatsep}{2ex}



\begin{document}
\VerbatimFootnotes

%\begin{titlepage}
%  \maketitle
%\vspace{2in}
%
%\begin{center}
%  \begin{large}
%    PROPOSAL WHITE PAPER
%
%BAA 14-013
%
%Can a Hausa Language Television Station Change Norms about Violence in Northern Nigeria? A Randomized Study of Media Effects on Violent Extremism
%
%Jake Bowers
%
%University of Illinois @ Urbana-Champaign (jwbowers@illinois.edu)
%
%\url{http://jakebowers.org}
%
%Phone: +12179792179
%
%Topic Number: 1
%
%Topic Title: Identity, Influence and Mobilization
%
%\end{large}
%\end{center}
%\end{titlepage}

\maketitle

\begin{abstract}

Group conflict should be avoidable through intergroup bargaining, yet group conflict is commonplace throughout the world.  We argue that psychological phenomena account for many bargaining failures by inducing barriers to bargaining through two psychological processes.  First, psychological phenomena prevent groups from developing intergroup trust by biasing interpretations of the behavior of the other side.  Second, psychological phenomena add costs to compromise because group members derive self-esteem from feelings of group superiority.  Interventions that abate these psychological barriers to bargaining should allow the groups to avoid violent conflict.  We test the ability of a contact-based intervention to promote peace between conflicting groups with a field experiment in Nigeria, where farmer and pastoralist communities are embroiled in a deadly conflict over land use.  We evaluate the program with surveys, direct observation of behavior, and a behavioral game.  We find that participation in the program increases intergroup contact, intergroup trust, and perceptions of physical security among group members; many of the program's effects also diffuse to group members who did not directly participate in the program but who lived alongside participants.  These results suggest that reducing barriers to bargaining between conflicting groups is possible, and that structured intergroup contact is a promising method to do so.

\end{abstract}

\hypertarget{introduction}{%
\section{Introduction}\label{introduction}}

Violent conflicts are one of most crucial phenomena for humans to
overcome. As of 2018, violent conflicts forcibly displaced over 70
million people from their homes (UNHCR statistical yearbook 2019),
caused 2 million deaths since the year 2000 (Sundberg and Melander
2013), threatened food supplies in numerous countries (Verwimp and
others 2012), and extracted a psychological toll on participants and
victims. Despite extensive research, no means of consistently preventing
violent conflict between groups is known.

To understand what causes some groups to engage in violent conflict
while others maintain peaceful relations, we join many conflict scholars
in conceptualizing group conflict as a bargaining failure (Fearon 1994b,
1995; Powell 2006). In the bargaining framework, both groups want some
resource -- land, power, etc -- and must decide how to distribute that
resource. The groups can either bargain and split the resource or groups
can fight to claim all of the resource. Fighting is costly, so both
groups are better off finding a bargained solution than fighting. Though
both groups materially benefit from a bargained peace, many peace
negotiations fail. The literature points to a lack of intergroup trust
as the primary obstacle to peace: bargaining fails if neither group
trusts the other side to be truthful or to honor bargained agreements
(Kydd 2000; Rohner, Thoenig, and Zilibotti 2013). Without a reason to
trust each other, groups are likely to fight despite the costs to both
sides.

A number of psychological phenomena complicate trust-building between
opposing groups. First, these groups hold biased perceptions of their
own behavior and the behavior of the other side (Duncan 1976; Vallone,
Ross, and Lepper 1985; Ward et al. 1997). Groups perceive their own
belligerent actions as defensive and justified, and perceive the
defensive actions of the other side as belligerent and gratuitous. These
perceptual biases create and reinforce outgroup mistrust and other
negative stereotypes (Hewstone 1990). Second, groups in conflict derive
psychological benefits from feelings of moral superiority over the
outgroup (Cikara et al. 2014; Fein and Spencer 1997; Tajfel and Turner
1979). These feelings of moral superiority add costs to improving
attitudes about the outgroup and to cooperating with the outgroup. These
two psychological phenomena sabotage intergroup bargaining by causing
the groups to have inaccurate beliefs about each other and by limiting
peace agreements acceptable to both sides.

Groups in conflict have few natural opportunities to build trust and
many to degrade it. We argue that cooperative intergroup contact
provides opportunities to build intergroup trust and can improve the
prospects for peaceful negotiation. Directly, cooperative face-to-face
contact provides groups the opportunity to send costly signals about
their trustworthiness and preference for peace (Bohnet and Frey 1999;
Kydd 2000; Lupia, McCubbins, and Arthur 1998; Rohner, Thoenig, and
Zilibotti 2013). Indirectly, cooperative contact ameliorates the
psychological biases that breed mistrust by reducing the anxiety felt
from intergroup interaction (Lee 2001; Page-Gould, Mendoza-Denton, and
Tropp 2008), increasing empathy for members of the other group
(Broockman and Kalla 2016; Pettigrew and Tropp 2008), and providing
information about the other group that replaces negative stereotypes
(Allport 1954; Pettigrew and Tropp 2008). Cooperative contact can also
demonstrate to groups the material benefits of cooperation (Gaertner et
al. 2000; Sherif 1958) and can form a superordinate identity that
encompasses both groups (Gaertner and Dovidio 2014). Cooperative contact
for some group members can diffuse throughout the entire group through
the process of indirect contact (knowledge of friendships between
ingroup and outgroup members) and changes to social norms (Crisp and
Turner 2009; Paolini et al. 2004; Sen and Airiau 2007; Yablon 2012).

To determine if cooperative contact improves intergroup attitudes, we
conducted a field experiment with conflicting farmer and pastoralist
communities in Nigeria. More than an occupational difference, farmers
who cultivate crops and pastoralists who graze cattle define a major
social cleavage in many parts of the world. These groups conflict over
land rights, which define both of their livelihoods. Farmer-pastoralist
conflict has escalated throughout the Sahel in recent years, and nowhere
more than in Nigeria. The most recent conflict escalation has caused
7,000 deaths in the past five years, displaced hundreds of thousands of
people from their homes, and costs \$13 billion annually in lost
economic productivity (Akinwotu 2018; Daniel 2018; Harwood 2019;
McDougal et al. 2015).

We randomly assigned communities with ongoing farmer-pastoralist
violence to receive a contact-based peacebuilding intervention or serve
as a control group. The intervention formed mixed-group committees and
provided them with funds to build infrastructure that would benefit both
communities; committees then collaboratively chose and constructed
infrastructure projects.\footnote{The communities built boreholes,
  market stalls, primary health care facilities, etc.} The program also
provided mediation training to each community's leaders and held forums
where the groups discussed the underlying drivers of conflict. To
measure the effects of the intervention, we conducted pre- and
post-intervention surveys, a post-intervention natural public goods
behavioral game,\footnote{In a public goods game (PGG), research
  subjects are given money and told they can keep the money or donate it
  to a public fund. Money donated to the public fund is multiplied by
  some amount and then shared with all subjects. Our PGG is
  \emph{natural} because it was conducted in a natural setting, rather
  than a lab. The funding for the PGG came from the National Science
  Foundation under Grant No.~1656871.} and twelve months of systematic
observations in markets and social events during the intervention.

We find that the program increased intergroup trust, intergroup contact,
and perceptions of physical security. We see signs of the positive
effects in fieldwork as well as in data -- in one of the treatment
sites, farmers defended pastoralists from a group of anti-pastoralist
vigilantes, rather than assist the vigilantes in removing the
pastoralists and claiming their land. Our results also show that the
intervention affected communities as a whole, not just community members
directly involved in the intergroup contact. Individuals who directly
engaged in intergroup contact changed the most positively from baseline
to endline, but we also observe positive spillovers of trust to group
members for whom we did not exogenously increase intergroup contact.

This study expands our knowledge about intergroup conflict in several
ways. First, this study teaches us about the capacity of intergroup
contact to improve intergroup relations and reduce conflict.
Peacebuilding organizations implement numerous contact-based
interventions in violent contexts each year (Ditlmann, Samii, and
Zeitzoff 2017), but its efficacy to improve intergroup attitudes amid
real-world conflict is an open question (Ditlmann, Samii, and Zeitzoff
2017; Paluck, Green, and Green 2017). To our knowledge this is the first
field experimental test of a contact-based intervention implemented
during an active conflict. Each of the groups in our study were part of
an active conflict, with members of each side being killed within one
year of the intervention's onset. We evaluated the program's effects on
both attitudinal and behavioral outcomes. The results suggest that
contact-based peacebuilding programs can effectively improve
relationships between conflicting groups and is especially relevant to
conflict resolution in the cases of intergroup and intercommunal
conflicts.

Second, we contribute to the literature about informal structures, such
as social norms, in solving collective action problems. In some
contexts, formal institutions ensure collective action by punishing
groups and individuals who ``defect'' on agreements. In many contexts,
such as rural Nigeria, no formal institutions exist to encourage such
behavior and so groups in those contexts develop informal structures to
achieve collective action (Ostrom 2000). This peacebuilding intervention
showed how informal structures governing interactions between groups can
develop through repeated intergroup interaction. Creating informal
structures that diffuse the effects of contact are a way of scaling up
peacebuilding interventions.

Third, this paper teaches us about settling disputes between sedentary
peoples and nomadic peoples. Violent conflict between settled peoples
and nomadic peoples is on the rise throughout Africa (Kuusaana and
Bukari 2015; Mwamfupe 2015; Nnoko-Mewanu 2018). This study focuses on
the Fulani, the largest semi-nomadic people on Earth (Encyclopedia
2017). Their way of life makes them targets for violence throughout
Africa. Along with this conflict in Nigeria, Fulani in Mali have been
the targets of violence so severe that researchers at Armed Conflict
Location \& Event Data Project called it ``ethnic cleansing'' (Economist
2019). Understanding how to prevent violent conflict between Fulani and
settled peoples can help prevent violence that targets other nomadic and
semi-nomadic peoples, such as the Tuaregs in West Africa, Uyghurs in
Central Asia, Kochi in Afghanistan, and Khoisan of Southern Africa.
Preventing such violence could help preserve a dying way of life.

In the next section we provide a theoretical framework for how and why
opposing groups struggle to solve their disagreements through bargaining
and negotiation, and argue that cooperative contact can help these
groups resolve disagreements by improving intergroup trust. We then
discuss Nigeria's farmer-pastoralist conflict, our experimental
intervention, and two designs to evaluate the effect of the
intervention. Last we present the results of the study and conclude by
discussing their implications for intergroup contact and group conflict.

\hypertarget{theory}{%
\section{Theory}\label{theory}}

\hypertarget{intergroup-conflict-as-a-bargaining-problem}{%
\subsection{Intergroup Conflict as a Bargaining
Problem}\label{intergroup-conflict-as-a-bargaining-problem}}

Intergroup conflict is most often conceptualized as a bargaining problem
(Fearon 1994b; Powell 2006), and most solutions to reducing intergroup
conflict strive to help the groups overcome barriers to bargaining (Di
Salvatore and Ruggeri 2017) . Intergroup conflict is a bargaining
problem because both groups want some resource -- land, power, etc --
but cannot reach an agreement about how to distribute that resource
peacefully. Because fighting is costly, the groups are better off
reaching a bargained compromise than fighting. However, two problems
prevent successful bargaining: information problems and commitment
problems (Fearon 1995). Groups are less likely to successfully bargain
without (1) accurate information about each other's strengths and
preferences, and/or (2) assurance that each side will abide by its
agreements.

An \emph{information problem} arises when neither group possesses
accurate information about the other. This problem can occur for many
reasons, but often surfaces because independent and accurate information
is unavailable and because groups have an incentive to portray
themselves as stronger, more willing to fight, and less willing to make
concessions than they truly are in order to achieve an advantageous
bargaining outcome (Fearon 1995). An information problem causes
bargaining failures because neither group knows what agreements the
other side is willing to accept or what their side should receive from
bargaining.

A \emph{commitment problem} arises when neither group can credibly
commit to honor bargained agreements. This problem can also occur for
many reasons, but often surfaces due to the potential for bargaining
power to shift after an agreement. If bargaining power shifts after an
agreement, one side will have an incentive to renege on that agreement
to achieve a better outcome. A commitment problem causes bargaining
failures because neither group will commit to agreements today that they
believe will be broken tomorrow.\footnote{The canonical example of a
  commitment problem is the prisoners' dilemma: why stay silent if you
  expect your partner to rat you out?{[}chris: should I remove this
  prisoners' dilemma sentence, footnote PD, or ignore PD?{]}}

Groups in conflict overcome information and commitment problems in two
main ways. The first way is through third-parties. Third parties can
solve information problems by mediating disputes, providing accurate
information to both sides (Kydd 2006). Third parties can also solve
commitment problems by intervening to punish defection from agreements.
Though each group may have an incentive to defect on an agreement after
it is made, the groups have less incentive to defect if a strong third
party is capable of and willing to punish defection from bargained
agreements (Fearon 1995).

The second way groups solve information and commitment problems is
through forming trustworthy reputations and the possibility of future
interactions (Kydd 2000). Reputations and future interactions can solve
information and commitment problems because dishonesty and defection
damage a group's reputation, worsening that group's prospects for future
bargaining. Groups with reputations for cooperation are likely to elicit
cooperation in future interactions; groups with reputations for
defection are likely to elicit defection in future interactions (Axelrod
and Hamilton 1981; Kydd 2000; Ostrom and Walker 2003). Though each group
may improve its current outcome by defecting on the current agreement,
the groups have less incentive to defect if defection will worsen future
bargaining outcomes. These mechanisms apply to future interactions
between the same two groups and to future interactions with other
potential bargaining partners.\footnote{We discuss repeated interactions
  and reputations together, but these mechanisms are subtly different.
  The repeated interaction mechanism is generally mobilized for contexts
  with just two groups. In those contexts, cooperation in previous
  interactions assists in obtaining cooperative behavior in the future
  from the same partner; other potential partners are unnecessary. The
  reputation mechanism is generally mobilized for contexts in which many
  groups observe the behavior of many other groups. In those contexts a
  good reputation assists in obtaining \emph{other} cooperative
  partners; repeated interactions with the same partner are unnecessary.
  We discuss these mechanisms together because both rely on creating
  cooperative expectations from bargaining partners in future
  interactions.}

\hypertarget{the-persistence-of-intergroup-conflict}{%
\subsection{The Persistence of Intergroup
Conflict}\label{the-persistence-of-intergroup-conflict}}

If scholars and practitioners know how to solve the information and
commitment problems that cause intergroup conflict, why does conflict
persist? First, it is not possible for third parties to mediate or
intervene into every conflict. Sometimes no third party is willing or
able to become involved (Fey and Ramsay 2010, 530), and sometimes the
conflict is too decentralized for a third party to effectively mediate
or intervene. This situation is common for internal conflicts in weak
states, where conflicts are diffuse and the state lacks the capacity to
mediate or intervene into the conflict. Second, third party mediation
and intervention are not effective in all contexts (Autesserre 2017;
Beardsley 2008; Beber 2012; Weinstein 2005).

Where third parties cannot effectively intervene, conflicting groups
could negotiate based on reputation, but reputations are more likely to
hinder than to help the bargaining of groups in conflict. Psychological
phenomena introduce significant friction in the ability of groups to
build trusting reputations. Group members interpret the defensive action
by outgroup members as belligerent and threatening, while they see their
own belligerent actions as defensive and justified (Duncan 1976;
Vallone, Ross, and Lepper 1985; Ward et al. 1997). Even positive actions
by outgroup members may be re-interpreted as negative to avoid cognitive
dissonance (Gubler 2013; Paolini, Harwood, and Rubin 2010). Group
members then attribute those negative outgroup behaviors to internal
characteristics that define all outgroup members (Hewstone 1990). At the
same time, they under-generalize positive behaviors as exceptional to
the outgroup and selectively remember events that corroborate their
pre-existing negative perceptions (De Dreu, Nijstad, and Knippenberg
2008). These tendencies lead group members to ascribe negative traits,
like immorality and untrustworthiness, to outgroups (Brewer 1999;
Eidelson and Eidelson 2003; LeVine and Campbell 1972; Tajfel 1981).

Biased perceptions of outgroup members prevent conflicting groups from
developing positive reputations, even when both groups are motivated to
end the conflict. The primary way that groups reassure each other of
cooperative intentions is through costly signals, but groups in conflict
may interpret these costly signals inaccurately and fail to update their
perception of the other side's willingness to cooperate (Kydd 2000;
Rohner, Thoenig, and Zilibotti 2013). Groups also reactively devalue any
compromise the outgroup is willing to concede, lowering the perceived
costliness of any cooperative signal (Ross and Stillinger 1991; Ward et
al. 1997). The inability to cultivate positive reputations through
costly signals prevents groups from forming the trusting relationships
needed to solve commitment problems. These biases also exacerbate
information problems by preventing accurate perceptions of the
preferences of the outgroup and by distorting perceptions of which group
is in the wrong and should make concessions (Ward et al. 1997). These
perceptions, once arisen, are very difficult to correct (Kunda 1990;
Nyhan and Reifler 2010).

Compounding these perceptual problems, reputations for trustworthiness
are hampered by a lack of opportunities to learn about the other side
from observing their interactions with other groups. Even if
opportunities are available, few of the outgroup's interactions will be
with groups that are relevant for predicting the outgroup's behavior
towards the ingroup, reducing the usefulness of observational learning
(Kazdin 1974; Yang 2013). The main opportunity to observe outgroup
behavior and learn their reputation is the ingroup's own interactions
with the outgroup. For groups in conflict, these opportunities are
likely rare and almost always adversarial.

If groups overcome their biased perceptions of the other side and solve
information and commitment problems, psychological phenomena provide
further barriers to the peaceful resolution of conflict. From a purely
material standpoint, groups should prefer peace to conflict because they
expect conflict to be more costly than peace. However, when forming
preferences for conflict or peace, groups also weigh psychological and
social factors. For groups in conflict, these psychological and social
factors add costs to cooperation and benefits to conflict. Factoring in
these costs and benefits, groups may be prevented from reaching a
bargain because there is no bargain that both groups will accept.

Group members suffer psychological costs from bargains that do not favor
their group at the expense of the outgroup. Group members derive
self-esteem from positively comparing their group to a rival group, and
any agreement that acknowledges both groups as equals challenges that
group-based boost to self-esteem (Brewer 1999; Fein and Spencer 1997;
Tajfel 1981; Tajfel and Turner 1979; Wood 2000). As a result, group
members are willing to accept lower absolute gains to increase relative
gain over the outgroup (Turner, Brown, and Tajfel 1979). Group members
may also receive psychological benefit from \emph{harming} the outgroup
(Cikara et al. 2014; Weisel and Böhm 2015), and a peace agreement
prohibits group members from receiving those benefits. These
psychological phenomena add costs to peace agreements.

Along with psychological factors that prevent peace, social factors push
groups towards conflict. Group members receive rewards for conforming to
group norms and punishments for violating group norms (Abrams et al.
1990; Bendor and Swistak 2001; Hogg and Reid 2006).\footnote{The rewards
  for following injunctive norms and punishments for violating them can
  come either internally, from the individual following or violating the
  norm (Abrams and Hogg 1990; Onwezen, Antonides, and Bartels 2013) or
  from other members of an individual's group (Bendor and Swistak 2001).
  Internal reward/punishment comes in the form of feelings like
  correctness, pride, and shame; external reward punishment come in the
  form of material and social status.} In the context of group
competition, group norms can encourage aggressive behaviors towards the
outgroup and deter intergroup cooperation (Dreu 2010). Group members and
leaders disposed to negotiation may find themselves unable to compromise
with the other side due to social pressure (Fearon 1994a). Groups in
active conflict might also not engage in ingroup policing, a strong,
costly signal to the other side that your group will uphold its peace
agreements (Fearon and Laitin 1996).\footnote{In addition to reducing
  the likelihood of ingroup policing, strong anti-outgroup norms might
  encourage group members to defect on peace agreements that their
  leaders have agreed to. This individual-level defection can cause
  commitment problems because leaders cannot credibly commit to control
  the aggressive behaviors of group members.} Social sanctions for
cooperative behavior, and encouragement for antagonistic behavior,
reduce the benefits and increase the costs of peace.

Group conflict is solvable by bargaining. However, group conflict
persists because psychological biases discourage intergroup trust and
color each side's perceptions of the facts of the conflict.
Psychological evaluations and social pressures also add non-material
costs to peace and non-material benefits to conflict, reducing the
number of peace agreements acceptable to both sides. Together, these
psychological barriers to peace reduce the likelihood of successful
bargaining and makes violent conflict more likely. Removing these
psychological barriers may be key for reducing group conflict. The next
section proposes structured intergroup contact as a method for removing
those barriers and promoting peace.

\hypertarget{intergroup-contact-interventions-to-improve-intergroup-attitudes-and-reduce-conflict}{%
\subsection{Intergroup Contact Interventions to Improve Intergroup
Attitudes and Reduce
Conflict}\label{intergroup-contact-interventions-to-improve-intergroup-attitudes-and-reduce-conflict}}

If poor relations between groups cause psychological barriers that
prevent groups from bargaining successfully, then improving intergroup
relations should remove those barriers and allow groups to reach
peaceful agreements through bargaining. One prominent approach to
improving relations between groups comes from intergroup contact theory
(Allport 1954). Intergroup contact theory hypothesizes that intergroup
relations can be improved through interactions in which group members
(1) cooperate (2) with equal status (3) to achieve shared goals (4) with
the support of elites. Improving relations -- especially improving trust
-- can help groups overcome information and commitment problems and
reduce the likelihood of violence.\footnote{Important to differentiate
  structured vs unstructured contact. Contact under other conditions --
  such as incidental contact, intergroup competition, interactions in
  which one group has power over the other, elite disapproval of
  intergroup contact -- is unlikely to improve intergroup relations
  (Enos 2014; Forbes 1997).{]} This concern is heightened from groups in
  conflict, whose incidental interactions are likely antagonistic.}

Structured intergroup contact improves intergroup relations for several
reasons.\footnote{A full accounting of the mechanisms through which
  intergroup contact can improve intergroup relations is beyond the
  scope of this article. We discuss the mechanisms mostly likely to
  affect prejudice for groups in active conflict. For thorough reviews,
  see Pettigrew et al. (2011), Pettigrew and Tropp (2008), or Pettigrew
  and Tropp (2013).{]} Second, intergroup contact can change emotional
  reactions to group members and reduce feelings of threat and anxiety
  Feelings of threat and anxiety often arise from fear of the unknown,
  and feelings of threat and anxiety reduce as familiarity with outgroup
  members increases (Lee 2001; Page-Gould, Mendoza-Denton, and Tropp
  2008; Paolini et al. 2004). } First, intergroup contact provides an
opportunity for groups to learn about each other and update opinions
based on personal experience, thereby dispelling stereotypes (Allport
1954). In the context of conflict resolution, the primary barrier to
peace is the stereotype that the other group is untrustworthy.
Intergroup contact provides the opportunity to signal trustworthiness
and preferences for cooperation to the other group (Kydd 2000; Rohner,
Thoenig, and Zilibotti 2013) and has been linked with increased trust
even in contexts of ongoing conflict (Hewstone et al. 2006).

Third, intergroup contact can show that working together is materially
beneficial for both groups. Intergroup contact often entails reciprocal
exchanges like trade, which can provide tangible evidence that both
groups are materially better off cooperating than fighting, increasing
incentives to reach a bargain and avoid conflict (Rohner, Thoenig, and
Zilibotti 2013).

Fourth, intergroup contact can reduce the perceived social distance
between the groups and even expand the ingroup to include the former
outgroup. Contact can make salient many similarities between the groups,
reducing feelings of social distance and even helping to craft a joint
identity that encompasses both groups (Gaertner and Dovidio 2014).

The effectiveness of intergroup contact has been demonstrated in a
variety of contexts and using a variety of methodological approaches
(Paluck, Green, and Green 2017; Pettigrew and Tropp 2006). Notably,
intergroup contacted programs improved relations between white people
and black people in the U.S. South Africa, and Norway (Burns, Corno, and
La Ferrara 2015; Carrell, Hoekstra, and West 2015; Finseraas and
Kotsadam 2017; Marmaros and Sacerdote 2006), Jews and Arabs (Ditlmann
and Samii 2016; Yablon 2012), and Hindus and Muslims in India (Barnhardt
2009). In Nigeria, a recent study found that intergroup contact between
Muslims and Christians decreased discrimination relative to a group that
experienced \emph{intragroup} contact, suggesting that intergroup
contact can work by countering the adverse effects of ingroup-only
experiences (Scacco and Warren 2018). This leads to our main hypothesis:

\begin{itemize}
\tightlist
\item
  \(\mathbf{H_1a}:\) \textbf{Intergroup contact will help.} 
\end{itemize}

But intergroup contact might not change attitudes or reduce conflict.
Group animosity often arises due to the competition for resources that
both groups claim or desire (Sherif 1958). If competition for resources
itself causes poor intergroup relations/negative intergroup attitudes,
then competition for resources will cause conflict and intergroup
contact will have no effect because it doesn't change underlying
resource competition. With or without structured intergroup contact, the
groups are still engaged in competition for resources, which breeds
conflict (Campbell 1965; Sherif et al. 1988).

This concern is heightened for groups in active conflict. Their negative
intergroup attitudes come from real conflict, not the type of prejudice
that Allport and others theorized about. Through contact, you might
learn that the other group \emph{does} favor itself. Contact might show
you that the other group \emph{will} defect on agreements. Learning
about the other side is not going to improve relations if incentives are
completely misaligned and the other side \emph{does} want to harm you.
Additionally, in contexts of ongoing conflict, there may be few norms
against prejudice, and possibly group norms that support this particular
intergroup prejudice. If intergroup contact works by activating ``norms
against prejudice'', contact is unlikely to have an effect for groups in
conflict.

This leads to our alternative hypothesis:

\begin{itemize}
\tightlist
\item
  \(\mathbf{H_1b}:\) \textbf{Intergroup contact will not help.}
\end{itemize}

The effect may also be conditional on group status. Intergroup contact
may only reduce prejudice from high-status group to low-status group.
Ditlmann and Samii (2016) contact-based intervention did not affect the
disadvantaged, minority group.

\begin{itemize}
\tightlist
\item
  \(\mathbf{H_1b}:\) \textbf{Intergroup contact will improve attitudes
  of farmers towards pastoralists but not pastoralists towards farmers.}
\end{itemize}

\hypertarget{context}{%
\section{Context}\label{context}}

Farmer-pastoralist conflict in Nigeria's Middle Belt provides a context
to learn about intergroup conflict. Nigeria's Middle Belt divides the
country between north and south, and houses a blend of various ethnic
groups, with no clear majority. The south comprises primarily Christian
farmers from various ethnic groups, while Muslims from the Hausa and
Fulani ethnic groups --- including the mostly Fulani pastoralists ---
dominate the north of the country. These religious, ethnic, and
occupational identities intersect and create deep fault lines between
communities. Historically, these communities interacted through trade
and shared access to land. However, in recent years, interconnected
factors have made these interactions more contentious. The Middle Belt
has been stressed by climate change, population booms, migration, and
government policies perceived to favor some groups over others. These
stressors have sparked violent conflict between farmers and pastoralists
in recent years.

Climate change contributes to the conflict in two ways. First, changing
climate has reduced available water sources and land conducive to
farming or grazing. As the Sahara expands southward (Thomas and Nigam
2018) and water sources dry (Okpara et al. 2015), farmers and
pastoralists have to share fewer resources. Second, changing climate
pushes more pastoralists southward into the Middle Belt.\footnote{In
  addition to the role of resource scarcity, Hoffmann and Mohammed
  (2004) also suggests that southward transhumance by Tuareg groups
  pushes Fulani pastoralists further south.} Their southward migration
of pastoralists who formerly resided in northern Nigeria causes even
more demand for land and water as those resources are becoming scarce.

Land scarcity is exacerbated by Nigeria's population boom. Nigeria's
population increased from 73 million people in 1980 to an estimated 200
million people in 2019. The rapidly growing population means increased
demand for farming and grazing land, simultaneous with the decreasing
supply of farming and grazing land. The dual pressures of increasing
demand and decreasing supply have fueled farmer-pastoralist conflict in
recent years (Unah 2018). Sharing land is easier when people are scarce
and land is plentiful; it is not so easy when land is scarce and people
are plentiful.

Land scarcity and new migrants jeopardize traditional agreements that
have managed farmer-pastoralist interactions for decades. Farmers and
pastoralists residing in Nigeria's Middle Belt developed agreements
about seasonal land sharing, exchanges for crop residue and animal
manure, compensation for damage to crops by livestock, and other
arrangements that helped them share resources and avoid conflict. As
land scarcity increased, these traditional agreements were increasingly
broken (Cotula et al. 2004; Kuusaana and Bukari 2015). These agreements
were also less likely to be implemented by the pastoralists migrating
into the Middle Belt, who were unaware of these agreements and not
involved in forging them. Their use of land beyond these traditional
agreements has further ignited tensions.

Grievances related to access land and water points are compounded by
government policies perceived to favor farmers. Land privatization
enacted in recent decades encouraged farmers to plant crops that occupy
land continuously, like orchards, and effectively nullified
farmer-pastoralist land sharing agreements (Bassett 2009). Official
cattle routes and reserves for moving herds are rarely enforced by the
government, leading farmers to plant crops in once-protected areas,
which further limits pastoralists' available grazing space. The
``indigene versus settler'' policy limits land ownership and other
rights, including political representation, to certain ethnic groups in
each state (Network 2014). Certain communities -- often pastoralists,
who are seen as ``settlers'' -- are denied the right to run for public
office, limiting the incorporation of their views into local policies.

Compounding matters, the government of Benue State enacted an anti--open
grazing law in November 2017, sparking more violence because many
pastoralists reasonably viewed the law as biased against their way of
life. Benue mobilized state-sanctioned vigilante groups called
``livestock guard'' to enforce the law, but the livestock guard have
sometimes sought out pastoralists, rather than guard farmland.{[}chris:
Duru (2018) benue police arresting livestock guard and benue courts
releasing the guard; cite other news articles about livestock guard{]}

These factors -- climate change, increasing population, migration, and
government policy -- have broken the agreements that previously governed
interactions between farmers and pastoralists. These factors have
challenged the interdependence among the groups and the benefits and
costs of reaching agreements. As farmers in Nigeria began to raise their
own livestock (Hoffmann and Mohammed 2004), the need for manure from
pastoralists decreases. As governments allocate private property to
settled people, pastoralists increasingly destroy crops when using
ancestral grazing routes. And as demand for agriculture products rise,
prices for crop destruction must increase to account for lost revenue.
These changes could seem like ``defecting'' on previous agreements,
suggesting to each side that the other is untrustworthy. Perceived
untrustworthy behavior begets a cycle of distrust, culminating in the
violent farmer-pastoralist conflict of recent years.

This persistent violence continues to have debilitating effects on
Nigerians and the economy. First, it has taken many lives. In 2013
alone, Plateau, Kaduna, Nasarawa, and Benue states registered more than
100 incidents of violent conflict, accounting for more than 1,050 deaths
(Mercy Corps 2018). The more recent violence left 300,000 displaced
(Akinwotu 2018) and more than 1,476 dead in 2018 (Harwood 2019). The
Middle Belt is considered Nigeria's ``food basket'' and is central to
key value chains throughout the country, including beef, dairy, and cash
crops such as cassava. This violence has impeded food production and
threatens to create a food shortage (Hailemariam 2018). Before the
latest surge in violence, the conflict was costing the Nigerian economy
\$13.7 billion a year (McDougal et al. 2015). As one reporter noted,
``The countryside is littered with the charred ruins of homes, schools,
police stations, mosques and churches.'' (McDonnel 2017).

Farmer-pastoralist conflict in the Middle Belt is not dissimilar from
current and past conflicts in other parts of the world. Throughout Sahel
-- farmer-pastoralist clashes are a persistent problem throughout the
Sahel and savanna areas of Africa, including Mali, the Ivory Coast
(Bassett 1988, 2009), Niger (Thebaud and Batterbury 2001), and Ghana
(Tonah 2002). Farmer-pastoralist clashes are destabilizing to these
countries politically, socially, and economically. In South Sudan,
Myanmar, Bosnia. ``Range wars'' between farmers and ranchers in 19th and
early 20th century American West. Can learn about intergroup conflict
generally from farmer-pastoralist conflict in Nigeria's Middle Belt.
These conflicts are primarily rural and the state does not project power
to these areas. It is often viewed as biased for or against one group.

\hypertarget{intervention}{%
\subsection{Intervention}\label{intervention}}

The intervention formed mixed-group committees with equal numbers of
farmers and pastoralists and provided them with funds to build
infrastructure that would benefit both communities; committees then
collaboratively chose and constructed infrastructure projects. It
started with a series of community meetings, beginning with separate
farmer and pastoralist meetings that built up to joint decision-making
meetings with the two groups together. Each joint project committee
included an even number of farmers and pastoralists, as well as women
and youth representatives, and totaled between 12 and 15 members. Each
committee received two grants, one for quick-impact projects, of
approximately \$2,000, and one for joint projects, of approximately
\$25,000.

The quick-impact projects were conceived as a trust-building initiative,
intended to let community members see that cooperation was possible.
Projects, managed by both farmers and pastoralists, included hand pumps,
construction or rehabilitation of market stalls and schools,
rehabilitation of health centers, and construction of fences along
grazing routes to protect farmlands. The joint economic development
projects aimed to address an underlying issue related to the conflict:
sharing of resources that impact livelihoods. Pollution of water,
affecting both farming and livestock, was the primary issue people
raised. As a result, each site received a new borehole well, with farmer
and pastoralist youth helping to construct the wells.

The program also provided mediation training to each community's leaders
and held forums where the groups discussed the underlying drivers of
conflict.

These projects were designed with the conditions of Contact Theory in
mind. Groups (1) cooperated with (2) equal status to achieve (3) shared
goals with (4) support of local authorities. These projects were meant
to help the groups solve some underlying resource problems that drove
conflict. Collectively, these project give groups the opportunity so
send costly signals about their willingness to cooperate (Kydd 2000;
Rohner, Thoenig, and Zilibotti 2013).

\hypertarget{references}{%
\section*{References}\label{references}}
\addcontentsline{toc}{section}{References}

\hypertarget{refs}{}
\leavevmode\hypertarget{ref-abrams1990social}{}%
Abrams, Dominic, and Michael A Hogg. 1990. ``Social Identification,
Self-Categorization and Social Influence.'' \emph{European review of
social psychology} 1(1): 195--228.

\leavevmode\hypertarget{ref-abrams1990knowing}{}%
Abrams, Dominic et al. 1990. ``Knowing What to Think by Knowing Who You
Are: Self-Categorization and the Nature of Norm Formation, Conformity
and Group Polarization.'' \emph{British Journal of Social Psychology}
29(2): 97--119.

\leavevmode\hypertarget{ref-nyt2018nigeria}{}%
Akinwotu, Emmanuel. 2018. ``Nigeria's Farmers and Herders Fight a Deadly
Battle for Scarce Resources.'' \emph{New York Times}.
\url{https://www.nytimes.com/2018/06/25/world/africa/nigeria-herders-farmers.html}.

\leavevmode\hypertarget{ref-allport1954prejudice}{}%
Allport, Gordon. 1954. ``The Nature of Prejudice.'' \emph{Garden City,
NJ Anchor}.

\leavevmode\hypertarget{ref-autesserre2017international}{}%
Autesserre, Severine. 2017. ``International Peacebuilding and Local
Success: Assumptions and Effectiveness.'' \emph{International Studies
Review} 19(1): 114--32.

\leavevmode\hypertarget{ref-axelrod1981evolution}{}%
Axelrod, Robert, and William D Hamilton. 1981. ``The Evolution of
Cooperation.'' \emph{science} 211(4489): 1390--6.

\leavevmode\hypertarget{ref-barnhardt2009near}{}%
Barnhardt, Sharon. 2009. ``Near and Dear? Evaluating the Impact of
Neighbor Diversity on Inter-Religious Attitudes.'' \emph{Unpublished
working paper}.

\leavevmode\hypertarget{ref-bassett2009mobile}{}%
Bassett, Thomas J. 2009. ``Mobile Pastoralism on the Brink of Land
Privatization in Northern côte d'Ivoire.'' \emph{Geoforum} 40(5):
756--66.

\leavevmode\hypertarget{ref-beardsley2008agreement}{}%
Beardsley, Kyle. 2008. ``Agreement Without Peace? International
Mediation and Time Inconsistency Problems.'' \emph{American journal of
political science} 52(4): 723--40.

\leavevmode\hypertarget{ref-beber2012international}{}%
Beber, Bernd. 2012. ``International Mediation, Selection Effects, and
the Question of Bias.'' \emph{Conflict Management and Peace Science}
29(4): 397--424.

\leavevmode\hypertarget{ref-bendor2001evolution}{}%
Bendor, Jonathan, and Piotr Swistak. 2001. ``The Evolution of Norms.''
\emph{American Journal of Sociology} 106(6): 1493--1545.

\leavevmode\hypertarget{ref-bohnet1999social}{}%
Bohnet, Iris, and Bruno S Frey. 1999. ``Social Distance and
Other-Regarding Behavior in Dictator Games: Comment.'' \emph{American
Economic Review} 89(1): 335--39.

\leavevmode\hypertarget{ref-brewer1999ingroupOutgroup}{}%
Brewer, Marilynn B. 1999. ``The Psychology of Prejudice: Ingroup Love
and Outgroup Hate?'' \emph{Journal of social issues} 55(3): 429--44.

\leavevmode\hypertarget{ref-broockman2016durably}{}%
Broockman, David, and Joshua Kalla. 2016. ``Durably Reducing
Transphobia: A Field Experiment on Door-to-Door Canvassing.''
\emph{Science} 352(6282): 220--24.

\leavevmode\hypertarget{ref-burns2015interaction}{}%
Burns, Justine, Lucia Corno, and Eliana La Ferrara. 2015.
\emph{Interaction, Prejudice and Performance. Evidence from South
Africa}. Working paper.

\leavevmode\hypertarget{ref-campbell1965ethno}{}%
Campbell, Donald T. 1965. ``Ethnocentric and Other Altruistic Motives.''
In \emph{Nebraska Symposium on Motivation}, 283--311.

\leavevmode\hypertarget{ref-carrell2015impact}{}%
Carrell, Scott E, Mark Hoekstra, and James E West. 2015. \emph{The
Impact of Intergroup Contact on Racial Attitudes and Revealed
Preferences}. National Bureau of Economic Research.

\leavevmode\hypertarget{ref-cikara2014their}{}%
Cikara, Mina, Emile Bruneau, Jay J Van Bavel, and Rebecca Saxe. 2014.
``Their Pain Gives Us Pleasure: How Intergroup Dynamics Shape Empathic
Failures and Counter-Empathic Responses.'' \emph{Journal of experimental
social psychology} 55: 110--25.

\leavevmode\hypertarget{ref-cotula2004land}{}%
Cotula, Lorenzo, Camilla Toulmin, Ced Hesse, and others. 2004.
\emph{Land Tenure and Administration in Africa: Lessons of Experience
and Emerging Issues}. International Institute for Environment;
Development London.

\leavevmode\hypertarget{ref-crisp2009imagined}{}%
Crisp, Richard J, and Rhiannon N Turner. 2009. ``Can Imagined
Interactions Produce Positive Perceptions?: Reducing Prejudice Through
Simulated Social Contact.'' \emph{American Psychologist} 64(4): 231.

\leavevmode\hypertarget{ref-daniel2018anti}{}%
Daniel, Soni. 2018. ``Anti-Open Grazing Law: Nass, Benue, Kwara, Taraba
Tackle Defence Minister.'' \emph{Vanguard}.
\url{https://www.vanguardngr.com/2018/06/anti-open-grazing-law-nass-benue-kwara-taraba-tackle-defence-minister/}.

\leavevmode\hypertarget{ref-de2008motivated}{}%
De Dreu, Carsten KW, Bernard A Nijstad, and Daan van Knippenberg. 2008.
``Motivated Information Processing in Group Judgment and Decision
Making.'' \emph{Personality and Social Psychology Review} 12(1): 22--49.

\leavevmode\hypertarget{ref-di2017effectiveness}{}%
Di Salvatore, Jessica, and Andrea Ruggeri. 2017. ``Effectiveness of
Peacekeeping Operations.'' \emph{Oxford Research Encyclopedia of
Politics}.

\leavevmode\hypertarget{ref-ditlmann2016can}{}%
Ditlmann, Ruth K, and Cyrus Samii. 2016. ``Can Intergroup Contact Affect
Ingroup Dynamics? Insights from a Field Study with Jewish and
Arab-Palestinian Youth in Israel.'' \emph{Peace and Conflict: Journal of
Peace Psychology} 22(4): 380.

\leavevmode\hypertarget{ref-ditlmann2017addressing}{}%
Ditlmann, Ruth K, Cyrus Samii, and Thomas Zeitzoff. 2017. ``Addressing
Violent Intergroup Conflict from the Bottom up?'' \emph{Social Issues
and Policy Review} 11(1): 38--77.

\leavevmode\hypertarget{ref-dreu2010social}{}%
Dreu, Carsten KW de. 2010. ``Social Value Orientation Moderates Ingroup
Love but Not Outgroup Hate in Competitive Intergroup Conflict.''
\emph{Group Processes \& Intergroup Relations} 13(6): 701--13.

\leavevmode\hypertarget{ref-duncan1976differential}{}%
Duncan, Birt L. 1976. ``Differential Social Perception and Attribution
of Intergroup Violence: Testing the Lower Limits of Stereotyping of
Blacks.'' \emph{Journal of personality and social psychology} 34(4):
590.

\leavevmode\hypertarget{ref-duru2018court}{}%
Duru, Peter. 2018. ``Court Stops Inspector General from Proscribing
Benue Livestock Guard.'' \emph{Vanguard}.
\url{https://www.vanguardngr.com/2018/11/court-stops-ig-from-proscribing-benue-livestock-guards/}.

\leavevmode\hypertarget{ref-economist2019militias}{}%
Economist, The. 2019. ``Malicious Malitias: States in the Sahel Have
Unleashed Ethnic Gangs with Guns.'' \emph{The Economist}.
\url{https://www.economist.com/middle-east-and-africa/2019/05/04/states-in-the-sahel-have-unleashed-ethnic-gangs-with-guns}.

\leavevmode\hypertarget{ref-eidelson2003dangerous}{}%
Eidelson, Roy J, and Judy I Eidelson. 2003. ``Dangerous Ideas: Five
Beliefs That Propel Groups Toward Conflict.'' \emph{American
Psychologist} 58(3): 182.

\leavevmode\hypertarget{ref-fulanisize2017}{}%
Encyclopedia, New World. 2017. ``Fulani --- New World Encyclopedia,''
\url{//www.newworldencyclopedia.org/p/index.php?title=Fulani\&oldid=1004777}.

\leavevmode\hypertarget{ref-enos2014causal}{}%
Enos, Ryan D. 2014. ``Causal Effect of Intergroup Contact on
Exclusionary Attitudes.'' \emph{Proceedings of the National Academy of
Sciences} 111(10): 3699--3704.

\leavevmode\hypertarget{ref-fearon1994domestic}{}%
Fearon, James D. 1994a. ``Domestic Political Audiences and the
Escalation of International Disputes.'' \emph{American political science
review} 88(3): 577--92.

\leavevmode\hypertarget{ref-fearon1994ethnic}{}%
---------. 1994b. ``Ethnic War as a Commitment Problem.'' In
\emph{Annual Meetings of the American Political Science Association},
2--5.

\leavevmode\hypertarget{ref-fearon1995rationalist}{}%
---------. 1995. ``Rationalist Explanations for War.''
\emph{International organization} 49(3): 379--414.

\leavevmode\hypertarget{ref-fearon1996explaining}{}%
Fearon, James D, and David D Laitin. 1996. ``Explaining Interethnic
Cooperation.'' \emph{American political science review} 90(4): 715--35.

\leavevmode\hypertarget{ref-fein1997prejudice}{}%
Fein, Steven, and Steven J Spencer. 1997. ``Prejudice as Self-Image
Maintenance: Affirming the Self Through Derogating Others.''
\emph{Journal of personality and Social Psychology} 73(1): 31.

\leavevmode\hypertarget{ref-fey2010shuttle}{}%
Fey, Mark, and Kristopher W Ramsay. 2010. ``When Is Shuttle Diplomacy
Worth the Commute? Information Sharing Through Mediation.'' \emph{World
Politics} 62(4): 529--60.

\leavevmode\hypertarget{ref-finseraas2017does}{}%
Finseraas, Henning, and Andreas Kotsadam. 2017. ``Does Personal Contact
with Ethnic Minorities Affect Anti-Immigrant Sentiments? Evidence from a
Field Experiment.'' \emph{European Journal of Political Research} 56(3):
703--22.

\leavevmode\hypertarget{ref-forbes1997ethnic}{}%
Forbes, Hugh Donald. 1997. \emph{Ethnic Conflict: Commerce, Culture, and
the Contact Hypothesis}. Yale University Press.

\leavevmode\hypertarget{ref-gaertner2014reducing}{}%
Gaertner, Samuel L, and John F Dovidio. 2014. \emph{Reducing Intergroup
Bias: The Common Ingroup Identity Model}. Psychology Press.

\leavevmode\hypertarget{ref-gaertner2000reducing}{}%
Gaertner, Samuel L et al. 2000. ``Reducing Intergroup Conflict: From
Superordinate Goals to Decategorization, Recategorization, and Mutual
Differentiation.'' \emph{Group Dynamics: Theory, Research, and Practice}
4(1): 98.

\leavevmode\hypertarget{ref-gubler2013humanizing}{}%
Gubler, Joshua R. 2013. ``When Humanizing the Enemy Fails: The Role of
Dissonance and Justification in Intergroup Conflict.'' In \emph{Annual
Meeting of the American Political Science Association},

\leavevmode\hypertarget{ref-frontera2018nigeria}{}%
Hailemariam, Adium. 2018. ``Nigeria: Violence in the Middle Belt Becomes
Major Concern for President Buhari.'' \emph{Frontera}.
\url{https://frontera.net/news/africa/nigeria-violence-in-the-middle-belt-becomes-major-concern-for-president-buhari/}.

\leavevmode\hypertarget{ref-council2019nigeria}{}%
Harwood, Asch. 2019. ``Update: The Numbers Behind Sectarian Violence in
Nigeria.'' \emph{Council on Foreign Relations}.
\url{https://www.cfr.org/blog/update-numbers-behind-sectarian-violence-nigeria}.

\leavevmode\hypertarget{ref-hewstone1990ultimate}{}%
Hewstone, Miles. 1990. ``The `Ultimate Attribution Error'? A Review of
the Literature on Intergroup Causal Attribution.'' \emph{European
Journal of Social Psychology} 20(4): 311--35.

\leavevmode\hypertarget{ref-hewstone2006intergroup}{}%
Hewstone, Miles et al. 2006. ``Intergroup Contact, Forgiveness, and
Experience of `the Troubles' in Northern Ireland.'' \emph{Journal of
Social Issues} 62(1): 99--120.

\leavevmode\hypertarget{ref-hoffmann2004role}{}%
Hoffmann, Irene, and Isiaka Mohammed. 2004. ``THE Role of Nomadic Camels
for Manuring Farmers'FIELDS in the Sokoto Close Settled Zone, Northwest
Nigeria.'' \emph{Nomadic Peoples} 8(1): 99--112.

\leavevmode\hypertarget{ref-hogg2006social}{}%
Hogg, Michael A, and Scott A Reid. 2006. ``Social Identity,
Self-Categorization, and the Communication of Group Norms.''
\emph{Communication theory} 16(1): 7--30.

\leavevmode\hypertarget{ref-kazdin1974covertModeling}{}%
Kazdin, Alan E. 1974. ``Covert Modeling, Model Similarity, and Reduction
of Avoidance Behavior.'' \emph{Behavior Therapy} 5(3): 325--40.

\leavevmode\hypertarget{ref-kunda1990motivatedReasoning}{}%
Kunda, Ziva. 1990. ``The Case for Motivated Reasoning.''
\emph{Psychological bulletin} 108(3): 480.

\leavevmode\hypertarget{ref-kuusaana2015land}{}%
Kuusaana, Elias Danyi, and Kaderi Noagah Bukari. 2015. ``Land Conflicts
Between Smallholders and Fulani Pastoralists in Ghana: Evidence from the
Asante Akim North District (Aand).'' \emph{Journal of rural studies} 42:
52--62.

\leavevmode\hypertarget{ref-kydd2000trust}{}%
Kydd, Andrew. 2000. ``Trust, Reassurance, and Cooperation.''
\emph{International Organization} 54(2): 325--57.

\leavevmode\hypertarget{ref-kydd2006can}{}%
Kydd, Andrew H. 2006. ``When Can Mediators Build Trust?'' \emph{American
Political Science Review} 100(3): 449--62.

\leavevmode\hypertarget{ref-lee2001mere}{}%
Lee, Angela Y. 2001. ``The Mere Exposure Effect: An Uncertainty
Reduction Explanation Revisited.'' \emph{Personality and Social
Psychology Bulletin} 27(10): 1255--66.

\leavevmode\hypertarget{ref-levine1972ethnocentrism}{}%
LeVine, Robert A, and Donald T Campbell. 1972. ``Ethnocentrism: Theories
of Conflict, Ethnic Attitudes, and Group Behavior.''

\leavevmode\hypertarget{ref-lupia1998democratic}{}%
Lupia, Arthur, Mathew D McCubbins, and Lupia Arthur. 1998. \emph{The
Democratic Dilemma: Can Citizens Learn What They Need to Know?}
Cambridge University Press.

\leavevmode\hypertarget{ref-marmaros2006friendships}{}%
Marmaros, David, and Bruce Sacerdote. 2006. ``How Do Friendships Form?''
\emph{The Quarterly Journal of Economics} 121(1): 79--119.

\leavevmode\hypertarget{ref-mcdonnel2017graze}{}%
McDonnel, Tim. 2017. ``Why It's Now a Crime to Let Cattle Graze Freely
in 2 Nigerian States.'' \emph{National Public Radio (NPR)}.
\url{https://www.npr.org/sections/goatsandsoda/2017/12/12/569913821/why-its-now-a-crime-to-let-cattle-graze-freely-in-2-nigerian-states}.

\leavevmode\hypertarget{ref-mcdougal2015effect}{}%
McDougal, Topher L et al. 2015. ``The Effect of Farmer-Pastoralist
Violence on Income: New Survey Evidence from Nigeria's Middle Belt
States.'' \emph{Economics of Peace and Security Journal} 10(1): 54--65.

\leavevmode\hypertarget{ref-mwamfupe2015persistence}{}%
Mwamfupe, Davis. 2015. ``Persistence of Farmer-Herder Conflicts in
Tanzania.'' \emph{International Journal of Scientific and Research
Publications} 5(2): 1--8.

\leavevmode\hypertarget{ref-nigeria2014freedom}{}%
Network, Nigeria Research. 2014. ``Indigeneity, Belonging, and Religious
Freedom in Nigeria: Citizens' Views from the Street.'' \emph{5. NRN
Policy Brief}.

\leavevmode\hypertarget{ref-hrc2018farmer}{}%
Nnoko-Mewanu, Juliana. 2018. ``Farmer-Herder Conflicts on the Rise in
Africa.'' \emph{Human Rights Watch}.

\leavevmode\hypertarget{ref-nyhan2010corrections}{}%
Nyhan, Brendan, and Jason Reifler. 2010. ``When Corrections Fail: The
Persistence of Political Misperceptions.'' \emph{Political Behavior}
32(2): 303--30.

\leavevmode\hypertarget{ref-okpara2015conflicts}{}%
Okpara, Uche T, Lindsay C Stringer, Andrew J Dougill, and Mohammed D
Bila. 2015. ``Conflicts About Water in Lake Chad: Are Environmental,
Vulnerability and Security Issues Linked?'' \emph{Progress in
Development Studies} 15(4): 308--25.

\leavevmode\hypertarget{ref-onwezen2013norm}{}%
Onwezen, Marleen C, Gerrit Antonides, and Jos Bartels. 2013. ``The Norm
Activation Model: An Exploration of the Functions of Anticipated Pride
and Guilt in Pro-Environmental Behaviour.'' \emph{Journal of Economic
Psychology} 39: 141--53.

\leavevmode\hypertarget{ref-ostrom2000collective}{}%
Ostrom, Elinor. 2000. ``Collective Action and the Evolution of Social
Norms.'' \emph{Journal of economic perspectives} 14(3): 137--58.

\leavevmode\hypertarget{ref-ostrom2003trust}{}%
Ostrom, Elinor, and James Walker. 2003. \emph{Trust and Reciprocity:
Interdisciplinary Lessons for Experimental Research}. Russell Sage
Foundation.

\leavevmode\hypertarget{ref-page2008little}{}%
Page-Gould, Elizabeth, Rodolfo Mendoza-Denton, and Linda R Tropp. 2008.
``With a Little Help from My Cross-Group Friend: Reducing Anxiety in
Intergroup Contexts Through Cross-Group Friendship.'' \emph{Journal of
personality and social psychology} 95(5): 1080.

\leavevmode\hypertarget{ref-paluck2017contact}{}%
Paluck, Elizabeth Levy, Seth Green, and Donald P Green. 2017. ``The
Contact Hypothesis Revisited.''

\leavevmode\hypertarget{ref-paolini2010negative}{}%
Paolini, Stefania, Jake Harwood, and Mark Rubin. 2010. ``Negative
Intergroup Contact Makes Group Memberships Salient: Explaining Why
Intergroup Conflict Endures.'' \emph{Personality and Social Psychology
Bulletin} 36(12): 1723--38.

\leavevmode\hypertarget{ref-paolini2004effects}{}%
Paolini, Stefania, Miles Hewstone, Ed Cairns, and Alberto Voci. 2004.
``Effects of Direct and Indirect Cross-Group Friendships on Judgments of
Catholics and Protestants in Northern Ireland: The Mediating Role of an
Anxiety-Reduction Mechanism.'' \emph{Personality and Social Psychology
Bulletin} 30(6): 770--86.

\leavevmode\hypertarget{ref-pettigrew2006meta}{}%
Pettigrew, Thomas F, and Linda R Tropp. 2006. ``A Meta-Analytic Test of
Intergroup Contact Theory.'' \emph{Journal of personality and social
psychology} 90(5): 751.

\leavevmode\hypertarget{ref-pettigrew2008does}{}%
---------. 2008. ``How Does Intergroup Contact Reduce Prejudice?
Meta-Analytic Tests of Three Mediators.'' \emph{European Journal of
Social Psychology} 38(6): 922--34.

\leavevmode\hypertarget{ref-pettigrew2013groups}{}%
---------. 2013. \emph{When Groups Meet: The Dynamics of Intergroup
Contact}. psychology press.

\leavevmode\hypertarget{ref-pettigrew2011advances}{}%
Pettigrew, Thomas F, Linda R Tropp, Ulrich Wagner, and Oliver Christ.
2011. ``Recent Advances in Intergroup Contact Theory.''
\emph{International Journal of Intercultural Relations} 35(3): 271--80.

\leavevmode\hypertarget{ref-powell2006war}{}%
Powell, Robert. 2006. ``War as a Commitment Problem.''
\emph{International organization} 60(1): 169--203.

\leavevmode\hypertarget{ref-rohner2013war}{}%
Rohner, Dominic, Mathias Thoenig, and Fabrizio Zilibotti. 2013. ``War
Signals: A Theory of Trade, Trust, and Conflict.'' \emph{Review of
Economic Studies} 80(3): 1114--47.

\leavevmode\hypertarget{ref-ross1991barriers}{}%
Ross, Lee, and Constance Stillinger. 1991. ``Barriers to Conflict
Resolution.'' \emph{Negotiation journal} 7(4): 389--404.

\leavevmode\hypertarget{ref-scacco2018nigeria}{}%
Scacco, Alexandra, and Shana S Warren. 2018. ``Can Social Contact Reduce
Prejudice and Discrimination? Evidence from a Field Experiment in
Nigeria.'' \emph{American Political Science Review} 112(3): 654--77.

\leavevmode\hypertarget{ref-sen2007emergence}{}%
Sen, Sandip, and Stéphane Airiau. 2007. ``Emergence of Norms Through
Social Learning.'' In \emph{IJCAI}, 1512.

\leavevmode\hypertarget{ref-sherif1958superordinate}{}%
Sherif, Muzafer. 1958. ``Superordinate Goals in the Reduction of
Intergroup Conflict.'' \emph{American journal of Sociology} 63(4):
349--56.

\leavevmode\hypertarget{ref-Sherif1988robbersCave}{}%
Sherif, Muzafer et al. 1988. ``Intergroup Conflict and Cooperation: The
Robbers Cave Experiment, Norman: Institute of Group Relations,
University of Oklahoma.''

\leavevmode\hypertarget{ref-ucdp}{}%
Sundberg, Ralph, and Erik Melander. 2013. ``Introducing the Ucdp
Georeferenced Event Dataset.'' \emph{Journal of Peace Research} 50(4):
523--32.

\leavevmode\hypertarget{ref-tajfel1981groups}{}%
Tajfel, Henri. 1981. \emph{Human Groups and Social Categories: Studies
in Social Psychology}. CUP Archive.

\leavevmode\hypertarget{ref-tajfel1979integrative}{}%
Tajfel, Henri, and John C Turner. 1979. ``An Integrative Theory of
Intergroup Conflict.'' \emph{The social psychology of intergroup
relations} 33(47): 74.

\leavevmode\hypertarget{ref-thomas2018sahara}{}%
Thomas, Natalie, and Sumant Nigam. 2018. ``Twentieth-Century Climate
Change over Africa: Seasonal Hydroclimate Trends and Sahara Desert
Expansion.'' \emph{Journal of Climate} 31(9): 3349--70.

\leavevmode\hypertarget{ref-turner1979social}{}%
Turner, John C, Rupert J Brown, and Henri Tajfel. 1979. ``Social
Comparison and Group Interest in Ingroup Favouritism.'' \emph{European
journal of social psychology} 9(2): 187--204.

\leavevmode\hypertarget{ref-unah2018nigeria}{}%
Unah, Linus. 2018. ``In Nigeria's Diverse Middle Belt, a Drying
Landscape Deepens Violent Divides.'' \emph{Christian Science Minitor}.

\leavevmode\hypertarget{ref-unhcr2019}{}%
\emph{UNHCR Statistical Yearbook}. 2019.
https://www.unhcr.org/en-us/figures-at-a-glance.html: United Nations
High Commission for Refugees.

\leavevmode\hypertarget{ref-vallone1985hostileMedia}{}%
Vallone, Robert P, Lee Ross, and Mark R Lepper. 1985. ``The Hostile
Media Phenomenon: Biased Perception and Perceptions of Media Bias in
Coverage of the Beirut Massacre.'' \emph{Journal of personality and
social psychology} 49(3): 577.

\leavevmode\hypertarget{ref-verwimp2012food}{}%
Verwimp, Philip, and others. 2012. ``Food Security, Violent Conflict and
Human Development: Causes and Consequences.'' \emph{United Nations
Development Programme Working Paper}: 1--13.

\leavevmode\hypertarget{ref-ward1997naive}{}%
Ward, Andrew et al. 1997. ``Naive Realism in Everyday Life: Implications
for Social Conflict and Misunderstanding.'' \emph{Values and knowledge}:
103--35.

\leavevmode\hypertarget{ref-weinstein2005autonomous}{}%
Weinstein, Jeremy M. 2005. ``Autonomous Recovery and International
Intervention in Comparative Perspective.'' \emph{Available at SSRN
1114117}.

\leavevmode\hypertarget{ref-weisel2015ingroup}{}%
Weisel, Ori, and Robert Böhm. 2015. ```Ingroup Love' and `Outgroup Hate'
in Intergroup Conflict Between Natural Groups.'' \emph{Journal of
experimental social psychology} 60: 110--20.

\leavevmode\hypertarget{ref-wood2000attitude}{}%
Wood, Wendy. 2000. ``Attitude Change: Persuasion and Social Influence.''
\emph{Annual review of psychology} 51(1): 539--70.

\leavevmode\hypertarget{ref-yablon2012we}{}%
Yablon, Yaacov B. 2012. ``Are We Preaching to the Converted? The Role of
Motivation in Understanding the Contribution of Intergroup Encounters.''
\emph{Journal of Peace Education} 9(3): 249--63.

\leavevmode\hypertarget{ref-yang2013similarity}{}%
Yang, Nianhua. 2013. ``A Similarity Based Trust and Reputation
Management Framework for Vanets.'' \emph{International Journal of Future
Generation Communication and Networking} 6(2): 25--34.

\end{document}
