%\documentclass[]{article}
\documentclass[11pt]{article}
\usepackage[usenames,dvipsnames]{xcolor}

\usepackage[T1]{fontenc}
%\usepackage{lmodern}
\usepackage{tgtermes}
\usepackage{amssymb,amsmath}
%\usepackage[margin=1in]{geometry}
\usepackage[letterpaper,bottom=1in,top=1in,right=1.25in,left=1.25in,includemp=FALSE]{geometry}
\usepackage{pdfpages}
\usepackage[small]{caption}

\usepackage{ifxetex,ifluatex}
\usepackage{fixltx2e} % provides \textsubscript
% use microtype if available
\IfFileExists{microtype.sty}{\usepackage{microtype}}{}
\ifnum 0\ifxetex 1\fi\ifluatex 1\fi=0 % if pdftex
\usepackage[utf8]{inputenc}
\else % if luatex or xelatex
\usepackage{fontspec}
\ifxetex
\usepackage{xltxtra,xunicode}
\fi
\defaultfontfeatures{Mapping=tex-text,Scale=MatchLowercase}
\newcommand{\euro}{€}
\fi
%

\usepackage{fancyvrb}

\usepackage{ctable,longtable}

\usepackage[section]{placeins}
\usepackage{float} % provides the H option for float placement
\restylefloat{figure}
\usepackage{dcolumn} % allows for different column alignments
\newcolumntype{.}{D{.}{.}{1.2}}

\usepackage{booktabs} % nicer horizontal rules in tables

%Assume we want graphics always
\usepackage{graphicx}
% We will generate all images so they have a width \maxwidth. This means
% that they will get their normal width if they fit onto the page, but
% are scaled down if they would overflow the margins.
%% \makeatletter
%% \def\maxwidth{\ifdim\Gin@nat@width>\linewidth\linewidth
%%   \else\Gin@nat@width\fi}
%% \makeatother
%% \let\Oldincludegraphics\includegraphics
%% \renewcommand{\includegraphics}[1]{\Oldincludegraphics[width=\maxwidth]{#1}}
\graphicspath{{.}{../Soccom_Code/socom_2013/}}


%% \ifxetex
%% \usepackage[pagebackref=true, setpagesize=false, % page size defined by xetex
%% unicode=false, % unicode breaks when used with xetex
%% xetex]{hyperref}
%% \else
\usepackage[pagebackref=true, unicode=true, bookmarks=true, pdftex]{hyperref}
% \fi


\hypersetup{breaklinks=true,
  bookmarks=true,
  pdfauthor={Christopher Grady, Rebecca Wolfe, Danjuma Dawop, and Lisa Inks},
  pdftitle={Promoting Peace Amidst Group Conflict: An Intergroup Contact Field Experiment in Nigeria},
  colorlinks=true,
  linkcolor=BrickRed,
  citecolor=blue, %MidnightBlue,
  urlcolor=BrickRed,
  % urlcolor=blue,
  % linkcolor=magenta,
  pdfborder={0 0 0}}

%\setlength{\parindent}{0pt}
%\setlength{\parskip}{6pt plus 2pt minus 1pt}
\usepackage{parskip}
\setlength{\emergencystretch}{3em}  % prevent overfull lines
\providecommand{\tightlist}{%
  \setlength{\itemsep}{0pt}\setlength{\parskip}{0pt}}

%% Insist on this.
\setcounter{secnumdepth}{2}

\VerbatimFootnotes % allows verbatim text in footnotes

\title{Promoting Peace Amidst Group Conflict: An Intergroup Contact Field
Experiment in Nigeria}

\author{
Christopher Grady, Rebecca Wolfe, Danjuma Dawop, and Lisa Inks
}


\date{May 18, 2019}


\usepackage{versions}
\makeatletter
\renewcommand*\versionmessage[2]{\typeout{*** `#1' #2. ***}}
\renewcommand*\beginmarkversion{\sffamily}
  \renewcommand*\endmarkversion{}
\makeatother

\excludeversion{comment}

%\usepackage[margins=1in]{geometry}

\usepackage[compact,bottomtitles]{titlesec}
%\titleformat{ ⟨command⟩}[⟨shape⟩]{⟨format⟩}{⟨label⟩}{⟨sep⟩}{⟨before⟩}[⟨after⟩]
\titleformat{\section}[hang]{\Large\bfseries}{\thesection}{.5em}{\hspace{0in}}[\vspace{-.2\baselineskip}]
\titleformat{\subsection}[hang]{\large\bfseries}{\thesubsection}{.5em}{\hspace{0in}}[\vspace{-.2\baselineskip}]
%\titleformat{\subsubsection}[hang]{\bfseries}{\thesubsubsection}{.5em}{\hspace{0in}}[\vspace{-.2\baselineskip}]
\titleformat{\subsubsection}[hang]{\bfseries}{\thesubsubsection}{1ex}{\hspace{0in}}[\vspace{-.2\baselineskip}]
\titleformat{\paragraph}[runin]{\bfseries\itshape}{\theparagraph}{1ex}{}{\vspace{-.2\baselineskip}}
%\titleformat{\paragraph}[runin]{\itshape}{\theparagraph}{1ex}{}{\vspace{-.2\baselineskip}}

%%\titleformat{\subsection}[hang]{\bfseries}{\thesubsection}{.5em}{\hspace{0in}}[\vspace{-.2\baselineskip}]
%%%\titleformat*{\subsection}{\bfseries\scshape}
%%%\titleformat{\subsubsection}[leftmargin]{\footnotesize\filleft}{\thesubsubsection}{.5em}{}{}
%%\titleformat{\subsubsection}[hang]{\small\bfseries}{\thesubsubsection}{.5em}{\hspace{0in}}[\vspace{-.2\baselineskip}]
%%\titleformat{\paragraph}[runin]{\itshape}{\theparagraph}{1ex}{}{\vspace{-.5\baselineskip}}

%\titlespacing*{ ⟨command⟩}{⟨left⟩}{⟨beforesep⟩}{⟨aftersep⟩}[⟨right⟩]
\titlespacing{\section}{0pc}{1.5ex plus .1ex minus .2ex}{.5ex plus .1ex minus .1ex}
\titlespacing{\subsection}{0pc}{1.5ex plus .1ex minus .2ex}{.5ex plus .1ex minus .1ex}
\titlespacing{\subsubsection}{0pc}{1.5ex plus .1ex minus .2ex}{.5ex plus .1ex minus .1ex}



%% These next lines tell latex that it is ok to have a single graphic
%% taking up most of a page, and they also decrease the space around
%% figures and tables.
\renewcommand\floatpagefraction{.9}
\renewcommand\topfraction{.9}
\renewcommand\bottomfraction{.9}
\renewcommand\textfraction{.1}
\setcounter{totalnumber}{50}
\setcounter{topnumber}{50}
\setcounter{bottomnumber}{50}
\setlength{\intextsep}{2ex}
\setlength{\floatsep}{2ex}
\setlength{\textfloatsep}{2ex}



\begin{document}
\VerbatimFootnotes

%\begin{titlepage}
%  \maketitle
%\vspace{2in}
%
%\begin{center}
%  \begin{large}
%    PROPOSAL WHITE PAPER
%
%BAA 14-013
%
%Can a Hausa Language Television Station Change Norms about Violence in Northern Nigeria? A Randomized Study of Media Effects on Violent Extremism
%
%Jake Bowers
%
%University of Illinois @ Urbana-Champaign (jwbowers@illinois.edu)
%
%\url{http://jakebowers.org}
%
%Phone: +12179792179
%
%Topic Number: 1
%
%Topic Title: Identity, Influence and Mobilization
%
%\end{large}
%\end{center}
%\end{titlepage}

\maketitle

\begin{abstract}

Intergroup conflict is responsible for many of the worst displays of human nature.  In this paper we test the ability of psychological reconciliation through intergroup contact to contribute to peace between groups involved in violent conflict.  Intergroup conflict is often mitigated by international intervention, like foreign military intervention and third party mediation.  However, international intervention is not always effective at reducing intergroup conflicts.  We propose psychological reconciliation through intergroup contact as an alternative to large-scale international intervention.  Groups in conflict often share mutual distrust, which make peace impossible unlikely by sabotaging intergroup bargaining.  Outside intervention often serve as substitutes for intergroup trust, but psychological reconciliation could improve intergroup trust directly.  Improved intergroup trust should allow the conflicting groups to settle disputes even in the absence of outside intervention.  We test the ability of psychological reconciliation programs to promote peace between violently conflicting groups with a field experiment in Nigeria, where farmer and pastoralist communities are embroiled in a deadly conflict over land use.  We find that the programs increase intergroup trust and perceptions of physical security of group members.  These results suggest that psychological reconciliation can promote peace between groups in conflict.

\end{abstract}

\section{Introduction}\label{introduction}

Intergroup conflict is responsible for many of the worst displays of
human nature. In Nigeria, intergroup conflict between farmers and
pastoralists has claimed countless lives and destroyed countless
communities over the past two decades. As the conflict escalated, groups
of anti-pastoralist vigilantes mobilized to pre-emptively prevent
pastoralists from encroaching on land claimed by farmers (Duru 2018;
McDonnel 2017). These groups, dubbed the ``livestock guard'', ransacked
pastoralist settlements and violently drove pastoralists from their
homes, often with the assistance of the local farming community.
Likewise, pastoralist groups enacted vigilante justice, raiding and
burning down farming villages seen to encroach on land claimed by
pastoralists. Attacks such as these forced up to 180,000 people from
their homes in 2018 (Daniel 2018) and farmer-pastoralist conflict costs
Nigeria \$13 billion of lost economic productivity annually (McDougal et
al. 2015).

Though farmer-pastoralist conflict was widespread, mass violence did not
break out between these groups in all conflicting communities. In one
village, farmers and pastoralists defended each other from hostile
neighbors. When a group of livestock guard came for one pastoralist
settlement, the neighboring farming village arrested them to protect the
pastoralists. After the arrest, farmers and pastoralists convened to
decide what should be done with the prisoners. They agreed that the
group of livestock guard should not be punished, but should be disarmed
and released home -- a proposition proposed by \emph{the pastoralists}.
These farmers and pastoralists had also struggled with conflict, and
people on both sides had died in disputes over farmland and grazing
land. But their disputes had not escalated to the point that each side
wanted the other removed by any means necessary. They had created
structures that allowed them to settle disputes, and the same structures
allowed them to reach a solution about the livestock guard.

Why were some farmer and pastoralist groups able to keep peace whereas
others were swallowed by the escalating conflict? Why were some
communities able to overcome their intergroup disputes whereas others
were destroyed by them? These questions are not unique to Nigeria, and
understanding the factors that help groups resolve their disputes is
important for mitigating and preventing the myriad intergroup conflicts
around the world. Using the framework of intergroup conflict as a
bargaining failure, we argue that reducing intergroup prejudice during
an ongoing conflict can increase intergroup trust and reduce violent
conflict. In situations of high prejudice, groups are essentially locked
into a mutually-defecting Prisoner's Dilemma because each side assumes
the other will defect, even if one side cooperated first. Interventions
that reduce group prejudice could help the groups achieve peace by
helping the groups update their perceptions of each other's
trustworthiness. Intergroup trust ameliorates bargaining problems and
increases the likelihood of the groups solving disputes through
bargaining instead of violence. Peacebuilding programs can help
conflicting groups achieve trust through intergroup contact and
superordinate goals.

Intergroup conflict is often conceptualized as a bargaining problem
(Fearon 1994b; Powell 2006). Both groups want some resource -- land,
power, etc -- and must decide how to distribute that resource. Groups
can either bargain and split the resource, or groups can fight to claim
all of the resource or to increase their bargaining position later.
Fighting is costly, so both groups are better off finding a bargained
solution than fighting. However, bargaining fails if neither group
trusts the other side to be truthful or to honor bargained agreements
(Kydd 2000; Rohner, Thoenig, and Zilibotti 2013, 2013). Without a reason
to trust in the other side, groups are likely to remain in conflict.

How can groups in conflict escape this spiral of distrust and conflict?
The classic answer is a strong external actor that can increase the cost
of fighting, punish defections from bargained agreements, and facilitate
information flows (Doyle and Sambanis 2000; Fearon 1994b; Ostrom and
Walker 2003; Powell 2006). For many conflicts, international actors
fulfill those role through military intervention and mediation (Di
Salvatore and Ruggeri 2017; Doyle and Sambanis 2000). By providing
reliable information and punishing defection from agreements,
international intervention helps conflicting actors overcome the
bargaining failures that lead to violent conflict (Fearon 1994b, 1995;
Powell 2006; Smith and Stam 2003). If a strong external actor punishes
groups for defecting, each side can trust the other to abide by their
agreement because abiding by their agreement is in each group's
self-interest. A strong external actor serves as a substitute for the
trust the groups lack.

While not universally successful, military intervention and mediation
are effective strategies to promote peace in many conflicts (Doyle and
Sambanis 2000; Gartner 2011; Hartzell, Hoddie, and Rothchild 2001;
Wallensteen and Svensson 2014; Walter 2002). But strong external actors
guaranteeing peace are not desirable or possible everywhere. Intervening
as an external actor is costly, and often there are no external actors
strong enough or interested enough to intervene (Fey and Ramsay 2010;
Kydd 2006). In many conflicts, the conflicting groups are not united and
therefore it is not clear who is involved in bargaining and who the
external actor should punish if one side defects. Where an external
actor is available, its presence is a short-term peace solution and its
effects do not endure with the external actor's departure (Beardsley
2008; Rohner, Thoenig, and Zilibotti 2013; Weinstein 2005). With or
without external actors to halt the conflict, the groups must eventually
build trust rather than find a substitute for trust.

Increasing trust amidst conflict is difficult, however, for
psychological reasons. Group conflicts are often driven and perpetuated
by intergroup animosity and prejudice long after the original grievance
is forgotten {[}McDonnel (2017); cite more about intergroup conflict \&
intergroup prejudice{]}. Intergroup animosity and intergroup prejudice
can be insurmountable barriers to peace for direct and indirect reasons.
Directly, highly prejudiced groups are less likely to trust information
they receive from the other side or any peace commitment they get from
the other side. In this way, prejudice directly inhibits intergroup
bargaining and prevents peaceful resolution of the conflict.

Indirectly, prejudice biases interpretations of ingroup and outgroup
behavior, preventing each group from updating and improving their
attitudes about the other. Ingroups will perceive their own belligerent
actions as defensive and justified, whereas behavior by outgroup members
may be perceived as more threatening and more malicious than the same
behavior committed by a neutral party {[}Hewstone (1990); cite
fundamental/ultimate attribution error, motivated reasoning,
confirmation bias, anchoring bias{]}. The perceived negative behavior
may be seen as \emph{defining} of the group, whereas any perceived
positive behavior may be seen as the \emph{exception} to the group
(Allison and Messick 1985; Pettigrew 1979). This biased information
processing reinforces negative group stereotypes and can subvert the
groups' own attempts to foster peace. It can also sabotage intergroup
bargaining by causing the groups to have inaccurate beliefs about each
other and each other's willingness to make peace, adding to information
and commitment problems that cause conflict.

Existing strategies to reduce intergroup conflict do not deal with
prejudice reduction. One approach to reducing intergroup prejudice comes
from intergroup contact theory. Intergroup contact theory hypothesizes
that interactions in which group members cooperate to achieve shared
goals will reduce prejudice and the likelihood of violence. Intergroup
contact works by demystifying the outgroup, presenting the other group's
perspective, and replacing imagined stereotypes with firsthand knowledge
(Allport 1954; Pettigrew and Tropp 2008). This type of structured
face-to-face contact also provides groups the opportunity to send costly
signals about their willingness to cooperate (Kydd 2000; Lupia,
McCubbins, and Arthur 1998; Rohner, Thoenig, and Zilibotti 2013),
helping each group update its opinion of the other side's preference for
peace. Intergroup contact is likely especially good at reducing
intergroup conflict when groups cooperate to achieve superordinate goals
-- goals that require the cooperation of both groups and benefit both
groups -- because groups experience the material benefits of cooperation
(Gaertner et al. 2000; Sherif 1958).

The prejudice-reducing effects of intergroup contact have been
demonstrated in a variety of countries and using a variety of
methodological approaches (Paluck, Green, and Green 2017; Pettigrew and
Tropp 2006). Notably, intergroup contacted programs reduced prejudice
between white people and black people in the U.S. and South Africa
(Burns, Corno, and La Ferrara 2015; Marmaros and Sacerdote 2006), Jews
and Arabs (Ditlmann and Samii 2016; Yablon 2012), and Hindus and Muslims
(Barnhardt 2009). Peacebuilding programs utilizing intergroup contact
and superordinate goals are increasingly used to reduce intergroup
conflict by peacebuilding organizations (Ditlmann, Samii, and Zeitzoff
2017).

Although research shows support for intergroup contact theory generally,
its efficacy to reduce prejudice amid real-world conflict is an open
question. First, most research about intergroup contact uses
correlational studies or lab experiments, both of which have
methodological weaknesses. Correlational studies cannot demonstrate
causal effects, and results from lab experiments may not apply to real
world conflicts, where groups compete for resources and share a history
of intergroup violence (Ditlmann, Samii, and Zeitzoff 2017). Second, no
prior studies of intergroup contact have involved groups engaged in
active intergroup violence\footnote{Previous studies have involved
  groups with a history of violent conflict but not groups involved in
  an active conflict. See Scacco and Warren (2018) and Ditlmann and
  Samii (2016) for examples.}, and some studies suggest that intergroup
contact in violent settings could be ineffective or even backfire.
Negative experiences with outgroups increase prejudice, and the most
prejudiced individuals are most likely to interpret intergroup contact
negatively (Gubler 2013; Paolini, Harwood, and Rubin 2010). Despite a
lack of evidence about the effects of contact-based peacebuilding
programs in violent contexts, and the risks of programs going badly,
peacebuilding organizations implement numerous contact-based
interventions in violent contexts each year. These peacebuilding
programs might defuse intergroup conflict, but these programs also might
do more harm than good.

To study the effect of bottom-up psychological interventions on violent
conflict, we conduct a field experiment with farmer and pastoralist
communities in Nigeria to determine if an intergroup contact-based
program effectively increases intergroup trust and increases the
physical security of group members. We randomly assigned communities
with farmer-pastoralist violence to receive the peacebuilding
intervention or serve as a control group. The intervention formed
mixed-group committees and provided them with funds to build
infrastructure that would benefit both communities; committees then
collaboratively chose and constructed infrastructure projects.\footnote{The
  communities built boreholes, market stalls, primary health care
  facilities, etc.} The program also provided mediation training to each
community's leaders. To measure the effects of the intervention, we
conducted pre- and post-intervention surveys, a post-intervention
natural public goods behavioral game,\footnote{In a public goods game
  (PGG), research subjects are given money and told they can keep the
  money or donate it to a public fund. Money donated to the public fund
  is multiplied by some amount and then shared with all subjects. Our
  PGG is \emph{natural} because it was conducted in a natural setting,
  rather than a lab. The funding for the PGG came from the National
  Science Foundation under Grant No. 1656871.} and twelve months of
systematic observations in markets and social events during the
intervention.

We find that the program increased trust between groups and decreased
perceptions of physical insecurity. Compared to the control group, the
treatment group expressed more outgroup trust and more willingness to
interact with outgroup members. The treatment group was also less
prevented by violence from engaging in routine tasks, such as working,
going to the market, and getting water. The results also suggests that
intergroup contact for a relatively small percentage of a group can
indirectly affect attitudes of group members with no exogenous increase
in contact with the outgroup. We observe the most positive changes from
individuals directly involved in the intergroup committees, but we also
observe positive spillover to group members who were not involved in the
intergroup contact intervention.

This study expands our knowledge about intergroup conflict in several
ways. First, this study teaches us about the capacity of
psychologically-based peacebuilding programs to improve intergroup
relations. To our knowledge this is the first field experimental test of
a psychologically-based peacebuilding program implemented during an
active conflict. We evaluated the program's effects on both attitudinal
and behavioral outcomes. The results suggest that bottom-up
psychologically-based peacebuilding programs can effectively reduce
conflict and is especially relevant to conflict resolution in the cases
of intergroup and intercommunal conflicts.

Second, this study is one of the only field experimental interventions
to test intergroup contact theory with groups actively engaged in
violence. Each of the groups in our study were part of an active and
escalating conflict, with members of each side being killed within one
year of the intervention's onset. Even in such a context, community
members who engaged in direct interpersonal interaction with an outgroup
member changed more positively than other community members.
Importantly, the intergroup contact involved achieving superordinate
goals that benefited both groups materially. This suggests that contact
with superordinate goals is robust to actively violent contexts.

Third, we contribute to the literature about the role of social
diffusion and social institutions in shaping attitudes and behaviors.
Bottom-up peacebuilding interventions seek to provide a structure in
which groups can solve their own conflicts, and those structures are
social rather than coercive. Understanding how those structures form and
are maintained is relevant to other institutional setups attempting to
influence behavior. In this study, though the greatest changes in
attitudes and behaviors were from individuals directly interacting with
outgroup members, the attitudes of other community members also
improved. This type of ``social effect'' is also a potential way to
``scale up'' the effects of intergroup contact.

Fourth, this paper's focus on farmer-pastoralist conflict is especially
important because the pastoralists are of the Fulani ethnic group. The
Fulani are the largest semi-nomadic people on Earth, but their way of
life makes them targets for violence throughout Africa. Along with this
conflict in Nigeria, Fulani in Mali have been the targets of violence so
severe that researchers at Armed Conflict Location \& Event Data Project
called it ``ethnic cleansing'' (Economist 2019). Understanding how to
prevent conflict between Fulani and settled peoples can help prevent the
eradication of a people and their way of life.

In the next section we provide a theoretical framework for how and why
bottom-up interventions that focus on prejudice reduction can reduce
intergroup violence. We then discuss Nigeria's farmer-pastoralist
conflict, our experimental intervention, and two designs to evaluate the
effect of the intervention. Last we present the results of the study and
conclude by connecting these findings to theories of group conflict and
prejudice.

\section{Theory}\label{theory}

\subsection{Intergroup Conflict as a Bargaining
Problem}\label{intergroup-conflict-as-a-bargaining-problem}

Intergroup conflict is most often conceptualized as a bargaining problem
(Fearon 1994b; Powell 2006), and most solutions to reducing intergroup
conflict strive to help the groups overcome those bargaining problems
{[}Di Salvatore and Ruggeri (2017); chris: this cite is just for
peacekeeping/intervention{]}. Intergroup conflict is a bargaining
problem because both groups want some resource -- land, power, etc --
but cannot reach an agreement about how to distribute that resource
peacefully. Because fighting is costly, the groups are better off
reaching a bargained compromise and not fighting. However, two
bargaining problems prevent this: information problems and commitment
problems. To successfully bargain, the groups need (1) accurate
information about each other, and/or (2) the assurance that each side
will abide by its agreements.

An \emph{information problem} arises because neither group possesses
accurate information about the other, and both groups have an incentive
to deceive the other in order to achieve an advantageous bargaining
outcome. Groups have an incentive to portray themselves as stronger,
more willing to fight, and less willing to make concessions than they
truly are (Fearon 1995). This causes bargaining failures because neither
group knows what agreements the other side is willing to accept or what
their side should receive from bargaining. A \emph{comittment problem}
arises because neither group can credibly commit to honor bargained
agreements if bargaining power shifts in the future. If bargaining power
shifts, one side will have an incentive to renege on the status quo
agreement to achieve a better agreement. Neither group can commit to
honor agreements made today when both groups know that bargaining power
may shift tomorrow.\footnote{Power between the groups can shift due to
  factors that affect each group's preferences and capabilities.
  Internally, one group may grow in power or size, one group may disarm
  before the other, the groups may have factions that reject the
  agreement, the groups may change leaders, or group members attitudes
  may change for other reasons. Externally, the groups may gain or lose
  de facto or de jure state support, other external actors may switch
  allegiances, or some other shock may change each group's power
  relative to the other {[}Fearon (2004); Reed et al. (2016); chris:
  need to cite and add more{]}. Anything that will change group power in
  the future can cause commitment problems in the present.} Without the
ability to commit to agreements, bargaining will not be successful.

Groups in conflict overcome these bargaining problems in several ways.
Groups can overcome information problems through fighting, as they learn
about each others capacity and willingness to fight (Smith and Stam
2003). Groups can also overcome information problems through mediation.
An interested third party mediator with no stake in the conflict can
provide accurate information to both sides, helping each side reach a
bargain (Beber 2012; Kydd 2006; Ott 1972)

Even if groups overcome information problems, commitment problems could
prevent groups from reaching an agreement. The main way that groups
overcome commitment problems is through strong third parties that add
large costs to reneging on agreements (Doyle and Sambanis 2000; Fearon
1998; Powell 2006). Though each group may have an incentive to defect on
their agreement if bargaining power changes in a vacuum, the groups have
no incentive to defect if a strong third party is capable and willing to
punish defection from bargained agreements. With a third party punishing
defection, the groups can bargain in good faith knowing that the other
will abide by its word.

\subsection{The Persistence of Intergroup
Conflict}\label{the-persistence-of-intergroup-conflict}

If we know how to resolve intergroup conflict, why does conflict
persist? International mediation and intervention are dogged by two
issues of motivation. First, mediators are usually motivated for peace.
This motivation implies that mediator's provide information that is
biased towards encouraging the groups to negotiate a peace agreement.
Groups engaged in bargaining must believe that mediators provide
accurate, unbiased information for mediators to reduce information
problems. Since both groups know that mediators are biased towards
peace, mediators may not help groups achieve peace (Fey and Ramsay 2010;
Kydd 2006; Smith and Stam 2003). Second, international actors may lack
the motivation to mediate or to intervene into the conflict, even if one
group reneges on its agreement. Intervention is costly and so
international actors cannot credibly commit to intervene into the
conflict unless they have a stake in the conflict (Kydd 2006).

This lack of motivation is especially relevant for intergroup conflict
between groups within a state. Most international actors and strong
third parties lack an incentive to intervene into intrastate intergroup
conflicts, and these disputes tend to take place within states that lack
the capacity to compel both sides themselves. Since intervention is
costly, strong third parties have an incentive to intervene only
\emph{after} fighting escalates, so intervention will generally not be
used to prevent conflict from escalating or to reduce the persistent,
smaller-scale violence that plague many countries. The lack of a strong
third party to prevent the intergroup bargaining failures that causes
persistent intergroup conflict is a serious barrier to peace.

Conflict also persists due to intergroup prejudice. Groups in
competition and conflict develop prejudiced attitudes that make peace
difficult to attain (Allport 1954; Sherif 1958). Conflicting groups
think the outgroup is untrustworthy, selfishly motivated, and innately
bad {[}cite: dehumanization, tajfel1981, Gutsell and Inzlitch 2010 for
brain mechanism, Schaller\_Neuberg\_2008{]}. Intergroup conflict fuels
and is fueled by intergroup prejudice. Intergroup prejudice prevents
peaceful resolution of conflict directly, through exacerbating
bargaining problems, and indirectly, through biasing perceptions of
ingroup and outgroup behavior and through changing each group's
preferences for peace and war.

Directly, intergroup prejudice prevents peace by exacerbating bargaining
problems. At their heart, information and commitment problems are both
problems of trust. For information problems, groups do not trust the
information they get from the other group. For commitment problems,
groups do not trust the other group to abide by its agreements. Highly
prejudiced groups are less likely to trust information they receive from
the other side or any peace commitment they get from the other side. As
a result, prejudiced groups are unlikely to overcome bargaining problems
and more likely to engage in violent conflict.

Indirectly, intergroup prejudice prevents accurate perceptions about the
attitudes and preferences of the outgroup. Prejudice biases our
interpretation of ingroup and outgroup behavior. Ingroups will perceive
their own belligerent actions as defensive and justified, and are more
likely to perceive outgroup actions as aggressive, negatively motivated,
and unjustified {[}Hewstone (1990); Amir (1969); chris: fundamental
attribution error, ultimate/group attribution error, confirmation bias,
anchoring bias, Hunter et al 1991{]}. The perceived negative behavior
may be seen as \emph{defining} of the group, whereas any perceived
positive behavior may be seen as the \emph{exception} to the group
(Allison and Messick 1985; Pettigrew 1979). Even positive intergroup
interactions may be re-interpreted as negative to avoid cognitive
dissonance (Festinger 1962; Gubler 2013; Paolini, Harwood, and Rubin
2010). Interpreting interactions negatively saps the power of each group
to reassure the other with costly signals of willingness to cooperate in
future interactions (Kydd 2000, Rohner, Thoenig, and Zilibotti (2013)).
Adds to information problems as groups will hold inaccurate views about
each other's willingness to cooperate and likelihood of upholding
agreements.

This indirect mechanism poses problems for groups to develop reputations
as trustworthy. When there is no strong third party to enforce
bargaining agreements, commitment problems are often overcome by
reputations.\footnote{The reputation argument is generally mobilized for
  contexts in which many groups observe the behavior of each other
  group. Because there are many potential partners in the future, a good
  reputation has high payoff in the form of many potential cooperative
  partners in the future. In our case, there are two main sides forming
  perceptions about the reputation of each other. This closely mimics
  repeating prisoner's dilemma interactions. As shown in Axelrod (1980a)
  and Axelrod (1980b), beliefs about the ``type'' of player my partner
  is influence my decision to cooperate or defect under several decision
  rules designed to mimic human decision-making.} Reputations, beliefs
that groups have about the likely behavior of other actors, allow groups
to bargain and negotiate even without an organization willing to punish
defection from agreements. Though defecting may be beneficial in one
specific instance, groups may not defect because getting a reputation
for defecting on agreements harms a group's ability to achieve
agreements in the future. By relying on reputation and the knowledge
that groups desire good reputations, groups can coordinate in the
absence of a strong third party.

The reputation mechanism could prevent intergroup conflict but
reputations, too, are hampered by prejudice. Prejudice biases
interpretations of ingroup and outgroup behavior, which makes it very
difficult for a group to develop a positive reputation with a group they
are in conflict with, even when both groups are motivated to end the
conflict. This bias likely pushes each group's perception of the other
side's willingness to make peace further from their true willingness to
make peace and so reputations hinder, rather than aid, intergroup
bargaining processes.

Reputations to prevent conflict are also hampered by a lack of
opportunities for groups to observe each others behavior and to learn
the outgroup's reputation. Compounding that problem, few of the
outgroup's interactions will be with groups that are relevant for
predicting the outgroup's behavior towards my group. This means that the
main opportunity to observe outgroup behavior and learn their reputation
is your own interactions with the outgroup. For groups in conflict,
these opportunities are likely rare and almost always adversarial.

These negative attitudes also change the each groups preference for
peace or war, both for leaders and for individuals who encounter
disagreements with outgroup members. The utility an individual gets for
attitudes and behaviors depends largely on how those attitudes and
behaviors are received by their ingroup (Wood 2000). These ``social
payoffs'' constrain the actions of leaders and individual group members.
In the context of outgroup prejudice, the group might punish a leader
for cooperating or compromising with the outgroup (Fearon 1994a).
Individuals might get similarly socially punished for cooperating
instead of taking a hard stand against the other side's perceived
transgressions. Leaders of highly prejudiced groups also cannot credibly
commit to keep their group members in check. Individuals in highly
prejudiced groups might not engage in ingroup policing, a strong, costly
signal to the other side that your group will uphold its peace
agreements (Fearon and Laitin 1996). While cooperation and ingroup
policing might be punished, aggressive actions may be seen as righteous
self-defense of the ingroup and rewarded.

Along with social benefits from aggressive attitudes and behavior,
individuals might receive psychological benefits from conflict and from
social differentiation with the outgroup (Wood 2000). Many groups define
``us'' by positive differences with a ``them'', and an individual can
derive self-esteem from positively comparing their group identity to a
rival group (Brewer 1999; Tajfel 1981). When group members derive
self-esteem from feeling superior to an outgroup, group members may
reject rhetoric about group similarity due to cognitive dissonance it
causes them. Group members may also reject actions that recognize the
outgroup as equals.

\subsection{Psychological Reconciliation to Reduce Prejudice and
Conflict}\label{psychological-reconciliation-to-reduce-prejudice-and-conflict}

The problems of negative intergroup attitudes suggests that improving
those attitudes could lead to peace-promoting behaviors and reduce
conflict. One of the most promising approaches to improving intergroup
attitudes comes from intergroup contact theory. Intergroup contact
theory hypothesizes that interactions in which group members cooperate
to achieve shared goals will reduce prejudice. Reducing prejudice can
help groups overcome bargaining problems and reduce the likelihood of
violence.

When groups are in conflict, most incidental intergroup contact will be
highly adversarial. Intergroup contact theory posits several conditions
necessary for intergroup contact to reduce prejudice. Groups must
cooperate with equal status to achieve shared goals with the support of
elites. Intergroup contact under these conditions gives the ability give
strong costly signals about a group's reputation and the ``group type''
as a conditional complier: we will honor our agreements if you honor
yours. And about perceptual fairness/unbiasedness: we will interpret
things you do fairly and we will interpret things we do fairly (i.e.~we
will not think everything we do is justified and everything you do is
unjustified).

It is important for the intergroup contact to focus on superordinate
goals ({\textbf{???}}).

Could improve reputations/remove stereotypes. Increase perception of
trustworthiness/intergroup trust. Contact gives opportunity for costly
signals. Reputations will help groups overcome information and
commitment problems. Groups trust information they get from the other
group. Groups trust the other group to abide by agreements because they
believe the other group is also motivated by peace.

Can give group members interactions that reduce perceptual biases. Can
decrease prejudice and encourage ingroup policing. Decrease prejudice
also allows leaders to credibly commit entire group to peace. Encourage
ingroup policing. Create norms against prejudice and violence, or
connect existing norms to the outgroup by humanizing the outgroup.

Could improve dispute resolution and reduce conflict over resources.
Encourage sharing resources \emph{or} separation of resources. Mediation
training -- help community leaders resolve disputes.

Could decrease threat, though in cases of active conflict this might
exacerbate rather than solve conflict (information about reduced
outgroup threat == higher chance of my group winning).

Could increase empathy. Seeing the other side's argument, seeing their
motivations. Increase empathy/less prejudice helps interpret information
in non-biased way. {[}Chris: cite literature on empathy making me
interpret information about other people more accurately. Kertzer?{]}

Could expand ingroup/make group similarities salient. Seeing our two
groups as one, seeing our incentives as aligned.

Groups work together to gain more material resources, improves us now,
makes me think working together in the \emph{future} would be good.

Can give group members the capacity to affect conflict.

As an alternative to top-down international intervention to reduce
conflict, bottom-up peacebuilding programs can reduce conflict by
focusing on both the immediate economic concerns contributing to
conflict (superordinate goals?) and the psychological attitudes
contributing to conflict. Bottom-up psychological reconciliation
programs should work for the types of conflicts for which international
intervention is rarely used and is unlikely to be successful. Persistent
intergroup conflict that plagues many countries. Low-level conflict
before it builds to large-scale conflict. There are XX conflicts of this
type accounting for XX deaths each year.

\subsection{Why Reducing Prejudice Could
Fail}\label{why-reducing-prejudice-could-fail}

Yet reducing prejudice generally, and intergroup contact specifically,
may not work in contexts of ongoing violence. As with mediation and
intervention, bottom-up reconciliation programs may be ineffective while
conflict is going on and only effective at maintaining peace once the
formal conflict has ended. Many reasons contact may not work in this
context.

Intergroup conflict is commitment problem solved by strong third party
(Fearon 1994b; Powell 2006) and bottom-up programs do not provide a
strong third party that can enforce commitments. Psychological
reconciliation is not how we've thought of solving commitment problems
between conflicting groups or improving reputations. Can psych
reconciliation improve reputations in contexts of ongoing violence?

Psychological reconciliation can prevent resumption of conflict but
requires conflict to already be resolved (Bar-Tal 2000). Active conflict
produces many grievances and high prejudice; cognitive dissonance causes
backlash for the most prejudiced people (Festinger 1962; Gubler 2011).
Negative contact experiences reinforce negative stereotypes (Paolini,
Harwood, and Rubin 2010). Motivated reasoning for attitudes towards
others (Klein and Kunda 1992). If you don't want to like someone, you
will find a reason not to.

Underlying cause of conflict still present. Groups still engage in
competition for resources, which breeds conflict (Campbell 1965; Sherif
et al. 1988).

Few norms against prejudice, possibly group norms that support this
particular intergroup prejudice. If intergroup contact works by
activating ``norms against prejudice'', unlikely to work here.

Intergroup contact may only reduce prejudice from high-status group to
low-status group.

\section*{References}\label{references}
\addcontentsline{toc}{section}{References}

\hypertarget{refs}{}
\hypertarget{ref-allison1985group}{}
Allison, Scott T, and David M Messick. 1985. ``The Group Attribution
Error.'' \emph{Journal of Experimental Social Psychology} 21(6):
563--79.

\hypertarget{ref-allport1954prejudice}{}
Allport, Gordon. 1954. ``The Nature of Prejudice.'' \emph{Garden City,
NJ Anchor}.

\hypertarget{ref-amir1969contact}{}
Amir, Yehuda. 1969. ``Contact Hypothesis in Ethnic Relations.''
\emph{Psychological bulletin} 71(5): 319.

\hypertarget{ref-axelrod1980effective}{}
Axelrod, Robert. 1980a. ``Effective Choice in the Prisoner's Dilemma.''
\emph{Journal of conflict resolution} 24(1): 3--25.

\hypertarget{ref-axelrod1980more}{}
---------. 1980b. ``More Effective Choice in the Prisoner's Dilemma.''
\emph{Journal of Conflict Resolution} 24(3): 379--403.

\hypertarget{ref-bar2000intractable}{}
Bar-Tal, Daniel. 2000. ``From Intractable Conflict Through Conflict
Resolution to Reconciliation: Psychological Analysis.'' \emph{Political
Psychology} 21(2): 351--65.

\hypertarget{ref-barnhardt2009near}{}
Barnhardt, Sharon. 2009. ``Near and Dear? Evaluating the Impact of
Neighbor Diversity on Inter-Religious Attitudes.'' \emph{Unpublished
working paper}.

\hypertarget{ref-beardsley2008agreement}{}
Beardsley, Kyle. 2008. ``Agreement Without Peace? International
Mediation and Time Inconsistency Problems.'' \emph{American journal of
political science} 52(4): 723--40.

\hypertarget{ref-beber2012international}{}
Beber, Bernd. 2012. ``International Mediation, Selection Effects, and
the Question of Bias.'' \emph{Conflict Management and Peace Science}
29(4): 397--424.

\hypertarget{ref-brewer1999ingroupOutgroup}{}
Brewer, Marilynn B. 1999. ``The Psychology of Prejudice: Ingroup Love
and Outgroup Hate?'' \emph{Journal of social issues} 55(3): 429--44.

\hypertarget{ref-burns2015interaction}{}
Burns, Justine, Lucia Corno, and Eliana La Ferrara. 2015.
\emph{Interaction, Prejudice and Performance. Evidence from South
Africa}. Working paper.

\hypertarget{ref-campbell1965ethno}{}
Campbell, Donald T. 1965. ``Ethnocentric and Other Altruistic Motives.''
In \emph{Nebraska Symposium on Motivation}, 283--311.

\hypertarget{ref-daniel2018anti}{}
Daniel, Soni. 2018. ``Anti-Open Grazing Law: Nass, Benue, Kwara, Taraba
Tackle Defence Minister.'' \emph{Vanguard}.
\url{https://www.vanguardngr.com/2018/06/anti-open-grazing-law-nass-benue-kwara-taraba-tackle-defence-minister/}.

\hypertarget{ref-di2017effectiveness}{}
Di Salvatore, Jessica, and Andrea Ruggeri. 2017. ``Effectiveness of
Peacekeeping Operations.'' \emph{Oxford Research Encyclopedia of
Politics}.

\hypertarget{ref-ditlmann2016can}{}
Ditlmann, Ruth K, and Cyrus Samii. 2016. ``Can Intergroup Contact Affect
Ingroup Dynamics? Insights from a Field Study with Jewish and
Arab-Palestinian Youth in Israel.'' \emph{Peace and Conflict: Journal of
Peace Psychology} 22(4): 380.

\hypertarget{ref-ditlmann2017addressing}{}
Ditlmann, Ruth K, Cyrus Samii, and Thomas Zeitzoff. 2017. ``Addressing
Violent Intergroup Conflict from the Bottom up?'' \emph{Social Issues
and Policy Review} 11(1): 38--77.

\hypertarget{ref-doyle2000international}{}
Doyle, Michael W, and Nicholas Sambanis. 2000. ``International
Peacebuilding: A Theoretical and Quantitative Analysis.'' \emph{American
political science review} 94(4): 779--801.

\hypertarget{ref-duru2018court}{}
Duru, Peter. 2018. ``Court Stops Inspector General from Proscribing
Benue Livestock Guard.'' \emph{Vanguard}.
\url{https://www.vanguardngr.com/2018/11/court-stops-ig-from-proscribing-benue-livestock-guards/}.

\hypertarget{ref-economist2019militias}{}
Economist, The. 2019. ``Malicious Malitias: States in the Sahel Have
Unleashed Ethnic Gangs with Guns.'' \emph{The Economist}.
\url{https://www.economist.com/middle-east-and-africa/2019/05/04/states-in-the-sahel-have-unleashed-ethnic-gangs-with-guns}.

\hypertarget{ref-fearon1994domestic}{}
Fearon, James D. 1994a. ``Domestic Political Audiences and the
Escalation of International Disputes.'' \emph{American political science
review} 88(3): 577--92.

\hypertarget{ref-fearon1994ethnic}{}
---------. 1994b. ``Ethnic War as a Commitment Problem.'' In
\emph{Annual Meetings of the American Political Science Association},
2--5.

\hypertarget{ref-fearon1995rationalist}{}
---------. 1995. ``Rationalist Explanations for War.''
\emph{International organization} 49(3): 379--414.

\hypertarget{ref-fearon1998commitment}{}
---------. 1998. ``Commitment Problems and the Spread of Ethnic
Conflict.'' \emph{The international spread of ethnic conflict} 107.

\hypertarget{ref-fearon2004civil}{}
---------. 2004. ``Why Do Some Civil Wars Last so Much Longer Than
Others?'' \emph{Journal of peace research} 41(3): 275--301.

\hypertarget{ref-fearon1996explaining}{}
Fearon, James D, and David D Laitin. 1996. ``Explaining Interethnic
Cooperation.'' \emph{American political science review} 90(4): 715--35.

\hypertarget{ref-festinger1962cognitiveDissonance}{}
Festinger, Leon. 1962. 2 \emph{A Theory of Cognitive Dissonance}.
Stanford university press.

\hypertarget{ref-fey2010shuttle}{}
Fey, Mark, and Kristopher W Ramsay. 2010. ``When Is Shuttle Diplomacy
Worth the Commute? Information Sharing Through Mediation.'' \emph{World
Politics} 62(4): 529--60.

\hypertarget{ref-gaertner2000reducing}{}
Gaertner, Samuel L et al. 2000. ``Reducing Intergroup Conflict: From
Superordinate Goals to Decategorization, Recategorization, and Mutual
Differentiation.'' \emph{Group Dynamics: Theory, Research, and Practice}
4(1): 98.

\hypertarget{ref-gartner2011signs}{}
Gartner, Scott Sigmund. 2011. ``Signs of Trouble: Regional Organization
Mediation and Civil War Agreement Durability.'' \emph{The Journal of
Politics} 73(2): 380--90.

\hypertarget{ref-gubler2011diss}{}
Gubler, Joshua R. 2011. ``The Micro-Motives of Intergroup Aggression: A
Case Study in Israel.'' PhD thesis.

\hypertarget{ref-gubler2013humanizing}{}
---------. 2013. ``When Humanizing the Enemy Fails: The Role of
Dissonance and Justification in Intergroup Conflict.'' In \emph{Annual
Meeting of the American Political Science Association},

\hypertarget{ref-hartzell2001stabilizing}{}
Hartzell, Caroline, Matthew Hoddie, and Donald Rothchild. 2001.
``Stabilizing the Peace After Civil War: An Investigation of Some Key
Variables.'' \emph{International organization} 55(1): 183--208.

\hypertarget{ref-hewstone1990ultimate}{}
Hewstone, Miles. 1990. ``The `Ultimate Attribution Error'? A Review of
the Literature on Intergroup Causal Attribution.'' \emph{European
Journal of Social Psychology} 20(4): 311--35.

\hypertarget{ref-klein1992motivated}{}
Klein, William M, and Ziva Kunda. 1992. ``Motivated Person Perception:
Constructing Justifications for Desired Beliefs.'' \emph{Journal of
experimental social psychology} 28(2): 145--68.

\hypertarget{ref-kydd2000trust}{}
Kydd, Andrew. 2000. ``Trust, Reassurance, and Cooperation.''
\emph{International Organization} 54(2): 325--57.

\hypertarget{ref-kydd2006can}{}
Kydd, Andrew H. 2006. ``When Can Mediators Build Trust?'' \emph{American
Political Science Review} 100(3): 449--62.

\hypertarget{ref-lupia1998democratic}{}
Lupia, Arthur, Mathew D McCubbins, and Lupia Arthur. 1998. \emph{The
Democratic Dilemma: Can Citizens Learn What They Need to Know?}
Cambridge University Press.

\hypertarget{ref-marmaros2006friendships}{}
Marmaros, David, and Bruce Sacerdote. 2006. ``How Do Friendships Form?''
\emph{The Quarterly Journal of Economics} 121(1): 79--119.

\hypertarget{ref-mcdonnel2017graze}{}
McDonnel, Tim. 2017. ``Why It's Now a Crime to Let Cattle Graze Freely
in 2 Nigerian States.'' \emph{National Public Radio (NPR)}.
\url{https://www.npr.org/sections/goatsandsoda/2017/12/12/569913821/why-its-now-a-crime-to-let-cattle-graze-freely-in-2-nigerian-states}.

\hypertarget{ref-mcdougal2015effect}{}
McDougal, Topher L et al. 2015. ``The Effect of Farmer-Pastoralist
Violence on Income: New Survey Evidence from Nigeria's Middle Belt
States.'' \emph{Economics of Peace and Security Journal} 10(1): 54--65.

\hypertarget{ref-ostrom2003trust}{}
Ostrom, Elinor, and James Walker. 2003. \emph{Trust and Reciprocity:
Interdisciplinary Lessons for Experimental Research}. Russell Sage
Foundation.

\hypertarget{ref-ott1972mediation}{}
Ott, Marvin C. 1972. ``Mediation as a Method of Conflict Resolution: Two
Cases.'' \emph{International Organization} 26(4): 595--618.

\hypertarget{ref-paluck2017contact}{}
Paluck, Elizabeth Levy, Seth Green, and Donald P Green. 2017. ``The
Contact Hypothesis Revisited.''

\hypertarget{ref-paolini2010negative}{}
Paolini, Stefania, Jake Harwood, and Mark Rubin. 2010. ``Negative
Intergroup Contact Makes Group Memberships Salient: Explaining Why
Intergroup Conflict Endures.'' \emph{Personality and Social Psychology
Bulletin} 36(12): 1723--38.

\hypertarget{ref-pettigrew1979ultimate}{}
Pettigrew, Thomas F. 1979. ``The Ultimate Attribution Error: Extending
Allport's Cognitive Analysis of Prejudice.'' \emph{Personality and
social psychology bulletin} 5(4): 461--76.

\hypertarget{ref-pettigrew2006meta}{}
Pettigrew, Thomas F, and Linda R Tropp. 2006. ``A Meta-Analytic Test of
Intergroup Contact Theory.'' \emph{Journal of personality and social
psychology} 90(5): 751.

\hypertarget{ref-pettigrew2008does}{}
---------. 2008. ``How Does Intergroup Contact Reduce Prejudice?
Meta-Analytic Tests of Three Mediators.'' \emph{European Journal of
Social Psychology} 38(6): 922--34.

\hypertarget{ref-powell2006war}{}
Powell, Robert. 2006. ``War as a Commitment Problem.''
\emph{International organization} 60(1): 169--203.

\hypertarget{ref-reed2016bargaining}{}
Reed, William, David Clark, Timothy Nordstrom, and Daniel Siegel. 2016.
``Bargaining in the Shadow of a Commitment Problem.'' \emph{Research \&
Politics} 3(3): 2053168016666848.

\hypertarget{ref-rohner2013war}{}
Rohner, Dominic, Mathias Thoenig, and Fabrizio Zilibotti. 2013. ``War
Signals: A Theory of Trade, Trust, and Conflict.'' \emph{Review of
Economic Studies} 80(3): 1114--47.

\hypertarget{ref-scacco2018nigeria}{}
Scacco, Alexandra, and Shana S Warren. 2018. ``Can Social Contact Reduce
Prejudice and Discrimination? Evidence from a Field Experiment in
Nigeria.'' \emph{American Political Science Review} 112(3): 654--77.

\hypertarget{ref-sherif1958superordinate}{}
Sherif, Muzafer. 1958. ``Superordinate Goals in the Reduction of
Intergroup Conflict.'' \emph{American journal of Sociology} 63(4):
349--56.

\hypertarget{ref-Sherif1988robbersCave}{}
Sherif, Muzafer et al. 1988. ``Intergroup Conflict and Cooperation: The
Robbers Cave Experiment, Norman: Institute of Group Relations,
University of Oklahoma.''

\hypertarget{ref-smith2003mediation}{}
Smith, Alastair, and Allan Stam. 2003. ``Mediation and Peacekeeping in a
Random Walk Model of Civil and Interstate War.'' \emph{International
Studies Review} 5(4): 115--35.

\hypertarget{ref-tajfel1981groups}{}
Tajfel, Henri. 1981. \emph{Human Groups and Social Categories: Studies
in Social Psychology}. CUP Archive.

\hypertarget{ref-wallensteen2014talking}{}
Wallensteen, Peter, and Isak Svensson. 2014. ``Talking Peace:
International Mediation in Armed Conflicts.'' \emph{Journal of Peace
Research} 51(2): 315--27.

\hypertarget{ref-walter2002committing}{}
Walter, Barbara F. 2002. \emph{Committing to Peace: The Successful
Settlement of Civil Wars}. Princeton University Press.

\hypertarget{ref-weinstein2005autonomous}{}
Weinstein, Jeremy M. 2005. ``Autonomous Recovery and International
Intervention in Comparative Perspective.'' \emph{Available at SSRN
1114117}.

\hypertarget{ref-wood2000attitude}{}
Wood, Wendy. 2000. ``Attitude Change: Persuasion and Social Influence.''
\emph{Annual review of psychology} 51(1): 539--70.

\hypertarget{ref-yablon2012we}{}
Yablon, Yaacov B. 2012. ``Are We Preaching to the Converted? The Role of
Motivation in Understanding the Contribution of Intergroup Encounters.''
\emph{Journal of Peace Education} 9(3): 249--63.

\end{document}
