\documentclass[]{article}
\usepackage{lmodern}
\usepackage{amssymb,amsmath}
\usepackage{ifxetex,ifluatex}
\usepackage{fixltx2e} % provides \textsubscript
\ifnum 0\ifxetex 1\fi\ifluatex 1\fi=0 % if pdftex
  \usepackage[T1]{fontenc}
  \usepackage[utf8]{inputenc}
\else % if luatex or xelatex
  \ifxetex
    \usepackage{mathspec}
  \else
    \usepackage{fontspec}
  \fi
  \defaultfontfeatures{Ligatures=TeX,Scale=MatchLowercase}
\fi
% use upquote if available, for straight quotes in verbatim environments
\IfFileExists{upquote.sty}{\usepackage{upquote}}{}
% use microtype if available
\IfFileExists{microtype.sty}{%
\usepackage{microtype}
\UseMicrotypeSet[protrusion]{basicmath} % disable protrusion for tt fonts
}{}
\usepackage[margin=1in]{geometry}
\usepackage{hyperref}
\hypersetup{unicode=true,
            pdftitle={Promoting Peace Amid Escalating Conflict: An Intergroup Contact Field Experiment in Nigeria - Abstract},
            pdfauthor={Christopher Grady},
            pdfborder={0 0 0},
            breaklinks=true}
\urlstyle{same}  % don't use monospace font for urls
\usepackage{graphicx,grffile}
\makeatletter
\def\maxwidth{\ifdim\Gin@nat@width>\linewidth\linewidth\else\Gin@nat@width\fi}
\def\maxheight{\ifdim\Gin@nat@height>\textheight\textheight\else\Gin@nat@height\fi}
\makeatother
% Scale images if necessary, so that they will not overflow the page
% margins by default, and it is still possible to overwrite the defaults
% using explicit options in \includegraphics[width, height, ...]{}
\setkeys{Gin}{width=\maxwidth,height=\maxheight,keepaspectratio}
\IfFileExists{parskip.sty}{%
\usepackage{parskip}
}{% else
\setlength{\parindent}{0pt}
\setlength{\parskip}{6pt plus 2pt minus 1pt}
}
\setlength{\emergencystretch}{3em}  % prevent overfull lines
\providecommand{\tightlist}{%
  \setlength{\itemsep}{0pt}\setlength{\parskip}{0pt}}
\setcounter{secnumdepth}{5}
% Redefines (sub)paragraphs to behave more like sections
\ifx\paragraph\undefined\else
\let\oldparagraph\paragraph
\renewcommand{\paragraph}[1]{\oldparagraph{#1}\mbox{}}
\fi
\ifx\subparagraph\undefined\else
\let\oldsubparagraph\subparagraph
\renewcommand{\subparagraph}[1]{\oldsubparagraph{#1}\mbox{}}
\fi

%%% Use protect on footnotes to avoid problems with footnotes in titles
\let\rmarkdownfootnote\footnote%
\def\footnote{\protect\rmarkdownfootnote}

%%% Change title format to be more compact
\usepackage{titling}

% Create subtitle command for use in maketitle
\newcommand{\subtitle}[1]{
  \posttitle{
    \begin{center}\large#1\end{center}
    }
}

\setlength{\droptitle}{-2em}
  \title{Promoting Peace Amid Escalating Conflict: An Intergroup Contact Field
Experiment in Nigeria - Abstract}
  \pretitle{\vspace{\droptitle}\centering\huge}
  \posttitle{\par}
  \author{Christopher Grady}
  \preauthor{\centering\large\emph}
  \postauthor{\par}
  \predate{\centering\large\emph}
  \postdate{\par}
  \date{March 21, 2019}


\begin{document}
\maketitle

\section{Abstract}\label{abstract}

\subsection{Abstract 3}\label{abstract-3}

Intergroup conflict is responsible for many of the worst displays of
human nature. In this paper we test the ability of intergroup contact to
contribute to peace between groups involved in violent conflict.
Intergroup conflict is often characterized as a bargaining failure. Both
groups would benefit from compromise, but negotiations break down and
conflict ensues because neither group trusts the other to abide by their
agreements. The proposed solutions are coercive institutions that
prevent groups from defecting on bargained agreements. However, coercive
power cannot reach many group conflicts. We propose intergroup contact
as an alternative to coercive institutions. Coercive institutions
substitute for intergroup trust, but intergroup contact can foster
intergroup trust directly by creating situations in which groups can
signal their trustworthiness. Intergroup trust allows groups to bargain
even in the absence of coercive institutions. We test the ability of
intergroup contact to promote peace between violently conflicting groups
with a field experiment in Nigeria, where farmer and pastoralist
communities are embroiled in a deadly conflict over land use. We find
that the program increases intergroup trust and physical security. We do
not find evidence for alternative mechanisms through which intergroup
contact could reduce violence, such as reducing outgroup threat or
expanding the conception of ingroup to include the former outgroup.
These results suggest that intergroup trust via intergroup contact can
serve as social institution that solves commitment problems and promotes
peace between groups in conflict.


\end{document}
