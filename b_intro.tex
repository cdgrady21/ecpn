%\documentclass[]{article}
\documentclass[11pt]{article}
\usepackage[usenames,dvipsnames]{xcolor}

\usepackage[T1]{fontenc}
%\usepackage{lmodern}
\usepackage{tgtermes}
\usepackage{amssymb,amsmath}
%\usepackage[margin=1in]{geometry}
\usepackage[letterpaper,bottom=1in,top=1in,right=1.25in,left=1.25in,includemp=FALSE]{geometry}
\usepackage{pdfpages}
\usepackage[small]{caption}

\usepackage{ifxetex,ifluatex}
\usepackage{fixltx2e} % provides \textsubscript
% use microtype if available
\IfFileExists{microtype.sty}{\usepackage{microtype}}{}
\ifnum 0\ifxetex 1\fi\ifluatex 1\fi=0 % if pdftex
\usepackage[utf8]{inputenc}
\else % if luatex or xelatex
\usepackage{fontspec}
\ifxetex
\usepackage{xltxtra,xunicode}
\fi
\defaultfontfeatures{Mapping=tex-text,Scale=MatchLowercase}
\newcommand{\euro}{€}
\fi
%

\usepackage{fancyvrb}

\usepackage{ctable,longtable}

\usepackage[section]{placeins}
\usepackage{float} % provides the H option for float placement
\restylefloat{figure}
\usepackage{dcolumn} % allows for different column alignments
\newcolumntype{.}{D{.}{.}{1.2}}

\usepackage{booktabs} % nicer horizontal rules in tables

%Assume we want graphics always
\usepackage{graphicx}
% We will generate all images so they have a width \maxwidth. This means
% that they will get their normal width if they fit onto the page, but
% are scaled down if they would overflow the margins.
%% \makeatletter
%% \def\maxwidth{\ifdim\Gin@nat@width>\linewidth\linewidth
%%   \else\Gin@nat@width\fi}
%% \makeatother
%% \let\Oldincludegraphics\includegraphics
%% \renewcommand{\includegraphics}[1]{\Oldincludegraphics[width=\maxwidth]{#1}}
\graphicspath{{.}{../Soccom_Code/socom_2013/}}


%% \ifxetex
%% \usepackage[pagebackref=true, setpagesize=false, % page size defined by xetex
%% unicode=false, % unicode breaks when used with xetex
%% xetex]{hyperref}
%% \else
\usepackage[pagebackref=true, unicode=true, bookmarks=true, pdftex]{hyperref}
% \fi


\hypersetup{breaklinks=true,
  bookmarks=true,
  pdfauthor={Christopher Grady, Rebecca Wolfe, Danjuma Dawop, and Lisa Inks},
  pdftitle={Promoting Peace Amidst Group Conflict: An Intergroup Contact Field Experiment in Nigeria - Introduction},
  colorlinks=true,
  linkcolor=BrickRed,
  citecolor=blue, %MidnightBlue,
  urlcolor=BrickRed,
  % urlcolor=blue,
  % linkcolor=magenta,
  pdfborder={0 0 0}}

%\setlength{\parindent}{0pt}
%\setlength{\parskip}{6pt plus 2pt minus 1pt}
\usepackage{parskip}
\setlength{\emergencystretch}{3em}  % prevent overfull lines
\providecommand{\tightlist}{%
  \setlength{\itemsep}{0pt}\setlength{\parskip}{0pt}}

%% Insist on this.
\setcounter{secnumdepth}{2}

\VerbatimFootnotes % allows verbatim text in footnotes

\title{Promoting Peace Amidst Group Conflict: An Intergroup Contact Field
Experiment in Nigeria - Introduction}

\author{
Christopher Grady, Rebecca Wolfe, Danjuma Dawop, and Lisa Inks
}


\date{May 24, 2019}


\usepackage{versions}
\makeatletter
\renewcommand*\versionmessage[2]{\typeout{*** `#1' #2. ***}}
\renewcommand*\beginmarkversion{\sffamily}
  \renewcommand*\endmarkversion{}
\makeatother

\excludeversion{comment}

%\usepackage[margins=1in]{geometry}

\usepackage[compact,bottomtitles]{titlesec}
%\titleformat{ ⟨command⟩}[⟨shape⟩]{⟨format⟩}{⟨label⟩}{⟨sep⟩}{⟨before⟩}[⟨after⟩]
\titleformat{\section}[hang]{\Large\bfseries}{\thesection}{.5em}{\hspace{0in}}[\vspace{-.2\baselineskip}]
\titleformat{\subsection}[hang]{\large\bfseries}{\thesubsection}{.5em}{\hspace{0in}}[\vspace{-.2\baselineskip}]
%\titleformat{\subsubsection}[hang]{\bfseries}{\thesubsubsection}{.5em}{\hspace{0in}}[\vspace{-.2\baselineskip}]
\titleformat{\subsubsection}[hang]{\bfseries}{\thesubsubsection}{1ex}{\hspace{0in}}[\vspace{-.2\baselineskip}]
\titleformat{\paragraph}[runin]{\bfseries\itshape}{\theparagraph}{1ex}{}{\vspace{-.2\baselineskip}}
%\titleformat{\paragraph}[runin]{\itshape}{\theparagraph}{1ex}{}{\vspace{-.2\baselineskip}}

%%\titleformat{\subsection}[hang]{\bfseries}{\thesubsection}{.5em}{\hspace{0in}}[\vspace{-.2\baselineskip}]
%%%\titleformat*{\subsection}{\bfseries\scshape}
%%%\titleformat{\subsubsection}[leftmargin]{\footnotesize\filleft}{\thesubsubsection}{.5em}{}{}
%%\titleformat{\subsubsection}[hang]{\small\bfseries}{\thesubsubsection}{.5em}{\hspace{0in}}[\vspace{-.2\baselineskip}]
%%\titleformat{\paragraph}[runin]{\itshape}{\theparagraph}{1ex}{}{\vspace{-.5\baselineskip}}

%\titlespacing*{ ⟨command⟩}{⟨left⟩}{⟨beforesep⟩}{⟨aftersep⟩}[⟨right⟩]
\titlespacing{\section}{0pc}{1.5ex plus .1ex minus .2ex}{.5ex plus .1ex minus .1ex}
\titlespacing{\subsection}{0pc}{1.5ex plus .1ex minus .2ex}{.5ex plus .1ex minus .1ex}
\titlespacing{\subsubsection}{0pc}{1.5ex plus .1ex minus .2ex}{.5ex plus .1ex minus .1ex}



%% These next lines tell latex that it is ok to have a single graphic
%% taking up most of a page, and they also decrease the space around
%% figures and tables.
\renewcommand\floatpagefraction{.9}
\renewcommand\topfraction{.9}
\renewcommand\bottomfraction{.9}
\renewcommand\textfraction{.1}
\setcounter{totalnumber}{50}
\setcounter{topnumber}{50}
\setcounter{bottomnumber}{50}
\setlength{\intextsep}{2ex}
\setlength{\floatsep}{2ex}
\setlength{\textfloatsep}{2ex}



\begin{document}
\VerbatimFootnotes

%\begin{titlepage}
%  \maketitle
%\vspace{2in}
%
%\begin{center}
%  \begin{large}
%    PROPOSAL WHITE PAPER
%
%BAA 14-013
%
%Can a Hausa Language Television Station Change Norms about Violence in Northern Nigeria? A Randomized Study of Media Effects on Violent Extremism
%
%Jake Bowers
%
%University of Illinois @ Urbana-Champaign (jwbowers@illinois.edu)
%
%\url{http://jakebowers.org}
%
%Phone: +12179792179
%
%Topic Number: 1
%
%Topic Title: Identity, Influence and Mobilization
%
%\end{large}
%\end{center}
%\end{titlepage}

\maketitle

\subsection{Introduction}\label{introduction}

Intergroup conflict is responsible for many of the worst displays of
human nature. In Nigeria's Middlebelt, intergroup conflict between
farmers and pastoralists claimed 7000 lives in the past five years,
forced 180,000 people from their homes in 2018, and costs Nigeria \$13
billion of lost economic productivity annually (Daniel 2018; A. Harwood
2019; McDougal et al. 2015). In the most recent conflict escalation,
groups of anti-pastoralist vigilantes mobilized to pre-emptively prevent
pastoralists from encroaching on land claimed by farmers (Duru 2018;
McDonnel 2017). These groups, dubbed the ``livestock guard'', ransacked
pastoralist settlements and violently drove pastoralists from their
homes, often with the assistance of the local farming community.
Likewise, pastoralist groups enacted vigilante justice, raiding and
burning down farming villages seen to encroach on land claimed by
pastoralists.

Though farmer-pastoralist conflict was widespread, mass violence did not
break out between these groups in all Middlebelt communities. In one
village, farmers and pastoralists defended each other from hostile
neighbors. When a group of livestock guard came for one pastoralist
settlement, the neighboring farming village arrested them to protect the
pastoralists. After the arrest, farmers and pastoralists convened to
decide what should be done with the prisoners. They agreed that the
group of livestock guard should not be punished, but should be disarmed
and released home -- a proposition proposed by \emph{the pastoralists}.
These farmers and pastoralists had also struggled with conflict, and
people on both sides had died in past violence over farmland and grazing
land. But their recent disputes had not escalated to the point that each
side wanted the other removed by any means necessary. They had created
structures and relationships that allowed them to settle disputes, and
the same structures and relationships allowed them to reach a solution
about the livestock guard.

Why were some farmer and pastoralist groups able to keep peace whereas
others were swallowed by the escalating conflict? Why were some
communities able to overcome their disputes whereas others were
destroyed by them? These questions are not unique to Nigeria -- similar
intergroup dynamics spurred violence in South Sudan, Myanmar, and Bosnia
before those conflicts escalated into war. Understanding the factors
that help groups resolve their disputes is important to mitigate and
defuse the myriad intergroup conflicts around the world. Using the
framework of intergroup conflict as a bargaining failure, we argue that
increasing cooperative contact between groups during an ongoing conflict
can increase intergroup trust and physical security. In situations of
low trust, groups are essentially locked into a mutually-defecting
Prisoner's Dilemma because each side assumes the other will defect, even
if one side cooperated first. Interventions that increase opportunities
for cooperative contact can improve the prospects for peace by helping
the groups update their perceptions of each other's trustworthiness.
Intergroup trust ameliorates bargaining problems and increases the
likelihood of the groups resolving disputes through bargaining instead
of violence.

Intergroup conflict is often conceptualized as a bargaining problem
(Fearon 1994; Powell 2006). Both groups want some resource -- land,
power, etc -- and must decide how to distribute that resource. Groups
can either bargain and split the resource, or groups can fight to claim
all of the resource or to increase their bargaining position later.
Fighting is costly, so both groups are better off finding a bargained
solution than fighting. However, bargaining fails if neither group
trusts the other side to be truthful or to honor bargained agreements
(A. Kydd 2000; Rohner, Thoenig, and Zilibotti 2013, 2013). Without a
reason to trust in the other side, groups are likely to remain in
conflict.

How can groups in conflict escape this spiral of distrust and conflict?
The classic answer is a strong external actor that can increase the cost
of fighting, punish defections from bargained agreements, and facilitate
information flows (Doyle and Sambanis 2000; Fearon 1994; Ostrom and
Walker 2003; Powell 2006). For many conflicts, international actors
fulfill those role through military intervention and mediation (Di
Salvatore and Ruggeri 2017; Doyle and Sambanis 2000). By providing
reliable information and punishing defection from agreements,
international intervention helps conflicting actors overcome the
bargaining failures that lead to violent conflict (Fearon 1994, 1995;
Powell 2006; Smith and Stam 2003). If a strong external actor punishes
groups for defecting, each side can trust the other to abide by their
agreement because abiding by their agreement is in each group's
self-interest. A strong external actor serves as a substitute for the
trust the groups lack.

While not universally successful, military intervention and mediation
are effective strategies to promote peace in many conflicts (Doyle and
Sambanis 2000; Gartner 2011; Hartzell, Hoddie, and Rothchild 2001;
Wallensteen and Svensson 2014; Walter 2002). But strong external actors
guaranteeing peace are not desirable or possible everywhere. Intervening
as an external actor is costly, and often there are no external actors
strong enough or interested enough to intervene (Fey and Ramsay 2010; A.
H. Kydd 2006). This is especially true for the persistent intercommunal
and sectarian violence that plagues states without the power or interest
to intervene in the conflict. Where an external actor is available, its
presence is a short-term peace solution and its effects do not endure
with the external actor's departure (Beardsley 2008; Rohner, Thoenig,
and Zilibotti 2013; Weinstein 2005). With or without external actors to
halt the conflict, the groups must eventually build trust rather than
find a substitute for trust.

Increasing trust amidst conflict is difficult, however, for
psychological reasons. Group conflicts are often driven and perpetuated
by animosity and distrust long after the original grievance is forgotten
{[}McDonnel (2017); cite more about intergroup conflict{]}. Animosity
can be an insurmountable barrier to peace for direct and indirect
reasons. Directly, these groups are less likely to trust information
they receive from the other side or any peace commitment they get from
the other side. In this way, animosity directly inhibit intergroup
bargaining and prevents peaceful resolution of the conflict.

Indirectly, animosity biases interpretations of ingroup and outgroup
behavior, preventing each group from updating and improving their trust
in the other. Ingroups will perceive their own belligerent actions as
defensive and justified, whereas behavior by outgroup members may be
perceived as more threatening and more malicious than the same behavior
committed by a neutral party {[}Hewstone (1990); cite
fundamental/ultimate attribution error, motivated reasoning,
confirmation bias, anchoring bias{]}. The perceived negative behavior
may be seen as \emph{defining} of the group, whereas any perceived
positive behavior may be seen as the \emph{exception} to the group
(Allison and Messick 1985; Pettigrew 1979). This biased information
processing reinforces negative group stereotypes and can subvert the
groups' own attempts to foster peace. It can also sabotage intergroup
bargaining by causing the groups to have inaccurate beliefs about each
other and each other's willingness to make peace, adding to information
and commitment problems that cause conflict.

Existing strategies to reduce intergroup conflict do not foster trust.
One approach to improving intergroup trust comes from intergroup contact
theory. Intergroup contact theory hypothesizes that interactions in
which group members cooperate to achieve shared goals will improve
intergroup relations. Intergroup contact works by demystifying the
outgroup, presenting the other group's perspective, and replacing
imagined stereotypes with firsthand knowledge (Allport 1954; Pettigrew
and Tropp 2008). This type of structured face-to-face contact also
provides groups the opportunity to send costly signals about their
trustworthiness and preference for peace (A. Kydd 2000; Lupia,
McCubbins, and Arthur 1998; Rohner, Thoenig, and Zilibotti 2013).
Intergroup contact is likely especially good at reducing intergroup
conflict when groups cooperate to achieve superordinate goals -- goals
that require the cooperation of both groups and benefit both groups --
because groups experience the material benefits of cooperation (Gaertner
et al. 2000; Sherif 1958).

The effectiveness of intergroup contact has been demonstrated in a
variety of contexts and using a variety of methodological approaches
(Paluck, Green, and Green 2017; Pettigrew and Tropp 2006). Notably,
intergroup contacted programs reduced prejudice between white people and
black people in the U.S. South Africa, and Norway (Burns, Corno, and La
Ferrara 2015; Finseraas and Kotsadam 2017; Marmaros and Sacerdote 2006),
Jews and Arabs (Ditlmann and Samii 2016; Yablon 2012), and Hindus and
Muslims in India (Barnhardt 2009). In Nigeria, a recent study found that
intergroup contact between Muslims and Christians decreased
discrimination relative to a group that experienced \_intra\_group
contact, suggesting that intergroup contact can work by countering the
adverse effects of ingroup-only experiences (Scacco and Warren 2018).

Although research shows support for intergroup contact theory generally,
its efficacy to reduce prejudice amid real-world conflict is an open
question (Ditlmann, Samii, and Zeitzoff 2017). Negative experiences with
outgroups increase prejudice, and the most prejudiced individuals are
most likely to interpret intergroup contact negatively (Gubler 2013;
Paolini, Harwood, and Rubin 2010). Its impact on interracial and
interethnic attitudes has also been challenged by recent reviews
(Paluck, Green, and Green 2017). Despite a lack of evidence about the
effects of contact-based peacebuilding programs in violent contexts, and
the risks of programs going badly, peacebuilding organizations implement
numerous contact-based interventions in violent contexts each year
(Ditlmann, Samii, and Zeitzoff 2017). These peacebuilding programs might
defuse intergroup conflict, but these programs also might do more harm
than good.

To study the effect of bottom-up peacebuilding interventions on violent
conflict, we conduct a field experiment with farmer and pastoralist
communities in Nigeria to determine if an intergroup contact-based
program effectively increases intergroup trust and increases the
physical security of group members. We randomly assigned communities
with farmer-pastoralist violence to receive the peacebuilding
intervention or serve as a control group. The intervention formed
mixed-group committees and provided them with funds to build
infrastructure that would benefit both communities; committees then
collaboratively chose and constructed infrastructure projects.\footnote{The
  communities built boreholes, market stalls, primary health care
  facilities, etc.} The program also provided mediation training to each
community's leaders and held forums where the groups discussed the
underlying drivers of conflict. To measure the effects of the
intervention, we conducted pre- and post-intervention surveys, a
post-intervention natural public goods behavioral game,\footnote{In a
  public goods game (PGG), research subjects are given money and told
  they can keep the money or donate it to a public fund. Money donated
  to the public fund is multiplied by some amount and then shared with
  all subjects. Our PGG is \emph{natural} because it was conducted in a
  natural setting, rather than a lab. The funding for the PGG came from
  the National Science Foundation under Grant No. 1656871.} and twelve
months of systematic observations in markets and social events during
the intervention.

We find that the program increased intergroup trust, intergroup contact,
and perceptions of physical security. Compared to the control group, the
treatment group expressed more outgroup trust and more willingness to
interact with outgroup members. The treatment group was also less
prevented by violence from engaging in routine tasks, such as working,
going to the market, and getting water. The results also suggests that
intergroup contact for a relatively small percentage of a group can
indirectly affect attitudes of group members with no exogenous increase
in contact with the outgroup. We observe the most positive changes from
individuals directly involved in the intergroup committees, but we also
observe positive spillovers of trust to group members who were not
involved in the intergroup contact intervention.

This study expands our knowledge about intergroup conflict in several
ways. First, this study teaches us about the capacity of contact-based
peacebuilding programs to improve intergroup relations. To our knowledge
this is the first field experimental test of a contact-based
peacebuilding program implemented during an active conflict. We
evaluated the program's effects on both attitudinal and behavioral
outcomes. The results suggest that contact-based peacebuilding programs
can effectively improve relationships between conflicting groups and is
especially relevant to conflict resolution in the cases of intergroup
and intercommunal conflicts.

Second, this study is one of the few field experimental interventions to
test intergroup contact theory with groups actively engaged in violence.
Each of the groups in our study were part of an active and escalating
conflict, with members of each side being killed within one year of the
intervention's onset. Even in such a context, community members who
engaged in direct interpersonal interaction with an outgroup member
changed more positively than other community members. Importantly, the
intergroup contact involved achieving superordinate goals that benefited
both groups materially. This suggests that contact with superordinate
goals is robust to actively violent contexts.

Third, we contribute to the literature about the role of social
diffusion and informal institutions in shaping attitudes and behaviors.
Bottom-up peacebuilding interventions seek to provide a structure in
which groups can solve their own conflicts, and those structures are
informal rather than formal. Understanding how those informal structures
form and shape attitudes, norms, and behaviors of the wider population
addresses the questions of scale for these programs.

Fourth, this paper's focus on farmer-pastoralist conflict is especially
important because the pastoralists are of the Fulani ethnic group. The
Fulani are the largest semi-nomadic people on Earth, but their way of
life makes them targets for violence throughout Africa. Along with this
conflict in Nigeria, Fulani in Mali have been the targets of violence so
severe that researchers at Armed Conflict Location \& Event Data Project
called it ``ethnic cleansing'' (Economist 2019). Understanding how to
prevent conflict between Fulani and settled peoples can help prevent the
eradication of a people and their way of life.

In the next section we provide a theoretical framework for how and why
bottom-up interventions that focus on intergroup contact can improve
trust and security. We then discuss Nigeria's farmer-pastoralist
conflict, our experimental intervention, and two designs to evaluate the
effect of the intervention. Last we present the results of the study and
conclude by connecting these findings to theories of group conflict and
prejudice.

\section*{References}\label{references}
\addcontentsline{toc}{section}{References}

\hypertarget{refs}{}
\hypertarget{ref-allison1985group}{}
Allison, Scott T, and David M Messick. 1985. ``The Group Attribution
Error.'' \emph{Journal of Experimental Social Psychology} 21(6):
563--79.

\hypertarget{ref-allport1954prejudice}{}
Allport, Gordon. 1954. ``The Nature of Prejudice.'' \emph{Garden City,
NJ Anchor}.

\hypertarget{ref-barnhardt2009near}{}
Barnhardt, Sharon. 2009. ``Near and Dear? Evaluating the Impact of
Neighbor Diversity on Inter-Religious Attitudes.'' \emph{Unpublished
working paper}.

\hypertarget{ref-beardsley2008agreement}{}
Beardsley, Kyle. 2008. ``Agreement Without Peace? International
Mediation and Time Inconsistency Problems.'' \emph{American journal of
political science} 52(4): 723--40.

\hypertarget{ref-burns2015interaction}{}
Burns, Justine, Lucia Corno, and Eliana La Ferrara. 2015.
\emph{Interaction, Prejudice and Performance. Evidence from South
Africa}. Working paper.

\hypertarget{ref-daniel2018anti}{}
Daniel, Soni. 2018. ``Anti-Open Grazing Law: Nass, Benue, Kwara, Taraba
Tackle Defence Minister.'' \emph{Vanguard}.
\url{https://www.vanguardngr.com/2018/06/anti-open-grazing-law-nass-benue-kwara-taraba-tackle-defence-minister/}.

\hypertarget{ref-di2017effectiveness}{}
Di Salvatore, Jessica, and Andrea Ruggeri. 2017. ``Effectiveness of
Peacekeeping Operations.'' \emph{Oxford Research Encyclopedia of
Politics}.

\hypertarget{ref-ditlmann2016can}{}
Ditlmann, Ruth K, and Cyrus Samii. 2016. ``Can Intergroup Contact Affect
Ingroup Dynamics? Insights from a Field Study with Jewish and
Arab-Palestinian Youth in Israel.'' \emph{Peace and Conflict: Journal of
Peace Psychology} 22(4): 380.

\hypertarget{ref-ditlmann2017addressing}{}
Ditlmann, Ruth K, Cyrus Samii, and Thomas Zeitzoff. 2017. ``Addressing
Violent Intergroup Conflict from the Bottom up?'' \emph{Social Issues
and Policy Review} 11(1): 38--77.

\hypertarget{ref-doyle2000international}{}
Doyle, Michael W, and Nicholas Sambanis. 2000. ``International
Peacebuilding: A Theoretical and Quantitative Analysis.'' \emph{American
political science review} 94(4): 779--801.

\hypertarget{ref-duru2018court}{}
Duru, Peter. 2018. ``Court Stops Inspector General from Proscribing
Benue Livestock Guard.'' \emph{Vanguard}.
\url{https://www.vanguardngr.com/2018/11/court-stops-ig-from-proscribing-benue-livestock-guards/}.

\hypertarget{ref-economist2019militias}{}
Economist, The. 2019. ``Malicious Malitias: States in the Sahel Have
Unleashed Ethnic Gangs with Guns.'' \emph{The Economist}.
\url{https://www.economist.com/middle-east-and-africa/2019/05/04/states-in-the-sahel-have-unleashed-ethnic-gangs-with-guns}.

\hypertarget{ref-fearon1994ethnic}{}
Fearon, James D. 1994. ``Ethnic War as a Commitment Problem.'' In
\emph{Annual Meetings of the American Political Science Association},
2--5.

\hypertarget{ref-fearon1995rationalist}{}
---------. 1995. ``Rationalist Explanations for War.''
\emph{International organization} 49(3): 379--414.

\hypertarget{ref-fey2010shuttle}{}
Fey, Mark, and Kristopher W Ramsay. 2010. ``When Is Shuttle Diplomacy
Worth the Commute? Information Sharing Through Mediation.'' \emph{World
Politics} 62(4): 529--60.

\hypertarget{ref-finseraas2017does}{}
Finseraas, Henning, and Andreas Kotsadam. 2017. ``Does Personal Contact
with Ethnic Minorities Affect Anti-Immigrant Sentiments? Evidence from a
Field Experiment.'' \emph{European Journal of Political Research} 56(3):
703--22.

\hypertarget{ref-gaertner2000reducing}{}
Gaertner, Samuel L et al. 2000. ``Reducing Intergroup Conflict: From
Superordinate Goals to Decategorization, Recategorization, and Mutual
Differentiation.'' \emph{Group Dynamics: Theory, Research, and Practice}
4(1): 98.

\hypertarget{ref-gartner2011signs}{}
Gartner, Scott Sigmund. 2011. ``Signs of Trouble: Regional Organization
Mediation and Civil War Agreement Durability.'' \emph{The Journal of
Politics} 73(2): 380--90.

\hypertarget{ref-gubler2013humanizing}{}
Gubler, Joshua R. 2013. ``When Humanizing the Enemy Fails: The Role of
Dissonance and Justification in Intergroup Conflict.'' In \emph{Annual
Meeting of the American Political Science Association},

\hypertarget{ref-hartzell2001stabilizing}{}
Hartzell, Caroline, Matthew Hoddie, and Donald Rothchild. 2001.
``Stabilizing the Peace After Civil War: An Investigation of Some Key
Variables.'' \emph{International organization} 55(1): 183--208.

\hypertarget{ref-council2019nigeria}{}
Harwood, Asch. 2019. ``Update: The Numbers Behind Sectarian Violence in
Nigeria.'' \emph{Council on Foreign Relations}.
\url{https://www.cfr.org/blog/update-numbers-behind-sectarian-violence-nigeria}.

\hypertarget{ref-hewstone1990ultimate}{}
Hewstone, Miles. 1990. ``The `Ultimate Attribution Error'? A Review of
the Literature on Intergroup Causal Attribution.'' \emph{European
Journal of Social Psychology} 20(4): 311--35.

\hypertarget{ref-kydd2000trust}{}
Kydd, Andrew. 2000. ``Trust, Reassurance, and Cooperation.''
\emph{International Organization} 54(2): 325--57.

\hypertarget{ref-kydd2006can}{}
Kydd, Andrew H. 2006. ``When Can Mediators Build Trust?'' \emph{American
Political Science Review} 100(3): 449--62.

\hypertarget{ref-lupia1998democratic}{}
Lupia, Arthur, Mathew D McCubbins, and Lupia Arthur. 1998. \emph{The
Democratic Dilemma: Can Citizens Learn What They Need to Know?}
Cambridge University Press.

\hypertarget{ref-marmaros2006friendships}{}
Marmaros, David, and Bruce Sacerdote. 2006. ``How Do Friendships Form?''
\emph{The Quarterly Journal of Economics} 121(1): 79--119.

\hypertarget{ref-mcdonnel2017graze}{}
McDonnel, Tim. 2017. ``Why It's Now a Crime to Let Cattle Graze Freely
in 2 Nigerian States.'' \emph{National Public Radio (NPR)}.
\url{https://www.npr.org/sections/goatsandsoda/2017/12/12/569913821/why-its-now-a-crime-to-let-cattle-graze-freely-in-2-nigerian-states}.

\hypertarget{ref-mcdougal2015effect}{}
McDougal, Topher L et al. 2015. ``The Effect of Farmer-Pastoralist
Violence on Income: New Survey Evidence from Nigeria's Middle Belt
States.'' \emph{Economics of Peace and Security Journal} 10(1): 54--65.

\hypertarget{ref-ostrom2003trust}{}
Ostrom, Elinor, and James Walker. 2003. \emph{Trust and Reciprocity:
Interdisciplinary Lessons for Experimental Research}. Russell Sage
Foundation.

\hypertarget{ref-paluck2017contact}{}
Paluck, Elizabeth Levy, Seth Green, and Donald P Green. 2017. ``The
Contact Hypothesis Revisited.''

\hypertarget{ref-paolini2010negative}{}
Paolini, Stefania, Jake Harwood, and Mark Rubin. 2010. ``Negative
Intergroup Contact Makes Group Memberships Salient: Explaining Why
Intergroup Conflict Endures.'' \emph{Personality and Social Psychology
Bulletin} 36(12): 1723--38.

\hypertarget{ref-pettigrew1979ultimate}{}
Pettigrew, Thomas F. 1979. ``The Ultimate Attribution Error: Extending
Allport's Cognitive Analysis of Prejudice.'' \emph{Personality and
social psychology bulletin} 5(4): 461--76.

\hypertarget{ref-pettigrew2006meta}{}
Pettigrew, Thomas F, and Linda R Tropp. 2006. ``A Meta-Analytic Test of
Intergroup Contact Theory.'' \emph{Journal of personality and social
psychology} 90(5): 751.

\hypertarget{ref-pettigrew2008does}{}
---------. 2008. ``How Does Intergroup Contact Reduce Prejudice?
Meta-Analytic Tests of Three Mediators.'' \emph{European Journal of
Social Psychology} 38(6): 922--34.

\hypertarget{ref-powell2006war}{}
Powell, Robert. 2006. ``War as a Commitment Problem.''
\emph{International organization} 60(1): 169--203.

\hypertarget{ref-rohner2013war}{}
Rohner, Dominic, Mathias Thoenig, and Fabrizio Zilibotti. 2013. ``War
Signals: A Theory of Trade, Trust, and Conflict.'' \emph{Review of
Economic Studies} 80(3): 1114--47.

\hypertarget{ref-scacco2018nigeria}{}
Scacco, Alexandra, and Shana S Warren. 2018. ``Can Social Contact Reduce
Prejudice and Discrimination? Evidence from a Field Experiment in
Nigeria.'' \emph{American Political Science Review} 112(3): 654--77.

\hypertarget{ref-sherif1958superordinate}{}
Sherif, Muzafer. 1958. ``Superordinate Goals in the Reduction of
Intergroup Conflict.'' \emph{American journal of Sociology} 63(4):
349--56.

\hypertarget{ref-smith2003mediation}{}
Smith, Alastair, and Allan Stam. 2003. ``Mediation and Peacekeeping in a
Random Walk Model of Civil and Interstate War.'' \emph{International
Studies Review} 5(4): 115--35.

\hypertarget{ref-wallensteen2014talking}{}
Wallensteen, Peter, and Isak Svensson. 2014. ``Talking Peace:
International Mediation in Armed Conflicts.'' \emph{Journal of Peace
Research} 51(2): 315--27.

\hypertarget{ref-walter2002committing}{}
Walter, Barbara F. 2002. \emph{Committing to Peace: The Successful
Settlement of Civil Wars}. Princeton University Press.

\hypertarget{ref-weinstein2005autonomous}{}
Weinstein, Jeremy M. 2005. ``Autonomous Recovery and International
Intervention in Comparative Perspective.'' \emph{Available at SSRN
1114117}.

\hypertarget{ref-yablon2012we}{}
Yablon, Yaacov B. 2012. ``Are We Preaching to the Converted? The Role of
Motivation in Understanding the Contribution of Intergroup Encounters.''
\emph{Journal of Peace Education} 9(3): 249--63.

\end{document}
