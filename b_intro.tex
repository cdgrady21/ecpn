%\documentclass[]{article}
\documentclass[11pt]{article}
\usepackage[usenames,dvipsnames]{xcolor}

\usepackage[T1]{fontenc}
%\usepackage{lmodern}
\usepackage{tgtermes}
\usepackage{amssymb,amsmath}
%\usepackage[margin=1in]{geometry}
\usepackage[letterpaper,bottom=1in,top=1in,right=1.25in,left=1.25in,includemp=FALSE]{geometry}
\usepackage{pdfpages}
\usepackage[small]{caption}

\usepackage{ifxetex,ifluatex}
\usepackage{fixltx2e} % provides \textsubscript
% use microtype if available
\IfFileExists{microtype.sty}{\usepackage{microtype}}{}
\ifnum 0\ifxetex 1\fi\ifluatex 1\fi=0 % if pdftex
\usepackage[utf8]{inputenc}
\else % if luatex or xelatex
\usepackage{fontspec}
\ifxetex
\usepackage{xltxtra,xunicode}
\fi
\defaultfontfeatures{Mapping=tex-text,Scale=MatchLowercase}
\newcommand{\euro}{€}
\fi
%

\usepackage{fancyvrb}

\usepackage{ctable,longtable}

\usepackage[section]{placeins}
\usepackage{float} % provides the H option for float placement
\restylefloat{figure}
\usepackage{dcolumn} % allows for different column alignments
\newcolumntype{.}{D{.}{.}{1.2}}

\usepackage{booktabs} % nicer horizontal rules in tables

%Assume we want graphics always
\usepackage{graphicx}
% We will generate all images so they have a width \maxwidth. This means
% that they will get their normal width if they fit onto the page, but
% are scaled down if they would overflow the margins.
%% \makeatletter
%% \def\maxwidth{\ifdim\Gin@nat@width>\linewidth\linewidth
%%   \else\Gin@nat@width\fi}
%% \makeatother
%% \let\Oldincludegraphics\includegraphics
%% \renewcommand{\includegraphics}[1]{\Oldincludegraphics[width=\maxwidth]{#1}}
\graphicspath{{.}{../Soccom_Code/socom_2013/}}


%% \ifxetex
%% \usepackage[pagebackref=true, setpagesize=false, % page size defined by xetex
%% unicode=false, % unicode breaks when used with xetex
%% xetex]{hyperref}
%% \else
\usepackage[pagebackref=true, unicode=true, bookmarks=true, pdftex]{hyperref}
% \fi


\hypersetup{breaklinks=true,
  bookmarks=true,
  pdfauthor={Christopher Grady, Rebecca Wolfe, Danjuma Dawop, and Lisa Inks},
  pdftitle={Promoting Peace Amidst Group Conflict: An Intergroup Contact Field Experiment in Nigeria - Introduction},
  colorlinks=true,
  linkcolor=BrickRed,
  citecolor=blue, %MidnightBlue,
  urlcolor=BrickRed,
  % urlcolor=blue,
  % linkcolor=magenta,
  pdfborder={0 0 0}}

%\setlength{\parindent}{0pt}
%\setlength{\parskip}{6pt plus 2pt minus 1pt}
\usepackage{parskip}
\setlength{\emergencystretch}{3em}  % prevent overfull lines
\providecommand{\tightlist}{%
  \setlength{\itemsep}{0pt}\setlength{\parskip}{0pt}}

%% Insist on this.
\setcounter{secnumdepth}{2}

\VerbatimFootnotes % allows verbatim text in footnotes

\title{Promoting Peace Amidst Group Conflict: An Intergroup Contact Field
Experiment in Nigeria - Introduction}

\author{
Christopher Grady, Rebecca Wolfe, Danjuma Dawop, and Lisa Inks
}


\date{October 27, 2019}


\usepackage{versions}
\makeatletter
\renewcommand*\versionmessage[2]{\typeout{*** `#1' #2. ***}}
\renewcommand*\beginmarkversion{\sffamily}
  \renewcommand*\endmarkversion{}
\makeatother

\excludeversion{comment}

%\usepackage[margins=1in]{geometry}

\usepackage[compact,bottomtitles]{titlesec}
%\titleformat{ ⟨command⟩}[⟨shape⟩]{⟨format⟩}{⟨label⟩}{⟨sep⟩}{⟨before⟩}[⟨after⟩]
\titleformat{\section}[hang]{\Large\bfseries}{\thesection}{.5em}{\hspace{0in}}[\vspace{-.2\baselineskip}]
\titleformat{\subsection}[hang]{\large\bfseries}{\thesubsection}{.5em}{\hspace{0in}}[\vspace{-.2\baselineskip}]
%\titleformat{\subsubsection}[hang]{\bfseries}{\thesubsubsection}{.5em}{\hspace{0in}}[\vspace{-.2\baselineskip}]
\titleformat{\subsubsection}[hang]{\bfseries}{\thesubsubsection}{1ex}{\hspace{0in}}[\vspace{-.2\baselineskip}]
\titleformat{\paragraph}[runin]{\bfseries\itshape}{\theparagraph}{1ex}{}{\vspace{-.2\baselineskip}}
%\titleformat{\paragraph}[runin]{\itshape}{\theparagraph}{1ex}{}{\vspace{-.2\baselineskip}}

%%\titleformat{\subsection}[hang]{\bfseries}{\thesubsection}{.5em}{\hspace{0in}}[\vspace{-.2\baselineskip}]
%%%\titleformat*{\subsection}{\bfseries\scshape}
%%%\titleformat{\subsubsection}[leftmargin]{\footnotesize\filleft}{\thesubsubsection}{.5em}{}{}
%%\titleformat{\subsubsection}[hang]{\small\bfseries}{\thesubsubsection}{.5em}{\hspace{0in}}[\vspace{-.2\baselineskip}]
%%\titleformat{\paragraph}[runin]{\itshape}{\theparagraph}{1ex}{}{\vspace{-.5\baselineskip}}

%\titlespacing*{ ⟨command⟩}{⟨left⟩}{⟨beforesep⟩}{⟨aftersep⟩}[⟨right⟩]
\titlespacing{\section}{0pc}{1.5ex plus .1ex minus .2ex}{.5ex plus .1ex minus .1ex}
\titlespacing{\subsection}{0pc}{1.5ex plus .1ex minus .2ex}{.5ex plus .1ex minus .1ex}
\titlespacing{\subsubsection}{0pc}{1.5ex plus .1ex minus .2ex}{.5ex plus .1ex minus .1ex}



%% These next lines tell latex that it is ok to have a single graphic
%% taking up most of a page, and they also decrease the space around
%% figures and tables.
\renewcommand\floatpagefraction{.9}
\renewcommand\topfraction{.9}
\renewcommand\bottomfraction{.9}
\renewcommand\textfraction{.1}
\setcounter{totalnumber}{50}
\setcounter{topnumber}{50}
\setcounter{bottomnumber}{50}
\setlength{\intextsep}{2ex}
\setlength{\floatsep}{2ex}
\setlength{\textfloatsep}{2ex}



\begin{document}
\VerbatimFootnotes

%\begin{titlepage}
%  \maketitle
%\vspace{2in}
%
%\begin{center}
%  \begin{large}
%    PROPOSAL WHITE PAPER
%
%BAA 14-013
%
%Can a Hausa Language Television Station Change Norms about Violence in Northern Nigeria? A Randomized Study of Media Effects on Violent Extremism
%
%Jake Bowers
%
%University of Illinois @ Urbana-Champaign (jwbowers@illinois.edu)
%
%\url{http://jakebowers.org}
%
%Phone: +12179792179
%
%Topic Number: 1
%
%Topic Title: Identity, Influence and Mobilization
%
%\end{large}
%\end{center}
%\end{titlepage}

\maketitle

\hypertarget{introduction}{%
\subsection{Introduction}\label{introduction}}

Violent conflicts are one of most crucial phenomena for humans to
overcome. As of 2018, violent conflicts caused 2 million deaths since
the year 2000 (Sundberg and Melander 2013), were responsible for
forcibly displacing over 70 million people from their homes (UNHCR
statistical yearbook 2019), , threatened food supplies in numerous
countries (Verwimp and others 2012), and extracted a psychological toll
on participants and victims (chris: cite Schomerus and Rigterink from
ISA Toronto?). Despite extensive research, violent conflict is still
common and methods to prevent violent conflict between groups are
elusive.

There exists two main perspectives about group conflict. One
perspective, coming from psychology, identifies group conflict as a
problem of emotion and group identity. This perspective emphasizes
things like perceived threat ({\textbf{???}}; {\textbf{???}}), anger and
aggression ({\textbf{???}}; {\textbf{???}}; {\textbf{???}}; Halperin
2011), ethnic or religious hatred ({\textbf{???}}; {\textbf{???}};
Cikara et al. 2014), perceptual biases (Duncan 1976; Hewstone 1990; Ward
et al. 1997), and dehumanization ({\textbf{???}}; Haslam and Loughnan
2014) as the drivers of group violence. This perspective points to
intergroup contact {[}({\textbf{???}}); ({\textbf{???}});
({\textbf{???}})), perspective-taking ({\textbf{???}}), and expert
appeals ({\textbf{???}}) to reduce violent conflict.

The second perspective, coming from international relations, identifies
group conflict as a bargaining problem (Fearon 1994, 1995; Powell 2006).
Two groups want some resource -- land, power, etc -- and must decide how
to distribute that resource. Violent conflict occurs when one or both
sides expect their utility from fighting to be higher than their utility
from bargaining. This perspective emphasizes two factors: (1) the
\emph{bargaining range} (i.e.~the number of bargained outcomes
preferable to fighting) ({\textbf{???}}; {\textbf{???}}) and (2)
\emph{trust} that the other side will honor bargained agreements (Kydd
2000; Powell 2006; Reed et al. 2016). This perspective points to third
party mediation ({\textbf{???}}; Doyle and Sambanis 2000; Fearon 1998)
and \emph{costly signals of trustworthiness/reputations for
trustworthiness} (Kydd 2000; Rohner, Thoenig, and Zilibotti 2013) to
reduce violent conflict.\footnote{Note to lab: Two psych-oriented
  theories are similar to formal theory of conflict. (1) Realistic group
  conflict theory from Campbell 1965: group conflict because groups
  compete for resources and have ``incompatible goals''. (2) Nisbett and
  Cohen 1996 about a reputation for toughness. I am not sure if I should
  bring those }

The psychological perspective and the international relations
perspective each make valuable contributions to the study of group
conflict, but each is incomplete without the other. The IR perspective
provides a model that elucides the fundamental causes of group conflict:
few agreements acceptable to both sides and low trust that the other
side will honor those agreements. It is portable, parsimonious, and
provides a basis to understand how other factors affect group conflict
by influencing the bargaining range or the formation of intergroup
trust. The IR perspective, however, does not always consider the myriad
non-material factors that influence bargaining ranges or trust
formation.\footnote{chris note: This sentence does not feel right. Would
  like advice.}

The psychological perspective provides details about those other
factors. The acceptable bargaining range is affected by outgroup
dehumanization and hatred. Those factors provide non-material costs to
bargained compromises that limit the number of compromises preferable to
fighting. The formation of intergroup trust is affected by perceptual
biases that distort outgroup behavior. Perceptual biases prevent one
side's costly signals of trustworthiness from being perceived as such by
the other side. The psychological perspective, however, does not have a
unified model of conflict to formalize how these various factors affect
violent conflict.

\textbf{OR}

The psychological perspective and the international relations
perspective each make valuable contributions to the study of group
conflict, but each is incomplete without the other. The IR perspective
provides a model that elucides the fundamental cause of violent group
conflict: bargaining failure. It is portable, parsimonious, and provides
a basis to understand how other factors increase or decrease the
likelihood of violent conflict. The IR perspective, however, does not
always consider the psychological factors that make bargaining more or
less difficult. The psychological perspective theorizes about the
psychological factors that influence group conflict, like emotion and
perceptual biases. It does not, however, provide a model of group
conflict from which psychological factors can be understood.

These perspectives work together to provide a more complete
understanding of group conflict.\footnote{I don't like this sentence.
  Would it be better as ``\ldots{}work together to provide a model of
  group conflict and understanding of how psychological factors work
  within that model.''?} Violent group conflict occurs when two groups
fail to bargain, and bargaining fails when there are few agreements
acceptable to both sides and low trust that the other side will honor
their agreements. The number of agreements acceptable to both sides
depends on psychological factors like outgroup dehumanization and
hatred; those factors provide non-material costs to bargained
compromises that limit the number of compromises preferable to fighting.
The formation of intergroup trust is affected by perceptual biases that
distort outgroup behavior; perceptual biases prevent one side's costly
signals of trustworthiness from being perceived as such by the other
side. In short, the IR perspective tells us group conflict is a
bargaining problem and the psychological perspective tells us what makes
that bargaining more or less difficult.

Synthesizing these perspectives, we argue that cooperative intergroup
contact can reduce group conflict by increasing the bargaining range and
building intergroup trust. Cooperative contact -- interactions in which
both groups cooperate to fulfill joint goals -- humanizes the outgroup,
promotes positive affect towards outgroup members, dispels
stereotypes/misperceptions that drive perceptual biases, and provides
opportunities for unequivocal signals of trustworthiness. Humanizing the
outgroup and promoting positive affect increase the bargaining range by
reducing the psychological costs of bargaining with the other side.
Cooperative contact can also demonstrate to groups the material benefits
of cooperation, increasing the expected utility of peace. Correcting
perceptual biases and providing opportunities for costly signals helps
foster intergroup trust and ensures each side that the other will honor
bargained agreements. What's more, cooperative contact for some group
members can diffuse throughout the entire group through the process of
indirect contact (knowledge of friendships between ingroup and outgroup
members) and changes to social norms.\footnote{This paragraph is not
  right yet and, anticipating changes, I have not added the citations.}

To test the ability of cooperative contact to reduce violent conflict,
we conducted a field experiment with conflicting farmer and pastoralist
communities in Nigeria. More than an occupational difference, farmers
who cultivate crops and pastoralists who graze cattle define a major
social cleavage in many parts of the world. These groups conflict over
land rights, which define both of their livelihoods. Farmer-pastoralist
conflict has escalated througout the Sahel in recent years, and nowhere
more than in Nigeria. The most recent conflict escalation has caused
7,000 deaths in the past five years, displaced hundreds of thousands of
people from their homes, and costs \$13 billion annually in lost
economic productivity (Akinwotu 2018; Daniel 2018; Harwood 2019;
McDougal et al. 2015).

We randomly assigned communities with ongoing farmer-pastoralist
violence to receive a contact-based intervention or serve as a control
group. The intervention formed mixed-group committees and provided them
with funds to build infrastructure that would benefit both communities;
committees then collaboratively chose and constructed infrastructure
projects.\footnote{The communities built boreholes, market stalls,
  primary health care facilities, etc.} The program also provided
mediation training to each community's leaders and held forums where the
groups discussed the underlying drivers of conflict. To measure the
effects of the intervention, we conducted pre- and post-intervention
surveys, a post-intervention natural public goods behavioral
game,\footnote{In a public goods game (PGG), research subjects are given
  money and told they can keep the money or donate it to a public fund.
  Money donated to the public fund is multiplied by some amount and then
  shared with all subjects. Our PGG is \emph{natural} because it was
  conducted in a natural setting, rather than a lab. The funding for the
  PGG came from the National Science Foundation under Grant No.~1656871.}
and twelve months of systematic observations in markets and social
events during the intervention.

We find that the program increased intergroup contact, intergroup trust,
and perceptions of physical security. We see signs of the positive
effects in fieldwork as well as in data -- in one of the treatment
sites, farmers defended pastoralists from a group of anti-pastoralist
vigilantes, rather than assist the vigilantes in removing the
pastoralists and claiming their land. Our results also show that the
intervention affected communities as a whole, not just community members
directly involved in the intergroup contact. Individuals who directly
engaged in intergroup contact changed the most positively from baseline
to endline, but we also observe positive spillovers of trust to group
members for whom we did not exogenously increase intergroup contact.

This study expands our knowledge about group conflict in several ways.
First, this study synthesizes psychological and rationalist explanations
for violent group conflict, demonstrating how the two perspectives
complement each other. Rationalist explanations teach that group
conflict is a bargaining problem and violence is caused by incompatible
incentives and mistrust; psychological explanations teach that emotions
and group biases cause violent group conflict. We synthesize these
perspectives to explain how psychological factors affect intergroup
bargaining and increase or decrease the likelihood of successful
bargaining.

Second, this study teaches us about the capacity of intergroup contact
to improve intergroup relations and reduce conflict. Peacebuilding
organizations implement numerous contact-based interventions in violent
contexts each year (Ditlmann, Samii, and Zeitzoff 2017), but its
efficacy to improve intergroup attitudes amid real-world conflict is an
open question (Ditlmann, Samii, and Zeitzoff 2017; Paluck, Green, and
Green 2017). To our knowledge this is the first field experimental test
of a contact-based intervention implemented during an active conflict.
The results suggest that contact-based peacebuilding programs can
effectively improve relationships between conflicting groups and is
especially relevant to conflict resolution in the cases of intergroup
and intercommunal conflicts.

Third, we contribute to the literature about informal structures, such
as social norms, in solving collective action problems. In some
contexts, formal institutions ensure collective action by punishing
groups and individuals who ``defect'' on agreements. In many contexts,
such as rural Nigeria, no formal institutions exist to encourage such
behavior and so groups in those contexts develop informal structures to
achieve collective action (Ostrom 2000). This intervention showed how
informal structures governing interactions between groups can develop
through repeated intergroup interaction. Our intervention only engaged a
small percentage of each community, yet its effects diffused to other
community members. Creating informal structures that diffuse the effects
of contact are a way of scaling up peacebuilding
interventions.\footnote{chris: this paragraph is too long and not
  focused enough. Would appreciate comments.}

Fourth, this paper teaches us about settling disputes between sedentary
peoples and nomadic peoples. Violent conflict between settled peoples
and nomadic peoples is on the rise throughout Africa (Kuusaana and
Bukari 2015; Mwamfupe 2015; Nnoko-Mewanu 2018). This study focuses on
the Fulani, the largest semi-nomadic people on Earth (Encyclopedia
2017). Their way of life makes them targets for violence throughout
Africa. Along with this conflict in Nigeria, Fulani in Mali have been
the targets of violence so severe that researchers at Armed Conflict
Location \& Event Data Project called it ``ethnic cleansing'' (Economist
2019). Understanding how to prevent violent conflict between Fulani and
settled peoples can help prevent violence that targets other nomadic and
semi-nomadic peoples, such as the Tuaregs in West Africa, Uyghurs in
Central Asia, Kochi in Afghanistan, and Khoisan of Southern Africa.
Preventing such violence could help preserve a dying way of life.

\begin{center}\rule{0.5\linewidth}{\linethickness}\end{center}

\begin{center}\rule{0.5\linewidth}{\linethickness}\end{center}

\hypertarget{references}{%
\section*{References}\label{references}}
\addcontentsline{toc}{section}{References}

\hypertarget{refs}{}
\leavevmode\hypertarget{ref-nyt2018nigeria}{}%
Akinwotu, Emmanuel. 2018. ``Nigeria's Farmers and Herders Fight a Deadly
Battle for Scarce Resources.'' \emph{New York Times}.
\url{https://www.nytimes.com/2018/06/25/world/africa/nigeria-herders-farmers.html}.

\leavevmode\hypertarget{ref-cikara2014their}{}%
Cikara, Mina, Emile Bruneau, Jay J Van Bavel, and Rebecca Saxe. 2014.
``Their Pain Gives Us Pleasure: How Intergroup Dynamics Shape Empathic
Failures and Counter-Empathic Responses.'' \emph{Journal of experimental
social psychology} 55: 110--25.

\leavevmode\hypertarget{ref-daniel2018anti}{}%
Daniel, Soni. 2018. ``Anti-Open Grazing Law: Nass, Benue, Kwara, Taraba
Tackle Defence Minister.'' \emph{Vanguard}.
\url{https://www.vanguardngr.com/2018/06/anti-open-grazing-law-nass-benue-kwara-taraba-tackle-defence-minister/}.

\leavevmode\hypertarget{ref-ditlmann2017addressing}{}%
Ditlmann, Ruth K, Cyrus Samii, and Thomas Zeitzoff. 2017. ``Addressing
Violent Intergroup Conflict from the Bottom up?'' \emph{Social Issues
and Policy Review} 11(1): 38--77.

\leavevmode\hypertarget{ref-doyle2000international}{}%
Doyle, Michael W, and Nicholas Sambanis. 2000. ``International
Peacebuilding: A Theoretical and Quantitative Analysis.'' \emph{American
political science review} 94(4): 779--801.

\leavevmode\hypertarget{ref-duncan1976differential}{}%
Duncan, Birt L. 1976. ``Differential Social Perception and Attribution
of Intergroup Violence: Testing the Lower Limits of Stereotyping of
Blacks.'' \emph{Journal of personality and social psychology} 34(4):
590.

\leavevmode\hypertarget{ref-economist2019militias}{}%
Economist, The. 2019. ``Malicious Malitias: States in the Sahel Have
Unleashed Ethnic Gangs with Guns.'' \emph{The Economist}.
\url{https://www.economist.com/middle-east-and-africa/2019/05/04/states-in-the-sahel-have-unleashed-ethnic-gangs-with-guns}.

\leavevmode\hypertarget{ref-fulanisize2017}{}%
Encyclopedia, New World. 2017. ``Fulani --- New World Encyclopedia,''
\url{//www.newworldencyclopedia.org/p/index.php?title=Fulani\&oldid=1004777}.

\leavevmode\hypertarget{ref-fearon1994ethnic}{}%
Fearon, James D. 1994. ``Ethnic War as a Commitment Problem.'' In
\emph{Annual Meetings of the American Political Science Association},
2--5.

\leavevmode\hypertarget{ref-fearon1995rationalist}{}%
---------. 1995. ``Rationalist Explanations for War.''
\emph{International organization} 49(3): 379--414.

\leavevmode\hypertarget{ref-fearon1998commitment}{}%
---------. 1998. ``Commitment Problems and the Spread of Ethnic
Conflict.'' \emph{The international spread of ethnic conflict} 107.

\leavevmode\hypertarget{ref-halperin2011emotional}{}%
Halperin, Eran. 2011. ``The Emotional Roots of Intergroup Aggression:
The Distinct Roles of Anger and Hatred.''

\leavevmode\hypertarget{ref-council2019nigeria}{}%
Harwood, Asch. 2019. ``Update: The Numbers Behind Sectarian Violence in
Nigeria.'' \emph{Council on Foreign Relations}.
\url{https://www.cfr.org/blog/update-numbers-behind-sectarian-violence-nigeria}.

\leavevmode\hypertarget{ref-haslam2014dehumanization}{}%
Haslam, Nick, and Steve Loughnan. 2014. ``Dehumanization and
Infrahumanization.'' \emph{Annual review of psychology} 65: 399--423.

\leavevmode\hypertarget{ref-hewstone1990ultimate}{}%
Hewstone, Miles. 1990. ``The `Ultimate Attribution Error'? A Review of
the Literature on Intergroup Causal Attribution.'' \emph{European
Journal of Social Psychology} 20(4): 311--35.

\leavevmode\hypertarget{ref-kuusaana2015land}{}%
Kuusaana, Elias Danyi, and Kaderi Noagah Bukari. 2015. ``Land Conflicts
Between Smallholders and Fulani Pastoralists in Ghana: Evidence from the
Asante Akim North District (Aand).'' \emph{Journal of rural studies} 42:
52--62.

\leavevmode\hypertarget{ref-kydd2000trust}{}%
Kydd, Andrew. 2000. ``Trust, Reassurance, and Cooperation.''
\emph{International Organization} 54(2): 325--57.

\leavevmode\hypertarget{ref-mcdougal2015effect}{}%
McDougal, Topher L et al. 2015. ``The Effect of Farmer-Pastoralist
Violence on Income: New Survey Evidence from Nigeria's Middle Belt
States.'' \emph{Economics of Peace and Security Journal} 10(1): 54--65.

\leavevmode\hypertarget{ref-mwamfupe2015persistence}{}%
Mwamfupe, Davis. 2015. ``Persistence of Farmer-Herder Conflicts in
Tanzania.'' \emph{International Journal of Scientific and Research
Publications} 5(2): 1--8.

\leavevmode\hypertarget{ref-hrc2018farmer}{}%
Nnoko-Mewanu, Juliana. 2018. ``Farmer-Herder Conflicts on the Rise in
Africa.'' \emph{Human Rights Watch}.

\leavevmode\hypertarget{ref-ostrom2000collective}{}%
Ostrom, Elinor. 2000. ``Collective Action and the Evolution of Social
Norms.'' \emph{Journal of economic perspectives} 14(3): 137--58.

\leavevmode\hypertarget{ref-paluck2017contact}{}%
Paluck, Elizabeth Levy, Seth Green, and Donald P Green. 2017. ``The
Contact Hypothesis Revisited.''

\leavevmode\hypertarget{ref-powell2006war}{}%
Powell, Robert. 2006. ``War as a Commitment Problem.''
\emph{International organization} 60(1): 169--203.

\leavevmode\hypertarget{ref-reed2016bargaining}{}%
Reed, William, David Clark, Timothy Nordstrom, and Daniel Siegel. 2016.
``Bargaining in the Shadow of a Commitment Problem.'' \emph{Research \&
Politics} 3(3): 2053168016666848.

\leavevmode\hypertarget{ref-rohner2013war}{}%
Rohner, Dominic, Mathias Thoenig, and Fabrizio Zilibotti. 2013. ``War
Signals: A Theory of Trade, Trust, and Conflict.'' \emph{Review of
Economic Studies} 80(3): 1114--47.

\leavevmode\hypertarget{ref-ucdp}{}%
Sundberg, Ralph, and Erik Melander. 2013. ``Introducing the Ucdp
Georeferenced Event Dataset.'' \emph{Journal of Peace Research} 50(4):
523--32.

\leavevmode\hypertarget{ref-unhcr2019}{}%
\emph{UNHCR Statistical Yearbook}. 2019.
https://www.unhcr.org/en-us/figures-at-a-glance.html: United Nations
High Commission for Refugees.

\leavevmode\hypertarget{ref-verwimp2012food}{}%
Verwimp, Philip, and others. 2012. ``Food Security, Violent Conflict and
Human Development: Causes and Consequences.'' \emph{United Nations
Development Programme Working Paper}: 1--13.

\leavevmode\hypertarget{ref-ward1997naive}{}%
Ward, Andrew et al. 1997. ``Naive Realism in Everyday Life: Implications
for Social Conflict and Misunderstanding.'' \emph{Values and knowledge}:
103--35.

\end{document}
