%\documentclass[]{article}
\documentclass[11pt]{article}
\usepackage[usenames,dvipsnames]{xcolor}

\usepackage[T1]{fontenc}
%\usepackage{lmodern}
\usepackage{tgtermes}
\usepackage{amssymb,amsmath}
%\usepackage[margin=1in]{geometry}
\usepackage[letterpaper,bottom=1in,top=1in,right=1.25in,left=1.25in,includemp=FALSE]{geometry}
\usepackage{pdfpages}
\usepackage[small]{caption}

\usepackage{ifxetex,ifluatex}
\usepackage{fixltx2e} % provides \textsubscript
% use microtype if available
\IfFileExists{microtype.sty}{\usepackage{microtype}}{}
\ifnum 0\ifxetex 1\fi\ifluatex 1\fi=0 % if pdftex
\usepackage[utf8]{inputenc}
\else % if luatex or xelatex
\usepackage{fontspec}
\ifxetex
\usepackage{xltxtra,xunicode}
\fi
\defaultfontfeatures{Mapping=tex-text,Scale=MatchLowercase}
\newcommand{\euro}{€}
\fi
%

\usepackage{fancyvrb}

\usepackage{ctable,longtable}

\usepackage[section]{placeins}
\usepackage{float} % provides the H option for float placement
\restylefloat{figure}
\usepackage{dcolumn} % allows for different column alignments
\newcolumntype{.}{D{.}{.}{1.2}}

\usepackage{booktabs} % nicer horizontal rules in tables

%Assume we want graphics always
\usepackage{graphicx}
% We will generate all images so they have a width \maxwidth. This means
% that they will get their normal width if they fit onto the page, but
% are scaled down if they would overflow the margins.
%% \makeatletter
%% \def\maxwidth{\ifdim\Gin@nat@width>\linewidth\linewidth
%%   \else\Gin@nat@width\fi}
%% \makeatother
%% \let\Oldincludegraphics\includegraphics
%% \renewcommand{\includegraphics}[1]{\Oldincludegraphics[width=\maxwidth]{#1}}
\graphicspath{{.}{../Soccom_Code/socom_2013/}}


%% \ifxetex
%% \usepackage[pagebackref=true, setpagesize=false, % page size defined by xetex
%% unicode=false, % unicode breaks when used with xetex
%% xetex]{hyperref}
%% \else
\usepackage[pagebackref=true, unicode=true, bookmarks=true, pdftex]{hyperref}
% \fi


\hypersetup{breaklinks=true,
  bookmarks=true,
  pdfauthor={Christopher Grady, Rebecca Wolfe, Danjuma Dawop, and Lisa Inks},
  pdftitle={Promoting Peace Amidst Group Conflict: An Intergroup Contact Field Experiment in Nigeria - Introduction},
  colorlinks=true,
  linkcolor=BrickRed,
  citecolor=blue, %MidnightBlue,
  urlcolor=BrickRed,
  % urlcolor=blue,
  % linkcolor=magenta,
  pdfborder={0 0 0}}

%\setlength{\parindent}{0pt}
%\setlength{\parskip}{6pt plus 2pt minus 1pt}
\usepackage{parskip}
\setlength{\emergencystretch}{3em}  % prevent overfull lines
\providecommand{\tightlist}{%
  \setlength{\itemsep}{0pt}\setlength{\parskip}{0pt}}

%% Insist on this.
\setcounter{secnumdepth}{2}

\VerbatimFootnotes % allows verbatim text in footnotes

\title{Promoting Peace Amidst Group Conflict: An Intergroup Contact Field
Experiment in Nigeria - Introduction}

\author{
Christopher Grady, Rebecca Wolfe, Danjuma Dawop, and Lisa Inks
}


\date{June 08, 2019}


\usepackage{versions}
\makeatletter
\renewcommand*\versionmessage[2]{\typeout{*** `#1' #2. ***}}
\renewcommand*\beginmarkversion{\sffamily}
  \renewcommand*\endmarkversion{}
\makeatother

\excludeversion{comment}

%\usepackage[margins=1in]{geometry}

\usepackage[compact,bottomtitles]{titlesec}
%\titleformat{ ⟨command⟩}[⟨shape⟩]{⟨format⟩}{⟨label⟩}{⟨sep⟩}{⟨before⟩}[⟨after⟩]
\titleformat{\section}[hang]{\Large\bfseries}{\thesection}{.5em}{\hspace{0in}}[\vspace{-.2\baselineskip}]
\titleformat{\subsection}[hang]{\large\bfseries}{\thesubsection}{.5em}{\hspace{0in}}[\vspace{-.2\baselineskip}]
%\titleformat{\subsubsection}[hang]{\bfseries}{\thesubsubsection}{.5em}{\hspace{0in}}[\vspace{-.2\baselineskip}]
\titleformat{\subsubsection}[hang]{\bfseries}{\thesubsubsection}{1ex}{\hspace{0in}}[\vspace{-.2\baselineskip}]
\titleformat{\paragraph}[runin]{\bfseries\itshape}{\theparagraph}{1ex}{}{\vspace{-.2\baselineskip}}
%\titleformat{\paragraph}[runin]{\itshape}{\theparagraph}{1ex}{}{\vspace{-.2\baselineskip}}

%%\titleformat{\subsection}[hang]{\bfseries}{\thesubsection}{.5em}{\hspace{0in}}[\vspace{-.2\baselineskip}]
%%%\titleformat*{\subsection}{\bfseries\scshape}
%%%\titleformat{\subsubsection}[leftmargin]{\footnotesize\filleft}{\thesubsubsection}{.5em}{}{}
%%\titleformat{\subsubsection}[hang]{\small\bfseries}{\thesubsubsection}{.5em}{\hspace{0in}}[\vspace{-.2\baselineskip}]
%%\titleformat{\paragraph}[runin]{\itshape}{\theparagraph}{1ex}{}{\vspace{-.5\baselineskip}}

%\titlespacing*{ ⟨command⟩}{⟨left⟩}{⟨beforesep⟩}{⟨aftersep⟩}[⟨right⟩]
\titlespacing{\section}{0pc}{1.5ex plus .1ex minus .2ex}{.5ex plus .1ex minus .1ex}
\titlespacing{\subsection}{0pc}{1.5ex plus .1ex minus .2ex}{.5ex plus .1ex minus .1ex}
\titlespacing{\subsubsection}{0pc}{1.5ex plus .1ex minus .2ex}{.5ex plus .1ex minus .1ex}



%% These next lines tell latex that it is ok to have a single graphic
%% taking up most of a page, and they also decrease the space around
%% figures and tables.
\renewcommand\floatpagefraction{.9}
\renewcommand\topfraction{.9}
\renewcommand\bottomfraction{.9}
\renewcommand\textfraction{.1}
\setcounter{totalnumber}{50}
\setcounter{topnumber}{50}
\setcounter{bottomnumber}{50}
\setlength{\intextsep}{2ex}
\setlength{\floatsep}{2ex}
\setlength{\textfloatsep}{2ex}



\begin{document}
\VerbatimFootnotes

%\begin{titlepage}
%  \maketitle
%\vspace{2in}
%
%\begin{center}
%  \begin{large}
%    PROPOSAL WHITE PAPER
%
%BAA 14-013
%
%Can a Hausa Language Television Station Change Norms about Violence in Northern Nigeria? A Randomized Study of Media Effects on Violent Extremism
%
%Jake Bowers
%
%University of Illinois @ Urbana-Champaign (jwbowers@illinois.edu)
%
%\url{http://jakebowers.org}
%
%Phone: +12179792179
%
%Topic Number: 1
%
%Topic Title: Identity, Influence and Mobilization
%
%\end{large}
%\end{center}
%\end{titlepage}

\maketitle

\hypertarget{introduction}{%
\subsection{Introduction}\label{introduction}}

Intergroup conflict is responsible for many of the worst displays of
human nature. In Nigeria's Middlebelt, intergroup conflict between
sedentary farmers and semi-nomadic pastoralists is causing dire
consequences: 7,000 deaths in the past five years, 300,000 internally
displaces peoples from their homes in 2018, and \$13 billion of lost
economic productivity annually (Akinwotu 2018; Daniel 2018; Harwood
2019; McDougal et al. 2015). In the most recent conflict escalation
beginning in the 2010s, groups of anti-pastoralist vigilantes have
mobilized to preempt pastoralists from encroaching on land claimed by
farmers (Duru 2018; McDonnel 2017). These groups, dubbed the ``livestock
guard'', ransack pastoralist settlements and violently drive
pastoralists from their homes, often with the assistance of the local
farming community. Likewise, pastoralist groups enact vigilante justice,
raiding and burning down farming villages seen to encroach on land
claimed by pastoralists.

Though farmer-pastoralist conflict is widespread, mass violence between
these groups has not broken out in all Middlebelt communities, and some
farmers and pastoralists even defend each other. When a group of
livestock guard came for one pastoralist settlement, the neighboring
farming village arrested them to protect the pastoralists. After the
arrest, farmers and pastoralists convened to decide what should be done
with the prisoners. They agreed that the group of livestock guard should
not be punished, but should be disarmed and released home -- a
proposition proposed by \emph{the pastoralists}. These farmers and
pastoralists had struggled with conflict, and people on both sides had
died in past violence over farmland and grazing land. But their recent
disagreements had not escalated to the point that each side wanted the
other removed by any means necessary. The groups had created structures
and relationships that allowed them to settle disputes, and the same
structures and relationships allowed them to reach a solution about the
livestock guard.

Why were some farmer and pastoralist groups able to keep peace whereas
others were consumed by the escalating conflict? Why were some
communities able to overcome their disagreements whereas others were
destroyed by them? These questions are not unique to Nigeria -- similar
intergroup dynamics plagued South Sudan, Myanmar, and Bosnia before
those conflicts escalated into war. To understand why disagreements
between groups devolve into violent conflict, we use the framework of
intergroup conflict as a bargaining failure, which highlights trust
problems as the primary obstacle to peace between groups. Groups in
conflict have few opportunities to build trust and many to degrade it.
We argue that providing opportunities for trust-building through
cooperative intergroup contact improves the prospects for peace.
Intergroup trust ameliorates bargaining problems and increases the
likelihood of the groups resolving disputes through bargaining instead
of violence.

Treating intergroup conflict as a bargaining failure is common in
conflict studies (Fearon 1994; Powell 2006). Both groups want some
resource -- land, power, etc -- and must decide how to distribute that
resource. Groups can either bargain and split the resource, or groups
can fight to claim all of the resource or to increase their later
bargaining position. Fighting is costly, so both groups are better off
finding a bargained solution than fighting. However, bargaining fails if
neither group trusts the other side to be truthful or to honor bargained
agreements (Kydd 2000; Rohner, Thoenig, and Zilibotti 2013, 2013).
Without a reason to trust in the other side, groups are likely to fight
despite the costs to both sides.

A number of psychological mechanisms make intergroup trust amidst
conflict rare. First, conflicting groups have biased perceptions of
their own behavior and the behavior of the other side (Ward et al.
1997). Groups perceive their own belligerent actions as defensive and
justified, and perceive the defensive actions of the other side as
belligerent and gratuitous. Second, groups over-generalize negative
behaviors of outgroup members as representative of the entire outgroup
and under-generalize positive behaviors as exceptional to the outgroup
(Hewstone 1990). These over and under-generalizations create and
reinforce negative outgroup stereotypes. Together these two
psychological mechanisms sabotage intergroup bargaining by causing the
groups to have inaccurate beliefs about each other and each other's
willingness to make peace.

Many peacebuilding organizations utilize peacebuilding approaches
focused on improving intergroup attitudes. One such approach, intergroup
contact theory, hypothesizes that interactions in which group members
cooperate to achieve shared goals will improve intergroup attitudes.
Cooperative contact provides positive personal experience with the
outgroup, and those experiences reshape outgroup attitudes (Allport
1954; Pettigrew and Tropp 2008). This type of structured face-to-face
contact also provides groups the opportunity to send costly signals
about their trustworthiness and preference for peace (Kydd 2000; Lupia,
McCubbins, and Arthur 1998; Rohner, Thoenig, and Zilibotti 2013).
Intergroup contact is especially good at reducing intergroup conflict
when groups cooperate to achieve superordinate goals -- goals that
require the cooperation of both groups and benefit both groups --
because groups experience the material benefits of cooperation (Gaertner
et al. 2000; Sherif 1958).

Although research shows support for intergroup contact theory generally
(Pettigrew and Tropp 2006), its efficacy to reduce animosity amid
real-world conflict is an open question (Ditlmann, Samii, and Zeitzoff
2017). Negative experiences with outgroups worsen intergroup relations,
and individuals with the most negative attitudes are most likely to
interpret intergroup contact negatively (Gubler 2013; Paolini, Harwood,
and Rubin 2010). Its impact on interracial and interethnic attitudes has
also been challenged by recent reviews (Paluck, Green, and Green 2017).
Despite a lack of evidence about the effects of contact-based
peacebuilding programs in violent contexts, and the risks of programs
going badly, peacebuilding organizations implement numerous
contact-based interventions in violent contexts each year (Ditlmann,
Samii, and Zeitzoff 2017). These peacebuilding programs might defuse
intergroup conflict, but these programs also might do more harm than
good.

To determine if a contact-based peacebuilding interventions improves
intergroup trust, we conduct a field experiment with conflicting farmer
and pastoralist communities in Nigeria. We randomly assigned communities
with ongoing farmer-pastoralist violence to receive the peacebuilding
intervention or serve as a control group. The intervention formed
mixed-group committees and provided them with funds to build
infrastructure that would benefit both communities; committees then
collaboratively chose and constructed infrastructure projects.\footnote{The
  communities built boreholes, market stalls, primary health care
  facilities, etc.} The program also provided mediation training to each
community's leaders and held forums where the groups discussed the
underlying drivers of conflict. To measure the effects of the
intervention, we conducted pre- and post-intervention surveys, a
post-intervention natural public goods behavioral game,\footnote{In a
  public goods game (PGG), research subjects are given money and told
  they can keep the money or donate it to a public fund. Money donated
  to the public fund is multiplied by some amount and then shared with
  all subjects. Our PGG is \emph{natural} because it was conducted in a
  natural setting, rather than a lab. The funding for the PGG came from
  the National Science Foundation under Grant No.~1656871.} and twelve
months of systematic observations in markets and social events during
the intervention.

We find that the program increased intergroup trust, intergroup contact,
and perceptions of physical security. Compared to the control group, the
treatment group expressed more trust in outgroup members and more
willingness to interact with outgroup members. The treatment group was
also less affected by violence and more able to engage in routine tasks,
such as working, going to the market, and getting water. We see signs of
the positive effects in fieldwork as well as in the data -- the opening
story in which farmers defended pastoralists from the livestock guard
was a treatment site. The results also show that the intervention
affected communities as a whole, not just community members directly
involved in the intergroup contact. Individuals who directly engaged in
intergroup contact changed the most positively from baseline to endline,
but we also observe positive spillovers of trust to group members for
whom we did not exogenously increase intergroup contact.

This study expands our knowledge about intergroup conflict in several
ways. First, this study teaches us about the capacity of contact-based
peacebuilding programs to improve intergroup relations. To our knowledge
this is the first field experimental test of a contact-based
peacebuilding program implemented during an active conflict. Each of the
groups in our study were part of an active and escalating conflict, with
members of each side being killed within one year of the intervention's
onset. We evaluated the program's effects on both attitudinal and
behavioral outcomes. The results suggest that contact-based
peacebuilding programs can effectively improve relationships between
conflicting groups and is especially relevant to conflict resolution in
the cases of intergroup and intercommunal conflicts.

Second, we contribute to the literature about the role of social
diffusion and informal institutions in shaping attitudes and behaviors.
This peacebuilding intervention sought to provide a structure in which
groups can solve their own conflicts, and those structures are informal
rather than formal. Understanding how those informal structures form and
shape attitudes, norms, and behaviors of the wider population addresses
the questions of scale for these programs.

Third, this paper teaches us about settling disputes between sedentary
peoples and nomadic peoples. Violent conflict between settled peoples
and nomadic peoples is on the rise throughout the world {[}chris: need
cite{]}. This study focuses on the Fulani, the largest semi-nomadic
people on Earth. Their way of life makes them targets for violence
throughout Africa. Along with this conflict in Nigeria, Fulani in Mali
have been the targets of violence so severe that researchers at Armed
Conflict Location \& Event Data Project called it ``ethnic cleansing''
(Economist 2019). Understanding how to prevent violent conflict between
Fulani and settled peoples can help prevent violence that targets other
nomadic and semi-nomadic peoples, such as the Tuaregs in West Africa,
Uyghurs in Central Asia, Kochi in Afghanistan, and Khoisan of Southern
Africa. Preventing such violence could help preserve a dying way of
life.

In the next section we provide a theoretical framework for how and why
opposing groups struggle to solve their disagreements through bargaining
and negotiation, and elucidate how contact-based peacebuilding
interventions help these groups resolve disagreements by improving
intergroup trust. We then discuss Nigeria's farmer-pastoralist conflict,
our experimental intervention, and two designs to evaluate the effect of
the intervention. Last we present the results of the study and conclude
by connecting these findings to psychological and economic theories of
group conflict.

\hypertarget{references}{%
\section*{References}\label{references}}
\addcontentsline{toc}{section}{References}

\hypertarget{refs}{}
\leavevmode\hypertarget{ref-nyt2018nigeria}{}%
Akinwotu, Emmanuel. 2018. ``Nigeria's Farmers and Herders Fight a Deadly
Battle for Scarce Resources.'' \emph{New York Times}.
\url{https://www.nytimes.com/2018/06/25/world/africa/nigeria-herders-farmers.html}.

\leavevmode\hypertarget{ref-allport1954prejudice}{}%
Allport, Gordon. 1954. ``The Nature of Prejudice.'' \emph{Garden City,
NJ Anchor}.

\leavevmode\hypertarget{ref-daniel2018anti}{}%
Daniel, Soni. 2018. ``Anti-Open Grazing Law: Nass, Benue, Kwara, Taraba
Tackle Defence Minister.'' \emph{Vanguard}.
\url{https://www.vanguardngr.com/2018/06/anti-open-grazing-law-nass-benue-kwara-taraba-tackle-defence-minister/}.

\leavevmode\hypertarget{ref-ditlmann2017addressing}{}%
Ditlmann, Ruth K, Cyrus Samii, and Thomas Zeitzoff. 2017. ``Addressing
Violent Intergroup Conflict from the Bottom up?'' \emph{Social Issues
and Policy Review} 11(1): 38--77.

\leavevmode\hypertarget{ref-duru2018court}{}%
Duru, Peter. 2018. ``Court Stops Inspector General from Proscribing
Benue Livestock Guard.'' \emph{Vanguard}.
\url{https://www.vanguardngr.com/2018/11/court-stops-ig-from-proscribing-benue-livestock-guards/}.

\leavevmode\hypertarget{ref-economist2019militias}{}%
Economist, The. 2019. ``Malicious Malitias: States in the Sahel Have
Unleashed Ethnic Gangs with Guns.'' \emph{The Economist}.
\url{https://www.economist.com/middle-east-and-africa/2019/05/04/states-in-the-sahel-have-unleashed-ethnic-gangs-with-guns}.

\leavevmode\hypertarget{ref-fearon1994ethnic}{}%
Fearon, James D. 1994. ``Ethnic War as a Commitment Problem.'' In
\emph{Annual Meetings of the American Political Science Association},
2--5.

\leavevmode\hypertarget{ref-gaertner2000reducing}{}%
Gaertner, Samuel L et al. 2000. ``Reducing Intergroup Conflict: From
Superordinate Goals to Decategorization, Recategorization, and Mutual
Differentiation.'' \emph{Group Dynamics: Theory, Research, and Practice}
4(1): 98.

\leavevmode\hypertarget{ref-gubler2013humanizing}{}%
Gubler, Joshua R. 2013. ``When Humanizing the Enemy Fails: The Role of
Dissonance and Justification in Intergroup Conflict.'' In \emph{Annual
Meeting of the American Political Science Association},

\leavevmode\hypertarget{ref-council2019nigeria}{}%
Harwood, Asch. 2019. ``Update: The Numbers Behind Sectarian Violence in
Nigeria.'' \emph{Council on Foreign Relations}.
\url{https://www.cfr.org/blog/update-numbers-behind-sectarian-violence-nigeria}.

\leavevmode\hypertarget{ref-hewstone1990ultimate}{}%
Hewstone, Miles. 1990. ``The `Ultimate Attribution Error'? A Review of
the Literature on Intergroup Causal Attribution.'' \emph{European
Journal of Social Psychology} 20(4): 311--35.

\leavevmode\hypertarget{ref-kydd2000trust}{}%
Kydd, Andrew. 2000. ``Trust, Reassurance, and Cooperation.''
\emph{International Organization} 54(2): 325--57.

\leavevmode\hypertarget{ref-lupia1998democratic}{}%
Lupia, Arthur, Mathew D McCubbins, and Lupia Arthur. 1998. \emph{The
Democratic Dilemma: Can Citizens Learn What They Need to Know?}
Cambridge University Press.

\leavevmode\hypertarget{ref-mcdonnel2017graze}{}%
McDonnel, Tim. 2017. ``Why It's Now a Crime to Let Cattle Graze Freely
in 2 Nigerian States.'' \emph{National Public Radio (NPR)}.
\url{https://www.npr.org/sections/goatsandsoda/2017/12/12/569913821/why-its-now-a-crime-to-let-cattle-graze-freely-in-2-nigerian-states}.

\leavevmode\hypertarget{ref-mcdougal2015effect}{}%
McDougal, Topher L et al. 2015. ``The Effect of Farmer-Pastoralist
Violence on Income: New Survey Evidence from Nigeria's Middle Belt
States.'' \emph{Economics of Peace and Security Journal} 10(1): 54--65.

\leavevmode\hypertarget{ref-paluck2017contact}{}%
Paluck, Elizabeth Levy, Seth Green, and Donald P Green. 2017. ``The
Contact Hypothesis Revisited.''

\leavevmode\hypertarget{ref-paolini2010negative}{}%
Paolini, Stefania, Jake Harwood, and Mark Rubin. 2010. ``Negative
Intergroup Contact Makes Group Memberships Salient: Explaining Why
Intergroup Conflict Endures.'' \emph{Personality and Social Psychology
Bulletin} 36(12): 1723--38.

\leavevmode\hypertarget{ref-pettigrew2006meta}{}%
Pettigrew, Thomas F, and Linda R Tropp. 2006. ``A Meta-Analytic Test of
Intergroup Contact Theory.'' \emph{Journal of personality and social
psychology} 90(5): 751.

\leavevmode\hypertarget{ref-pettigrew2008does}{}%
---------. 2008. ``How Does Intergroup Contact Reduce Prejudice?
Meta-Analytic Tests of Three Mediators.'' \emph{European Journal of
Social Psychology} 38(6): 922--34.

\leavevmode\hypertarget{ref-powell2006war}{}%
Powell, Robert. 2006. ``War as a Commitment Problem.''
\emph{International organization} 60(1): 169--203.

\leavevmode\hypertarget{ref-rohner2013war}{}%
Rohner, Dominic, Mathias Thoenig, and Fabrizio Zilibotti. 2013. ``War
Signals: A Theory of Trade, Trust, and Conflict.'' \emph{Review of
Economic Studies} 80(3): 1114--47.

\leavevmode\hypertarget{ref-sherif1958superordinate}{}%
Sherif, Muzafer. 1958. ``Superordinate Goals in the Reduction of
Intergroup Conflict.'' \emph{American journal of Sociology} 63(4):
349--56.

\leavevmode\hypertarget{ref-ward1997naive}{}%
Ward, Andrew et al. 1997. ``Naive Realism in Everyday Life: Implications
for Social Conflict and Misunderstanding.'' \emph{Values and knowledge}:
103--35.

\end{document}
