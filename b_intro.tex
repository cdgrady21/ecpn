%\documentclass[]{article}
\documentclass[11pt]{article}
\usepackage[usenames,dvipsnames]{xcolor}

\usepackage[T1]{fontenc}
%\usepackage{lmodern}
\usepackage{tgtermes}
\usepackage{amssymb,amsmath}
%\usepackage[margin=1in]{geometry}
\usepackage[letterpaper,bottom=1in,top=1in,right=1.25in,left=1.25in,includemp=FALSE]{geometry}
\usepackage{pdfpages}
\usepackage[small]{caption}

\usepackage{ifxetex,ifluatex}
\usepackage{fixltx2e} % provides \textsubscript
% use microtype if available
\IfFileExists{microtype.sty}{\usepackage{microtype}}{}
\ifnum 0\ifxetex 1\fi\ifluatex 1\fi=0 % if pdftex
\usepackage[utf8]{inputenc}
\else % if luatex or xelatex
\usepackage{fontspec}
\ifxetex
\usepackage{xltxtra,xunicode}
\fi
\defaultfontfeatures{Mapping=tex-text,Scale=MatchLowercase}
\newcommand{\euro}{€}
\fi
%

\usepackage{fancyvrb}

\usepackage{ctable,longtable}

\usepackage[section]{placeins}
\usepackage{float} % provides the H option for float placement
\restylefloat{figure}
\usepackage{dcolumn} % allows for different column alignments
\newcolumntype{.}{D{.}{.}{1.2}}

\usepackage{booktabs} % nicer horizontal rules in tables

%Assume we want graphics always
\usepackage{graphicx}
% We will generate all images so they have a width \maxwidth. This means
% that they will get their normal width if they fit onto the page, but
% are scaled down if they would overflow the margins.
%% \makeatletter
%% \def\maxwidth{\ifdim\Gin@nat@width>\linewidth\linewidth
%%   \else\Gin@nat@width\fi}
%% \makeatother
%% \let\Oldincludegraphics\includegraphics
%% \renewcommand{\includegraphics}[1]{\Oldincludegraphics[width=\maxwidth]{#1}}
\graphicspath{{.}{../Soccom_Code/socom_2013/}}


%% \ifxetex
%% \usepackage[pagebackref=true, setpagesize=false, % page size defined by xetex
%% unicode=false, % unicode breaks when used with xetex
%% xetex]{hyperref}
%% \else
\usepackage[pagebackref=true, unicode=true, bookmarks=true, pdftex]{hyperref}
% \fi


\hypersetup{breaklinks=true,
  bookmarks=true,
  pdfauthor={Christopher Grady and Rebecca Wolfe},
  pdftitle={Promoting Peace Amidst Group Conflict: An Intergroup Contact Field Experiment in Nigeria - Introduction},
  colorlinks=true,
  linkcolor=BrickRed,
  citecolor=blue, %MidnightBlue,
  urlcolor=BrickRed,
  % urlcolor=blue,
  % linkcolor=magenta,
  pdfborder={0 0 0}}

%\setlength{\parindent}{0pt}
%\setlength{\parskip}{6pt plus 2pt minus 1pt}
\usepackage{parskip}
\setlength{\emergencystretch}{3em}  % prevent overfull lines
\providecommand{\tightlist}{%
  \setlength{\itemsep}{0pt}\setlength{\parskip}{0pt}}

%% Insist on this.
\setcounter{secnumdepth}{2}

\VerbatimFootnotes % allows verbatim text in footnotes

\title{Promoting Peace Amidst Group Conflict: An Intergroup Contact Field
Experiment in Nigeria - Introduction}

\author{
Christopher Grady and Rebecca Wolfe
}


\date{May 03, 2019}


\usepackage{versions}
\makeatletter
\renewcommand*\versionmessage[2]{\typeout{*** `#1' #2. ***}}
\renewcommand*\beginmarkversion{\sffamily}
  \renewcommand*\endmarkversion{}
\makeatother

\excludeversion{comment}

%\usepackage[margins=1in]{geometry}

\usepackage[compact,bottomtitles]{titlesec}
%\titleformat{ ⟨command⟩}[⟨shape⟩]{⟨format⟩}{⟨label⟩}{⟨sep⟩}{⟨before⟩}[⟨after⟩]
\titleformat{\section}[hang]{\Large\bfseries}{\thesection}{.5em}{\hspace{0in}}[\vspace{-.2\baselineskip}]
\titleformat{\subsection}[hang]{\large\bfseries}{\thesubsection}{.5em}{\hspace{0in}}[\vspace{-.2\baselineskip}]
%\titleformat{\subsubsection}[hang]{\bfseries}{\thesubsubsection}{.5em}{\hspace{0in}}[\vspace{-.2\baselineskip}]
\titleformat{\subsubsection}[hang]{\bfseries}{\thesubsubsection}{1ex}{\hspace{0in}}[\vspace{-.2\baselineskip}]
\titleformat{\paragraph}[runin]{\bfseries\itshape}{\theparagraph}{1ex}{}{\vspace{-.2\baselineskip}}
%\titleformat{\paragraph}[runin]{\itshape}{\theparagraph}{1ex}{}{\vspace{-.2\baselineskip}}

%%\titleformat{\subsection}[hang]{\bfseries}{\thesubsection}{.5em}{\hspace{0in}}[\vspace{-.2\baselineskip}]
%%%\titleformat*{\subsection}{\bfseries\scshape}
%%%\titleformat{\subsubsection}[leftmargin]{\footnotesize\filleft}{\thesubsubsection}{.5em}{}{}
%%\titleformat{\subsubsection}[hang]{\small\bfseries}{\thesubsubsection}{.5em}{\hspace{0in}}[\vspace{-.2\baselineskip}]
%%\titleformat{\paragraph}[runin]{\itshape}{\theparagraph}{1ex}{}{\vspace{-.5\baselineskip}}

%\titlespacing*{ ⟨command⟩}{⟨left⟩}{⟨beforesep⟩}{⟨aftersep⟩}[⟨right⟩]
\titlespacing{\section}{0pc}{1.5ex plus .1ex minus .2ex}{.5ex plus .1ex minus .1ex}
\titlespacing{\subsection}{0pc}{1.5ex plus .1ex minus .2ex}{.5ex plus .1ex minus .1ex}
\titlespacing{\subsubsection}{0pc}{1.5ex plus .1ex minus .2ex}{.5ex plus .1ex minus .1ex}



%% These next lines tell latex that it is ok to have a single graphic
%% taking up most of a page, and they also decrease the space around
%% figures and tables.
\renewcommand\floatpagefraction{.9}
\renewcommand\topfraction{.9}
\renewcommand\bottomfraction{.9}
\renewcommand\textfraction{.1}
\setcounter{totalnumber}{50}
\setcounter{topnumber}{50}
\setcounter{bottomnumber}{50}
\setlength{\intextsep}{2ex}
\setlength{\floatsep}{2ex}
\setlength{\textfloatsep}{2ex}



\begin{document}
\VerbatimFootnotes

%\begin{titlepage}
%  \maketitle
%\vspace{2in}
%
%\begin{center}
%  \begin{large}
%    PROPOSAL WHITE PAPER
%
%BAA 14-013
%
%Can a Hausa Language Television Station Change Norms about Violence in Northern Nigeria? A Randomized Study of Media Effects on Violent Extremism
%
%Jake Bowers
%
%University of Illinois @ Urbana-Champaign (jwbowers@illinois.edu)
%
%\url{http://jakebowers.org}
%
%Phone: +12179792179
%
%Topic Number: 1
%
%Topic Title: Identity, Influence and Mobilization
%
%\end{large}
%\end{center}
%\end{titlepage}

\maketitle

\subsection{Introduction}\label{introduction}

Intergroup conflict is responsible for many of the worst displays of
human nature. In Nigeria, intergroup conflict between farmers and
pastoralists has claimed countless lives and destroyed countless
communities over the past two decades. As the conflict escalated, groups
of anti-pastoralist vigilantes mobilized to pre-emptively prevent
pastoralists from encroaching on land claimed by farmers (Duru 2018;
McDonnel 2017). These groups, dubbed the ``livestock guard'', ransacked
pastoralist settlements and violently drove pastoralists from their
homes, often with the assistance of the local farming community.
Likewise, pastoralist groups enacted vigilante justice, raiding and
burning down farming villages seen to encroach on land claimed by
pastoralists. Attacks such as these forced up to 180,000 people from
their homes in 2018 (Daniel 2018) and farmer-pastoralist conflict costs
Nigeria \$13 billion of lost economic productivity annually (McDougal et
al. 2015).

Though farmer-pastoralist conflict was widespread, mass violence did not
break out in all communities with farmer-pastoralist conflict. In one
village, farmers and pastoralists defended each other from hostile
neighbors. When a group of livestock guard came for one pastoralist
settlement, the neighboring farming village arrested them to protect the
pastoralists. After the arrest, farmers and pastoralists convened to
decide what should be done with the prisoners. They agreed that the
group of livestock guard should not be punished, but should be disarmed
and released home -- a proposition proposed by \emph{the pastoralists}.
These farmers and pastoralists had also struggled with conflict, and
people on both sides had died in disputes over farmland and grazing
land. But their disputes had not escalated to the point that each side
wanted the other removed by any means necessary. They had created
structures that allowed them to settle disputes, and the same structures
allowed them to reach a solution about the livestock guard.

Why were some farmer and pastoralist groups able to keep peace whereas
others were swallowed by the escalating conflict? Why were some
communities able to overcome their intergroup disputes whereas others
were destroyed by them? Understanding the factors that help groups
resolve their disputes is important for mitigating and preventing the
myriad intergroup conflicts around the world. Intergroup conflict fuels
and is fueled by intergroup prejudice. We argue that psychological
reconciliation during an ongoing conflict can reduce intergroup
prejudice and mitigate violence. Bottom-up peacebuilding programs can
help conflicting groups achieve psychological reconciliation through
intergroup contact and superordinate goals (Sherif 1958).

Most literature and policy about peacebuilding and preventing intergroup
conflict revolves around international intervention (Severine Autesserre
2017; Regan 2002; Shannon 2009; Smith and Stam 2003; Walter 2002). The
international community, through military intervention and mediation,
can increase the cost of fighting, enforce or raise the costs of
defecting from peace agreements, and facilitate information flows (Di
Salvatore and Ruggeri 2017; Doyle and Sambanis 2000). By providing
information and punishing defection from agreements, international
intervention helps conflicting actors overcome the bargaining failures
that lead to violent conflict (Fearon 1994, 1995; Powell 2006; Smith and
Stam 2003). This type of peacebuilding has been employed by the
international institutions like UN in conflicts throughout the world,
including Namibia, Sierra Leone, Guatemala, Cambodia, and India and
Pakistan. Ideally, peacekeepers impose order to protect bystanders and
prevent further violence while mediation assists the two groups in
settling disputes peacefully. While not universally successful, research
shows that peacekeeping and mediation are effective strategies to
promote peace in many conflicts (Doyle and Sambanis 2000; Gartner 2011;
Hartzell, Hoddie, and Rothchild 2001; Wallensteen and Svensson 2014;
Walter 2002).

While international intervention has assisted in ending many conflicts,
it is most effective at maintaining peace in a post-conflict setting and
is relatively ineffective at ending ongoing conflicts (Bratt 1996; Doyle
and Sambanis 2006; Gilligan, Sergenti, and others 2008). Notably,
international intervention failed to stop intergroup conflicts in DRC
(Séverine Autesserre 2010), Somalia (Severine Autesserre 2019), Bosnia
(Fetherston, Ramsbotham, and Woodhouse 1994), Angola (Bratt 1996) and
Rwanda (Barnett 1997; Dallaire 2009). And even when international
intervention is successful, that success is often short-lived. Weinstein
(2005) estimated that 75\% of civil wars resume within 10 years of UN
intervention, and Beardsley (2008) showed that mediation reduces
conflict in the short-term but not the long-term. One reason for these
failures is that international intervention is often more influenced by
international norms about what strategies \emph{should} work than
finding the strategies that are most likely to work in a given context
(Paris 2003). But international intervention has also been criticized
for not solving the underlying problems that cause conflict -- providing
a band-aid when the conflict needs a doctor (Severine Autesserre 2019).

International intervention through military intervention and mediation
might be conducive to preventing different types of violence than
bottom-up peacebuilding programs. Whereas intervention is most
successful in the immediate aftermath of conflict and mainly reduces
violence in the short-term, bottom-up programs might be most effective
at mitigating ongoing conflict before it escalates and at ameliorating
the persistent ``no peace -- no war'' contexts that plague many
countries. Even for contexts when international intervention would be
successful, bottom-up interventions might be more feasible or timely.
Before international intervention can occur, the international community
must agree on an appropriate intervention and make large commitments of
forces and financing. As a result, international intervention often
comes late, costs billions of dollars, and is unlikely to be applied
unless preventing the conflict is an international priority. Therefore
international intervention is unlikely to target the persistent
intergroup conflicts that plagues many states and is unlikely to be used
until conflict escalates into large-scale violence.

While most attempts to reduce violent conflict between groups are
top-down, peacebuilding is going through a ``local turn'' as scholars
and practitioners call for bottom-up strategies to building peace
(Severine Autesserre 2016, 2017; Ditlmann, Samii, and Zeitzoff 2017;
Safunu 2012). Bottom-up strategies focus on engaging local actors and
assisting them in reducing group conflict. Conflicts at local levels are
often driven and perpetuated by intergroup animosity long after the
original grievance is forgotten {[}McDonnel (2017); cite more about
intergroup animosity{]}. Intergroup animosity and strongly negative
attitudes can be an insurmountable barrier to peace. Individuals often
motivatedly reason to rationalize their beliefs, which leads to biased
processing of information about outgroup members' behavior. Behavior by
outgroup members may be perceived as more threatening and more malicious
than the same behavior committed by a neutral party (cite fundamental
attribution error, motivated reasoning; Hewstone 1990). The perceived
negative behavior may be seen as \emph{defining} of the group, whereas
any perceived positive behavior may be seen as the \emph{exception} to
the group (Allison and Messick 1985; Pettigrew 1979). This biased
information processing reinforces negative group stereotypes and can
subvert the groups' own attempts to foster peace. It can also sabotage
intergroup bargaining by causing the groups to have inaccurate beliefs
about each other, adding to information and commitment problems that
cause conflict. Bottom-up strategies that help reduce group prejudice
could help the groups achieve peace without the large-scale
international intervention efforts most commonly used to build peace.

One of the most promising approaches to reducing intergroup prejudice
and violence comes from intergroup contact theory. Intergroup contact
theory hypothesizes that interactions in which group members cooperate
to achieve shared goals will reduce prejudice and the likelihood of
violence. Intergroup contact works by demystifying the outgroup,
presenting the other group's perspective, and replacing imagined
stereotypes with firsthand knowledge (Allport 1954; Pettigrew and Tropp
2008). Intergroup contact is likely especially good at reducing
intergroup conflict when groups cooperate to achieve superordinate goals
-- goals that require the cooperation of both groups and benefit both
groups (Gaertner et al. 2000; Sherif 1958) The prejudice-reducing
effects of intergroup contact have been demonstrated in a variety of
countries and using a variety of methodological approaches (Paluck,
Green, and Green 2017; Pettigrew and Tropp 2006). Notably, intergroup
contacted programs reduced prejudice between white people and black
people in the U.S. and South Africa (Burns, Corno, and La Ferrara 2015;
Marmaros and Sacerdote 2006), Jews and Arabs (Ditlmann and Samii 2016;
Yablon 2012), Muslims and Christians (Scacco and Warren 2016), and
Hindus and Muslims (Barnhardt 2009). Peacebuilding programs utilizing
intergroup contact and superordinate goals are increasingly used to
reduce intergroup conflict by peacebuilding organizations (Ditlmann,
Samii, and Zeitzoff 2017).

Although research shows support for intergroup contact theory generally,
its efficacy to reduce prejudice amid \emph{real-world} conflict is an
open question. First, most research about intergroup contact uses
correlational studies or lab experiments, both of which have
methodological weaknesses. Correlational studies cannot demonstrate
causal effects, and results from lab experiments may not apply to real
world conflicts, where groups compete for resources and share a history
of intergroup violence (Ditlmann, Samii, and Zeitzoff 2017). Second, no
prior studies of intergroup contact have involved groups engaged in
active intergroup violence\footnote{Previous studies have involved
  groups with a history of violence conflict -- see Scacco and Warren
  (2016) and Ditlmann and Samii (2016).}, and some studies suggest that
intergroup contact in violent settings could be ineffective or even
backfire. Negative experiences with outgroups increase prejudice, and
the most prejudiced individuals are most likely to interpret intergroup
contact negatively (Gubler 2013; Paolini, Harwood, and Rubin 2010).
Despite a lack of evidence about the effects of bottom-up peacebuilding
programs in violent contexts, and the risks of programs going badly,
peacebuilding organizations implement numerous bottom-up interventions
in violent contexts each year. Bottom-up peacebuilding programs might
defuse intergroup conflict, but these programs also might do more harm
than good.

To study the effect of bottom-up psychological interventions on violent
conflict, we conduct a field experiment with farmer and pastoralist
communities Nigeria to determine if an intergroup contact-based program
effectively reduces prejudice and violent conflict. We randomly assigned
communities with farmer-pastoralist violence to receive the
peacebuilding intervention or serve as a control group. The intervention
formed mixed-group committees and provided them with funds to build
infrastructure that would benefit both communities; committees then
collaboratively chose and constructed infrastructure projects.\footnote{The
  communities built boreholes, market stalls, primary health care
  facilities, etc.} The program also provided mediation training to each
community's leaders. To measure the effects of the intervention, we
conducted pre- and post-intervention surveys, a post-intervention
natural public goods behavioral game,\footnote{In a public goods game
  (PGG), research subjects are given money and told they can keep the
  money or donate it to a public fund. Money donated to the public fund
  is multiplied by some amount and then shared with all subjects. My PGG
  is \emph{natural} because it was conducted in a natural setting,
  rather than a lab. The funding for the PGG came from the National
  Science Foundation under Grant No. 1656871.} and twelve months of
systematic observations in markets and social events during the
intervention.

We find that the program assisted the groups in reducing their conflict.
Compared to the control group, the treatment group increased intergroup
contact and trust between the groups, and reduced physical violence and
insecurity experienced by group members. We find no evidence that the
program reduced outgroup threat or caused the groups to expand their
conception of ``ingroup'' to include the other side, two prominent
mechanisms for how intergroup contact reduces prejudice. The results
also suggests that intergroup contact for a relatively small percentage
of a group can indirectly affect attitudes of group members with no
exogenous increase in contact with the outgroup. We observe the most
positive changes from individuals directly involved in the intergroup
committees, but we also observe positive spillover to group members who
were not involved in the intergroup contact intervention.

This study expands our knowledge about intergroup conflict in several
ways. First, to our knowledge this is the first field experimental test
of a bottom-up peacebuilding program implemented by a peacebuilding
organization during an active conflict. We evaluated the program's
effects on both attitudinal and behavioral outcomes. The results suggest
that bottom-up peacebuilding programs can effectively reduce conflict
and is especially relevant to conflict resolution in the cases of
intergroup and intercommunal conflicts.

Second, this study is one of the only field experimental interventions
to test intergroup contact theory with groups actively engaged in
violence. Each of the groups in our study were part of an active and
escalating conflict, with members of each side being killed within one
year of the intervention's onset. Even in such a context, community
members who engaged in direct interpersonal interaction with an outgroup
member changed more positively than other community members.
Importantly, the intergroup contact involved achieving superordinate
goals that benefited both groups materially. This suggests that contact
with superordinate goals is robust to actively violent contexts.

Third, we contribute to the literature about the role of social
diffusion and social institutions in shaping attitudes and behaviors.
Bottom-up peacebuilding interventions seek to provide a structure in
which groups can solve their own conflicts, and those structures are
social rather than coercive. Understanding how those structures form and
are maintained is relevant to other institutional setups attempting to
influence behavior. In this study, though the greatest changes in
attitudes and behaviors were from individuals directly interacting with
outgroup members, the attitudes of other community members also
improved. This type of ``social effect'' is also a potential way to
``scale up'' the effects of intergroup contact.

In the next section we provide a theoretical framework for how and why
bottom-up interventions that focus on prejudice reduction can reduce
intergroup violence. We then discuss Nigeria's farmer-pastoralist
conflict, our experimental intervention, and two designs to evaluate the
effect of the intervention. Last we present the results of the study and
conclude by connecting these findings to theories of group conflict and
prejudice.

\section*{References}\label{references}
\addcontentsline{toc}{section}{References}

\hypertarget{refs}{}
\hypertarget{ref-allison1985group}{}
Allison, Scott T, and David M Messick. 1985. ``The Group Attribution
Error.'' \emph{Journal of Experimental Social Psychology} 21(6):
563--79.

\hypertarget{ref-allport1954prejudice}{}
Allport, Gordon. 1954. ``The Nature of Prejudice.'' \emph{Garden City,
NJ Anchor}.

\hypertarget{ref-autesserre2016failure}{}
Autesserre, Severine. 2016. ``The Responsibility to Protect in Congo:
The Failure of Grassroots Prevention.'' \emph{International
Peacekeeping} 23(1): 29--51.

\hypertarget{ref-autesserre2017international}{}
---------. 2017. ``International Peacebuilding and Local Success:
Assumptions and Effectiveness.'' \emph{International Studies Review}
19(1): 114--32.

\hypertarget{ref-autesserre2019crisis}{}
---------. 2019. ``The Crisis of Peacekeeping: Why the Un Can't End
Wars.'' \emph{Foreign Affairs}.
\url{https://www.foreignaffairs.com/articles/2018-12-11/crisis-peacekeeping}.

\hypertarget{ref-autesserre2010trouble}{}
Autesserre, Séverine. 2010. 115 \emph{The Trouble with the Congo: Local
Violence and the Failure of International Peacebuilding}. Cambridge
University Press.

\hypertarget{ref-barnett1997security}{}
Barnett, Michael N. 1997. ``The Un Security Council, Indifference, and
Genocide in Rwanda.'' \emph{Cultural Anthropology} 12(4): 551--78.

\hypertarget{ref-barnhardt2009near}{}
Barnhardt, Sharon. 2009. ``Near and Dear? Evaluating the Impact of
Neighbor Diversity on Inter-Religious Attitudes.'' \emph{Unpublished
working paper}.

\hypertarget{ref-beardsley2008agreement}{}
Beardsley, Kyle. 2008. ``Agreement Without Peace? International
Mediation and Time Inconsistency Problems.'' \emph{American journal of
political science} 52(4): 723--40.

\hypertarget{ref-bratt1996assessing}{}
Bratt, Duane. 1996. ``Assessing the Success of Un Peacekeeping
Operations.'' \emph{International Peacekeeping} 3(4): 64--81.

\hypertarget{ref-burns2015interaction}{}
Burns, Justine, Lucia Corno, and Eliana La Ferrara. 2015.
\emph{Interaction, Prejudice and Performance. Evidence from South
Africa}. Working paper.

\hypertarget{ref-dallaire2009shake}{}
Dallaire, Romeo. 2009. \emph{Shake Hands with the Devil: The Failure of
Humanity in Rwanda}. Vintage Canada.

\hypertarget{ref-daniel2018anti}{}
Daniel, Soni. 2018. ``Anti-Open Grazing Law: Nass, Benue, Kwara, Taraba
Tackle Defence Minister.'' \emph{Vanguard}.
\url{https://www.vanguardngr.com/2018/06/anti-open-grazing-law-nass-benue-kwara-taraba-tackle-defence-minister/}.

\hypertarget{ref-di2017effectiveness}{}
Di Salvatore, Jessica, and Andrea Ruggeri. 2017. ``Effectiveness of
Peacekeeping Operations.'' \emph{Oxford Research Encyclopedia of
Politics}.

\hypertarget{ref-ditlmann2016can}{}
Ditlmann, Ruth K, and Cyrus Samii. 2016. ``Can Intergroup Contact Affect
Ingroup Dynamics? Insights from a Field Study with Jewish and
Arab-Palestinian Youth in Israel.'' \emph{Peace and Conflict: Journal of
Peace Psychology} 22(4): 380.

\hypertarget{ref-ditlmann2017addressing}{}
Ditlmann, Ruth K, Cyrus Samii, and Thomas Zeitzoff. 2017. ``Addressing
Violent Intergroup Conflict from the Bottom up?'' \emph{Social Issues
and Policy Review} 11(1): 38--77.

\hypertarget{ref-doyle2000international}{}
Doyle, Michael W, and Nicholas Sambanis. 2000. ``International
Peacebuilding: A Theoretical and Quantitative Analysis.'' \emph{American
political science review} 94(4): 779--801.

\hypertarget{ref-doyle2006making}{}
---------. 2006. \emph{Making War and Building Peace: United Nations
Peace Operations}. Princeton University Press.

\hypertarget{ref-duru2018court}{}
Duru, Peter. 2018. ``Court Stops Inspector General from Proscribing
Benue Livestock Guard.'' \emph{Vanguard}.
\url{https://www.vanguardngr.com/2018/11/court-stops-ig-from-proscribing-benue-livestock-guards/}.

\hypertarget{ref-fearon1994ethnic}{}
Fearon, James D. 1994. ``Ethnic War as a Commitment Problem.'' In
\emph{Annual Meetings of the American Political Science Association},
2--5.

\hypertarget{ref-fearon1995rationalist}{}
---------. 1995. ``Rationalist Explanations for War.''
\emph{International organization} 49(3): 379--414.

\hypertarget{ref-fetherston1994unprofor}{}
Fetherston, A Betts, Oliver Ramsbotham, and Tom Woodhouse. 1994.
``UNPROFOR: Some Observations from a Conflict Resolution Perspective.''
\emph{International Peacekeeping} 1(2): 179--203.

\hypertarget{ref-gaertner2000reducing}{}
Gaertner, Samuel L et al. 2000. ``Reducing Intergroup Conflict: From
Superordinate Goals to Decategorization, Recategorization, and Mutual
Differentiation.'' \emph{Group Dynamics: Theory, Research, and Practice}
4(1): 98.

\hypertarget{ref-gartner2011signs}{}
Gartner, Scott Sigmund. 2011. ``Signs of Trouble: Regional Organization
Mediation and Civil War Agreement Durability.'' \emph{The Journal of
Politics} 73(2): 380--90.

\hypertarget{ref-gilligan2008interventions}{}
Gilligan, Michael J, Ernest J Sergenti, and others. 2008. ``Do Un
Interventions Cause Peace? Using Matching to Improve Causal Inference.''
\emph{Quarterly Journal of Political Science} 3(2): 89--122.

\hypertarget{ref-gubler2013humanizing}{}
Gubler, Joshua R. 2013. ``When Humanizing the Enemy Fails: The Role of
Dissonance and Justification in Intergroup Conflict.'' In \emph{Annual
Meeting of the American Political Science Association},

\hypertarget{ref-hartzell2001stabilizing}{}
Hartzell, Caroline, Matthew Hoddie, and Donald Rothchild. 2001.
``Stabilizing the Peace After Civil War: An Investigation of Some Key
Variables.'' \emph{International organization} 55(1): 183--208.

\hypertarget{ref-hewstone1990ultimate}{}
Hewstone, Miles. 1990. ``The `Ultimate Attribution Error'? A Review of
the Literature on Intergroup Causal Attribution.'' \emph{European
Journal of Social Psychology} 20(4): 311--35.

\hypertarget{ref-marmaros2006friendships}{}
Marmaros, David, and Bruce Sacerdote. 2006. ``How Do Friendships Form?''
\emph{The Quarterly Journal of Economics} 121(1): 79--119.

\hypertarget{ref-mcdonnel2017graze}{}
McDonnel, Tim. 2017. ``Why It's Now a Crime to Let Cattle Graze Freely
in 2 Nigerian States.'' \emph{National Public Radio (NPR)}.
\url{https://www.npr.org/sections/goatsandsoda/2017/12/12/569913821/why-its-now-a-crime-to-let-cattle-graze-freely-in-2-nigerian-states}.

\hypertarget{ref-mcdougal2015effect}{}
McDougal, Topher L et al. 2015. ``The Effect of Farmer-Pastoralist
Violence on Income: New Survey Evidence from Nigeria's Middle Belt
States.'' \emph{Economics of Peace and Security Journal} 10(1): 54--65.

\hypertarget{ref-paluck2017contact}{}
Paluck, Elizabeth Levy, Seth Green, and Donald P Green. 2017. ``The
Contact Hypothesis Revisited.''

\hypertarget{ref-paolini2010negative}{}
Paolini, Stefania, Jake Harwood, and Mark Rubin. 2010. ``Negative
Intergroup Contact Makes Group Memberships Salient: Explaining Why
Intergroup Conflict Endures.'' \emph{Personality and Social Psychology
Bulletin} 36(12): 1723--38.

\hypertarget{ref-paris2003peacekeeping}{}
Paris, Roland. 2003. ``Peacekeeping and the Constraints of Global
Culture.'' \emph{European Journal of International Relations} 9(3):
441--73.

\hypertarget{ref-pettigrew1979ultimate}{}
Pettigrew, Thomas F. 1979. ``The Ultimate Attribution Error: Extending
Allport's Cognitive Analysis of Prejudice.'' \emph{Personality and
social psychology bulletin} 5(4): 461--76.

\hypertarget{ref-pettigrew2006meta}{}
Pettigrew, Thomas F, and Linda R Tropp. 2006. ``A Meta-Analytic Test of
Intergroup Contact Theory.'' \emph{Journal of personality and social
psychology} 90(5): 751.

\hypertarget{ref-pettigrew2008does}{}
---------. 2008. ``How Does Intergroup Contact Reduce Prejudice?
Meta-Analytic Tests of Three Mediators.'' \emph{European Journal of
Social Psychology} 38(6): 922--34.

\hypertarget{ref-powell2006war}{}
Powell, Robert. 2006. ``War as a Commitment Problem.''
\emph{International organization} 60(1): 169--203.

\hypertarget{ref-regan2002third}{}
Regan, Patrick M. 2002. ``Third-Party Interventions and the Duration of
Intrastate Conflicts.'' \emph{Journal of Conflict Resolution} 46(1):
55--73.

\hypertarget{ref-safunu2012grassroots}{}
Safunu, Banchani John-Paul. 2012. ``Do Grassroots Approaches and
Mobilization for Development Contribute to Post-Conflict Peacebuilding?
The Experience of Northern Ghana.'' \emph{Nairobi: Africa Leadership
Center}.

\hypertarget{ref-Scacco2016}{}
Scacco, Alexandra, and Shana S Warren. 2016. ``Youth Vocational Training
and Conflict Mitigation: An Experimental Test of Social Contact Theory
in Nigeria.''

\hypertarget{ref-shannon2009preventing}{}
Shannon, Megan. 2009. ``Preventing War and Providing the Peace?
International Organizations and the Management of Territorial
Disputes.'' \emph{Conflict Management and Peace Science} 26(2): 144--63.

\hypertarget{ref-sherif1958superordinate}{}
Sherif, Muzafer. 1958. ``Superordinate Goals in the Reduction of
Intergroup Conflict.'' \emph{American journal of Sociology} 63(4):
349--56.

\hypertarget{ref-smith2003mediation}{}
Smith, Alastair, and Allan Stam. 2003. ``Mediation and Peacekeeping in a
Random Walk Model of Civil and Interstate War.'' \emph{International
Studies Review} 5(4): 115--35.

\hypertarget{ref-wallensteen2014talking}{}
Wallensteen, Peter, and Isak Svensson. 2014. ``Talking Peace:
International Mediation in Armed Conflicts.'' \emph{Journal of Peace
Research} 51(2): 315--27.

\hypertarget{ref-walter2002committing}{}
Walter, Barbara F. 2002. \emph{Committing to Peace: The Successful
Settlement of Civil Wars}. Princeton University Press.

\hypertarget{ref-weinstein2005autonomous}{}
Weinstein, Jeremy M. 2005. ``Autonomous Recovery and International
Intervention in Comparative Perspective.'' \emph{Available at SSRN
1114117}.

\hypertarget{ref-yablon2012we}{}
Yablon, Yaacov B. 2012. ``Are We Preaching to the Converted? The Role of
Motivation in Understanding the Contribution of Intergroup Encounters.''
\emph{Journal of Peace Education} 9(3): 249--63.

\end{document}
