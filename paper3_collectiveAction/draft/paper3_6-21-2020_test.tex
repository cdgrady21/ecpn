%\documentclass[]{article}
\documentclass[11pt]{article}
\usepackage[usenames,dvipsnames]{xcolor}

\usepackage[T1]{fontenc}
%\usepackage{lmodern}
\usepackage{tgtermes}
\usepackage{amssymb,amsmath}
%\usepackage[margin=1in]{geometry}
\usepackage[letterpaper,bottom=1in,top=1in,right=1in,left=1in,includemp=FALSE]{geometry}
\usepackage{pdfpages}
\usepackage[small,labelfont=bf]{caption}
\usepackage{subcaption}
\usepackage{multirow}
\usepackage{longtable}
\usepackage{pdflscape}
\usepackage{array}
\usepackage{amssymb}
\usepackage{graphicx}

\newcolumntype{P}[1]{>{\raggedright\arraybackslash}p{#1}}
\usepackage{tikz}
\usetikzlibrary{mindmap, positioning}


\usepackage{ifxetex,ifluatex}
\usepackage{fixltx2e} % provides \textsubscript
% use microtype if available
\IfFileExists{microtype.sty}{\usepackage{microtype}}{}
\ifnum 0\ifxetex 1\fi\ifluatex 1\fi=0 % if pdftex
\usepackage[utf8]{inputenc}
\else % if luatex or xelatex
\usepackage{fontspec}
\ifxetex
\usepackage{xltxtra,xunicode}
\fi
\defaultfontfeatures{Mapping=tex-text,Scale=MatchLowercase}
\newcommand{\euro}{€}
\fi
%

\usepackage{fancyvrb}

\usepackage{ctable,longtable}

\usepackage{float} % provides the H option for float placement
\usepackage{dcolumn} % allows for different column alignments
\newcolumntype{.}{D{.}{.}{1.2}}

\usepackage{booktabs} % nicer horizontal rules in tables

%Assume we want graphics always
\usepackage{graphicx}
% We will generate all images so they have a width \maxwidth. This means
% that they will get their normal width if they fit onto the page, but
% are scaled down if they would overflow the margins.
%% \makeatletter
%% \def\maxwidth{\ifdim\Gin@nat@width>\linewidth\linewidth
%%   \else\Gin@nat@width\fi}
%% \makeatother
%% \let\Oldincludegraphics\includegraphics
%% \renewcommand{\includegraphics}[1]{\Oldincludegraphics[width=\maxwidth]{#1}}
\graphicspath{{.}}


%% \ifxetex
%% \usepackage[pagebackref=true, setpagesize=false, % page size defined by xetex
%% unicode=false, % unicode breaks when used with xetex
%% xetex]{hyperref}
%% \else
\usepackage[pagebackref=true, unicode=true, bookmarks=true, pdftex]{hyperref}
% \fi


\hypersetup{breaklinks=true,
  bookmarks=true,
  pdfauthor={},
  pdftitle={Bargaining and Identity Perspectives on Group Conflict},
  pdfkeywords = {political psychology, bargaining, group conflict, intergroup contact},
  colorlinks=true,
  linkcolor=BrickRed,
  citecolor=blue, %MidnightBlue,
  urlcolor=BrickRed,
  % urlcolor=blue,
  % linkcolor=magenta,
  pdfborder={0 0 0}}

%\setlength{\parindent}{0pt}
%\setlength{\parskip}{6pt plus 2pt minus 1pt}
%\usepackage{parskip}
\setlength{\emergencystretch}{3em}  % prevent overfull lines
\providecommand{\tightlist}{%
  \setlength{\itemsep}{0pt}\setlength{\parskip}{0pt}}

%% Insist on this.
\setcounter{secnumdepth}{2}

\VerbatimFootnotes % allows verbatim text in footnotes

\title{Bargaining and Identity Perspectives on Group Conflict}

\author{\parbox{.7\linewidth}{\centering
Chris Grady\thanks{\href{mailto:cdgrady2@illinois.edu}{\nolinkurl{cdgrady2@illinois.edu}},
University of Illinois at Urbana-Champaign, Department of Political
Science.} \quad
}
}

\date{June 21, 2020}

\usepackage{versions}
\makeatletter
\renewcommand*\versionmessage[2]{\typeout{*** `#1' #2. ***}}
\renewcommand*\beginmarkversion{\sffamily}
  \renewcommand*\endmarkversion{}
\makeatother

\excludeversion{comment}

%\usepackage[margins=1in]{geometry}

\usepackage[compact,bottomtitles]{titlesec}
\setcounter{secnumdepth}{3}
%\titleformat{ ⟨command⟩}[⟨shape⟩]{⟨format⟩}{⟨label⟩}{⟨sep⟩}{⟨before⟩}[⟨after⟩]
\titleformat{\section}[hang]{\Large\bfseries}{\thesection}{.5em}{\hspace{0in}}[\vspace{-.2\baselineskip}]
\titleformat{\subsection}[hang]{\large\bfseries}{\thesubsection}{.5em}{\hspace{0in}}[\vspace{-.2\baselineskip}]
\titleformat{\subsubsection}[hang]{\bfseries}{\thesubsubsection}{.5em}{\hspace{0in}}[\vspace{-.2\baselineskip}]
%\titleformat{\subsubsection}[runin]{\bfseries}{\thesubsubsection}{1ex}{}[\vspace{-.2\baselineskip}]
\titleformat{\paragraph}[runin]{\bfseries\itshape}{\theparagraph}{1ex}{}{\vspace{-.2\baselineskip}}
%\titleformat{\paragraph}[runin]{\itshape}{\theparagraph}{1ex}{}{\vspace{-.2\baselineskip}}

%%\titleformat{\subsection}[hang]{\bfseries}{\thesubsection}{.5em}{\hspace{0in}}[\vspace{-.2\baselineskip}]
%%%\titleformat*{\subsection}{\bfseries\scshape}
%%%\titleformat{\subsubsection}[leftmargin]{\footnotesize\filleft}{\thesubsubsection}{.5em}{}{}
%%\titleformat{\subsubsection}[hang]{\small\bfseries}{\thesubsubsection}{.5em}{\hspace{0in}}[\vspace{-.2\baselineskip}]
%%\titleformat{\paragraph}[runin]{\itshape}{\theparagraph}{1ex}{}{\vspace{-.5\baselineskip}}

%\titlespacing*{ ⟨command⟩}{⟨left⟩}{⟨beforesep⟩}{⟨aftersep⟩}[⟨right⟩]
\titlespacing{\section}{0pc}{1.5ex plus .1ex minus .2ex}{.5ex plus .1ex minus .1ex}
\titlespacing{\subsection}{0pc}{1.5ex plus .1ex minus .2ex}{.5ex plus .1ex minus .1ex}
\titlespacing{\subsubsection}{0pc}{1.5ex plus .1ex minus .2ex}{.5ex plus .1ex minus .1ex}



%% These next lines tell latex that it is ok to have a single graphic
%% taking up most of a page, and they also decrease the space around
%% figures and tables.
\renewcommand\floatpagefraction{.9}
\renewcommand\topfraction{.9}
\renewcommand\bottomfraction{.9}
\renewcommand\textfraction{.1}
\setcounter{totalnumber}{50}
\setcounter{topnumber}{50}
\setcounter{bottomnumber}{50}
\setlength{\intextsep}{2ex}
\setlength{\floatsep}{2ex}
\setlength{\textfloatsep}{2ex}

\usepackage{setspace}
\doublespacing
\setlength{\parindent}{4em}

\begin{document}
\VerbatimFootnotes

%\begin{titlepage}
%  \maketitle
%\vspace{2in}
%
%\begin{center}
%  \begin{large}
%    PROPOSAL WHITE PAPER
%
%BAA 14-013
%
%Can a Hausa Language Television Station Change Norms about Violence in Northern Nigeria? A Randomized Study of Media Effects on Violent Extremism
%
%Jake Bowers
%
%University of Illinois @ Urbana-Champaign (jwbowers@illinois.edu)
%
%\url{http://jakebowers.org}
%
%Phone: +12179792179
%
%Topic Number: 1
%
%Topic Title: Identity, Influence and Mobilization
%
%\end{large}
%\end{center}
%\end{titlepage}


\maketitle

\begin{abstract}\singlespacing\noindent
There are two major perspectives on group conflict in political science.
One, which I refer to as the \emph{bargaining perspective}, considers
group conflict the result of information and commitment problems. The
other, which I refer to as the \emph{identity perspective}, considers
group conflict the result of the prejudice and other negative emotions
that naturally arise between groups or arise when groups compete for
resources. These two perspectives are rarely combined, but their
synthesis could benefit both literature's attempts to understand group
conflict and craft policies to reduce it. In this paper, I apply
insights about group biases from the identity perspective to the
bargaining perspective's framework of information and commitment
problems. I show how a solution originating from the identity
perspective, intergroup contact, alleviates information and and
commitment problems. I then use farmer-pastoralist conflict in Nigeria
as an illustrative example of how to understand group conflict using
both perspectives. I close by proposing six policy implications of
combining these perspectives and offering five avenues for future
research.
\\ \par\noindent\emph{Keywords}:political psychology, bargaining, group conflict, intergroup contact\par
\end{abstract}

\newcommand\blfootnote[1]{%
  \begingroup
  \renewcommand\thefootnote{}\footnote{#1}%
  \addtocounter{footnote}{-1}%
  \endgroup
}
\singlespacing\blfootnote{Thanks to Caglayan Baser, Ekrem Baser, Nuole Chen, and Alice Iannantuoni
for useful conversations and feedback on earlier drafts of this paper.
Thanks to Jake Bowers, Jim Kuklinski, Justin Rhodes, and Cara Wong for
feedback and guidance. Thanks to Danjuma Dawop, Lisa Inks, Rebecca
Wolfe, and the Mercy Corps Nigeria team for granting me the opportunity
to work with them on farmer-pastoralist conflict in Nigeria. Thanks to
Tahiru Ahmadu and Israel Okpe for their keen insights and knowledge
about group conflict between farmers and pastoralists.}

\newpage

\hypertarget{introduction}{%
\section{Introduction}\label{introduction}}

Groups fight for the resources -- land, wealth, power, or otherwise --
that fighting destroys. As such, groups could almost always divide the
resources before war in a way that benefits them more than their postwar
settlement. This is the enigma of violent conflict: why do groups engage
in violent conflict when violent conflict often destroys the very thing
the groups are fighting for? One answer is that groups suffer from a
bargaining failure when negotiating for peace (Fearon 1995; Powell 2006;
Reed et al. 2016). These bargaining failures can be caused by a lack of
trust (a commitment problem) or a lack of information (an information
problem). The commitment problem occurs when both sides are better off
negotiating a peaceful agreement but neither group trusts the other to
abide by negotiated agreements. The information problem occurs when
neither side knows the other side's power or preferences. If groups (1)
trusted each other to abide by agreements and (2) knew the strength and
preferences of the other side, they would not need to fight and each
group would enjoy superior outcomes relative to their outcomes after
fighting. I call this the \emph{bargaining perspective} of group
conflict.\footnote{Another common explanation for bargaining failures is
  \emph{issue indivisibility}: when the thing that groups would gain
  from fighting cannot be split easily or at all (the throne of a
  monarchy or control over holy land). Issue indivisibility is thought
  to rarely be a problem, however, because almost all things that groups
  fight for are divisible or can be substituted by something divisible
  (Fearon 1995). For this reason, issue indivisibility is sometimes seen
  as a commitment problem (Powell 2006). If the group being given the
  throne could commit to providing a substantial portion of the spoils
  to the other side, the group's could make that peace agreement and
  avoid fighting. The group given the throne cannot credibly make this
  commitment, however, because once in power nothing prevents that group
  from reneging on their agreement. I do not consider issue
  indivisibility as a distinct explanation for bargaining failures in
  this paper.}

Another answer to the question of why group's fight focuses on the
negative attitudes, emotions, and biases that explain individual group
members' lack of trust in and lack of accurate information about the
other side, as well as why groups might prefer fighting relative to a
peace deal even if fighting destroys resources (Campbell 1965; Tajfel et
al. 1971; Stephan, Stephan, and Oskamp 2000; Allport 1954; Runciman and
Runciman 1966; Sherif et al. 1988; Ross and Ward 1995; see Böhm, Rusch,
and Baron 2018 for a review of psychological theories of conflict). The
lack of trust can occur due to conflicts of interests and competition
for scarce resources or as a natural consequence of categorization into
ingroup and outgroup. The lack of accurate information is a consequence
of group's perceptual biases, outgroup stereotyping, and limited
interaction. Preferences for fighting over a peace deal the other side
would accept stem from self-esteem derived from feelings of group
superiority and considerations of relative gains over absolute gains. If
members of opposing groups can identify common interest or a common
identity, they can avoid the competition, emotions, and perceptual
biases that lead to conflict. I call this the \emph{identity
perspective} of group conflict.

In this paper, I argue that these perspectives are compatible and
attempt a synthesis by applying the identity perspective to the
bargaining perspective. There is growing interest in bridging political
psychology and political economy explanations for concepts like group
conflict, which is a concern for both literatures (Little and Zeitzoff
2017; Kertzer and Tingley 2018). At their heart, both perspectives focus
on preferences, trust, and information, and both consider preferences
based on material resources and non-material goods like values, esteem,
norms, and power\footnote{I use ``material'' to refer to things that
  exist in the world, like physical resources and infrastructure. I use
  ``non-material'' to refer to things that exist in the mind but not in
  the world, like values, emotions, trust, and reputations.}. The
identity perspective tends to highlight the formation of preferences and
trust and the perceptual biases that cause information asymmetries,
while the bargaining perspective tends to highlight how preferences,
mistrust, and information asymmetries cause conflict.\footnote{This
  characterization necessarily lacks the nuance to describe the entirety
  of these two perspectives. There is, of course, work in the identity
  perspective that considers the consequences of preferences {[}chris:
  example.{]} and work in the bargaining perspective that considers the
  formation of preferences {[}chris: cite endogenous preferences{]}. But
  I believe this characterization describes the main arguments of each
  perspective and the seminal articles on which subsequent work has been
  built.} Combining these perspectives will help scholars better
understand the causes of group conflict and practitioners better develop
policies to reduce group conflict.

I begin by reviewing the bargaining perspective on group conflict,
focusing on information problems, commitment problems, and solutions to
those problems. I follow that review by applying the identity
perspective to these bargaining problems and their solutions, focusing
on how group biases and other psychological factors contribute to
bargaining failures. In doing so, I propose intergroup contact as a way
to reduce group conflict by providing opportunities for the groups to
gain accurate information about each other and to build intergroup
trust, which alleviates information and commitment problems. I then
apply a synthesis of these perspectives to farmer-pastoralist conflict
in Nigeria, describing factors that push groups into conflict,
institutions groups have developed to avoid violence, and the
consequences when no such institutions exist. I conclude by describing
implications for peacebuilding programs and considering avenues for
future research.

\hypertarget{bargaining-perspective-on-group-conflict}{%
\section{Bargaining Perspective on Group
Conflict}\label{bargaining-perspective-on-group-conflict}}

The bargaining perspective considers conflict the result of a bargaining
failure. The two groups both want something and can either fight for it
or negotiate over it. Both groups are better off negotiating pre-war to
get their expected outcome post-war, without paying the cost of
fighting. But sides cannot agree to negotiated bargain due primarily to
commitment problems and information problems (Fearon 1995). These
problems cause a bargaining failure, and sides resort to
fighting.\newline

\noindent \textbf{Commitment Problem}: One reason sides resort to
fighting is that neither group can credibly commit to honor bargained
agreements. This ``commitment problem'' prevents any agreement from
being made because neither group will commit to agreements today that
they believe will be broken tomorrow. Because they cannot reach an
agreement, the groups fight.\footnote{A commitment problem can occur for
  many reasons. It often surfaces due to the potential for bargaining
  power to shift after an agreement. If bargaining power shifts after an
  agreement, one side will have an incentive to renege on that agreement
  to achieve a better outcome.} Commitment problems are often called
trust problems because the problem arises because neither group
\emph{trusts} the other to honor commitments. Commitment/Trust problems
are a common explanation for the inefficiency puzzle of violent conflict
(Fearon 1994, 1998; Powell 2006; Reed et al. 2016; Lake 2003)

The canonical example of a commitment problem is the prisoners' dilemma.
In the Prisoner's Dilemma, two criminals are arrested and interrogated
separately. The law enforcement officers lack the information to convict
the criminals for the full extent of their crimes but have enough
information to convict each criminal for a lesser charge. The officers
offer each criminal a deal: testify against your partner and you will be
set free; your partner will spend three years in prison. If both
criminals stay silent, they will each serve one year in prison. If they
both testify, they will each serve two years in prison. The prisoners
face a dilemma: why stay silent if you are best off testifying
regardless of what your partner does? The problem here is each side's
inability to commit to remaining silent -- a commitment problem. If they
could commit, they could get their second favorite outcome; without the
ability to commit, they get their second least-favorite outcome.

The Prisoner's Dilemma is formalized in Table \ref{tab:comProb}. The
criminals are referred to as ``players''. Staying silent is referred to
as ``cooperating'' with your partner; testifying is referred to as
``defecting'' against your partner. Each player wants their highest
payoff.

\begin{table}[h!]
\begin{center}
\setlength{\extrarowheight}{2pt}
\begin{tabular}{cc|c|c|}
    & \multicolumn{1}{c}{} & \multicolumn{2}{c}{Player $2$}\\
    & \multicolumn{1}{c}{} & \multicolumn{1}{c}{$Cooperate$}  & \multicolumn{1}{c}{$Defect$} \\\cline{3-4}
    \multirow{2}*{Player $1$}  & $Cooperate$ & $(2,2)$ & $(0,3)$ \\\cline{3-4}
      & $Defect$ & $(3,0)$ & $(1,1)$ \\\cline{3-4}
\end{tabular}
\caption{\label{tab:comProb}\textbf{Example of Commitment Problem.} Numbers represent payoffs to players.  The first number in each cell represents Player 1's payoffs, the second number represents Player 2's payoffs.  Player's want the highest payoff.}
\end{center}
\end{table}

Unless the criminals value cooperation above freedom or have some way of
punishing defection -- both of which change the above payoff structure
--, both criminals have an incentive to defect regardless of their
partners behavior. If Player 1 defects, Player 2 is best off defecting.
If Player 1 cooperates, Player 2 is also best off defecting. No one is
happy with the resulting outcome. The two sides would commit to mutual
cooperation to avoid mutual defection, but they cannot credibly commit
to cooperating because they both prefer to defect regardless of what the
other does.

This can be applied to group conflict where ``cooperation'' is abiding
by a peace agreement and ``defection'' is breaking it. If one side
abides by a peace agreement and disarms, for example, the other side has
an incentive to break the agreement by staying armed. Since neither can
trust the other to honor the peace agreement, both sides defect.\newline

\noindent \textbf{Information Problem}: Another reason groups fight is
that neither side knows the preferences of the other. This ``information
problem'' prevents peace agreements from being made, even if both sides
prefer mutual cooperation, because neither side knows that the other
would reciprocate cooperation. Because both sides fear defection if they
cooperate, the groups fight. The information problem of conflict is one
of the most common explanations for violent conflict (Fey and Ramsay
2011; Smith and Stam 2003; Fearon 1995; Kydd 2000; Moon and Souva 2016;
Rohner, Thoenig, and Zilibotti 2013; Wolford, Reiter, and Carrubba 2011)

We can observe an information problem by modifying the Prisoner's
Dilemma above. Imagine that each criminal is ``honorable'' and values
reciprocating cooperation. This bonus to cooperation can be seen as the
psychological value that each side places on being honorable; it is
represented by adding +2 to the payoff of joint cooperation. Each
criminal wants to cooperate if the other cooperates but wants to defect
if the other defects. Each criminal knows that they are honorable, but
neither knows that the other is also honorable. The problem here is a
lack of information -- an information problem.

This modified Prisoner's Dilemma, often called a Trust Game (Kydd 2000),
is formalized in Table \ref{tab:infoProb}. Each player again chooses
whether to cooperate or defect. Player 1 is honorable and prefers to
cooperate if Player 2 will cooperate but defect if Player 2 will defect.
Player 1 is unsure if Player 2 is also honorable.\footnote{In a Trust
  Game, Player 1 would believe Player 2 is honorable with probability
  \(p\), where \(p\) represents Player 1's trust in Player 2.} World A
represents the payoffs if Player 2 is honorable; World B represents the
payoffs if Player 2 is not. In World A there is only an information
problem: if groups knew each other's preferences, they have an incentive
to cooperate. In World B there is also a commitment problem: even if
groups knew each other's preferences, each still has an incentive to
defect.

\begin{table}[h!]
\begin{center}
\setlength{\extrarowheight}{2pt}
\begin{subtable}{0.48\linewidth}
\caption{\label{tab:info_a}\textbf{World A} \\Both sides prefer to cooperate if the other side cooperates but defect if the other side defects.}
\begin{tabular}{cc|c|c|}
    & \multicolumn{1}{c}{} & \multicolumn{2}{c}{Player $2$}\\
    & \multicolumn{1}{c}{} & \multicolumn{1}{c}{$Cooperate$}  & \multicolumn{1}{c}{$Defect$} \\\cline{3-4}
    \multirow{2}*{Player $1$}  & $Cooperate$ & $(4,4)$ & $(0,3)$ \\\cline{3-4}
      & $Defect$ & $(3,0)$ & $(1,1)$ \\\cline{3-4}
\end{tabular}
\end{subtable}%
\hfill
\begin{subtable}{0.48\linewidth}
\caption{\label{tab:info_b}\textbf{World B} \\Player 1 prefers mutual cooperation, but Player 2 prefers to defect regardless of Player 1's Behavior.}
\begin{tabular}{cc|c|c|}
    & \multicolumn{1}{c}{} & \multicolumn{2}{c}{Player $2$}\\
    & \multicolumn{1}{c}{} & \multicolumn{1}{c}{$Cooperate$}  & \multicolumn{1}{c}{$Defect$} \\\cline{3-4}
    \multirow{2}*{Player $1$}  & $Cooperate$ & $(4,2)$ & $(0,3)$ \\\cline{3-4}
      & $Defect$ & $(3,0)$ & $(1,1)$ \\\cline{3-4}
\end{tabular}
\end{subtable}
\caption{\label{tab:infoProb}\textbf{Example of Information Problem.} Numbers represent payoffs to players.  The first number in each cell represents Player 1's payoffs, the second number represents Player 2's payoffs.  Player's want the highest payoff.}
\end{center}
\end{table}

If Player 2 is honorable, Player 1 wants to cooperate. If Player 2 is
not honorable, then Player 1 is in the Prisoner's Dilemma above and
wants to defects. Player 1 cooperates or defects depending on her belief
that Player 2 is honorable. Without a way to for Player 2 to signal
honor, Player 1 defects. Again, no one is happy with the resulting
outcome. The two sides would prefer mutual cooperation even to betraying
the other player, but because neither side knows the other's preferences
both players defect.

This can again be applied to group conflict where ``cooperation'' is
abiding by a peace agreement and ``defection'' is breaking it. In the
case where both players are honorable, there is no commitment problem:
both players will cooperate if the other player cooperates. Conflict
could still occur due to information problems, however. Unless both
sides know that the other will reciprocate cooperation, both sides may
defect to avoid being betrayed. It is common in group conflict for each
group to claim that the other is the barrier to peace.

Another way to represent the information problem in a conflict setting
is to say that two groups need to split a resource, but neither knows
each other's fighting strength (capacity) or valuation of the resource
(willingness). Without knowing the other side's capacity and willingness
to fight, neither side knows what they should offer or receive from a
peace agreement. If, for example, each side believes (incorrectly) that
they are stronger than the other, that overconfidence prevent both sides
from finding a peace agreement in which they receive more than they
expect to receive through fighting (Johnson 2009). This information
problem can cause bargaining failures even if the two sides communicate.
Both groups have an incentive to portray themselves as stronger, more
willing to fight, and less willing to make concessions than they truly
are in order to achieve an advantageous bargaining outcome (Fearon
1995).

Groups can suffer from information problems and commitment problems
simultaneously. In this example of splitting a resource, groups may
fight even if both know each other's fighting strength and valuation of
the resource under dispute (i.e.~under complete information). With this
complete information, groups could presumably allocate a portion of that
resource to each side, with the higher-strength side taking a larger
portion. The problem, however, is that once a group takes a larger
portion of the resource, they now have more fighting strength and could
renegotiate an even more advantageous deal for themselves. Knowing this,
the weaker side may prefer to fight rather than acquiesce to a series of
peace agreements under weaker and weaker bargaining power.\newline

\noindent \textbf{No bargaining range}: A third reason groups fight is
that one or both sides prefers fighting to peace. In other words, there
is no bargaining range: peace agreements acceptable to both sides. I
will refer to this as a ``preferences problem''. This preferences
problem is not a bargaining failure; if one or both sides prefer
fighting to any peace deal, then groups are not bargaining for peace.
Nor is this a ``problem'' the way as commitment problems and information
problems are a problem. In those problems, conflict is a puzzle because
groups end up with outcomes neither group likes and which do not
maximize overall utility. If mutual fighting maximizes overall utility,
it is a problem normatively in that we normatively dislike fighting, but
it is only a puzzle if we think groups should not prefer fighting. For
these reasons, preferences for fighting over peace are an important
reason for group conflict but are discussed less frequently in published
literature (but see Coe 2012, @slantchev2012borrowed, and
@chang2013war).

We can observe a preferences problem by again modifying the Prisoner's
Dilemma. Imagine that the criminals hate each other and would regret
\emph{not} ratting out the other guy. This is represented by adding +1
to the payoff of joint defection and -1 to the payoff of joint
cooperation. Their payoffs would look like Table \ref{tab:prefProb}.

\begin{table}[h!]
\begin{center}
\setlength{\extrarowheight}{2pt}
\begin{tabular}{cc|c|c|}
    & \multicolumn{1}{c}{} & \multicolumn{2}{c}{Player $2$}\\
    & \multicolumn{1}{c}{} & \multicolumn{1}{c}{$Cooperate$}  & \multicolumn{1}{c}{$Defect$} \\\cline{3-4}
    \multirow{2}*{Player $1$}  & $Cooperate$ & $(1,1)$ & $(0,3)$ \\\cline{3-4}
      & $Defect$ & $(3,0)$ & $(2,2)$ \\\cline{3-4}
\end{tabular}
\caption{\label{tab:prefProb}\textbf{Example of Preferences Problem.} Numbers represent payoffs to players.  The first number in each cell represents Player 1's payoffs, the second number represents Player 2's payoffs.  Player's want the highest payoff.}
\end{center}
\end{table}

This setup, like the Prisoner's Dilemma, results in mutual defection.
Unlike the Prisoner's Dilemma, the players would not commit to
cooperation even if they could; nor does knowledge about each other's
preferences avoid defection. Mutual defection is the Pareto-optimal
outcome in that it maximizes the total utility of both sides.

Preferences for fighting can also be seen in group conflict. Each side
may prefer to fight because peace is relatively more costly, perhaps
because peace entails mounting armament costs (Coe 2012; Chang and Luo
2017) or because of debt payments that a state can only repay if they
win the conflict (Slantchev 2012). Sides may also prefer fighting to
peace because fighting is less costly than assumed due limited
destruction of resources (Chang and Luo 2013). Sides may also use
fighting today to improve their bargaining position with their current
adversary (Slantchev 2003) or potential future adversaries (Crescenzi
2007). \footnote{Lack of information and overconfidence (Johnson 2009)
  about likely victory in conflict can cause what looks like a
  ``preferences problem'' in that groups may prefer fighting relative to
  any peace deal the other side is willing to accept. The difference
  between an information problem and a preference problem is that the
  information problem disappears if both sides have complete information
  about the other, but complete information does not change the
  preferences problem.}

\hypertarget{solutions-to-group-conflict-in-the-bargaining-perspective}{%
\subsection{Solutions to Group Conflict in the Bargaining
Perspective}\label{solutions-to-group-conflict-in-the-bargaining-perspective}}

I will discuss third party intervention and reputation-building as the
primary solutions to commitment and information problems in the
bargaining perspective. Third party intervention can solve information
problems by providing accurate, credible information to both sides; it
can solve commitment problems by punishing defection so that it is in
both side's interests to cooperate. Reputation-building can solve
commitment problems and information problem by revealing the preferences
of each player and by the prospect of better future outcomes from a
cooperative reputation than a noncooperative reputation.\footnote{These
  problems can also be solved through signaling, where each side sends
  costly signals of their preferences to the other side. Because the
  signals are costly, non-cooperative types do not send costly signals.
  In this discussion, I consider signaling as a type of
  reputation-building.}

Third party intervention is a common mechanism for overcoming
information and commitment problems and resolving group conflict. Third
parties can solve information problems by mediating disputes, providing
accurate information about the preferences and strength of both sides
(Crescenzi et al. 2011; Gartner 2011; but see Kydd 2003, @kydd2006can,
and @beber2012international for the problem of mediator credibility).
Third parties can solve commitment problems by punishing defection,
which incentivizes each side to honor agreements. Though each group may
have an incentive to defect on an agreement after it is made, the groups
have less incentive to defect if a strong third party is capable of and
willing to punish defection from bargained agreements (Fearon 1994).
Third parties that facilitate information flows and enforce agreements
often effectively reduce conflict (Doyle and Sambanis 2000; Di Salvatore
and Ruggeri 2017; Walter 2002; Hartzell, Hoddie, and Rothchild 2001;
Wallensteen and Svensson 2014).\footnote{Some scholars have noted that
  third parties do not create trust between those groups that will last
  beyond the third party's presence (Gambetta and others 2000; Rohner,
  Thoenig, and Zilibotti 2013; Beardsley 2008).}

In many group conflicts, however, there exists no third party with the
capacity and incentive to intervene into the conflict (Fey and Ramsay
2010, 530). This situation is common for internal conflicts in weak
states, where conflicts are diffuse and the state lacks the capacity to
mediate or intervene effectively. Many group conflicts occur in such
states {[}chris: cite{]}.

In those cases, groups can resolve information and commitment problems
without relying on third party intervention. Groups can resolve those
problems by cultivating reputations for trustworthiness in repeated
interactions. Repeated interactions between groups, like third parties,
help overcome these problems by providing each group with the incentive
to honor agreements. Here the incentive comes not from fear of
punishment by a third party but the prospect of better future outcomes
from cooperation than defection. Though each group may have an incentive
to defect on an agreement today if the groups will not interact
tomorrow, the groups have an incentive to cooperate now if their
behavior today will be reciprocated by the other side in future
interactions (Ostrom and Walker 2003; Kydd 2000; Axelrod and Hamilton
1981).\footnote{I discuss reciprocity and reputations together, but
  these mechanisms are subtly different. The reciprocity mechanism is
  generally mobilized for contexts with just two groups. In those
  contexts, cooperation in previous interactions assists in obtaining
  cooperative behavior in the future from the same partner; other
  potential partners are unnecessary. The reputation mechanism is
  generally (though not always) mobilized for contexts in which many
  groups observe the behavior of many other groups. In those contexts a
  good reputation assists in obtaining \emph{other} cooperative
  partners; reciprocity from the same partner is unnecessary. I discuss
  these mechanisms together because both rely on creating cooperative
  expectations from bargaining partners in future interactions.} When
both groups stand to gain more from enduring cooperation than enduring
defection, but information and/or commitment problems would stymie
cooperation in one-shot interactions, repeated interactions provide
groups with the incentive to cooperate and provide groups with the
opportunity to signal their preference for cooperation.

In this section, I reviewed common explanations for and solutions to
group conflict under the bargaining perspective. In the next section, I
show how the identity perspective fits with and adds to the bargaining
perspective.

\hypertarget{applying-the-identity-perspective-to-bargaining}{%
\section{Applying the Identity Perspective to
Bargaining}\label{applying-the-identity-perspective-to-bargaining}}

The bargaining perspective explains formalizes the reasons groups can
fight: preferences (lack of bargaining range), information problems,
commitment problems. I argue that the identity perspective adds to these
explanations by describing how psychological and social factors (1)
decrease the bargaining range by providing incentives for groups to
fight, (2) cause commitment problems by preventing the formation of
trust, and (3) cause information problems by biasing information
processing about the other group and perceptions of their preferences.
By looking at groups as a collection of individual group members and
focusing on individuals' perceptual and cognitive biases, the identity
perspective helps explain when and why fighting will occur.

\hypertarget{costs-of-peace-and-benefits-of-fighting-reduce-the-bargaining-range}{%
\subsection{Costs of Peace and Benefits of Fighting Reduce the
Bargaining
Range}\label{costs-of-peace-and-benefits-of-fighting-reduce-the-bargaining-range}}

Psychological and social factors can contribute to group conflict by
reducing or erasing the range of peace agreements both sides will
accept. These factors reduce the bargaining range by adding costs to
peace and benefits to fighting. I identify three main ways that
psychological and social factors add costs to peace: (1) direct costs of
changing attitudes and social norms, (2) loss of self-esteem, and (3)
loss of sunk costs.\footnote{These three reasons are not an exhaustive
  list of the ways that psychological and social factors add costs to
  peace. Rather than give an exhaustive list, I seek to demonstrate that
  explicitly considering psychological and social factors helps explain
  why groups would find peace costly.}

First, direct costs to attitude change occur to the extent that
accepting peace requires individuals in the group to change attitudes
about the outgroup. Individuals use many strategies to maintain existing
attitudes -- searching for information that confirms pre-existing
beliefs, counter-arguing information divergent with their beliefs --
suggesting that costs to attitude change are not insubstantial
(Festinger 1962; Nickerson 1998; Kunda 1990). Changing negative
attitudes towards an enemy may be especially challenging because
negative outgroup attitudes are supported by a social norms and
justifications that feeds conflict (Bar-Tal 2007). At the group-level, a
peace agreement requires groups to dismantle the social institutions
that fed conflict and encouraged violence against the outgroup
(Bornstein 2003). Those institutions must be reformed to prevent, not
encourage, intergroup violence.

Second, peace agreements may cause a loss of self-esteem for individual
group members. Group members derive self-esteem from positively
comparing their group to a rival group, and and any agreement in which
their side acknowledges the legitimacy of the other challenges this
group-based boost to self-esteem (Tajfel 1981; Wood 2000; Tajfel and
Turner 1979; Fein and Spencer 1997; Martiny, Kessler, and Vignoles 2012;
Brown and Pehrson 2019). Group members may also pay psychological costs
to their self-esteem from losing their rationalization for engaging in
discriminatory or aggressive behaviors towards outgroup members. People
rationalize their behavior to maintain a moral self-image (Bandura 1999,
2014; Mazar, Amir, and Ariely 2008), and people who harmed the outgroup
must come to terms with past behaviors that are now deemed immoral.

Third, individuals are affected by sunk costs and want the gain from a
peace deal to make up for the cost of fighting even though those costs
were already born (Arkes and Blumer 1985). The desire to recoup sunk
costs induces both sides to demand that the other concede more in a
peace agreement. It would be ``dissonant for the disputing parties to
accept terms today that could have been achieved, without the ensuing
costs, at some earlier point in time'' (Ross and Ward 1995, 264). In
conjunction with biased perceptions and memories of conflict history
(discussed below), each group is also prone to believe that they are
morally entitled to more concessions than the other side (Ward et al.
1997). Accepting terms that could have been achieved without fighting is
also to admit that fighting was a mistake, and people are loathe to
admit their mistakes (Tavris and Aronson 2008). This partially explains
why ``emotions stemming from past interethnic violence serve as
impediments to peaceful resolutions to present-day conflicts'' (Little
and Zeitzoff 2017, 5; Horowitz 2001; Petersen 2002)

To see how these psychological costs to peace might sabotage peace
efforts, imagine that groups fighting over land sign a peace agreement
that assigns some land to each group. This agreement may make material
sense in that each group can enjoy the benefits of some land instead of
fighting for all of it. But in addition to the material cost of losing
potential access to land granted to the other side, each group must
begin punishing violence against outgroup members rather than
encouraging it. Individuals in the group must adjust their behavior to
conform to new norms and rethink the status of their group relative to
the outgroup; they can no longer ``bask in the reflected glory'' of
their group's superiority (Brown and Pehrson 2019, 312). Individuals who
felt justified in overcharged outgroup members for services or
physically harming outgroup members are now told that such behaviors are
wrong and immoral. Individuals must also accept that their sacrifices to
benefit fighting -- dead friends and loved ones, material deprivation,
time and energy spent fighting -- were for naught: the sacrifices did
not help the group win. Sustaining conflict against the outgroup allows
group members to maintain existing attitudes and behaviors and preserves
the possibility that their group will be victorious.

Other than avoiding these costs, group members may gain benefits from
fighting for two main reasons. First, group members may feel pleasure in
response to outgroup pain (Weisel and Böhm 2015; Cikara et al. 2014).
This pleasure adds psychological utility to participating in conflict
for individual group members. Second, individuals may like gaining glory
and other social rewards for participating in violent conflict.
Societies use various social carrots and sticks to encourage
participation in conflict (Gneezy and Fessler 2012), and group members
give up those rewards when conflict ends. These social rewards add
social utility to participating in conflict, which could make fighting
more attractive than peace. These benefits to fighting may work in
conjunction with each side's desire to signal strength and resolve to
future potential adversaries and each side's tendency to overestimate
their chances of victory, which can also promote preferences for
fighting in the bargaining perspective (Johnson 2009; Crescenzi 2007).

As a result of these costs to peace and benefits to conflict, there may
exist no peace agreement that both sides prefer to fighting. Each group
may prefer mutual defection and their payoff structure may resemble
Table \ref{tab:prefProb}, where fighting occurs because groups have a
``preferences problem''. Each side may only accept a peace agreement
that materially favor their group because the material gains from peace
must overcome its non-material costs; it is difficult for each side to
appreciate the non-material costs of the other (Ward et al. 1997).
Groups may also consider their gains relative to the outgroup, rather
than in absolute terms. Indeed, many group members are willing to accept
lower absolute gains to increase relative gain over the outgroup
{[}Turner, Brown, and Tajfel (1979); Waltz (2010); Halevy et al.
(2010);{]}. The preference for gains relative to the outgroup makes any
mutually beneficial peace agreement impossible.

\hypertarget{biased-perceptions-cause-information-and-commitment-problems}{%
\subsection{Biased Perceptions Cause Information and Commitment
Problems}\label{biased-perceptions-cause-information-and-commitment-problems}}

Though the above factors can increase the costs of peace and increase
benefits of conflict, the high cost of conflict may be such that both
sides still prefer peace. Even when both sides prefer peace,
psychological and social factors can contribute to the information and
commitment problems that prevent groups from negotiating peace
agreements. I identify two main ways that these factors contribute to
information and commitment problems. First, cognitive dissonance,
motivated reasoning, and confirmation bias\footnote{\emph{Cognitive
  dissonance} is the mental discomfort that occurs when one individual
  holds two contradictory beliefs. Individuals resolve this dissonance
  by rejecting one of the contradictory beliefs (Festinger 1962; Tavris
  and Aronson 2008). Individuals tend to resolve this dissonance by
  rejecting the newer or less central belief because it is costly to
  reject older and more central beliefs (Schwartz and Bilsky 1990;
  Converse 1970; Bryan, Yeager, and Hinojosa 2019). \emph{Motivated
  reasoning} and \emph{confirmation bias} are related concepts.
  Motivated reasoning is the tendency for individuals to ``reason'' in
  whatever way allows them to reach their desired conclusion (Kunda
  1990). Confirmation bias is the tendency for individuals to interpret
  new information and search memory for information to confirm existing
  beliefs (Nickerson 1998).} cause each side to interpret the other's
behavior in a way that supports existing negative attitudes. Second,
loss aversion and reactive devaluation\footnote{\emph{Loss aversion}
  refers to individuals' tendency to prefer avoiding losses more than
  they attaining equivalent gains (Kahneman and Tversky 2013). Reactive
  devaluation is the tendency for individuals to undervalue concessions
  and proposals from antagonists (Ward et al. 1997; Ross and Stillinger
  1991).} cause the two sides to overvalue concessions they give up and
undervalue concessions they receive. These phenomena introduce
significant friction in the ability of groups to accurately perceive the
other's preferences and to build trusting relationships.

Cognitive dissonance, motivated reasoning, and confirmation bias cause
group members to interpret and recall outgroup behavior in negative
ways. These phenomena prevent cross-group interactions from accurately
revealing each side's preferences because members of different groups
experience and interpret the same event differently (Ward et al. 1997).
Defensive action by outgroup members may be misperceived as belligerent
and threatening while belligerent action by one's own group is seen as
defensive and justified (Ward et al. 1997; Duncan 1976; Vallone, Ross,
and Lepper 1985). Positive actions by outgroup members may be
re-interpreted as negative to avoid cognitive dissonance (Gubler 2013;
Paolini, Harwood, and Rubin 2010; Good 2000). Unequivocally positive
behavior may be dismissed as an exception while negative attributes are
believed to define all outgroup members (Hewstone 1990). These
misperceptions can lead each group to believe the other side is
untrustworthy or prefers fighting to peace. And when recalling
information about the outgroup from memory, individuals selectively
recall events that corroborate their pre-existing negative perceptions
(De Dreu, Nijstad, and Knippenberg 2008).

Loss aversion and reactive devaluation can further sabotage the
trust-building process. Individuals tend to weigh losses more heavily
than equivalent gains, so anything their group gives up may be magnified
in importance (Kahneman and Tversky 2013). At the same time, individuals
tend to undervalue concessions from antagonists (Ward et al. 1997; Ross
and Stillinger 1991). These biased valuations of concessions made and
concessions given can prevent groups from reaching a mutually beneficial
agreement. For an abstract example, imagine that Group 1 offers Group 2
a concession that is objectively worth five ``negotiation units'', but
which Group 1 values as six due to loss aversion. Group 2 may interpret
that five as a four due to reactive devaluation, and then offer
something that is objectively only worth three but which they perceive
as worth four due to loss aversion. Each group could come away from that
encounter believing that the other is unreasonable and that there is no
peace agreement mutually satisfactory to both sides.

These phenomena can also interfere with groups sending and receiving
costly signals of cooperative intent, one of the main ways groups reveal
their preferences and solve commitment problems (Kydd 2000; Rohner,
Thoenig, and Zilibotti 2013; Gambetta and others 2000; Jervis 2017).
Costly signals are intended to reassure each side of the other's
trustworthiness. A costly signal from Group 1 to Group 2 can allow Group
2 to trust Group 1 because Group 1 would only pay the cost of the signal
if Group 1 was trustworthy. Perceptual biases like reactive devaluation,
however, may lower the perceived costliness of any cooperative signal.
Group 2 may perceive the signal as cheap and uninformative of Group 1's
preferences or intentions. Worse, Group 2 may perceive the signal as a
cynical attempt to manipulate Group 2 into cooperating so that Group 1
can take advantage of their naivete.

Compounding these perceptual problems, accurate information and
intergroup trust are hampered by a lack of opportunities to learn about
the other side. Groups in conflict tend to limit contact between the two
sides (Bornstein 2003), so most learning must come from observing the
outgroup's interactions with other groups. Even if those opportunities
are available, few of the outgroup's interactions will be with groups
that are relevant for predicting the outgroup's behavior towards the
ingroup, reducing the usefulness of observational learning (Kazdin 1974;
Yang 2013). The main opportunity to observe outgroup behavior and learn
their reputation is the ingroup's own interactions with the outgroup,
which are limited and hampered by perceptual biases.

\hypertarget{solutions-in-the-identity-perspective-applied-to-the-bargaining-perspective}{%
\subsection{Solutions in the Identity Perspective Applied to the
Bargaining
Perspective}\label{solutions-in-the-identity-perspective-applied-to-the-bargaining-perspective}}

I will discuss intergroup contact as the primary solution to group
conflict in the identity perspective. Intergroup contact, interactions
between group members in which members of different groups work together
to achieve common goals, is not the only solution for group conflict,
but I believe it is the most promising solution (see Böhm, Rusch, and
Baron 2018 for a summary of others.) Intergroup contact has the
potential to (1) change incentives for fighting and peace, (2) debias
perceptions of the outgroup, and (3) provide opportunities for costly
signaling. Put another way, intergroup contact can increase the
bargaining range and help solve information and commitment problems.

Intergroup contact can increase the value of peace and decrease the
value of fighting, thereby increasing the bargaining range and
increasing the likelihood that there are peace agreements both sides
prefer to fighting. Intergroup contact does this first by highlighting
the material benefits of cooperation. Few members of groups in conflict
are likely to have benefited from working with the outgroup, so they may
associate the outgroup with undesirable outcomes. When groups achieve a
goal through cooperation that is mutually beneficial to both, it adds
previously unforeseen benefits to cooperation. The prospect of benefits
through cooperation can motivate individuals to develop more positive
attitudes towards cooperation (Grady 2020a; Rohner, Thoenig, and
Zilibotti 2013). If cooperation is beneficial to the group, the group
may develop norms that encourage cooperation {[}Axelrod (1986); chris:
cite?{]}.

Along with increasing the perceived value of peace, intergroup contact
can also remove benefits to fighting for group members who gain
psychological utility or enjoy social rewards from harming the outgroup.
Intergroup contact humanizes and creates positive attitudes towards the
other side (Pettigrew 1998; Pettigrew and Tropp 2006), and group members
only gain utility from harming the outgroup if the outgroup is viewed
negatively and without empathy (Böhm, Rusch, and Gürerk 2016; Weisel and
Böhm 2015; Cikara et al. 2014). Likewise, groups norms that promote
cooperation will give social punishments, not social rewards, for
aggressive action towards the outgroup. When the outgroup is not hated,
harming them loses its luster.

As well as increasing the utility of cooperation relative to fighting,
for intergroup contact to help resolve group conflict it must help solve
information and commitment problems. Intergroup contact can help solve
these problems by assuring each group that the other side prefer
cooperation to fighting. It is not enough that each group prefers
cooperation, they must have the opportunity to signal that preference to
the other side, and the other side must accept the signal. Without this
shared knowledge, groups may remain in conflict because of information
and/or commitment problems.

The second way that intergroup contact helps reduce conflict is by
reducing perceptual biases that prevent groups from accurately
perceiving the other side's preferences and building trust. Removing
these perceptual biases helps solve information and commitment problems.
Intergroup contact reduces perceptual biases by dispelling stereotypes,
reducing feeling of threat and anxiety, engendering feelings of empathy,
and making group commonalities salient (Allport 1954; Pettigrew and
Tropp 2008; Page-Gould, Mendoza-Denton, and Tropp 2008; Batson et al.
1997; Broockman and Kalla 2016; Gaertner et al. 1993). By removing these
causes of perceptual biases, groups are more able to form accurate views
of each other's preferences. Without stereotypes and feelings of threat,
and with empathy and group commonalities, group members will not be
motivated to see the outgroup negatively or feel dissonance after
positive experiences with outgroup members. Groups are also less likely
to reactively devalue each other's concessions and more likely to have
empathy for each other's positions. Through these mechanisms, contact
builds trust even between members of conflicting groups (Grady 2020b;
Hewstone et al. 2006).

By debiasing perceptions, each group may identify that it is in each
group's interest to cooperate with each other. All of the mechanisms
above -- reducing stereotypes and threat, increasing empathy -- also
increase the likelihood that group members perceive cooperation to be in
their interest and in the interest of the other side. It is unlikely
that one side will expect cross-group interaction to be in their group's
interest if that side fears the outgroup and holds negative stereotypes
about the outgroup's work ethic and honesty. It is equally unlikely that
a group will expect the other side to cooperate with them if the other
side believes them to be lazy and dishonest. Likewise, achieving a joint
goal through cooperation reassures each group that the other side also
prefers cooperation. If each side prefers cooperation but neither side
is aware of the other side's preferences, groups may not cooperate to
avoid the costs of being betrayed. Each side can trust the other to
engage in cooperative behavior when both side's know that it is in each
side's interest to do so (Gambetta and others 2000). Removing these
perceptual biases helps each side see that cooperation is in the
interest of their group \emph{and} of the other group. It takes two to
mutually cooperate.

Along with increasing the utility of cooperation and debiasing
perceptions, the third way that intergroup contact can help resolve
conflict is by offering opportunities for costly signaling. Intergroup
contact allows groups members to learn about each other based on
personal experience, interacting and communicating directly. Direct
cross-group interaction allows group members to signal their
trustworthiness through communication with and behavior directly
observed by the other side. This type of direct communication reduces
competition and helps solve commitment problems in behavioral games like
the Prisoner's Dilemma (Bornstein et al. 1989; Ostrom 2006) and in
formal models (Rohner, Thoenig, and Zilibotti 2013). These interactions
can serve as confidence-building measures and allow groups to start
small and low-risk and gradually increasing as groups build trust over
time; trust is one of the few resources that increases with use
(Gambetta and others 2000). Importantly, intergroup contact gives each
group the opportunity to signal willingness to punish their own members
if those members jeopardize peace (Fearon and Laitin 1996).

In this section, I showed how the identity perspective works with the
bargaining perspective to explain group conflict. I also demonstrated
how a solution proposed by the identity perspective, intergroup contact,
works in the bargaining perspective. Intergroup contact can provide
incentives for cooperation, reduce perceptual biases, and provide
opportunities for costly signals of trustworthiness. Through incentives
for cooperation, it can give each side a preference for peace over
fighting. By reducing perceptual biases and providing opportunities for
costly signaling, contact can help solve information and commitment
problems. In the next section, I apply this joined perspective to
farmer-pastoralist conflict in Nigeria and describe the form that
institutions to improve intergroup cooperation have taken in that
context.

\hypertarget{application-farmer-pastoralist-conflict}{%
\section{Application: Farmer-Pastoralist
Conflict}\label{application-farmer-pastoralist-conflict}}

The conflict between farmers and pastoralists in Nigeria looks like a
textbook case of a commitment problem preventing peace. The groups
maintain complementary ways of life, making cooperation beneficial for
both. Pastoralists have an excess of protein in the form of meat, milk,
and other animal products, but they grow little in the way of grains,
tubers and vegetables; the farmers have an excess of grains, tubers, and
vegetables but they own few animals and have limited access to animal
products.\footnote{This is not to say that farmers have no independent
  access to protein and pastoralists have no independent access to
  grain. Farming villages typically stock chickens for eggs and meat as
  their main protein source, with a few goats and sheep. They do not
  have excess food year round to support large animals like cows. Most
  pastoralists are semi-migratory and few stay in one place long enough
  to cultivate crops. Those who stay in a ``home base'' will set aside
  some land for rice or yams, but most of the land is left for cattle to
  graze.} Farmers also want animals to graze on their lands after
harvest season to replenish the soil with animal waste, and pastoralists
want to graze their animals on crop residue (stalks, leaves, seed pods,
and other inedible parts of the plant) that is left on the fields after
harvest.

By all accounts, farmers and pastoralists benefited from their
complementarity for several generations (Tonah 2002; ``Herders Against
Farmers: Nigeria's Expanding Deadly Conflict'' 2017; Thébaud and
Batterbury 2001). There were, of course, disagreements between sedentary
farmers and mobile pastoralists, but their relationship was
characterized more by harmony than conflict. Recent decades, however,
have brought farmers and pastoralists into conflict. Historically,
farming was more common in southern Nigeria and pastoralism more common
in the north, but the two ways of life increasingly overlap
geographically. Farmers have moved north into marginal agricultural
lands due to the increasing food needs of Nigeria's booming population,
which grew from 50 million at Independence in 1960 to 200 million today
(Abbass 2012; Kuusaana and Bukari 2015). At the same time, pastoralists
have been pushed further south by the expansion of the Sahara, which
brought them to higher population density areas (Thomas and Nigam 2018;
Okpara et al. 2015). Less land and more people who depend on the land is
a recipe for conflict over land and resources.

Farmer-pastoralist conflict has exploded in recent years. The most
recent conflict escalation caused 7,000 deaths from 2014-2019 and
displaced hundreds of thousands of people from their homes (Harwood
2019; Daniel 2018; Ilo, Ier, and Adamolekun 2019; Akinwotu 2018). The
scale of economic damage is unknown, but farmer-pastoralist conflict
\emph{before} this escalation cost Nigeria \$13 billion annually in lost
economic productivity (McDougal et al. 2015). This violence has also
impeded food production, leading to an impending food crisis (Ilo, Ier,
and Adamolekun 2019; Hailemariam 2018; Unah 2018).

The proximate causes of violence are farmers sowing seeds on
pastoralists' grazing lands and pastoralists grazing their cattle on
farmers' crops. If either side retaliates -- a farmer by stealing cattle
from the pastoralists' herds, a pastoralist by grazing on more farmland
-- the scope of the conflict can rapidly expand. The farmer whose crops
were destroyed by cattle does not know which herd grazed on his land;
cattle he steals in revenge do not necessarily come from the
transgressing herd. Pastoralists, likewise, do not know which farmer
stole their cattle; the crops they destroy in revenge do not necessarily
come from the transgressing farm. From there, a little war often breaks
out. As one reporter noted, ``The countryside is littered with the
charred ruins of homes, schools, police stations, mosques and
churches.'' (McDonnel 2017). In one case I witnessed, a farmer took
revenge against cattle grazing on his farmland by poisoning the crop
residue left on his fields after harvest. After grazing on the residue,
the cattle of dozens of pastoralists became sick and died. More violence
followed.

The land conflict is exacerbated by ethnic and religious differences
between groups, which feeds mutual distrust. The pastoralists are almost
all from the Fulani ethnic group and practice Islam; the farmers are
from a non-Fulani ethnic groups and practice Islam and Christianity,
though the violence is worst where the farmers are homogeneously
Christian. Each group sees the other as biased towards their own side
for economic, cultural, and religious reasons. Each group also sees
their way of life as superior. Farmers see nomadic life as outdated,
backwards, and anti-progress; the pastoralists think that sedentary
farming makes one weak. One pastoralist commented to me that if he
dropped off a sedentary adult and pastoralist child in the forest, the
sedentary adult would depend on the child to survive.

Despite their cultural differences and competition for scarce land,
mutual complementarity remains. Pastoralists still have animal products
-- though more farmers have bought animals in recent years, pastoralists
still control roughly 90\% of Nigeria's livestock (``Herders Against
Farmers: Nigeria's Expanding Deadly Conflict'' 2017) -- and farmers
still have tubers, vegetables, grains, and the resultant crop residue.
The violence is extremely costly to both sides, so both have an
incentive to avoid conflict. Community leaders recognize that peace is
in the interest of their communities, but many have been unable to
prevent the violence. In interviews conducted in 2016 and 2019 by Grady
(2020b), community leaders from farming and pastoral communities
expressed their desire for peace between the two groups and blamed
deviants from the other side for ongoing violence. Farmers argue that
the local pastoralists do not prevent other pastoralists who migrate
through from destroying cropland, and pastoralists argue that farmers on
the outskirts of the farming village encroach on grazing routes more
each year.

Farmer and pastoralists groups lack the trust needed to make an
agreement, and the roots of mistrust -- deviants from both sides --
highlights that the mistrust is driven by a within-group collective
action problem, not a general mistrust of all outgroup members. Peace
benefits all members of a group, but contributing to peace is costly for
group members because they must refrain from engaging in behaviors that
benefit themselves at the expense of the outgroup. Achieving peace, as
with any public good, requires overcoming this collective action
problem: the group must compel group members to contribute to peace
despite the members' individual incentive to shirk and rely on others to
bear its cost. One group trusting the other to honor an agreement, then,
means not only trusting that a high proportion of outgroup members
desire peace, but also that those outgroup members can and will compel
less cooperative group members to honor a peace agreement (Fearon and
Laitin 1996). Peace can be derailed by a few radicals who do not reflect
the preferences of most group members (Sambanis and Shayo 2013; De
Sanctis and Galla 2009).

Peace between many farming and pastoral groups is prevented by a lack of
trust (i.e.~a commitment problem). Both sides are better off cooperating
peacefully than fighting, but neither side trusts the other to honor
agreements that would prevent fighting. The reasons each group does not
trust the other side, however, is not that they believe the other side
as a whole will defect on peace agreements, but rather that the farming
group cannot credibly commit to preventing all farmers from expanding
into grazing lands, and the pastoral group cannot credibly commit to
preventing all herders from grazing cattle on croplands. Their
between-group commitment problem is driven by a within-group collective
action problem because neither group trusts the other to punish their
own.

\hypertarget{resolving-farmer-pastoralist-conflict}{%
\subsection{Resolving farmer-pastoralist
conflict}\label{resolving-farmer-pastoralist-conflict}}

Though conflict between farming and pastoral groups is common, not all
farming and pastoral groups are in conflict. Many communities overcame
their trust problems and enjoy the complementarity of farming and
pastoral lifestyles. Communities that successfully navigated
farmer-pastoralist tension generally have a noticeable commonality: an
institution comprised of farmers and pastoralists to handle cases that
threaten intergroup relations. This institution can tap into ingroup
networks to identify transgressors from each side, much like an ingroup
policing institution (Fearon and Laitin 1996). In addition to the
benefit of ingroup networks, these joint institutions have another
benefit: since both groups are represented, neither the plaintiff, the
defendant, nor their group can credibly claim that the case was decided
unfairly due to group bias. The collaborative nature of the institution
helps curtail misperceptions, which can derail cooperative equilibria
(Jervis 2017; Wu and Axelrod 1995; Bendor, Kramer, and Stout 1991).

These joint institutions function much like trial by jury.
Representatives from each group meet, hear from the case's ``plaintiff''
and ``defendant'', and decide an appropriate punishment. For common
occurrences -- crop damage, cattle rustling, and the like -- there may
also be a code of laws: predecided penalties, such as a set cost per
acre of cropland destroyed or head of cattle stolen. In past decades,
agreements between the traditional leaders of each community function
similarly, by setting compensation for specific actions. Conflict is
often blamed on the breakdown of such institutions (Tonah 2002; Cotula
et al. 2004; Kuusaana and Bukari 2015).

These joint institutions are themselves a peace agreement. Each group
has conceded some local autonomy and agreed to an overarching legal
framework that will govern both farmers and pastoralists. To do so, the
groups must agree on several issues, like who serves on the joint
institution, how they decide punishment, what is the range of acceptable
punishments, and how punishments are enforced. This type of agreement is
only possible if both groups share a baseline level of trust. Each group
must also identify that the issue causing conflict between the two sides
is a commitment problem, not a preferences problem. Before an agreement
like joint punishment institutions can be reached, the groups first need
to learn enough about each other's preferences to establish that each
prefers mutual cooperation to mutual fighting. Put another way, groups
need to solve their information problem before they can use joint
punishment institutions to solve their commitment problem. A key
challenge, then, is for groups to signal their preferences. Intergroup
contact offers a means through which groups can send and receive those
signals.

Grady (2020b) evaluated a contact-based intervention to reduce conflict
between farmers and pastoralists. Though they do not measure the extent
to which the intervention affected institutional structures that govern
intergroup relations, they describe farmers and pastoralists who had
participated in the contact intervention meeting to jointly decide an
appropriate punishment for vigilante farmers who intended to harm
pastoralists. The would-be perpetrators, hailing from a nearby village,
were noticed and arrested by the hometown farmers while on their way to
steal the pastoralists' cattle. Rather than decide unilaterally how to
punish the vigilantes, leaders from the farming community called in
leaders from the pastoral community to jointly decide a punishment.
Together the groups decided to disarm the vigilantes and let them go
free without further punishment -- a solution proposed by the
\emph{pastoralists}. Had the farmers unilaterally decided to let the
vigilantes go free, the pastoralists may have interpreted the punishment
as too lenient and accused the farmers of bias. Since the pastoralists
had a say in the decision-making process, however, the group's were able
to build trust through cooperation. The pastoralists appreciated the
farmers calling a joint meeting, and the farmers appreciated the
pastoralists magnanimity in proposing a lenient punishment.

Grady (2020b) also describe a contrasting situation in which a farmers
and pastoralists who had not participated in the contact intervention
failed due to the lack of any joint institutions. In that situation,
both sides were participating in a public goods game in which money
raised would be jointly administered by both groups. These groups had no
preexisting structure to handle situations that concerned both sides,
and neither side trusted the other to hold the money. The pastoralists
claimed that the money would be squandered by the corruption of the
farming community's leader if a farmer held it; the farmers claimed that
the pastoralists would migrate away with the money if a pastoralist held
it. In the end, the groups agreed that the NGO administering the public
goods game would hold the money and disburse it in chunks when the
pastoralist leader and the farming leader agreed on its use.

In this section, I discussed farmer-pastoralist conflict in Nigeria,
focusing on causes of violence and solutions that groups developed to
curtail violence. In the next section I outline several policy
implications that follow from the farmer-pastoralist example and from
thinking about group conflict in the bargaining and identity
perspectives.

\hypertarget{policy-implications-for-peacebuilding-programs}{%
\section{Policy Implications for Peacebuilding
Programs}\label{policy-implications-for-peacebuilding-programs}}

I will discuss six policy implications from farmer-pastoralist conflict
and the bargaining and identity perspectives. I also discuss some
consequences of these implications. The implications are: (1) peace is
not intrinsically preferable to fighting; (2) preferences for
cooperation over fighting to not guarantee peace; (3) groups are
collections of individuals, not unitary actors; (4) Misperceptions are a
major threat to peace; (5) there is not one solution to group conflict;
and (6) some attempts to alleviate group conflict can backfire.

The first policy implication is that peace is not intrinsically
preferable to fighting. Some groups fight because the group members want
to fight. No information or signaling will reduce conflict in these
cases because the groups' preferences are misaligned. In those cases,
the first step to peace is each side coming to believe that cooperation
is superior to peace.

The best chance for groups to identify that cooperation is in their
material interest are distinct group-based comparative advantages.
Farmers and pastoralists each have group-based comparative advantages --
farmers in produce, pastoralists in animals -- that make identifying a
shared interest easy. Though cultural differences may seem like a
hindrance to peace, these groups have much to gain from cooperation. Two
groups of farmers have little to offer each other that the group's could
not obtain on their own, so there is less room for complementarity. In
general, increasing gains from cross-group interaction, and therefore
increasing the opportunity cost of fighting, is most likely to deter
conflict (Rohner, Thoenig, and Zilibotti 2013).

Material interest is not everything, however, and many motivators for
preferences are psychological and social. Groups may prefer fighting due
to psychological and social rewards and penalties. Group members may
resist peace if it sacrifices self-esteem and self-image. Some segment
of each group may embrace fighting because it offers them the
opportunity to garner social rewards. These costs and benefits can make
it so that there is no peace agreement each sides prefers to fighting.
Changing attitudes and norms can reduce the costs of peace and the
benefits of fighting. Attitude and norm change can potentially be
motivated by showing the groups that they can better accomplish their
goals through cooperation than conflict (Grady 2020a).

The second policy implication is that preferences for cooperation over
fighting do not necessarily end group conflict. Even if both groups
prefers mutual cooperation to mutual defection, lack of information
about the other side's preferences and an inability to commit to
cooperate can prevent peace. Cooperation is only in a group's interest
if the other group also plans to cooperate. Policies that promote peace
must therefore assure each side that the other will cooperate. The
easiest way to assure each side that the other will cooperate is to
impose external punishment for defection or external reward for
cooperation. That method of assurance, however, is not available in many
contexts and does not build the intergroup trust necessary for
cooperation once the external punishment mechanism is removed. It may
nonetheless serve as a short-term solution to reducing group conflict
and as a starting point for group cooperation.

In the absence of external rewards and punishment, intergroup contact in
which groups work together to achieve a joint goal can show both groups
that cooperation is in their interest and that the other side also sees
that. Achieving a joint goal is good for both groups, and both groups
see their group and the other side benefiting from cooperation, which
helps each side update about the other's preferences -- the other side
can be trusted to do what is in their own interest. Intergroup contact
also brings group members into close personal contact, which offers
opportunities for learning and costly signaling.

These first two policy implications suggest that contact-based programs
where the groups do not work towards a joint goal, like discussion
forums, may have no effect on group conflict. Discussion forums do not
directly change preferences; nor do they provide credible information
about the other side's preferences. It is possible that this minimal
form of contact could reduce perceptual biases by exposing each group to
the other's view points, but they mainly seem to be avenues for ``cheap
talk'' which both sides will discount.

The third policy implication is that groups are collections of group
members and the preferences of the group should not be taken as the
preferences of each group member. Group members preferences are likely
correlated with the preferences of the amorphous ``group'', but they
also deviate in important ways. For example, the group as a unitary
actor does not face a collective action problem, but the group as a
collection of individuals does because each individual wants to rely on
others to provide public goods. An important attribute of a group, then,
is its capacity to compel group members to contribute to the public good
of peace. Group's with the power to compel will be better at fighting
because they can compel group members to participate in violence
(Bornstein 2003) but should also be more able to credibly commit to
preventing their own from defecting.

Thinking of the group as a collection of group members also highlights
(1) the psychological and social factors that make conflict attractive
and (2) the perceptual and cognitive biases that color groups views of
each other. These type of individual-based factors are not readily
apparent when thinking of the groups as homogeneous units, but they
likely contribute to group conflict. They may also systematically cause
some group members to benefit more from peace than others, so peace as a
public good is not equally valuable or desirable to all group members.
The group is a useful but limited unit of analysis because it can only
influence group conflict through its effect on group members.

The fourth policy implication is that misperceptions are a major threat
to peace. It is easy to focus on changes to payoff structure and changes
to the value of resources under dispute as the driver of conflict. In
the farmer-pastoralist example, we readily blame population booms and
land scarcity for the recent surge in violence, and these factors no
doubt precipitated that surge. However, they do not necessarily result
in group conflict unless they reduce the bargaining range to zero. They
may instead cause conflict by creating the misperception that the other
side now prefers fighting for the chance to control all of the land to
cooperating to share the land between the two groups.

Misperceptions are also a significant cause of information and
commitment problems. If groups desire peace, the perception that the
other side does not desire peace can prevent cooperation. Once groups
are cooperating, misperceiving cooperation as defection can cause a
series of defections. Any institution that limits such misperceptions
should help maintain peace.

The fifth policy implication is that there are many ways to resolve
group conflict, so practitioners should not be wed to one ideological
idea about reducing group conflict. There are many ways that groups'
preferences can be aligned, credible information can be shared, and
credible commitments to specific behaviors can be made. Means to
accomplish each of these goals are likely to differ across contexts and
depend on characteristics of the group conflict. Intergroup contact may
not work everywhere. By being wed to one ideological approach instead of
assessing the situation, likely fail.

The sixth policy implication is that attempts to resolve group conflict
can backfire and cause or exacerbate group conflict. Contact programs,
for example, could easily be mismanaged in such a way that works against
cooperation. First, the groups may fail to achieve the goal, which could
cause scapegoating and reinforce the perspective that cooperation with
the other side is a waste of time. Second, the joint goal may not
benefit both sides equally. Differing gains can \emph{cause} a
commitment problem and violent conflict because the group gaining less
has an incentive to fight \emph{now} before the other side gets
stronger. The side gaining more cannot credibly commit to a deal based
on today's balance of power when their power will increase tomorrow.
Third, contact might be sabotaged by not involving influential members
of each group or by involving members of each group who have the least
to gain from peace.

Programs that incentivize cooperation by imposing external punishment or
reward can also backfire. Third party punishment could undermine each
group learning that the other prefers peace to fighting because each
group's cooperation might be due to fear of punishment that disappears
when the third party leaves. Any program imposing external carrots and
sticks for cooperation should combine that approach with a way for
group's to benefit from cooperation in a way that does not depend on the
external incentives.

Programs that do not explicitly seek to reduce group conflict can also
cause of exacerbate group conflict. First, any program that adds
resources to an area afflicted by ongoing conflict runs the risks of
those resources being captured by the conflicting groups (Findley et al.
2011). Second, programs designed to increase economic well-being and
increase the value of land can make land worth fighting for; they can
also upset traditional agreements that govern punishments for destroying
crops, as in the farmer-pastoralist case. And third, aid programs can
crowd out local employment opportunities and local institutions for
conflict management, which will be needed when the aid program ends.

\hypertarget{avenues-for-future-research}{%
\section{Avenues for Future
Research}\label{avenues-for-future-research}}

This paper sought to show how combining two perspectives on group
conflict, the bargaining perspective and the identity perspective, can
provide a better understanding of the causes and consequences of group
conflict. Doing so reveals several avenues for future research. The
first avenue is to synthesize models of group conflict from the numerous
disciplines that study it. Group conflict is an interdisciplinary
research topic with scholarship in political science (Lopez and Johnson
2017), economics (Kimbrough, Laughren, and Sheremeta 2017), psychology
(Böhm, Rusch, and Baron 2018), management science (McCarter et al.
2018), biology (Rusch and Gavrilets 2017), and evolutionary anthropology
(Glowacki, Wilson, and Wrangham 2017). Each discipline emphasizes
different causes and consequences of group conflict. A synthesis of
perspectives from these distinct viewpoints could provide insights into
causes of conflict and more effective methods to reduce it.

The second avenue is to further consider the role that collective action
problems play in group conflict. As we saw with farmer-pastoralist
conflict and as has been written about by Fearon and Laitin (1996) and
Sambanis and Shayo (2013), among others, peace can be derailed by a
small set of group members who refuse to ``play ball'' with the other
side. Without the ability to compel cooperative behavior from group
members, groups cannot credibly commit to honor negotiated agreements.
This points to within-group collective action problems as the driver of
some between-group commitment problems. Peace is a public good: group
members must forego behaviors that benefit them individually for peace
to be achieved, but once achieved all group members enjoy its benefits
whether or not they bore its costs.

Overcoming collective action problems is notoriously difficult. In the
case of intergroup conflict, collective action problems may be even more
difficult to solve because costs must be born before benefits are
obtained, and group members may be killed in the meantime and thus never
enjoy the benefits of their sacrifice. Individuals are notoriously bad
even at saving for retirement, where they can sacrifice now to benefit
themselves later in a low-risk environment (Benartzi and Thaler 2013;
Warner and Pleeter 2001). Individuals may rationally discount future
payoffs quite significantly in conflict settings where individuals may
never see the future and the benefits accrue to a group.

Bornstein (Bornstein (1992) and Bornstein (2003)) was among the first to
consider how groups dealt with the preferences of individual group
members in the context of group conflict. His focus, and the focus of
most subsequent scholarship, was how groups encourage group members to
participate in violence against the other side, since winning the fight
benefits all group members but only group members who fight bear the
costs. Most literature about collective action problems in conflict is
thus concerned with each group's attempt to compel group members to
\emph{fight} the other side or resist external pressure; relatively
little has been written about each group's attempt to compel group
members to \emph{cooperate} with the other side (Keefer 2012; Kaplan
2010; Rubin 2020).

Implicit in the discussion of collective action problems is that groups
are made up of individuals with preferences that differ from the group.
Following that logic, war may not be equally costly to all members of
society, and peace not equally beneficial. The public good of peace is
more beneficial for some group members than for others. In the case of
farmers and pastoralists, farmers on the outskirts of the community bear
the brunt of ``defections'' during peace time, so peace may not be more
costly than war for them. If group elites bear few costs from conflict
and may suffer in prestige from peace, they could resist peace
agreements. Future scholarship should consider who bears the costs of
conflict and who stands to benefit from peace.

A third avenue for future research is to study the manner in which
groups solve their own commitment problems, with an eye to the
within-group collective problem as the driver of the between-group
commitment problem. Fearon and Laitin (1996) showed that ingroup
policing, an institution wherein groups punish their own miscreants, can
help groups avoid an escalation of violence. They credit the relative
paucity of group conflict to such institutions. Ingroup policing is an
excellent example of within-group institutions that prevent violence
from escalating and for maintaining peace in many contexts.

Consideration of perceptual biases and mistrust between groups in or
with a recent history of conflict indicates a potential limitation of
ingroup policing. With ingroup policing, groups are restrained from
overly lenient punishment of their own people by the threat of violence
escalating if either group suspects the other of acting in bad faith.
Since ingroup policing depends on the perceptions of group members, if
those perceptions are biased such that members of each group suspect the
other is trying to take advantage of the ingroup policing institution,
then the institution could break down as each group does not trust the
other to allocate sufficient punishment to their own side. The risk of
cooperative behavior being misperceived as uncooperative behavior is one
of the main threats to enduring peace (Jervis 2017; Wu and Axelrod 1995;
Bendor, Kramer, and Stout 1991). Groups could account for the biases of
the other side and harshly punish their own to overcome the bias, but
overly harsh punishments could reduce the legitimacy of ingroup policing
among the ingroup. Future research should examine how misperceptions
caused by biases like motivated reasoning and reactive devaluation
affect the ingroup policing equilibrium.

If ingroup policing is unsustainable due to psychological biases, future
research should also consider how to modify that institution to minimize
and manage the misperceptions that result from those biases. A starting
point could be the collaborative ``joint-punishment'' institutions used
by some farming and pastoral groups to control their conflict. These
institutions tap into ingroup networks to identify wrongdoers, like
ingroup policing, but involve both sides in the punishment decisions to
increase transparency. They may also open up more lenient punishments,
as neither group has to worry that the other will misperceive a lenient
punishment as defecting from the agreement. Compared to ingroup
policing, however, a joint-punishment institution is likely more
difficult to initiate. Ingroup policing can be implemented by each group
alone, whereas a joint structure requires many agreements about the
structure's composition, decision-rules, and enforcement capacity.

The fourth avenue for future research is to use the bargaining and
identity perspectives to develop mechanisms through which conflict can
be resolved, and then to test those mechanisms in lab and field
experiments. One promising method to reduce conflict is intergroup
contact, but, to my knowledge, no contact-based intervention has
explicitly considered contact as a method to solve information and
commitment problems. Contact programs are beginning to be tested between
groups in or with a history of conflict, including Nigerian farmers and
pastoralists (Grady 2020b), Christians and Muslims in Iraq (Mousa 2018),
and Jewish and Arab-Palestinian youth in Israel (Ditlmann and Samii
2016), but these programs have thus far focused more on causal inference
(does contact work in these contexts) than causal mechanisms (how does
contact work in these contexts). Contact-based interventions would
benefit from considering how contact solves information and commitment
problems that cause conflict.

Based on the bargaining and identity perspective, intergroup contact may
reduce group conflict through the achievement of a joint goal. From this
viewpoint, the key mechanism is goal achievement and without it contact
will not improve attitudes or reduce conflict. Traditionally, goal
achievement is thought to ``further {[}the{]} process'' of attitude
change, but is not necessarily central to it (Pettigrew 1998, 66). To my
knowledge, only one lab experiment has tested the effect of intergroup
contact when it does or does not achieve a joint goal (Grady 2020a).
More experiments should test this mechanism with different types of
group dynamics and in more contexts.

Another mechanism to reduce group conflict is to increase the utility of
peace, be it materially (i.e.~financial benefit) or non-materially
(i.e.~alignment with values). Increasing peace's utility can solve the
preference problem if one exists, but it could also solve the
information problem if each side believes the preferences of the other
group have changed. I suspect increasing peace's material utility is
best for solving information problems because each side is more likely
to believe the other is motivated by material outcomes than non-material
outcomes (``If they had those values, why are they fighting us now?'').
Future research should test the efficacy of different means to increase
the utility of peace.

Interventions that reduce psychological and cognitive biases against the
outgroup should also help groups solve information and commitment
problems. Along with testing the efficacy of reducing these biases,
further research should test the best means through which such biases
can be reduced. Intergroup contact is one method, but it may be
difficult to implement in very violent contexts or in contexts with
strict segregation. Vicarious intergroup contact -- contact in which
members of each side observe ingroup members interacting with outgroup
members -- could reduce biases in these contexts.

The last avenue for future research I will discuss is how group dynamics
moderate the effect of interventions meant to reduce group conflict.
Fearon and Laitin (1996) noted that group size changes effective
strategies for deterring group conflict: large groups can use the threat
of intergroup conflict to incentivize the cooperation of a small group,
and small groups use ingroup policing to avoid the wrath of the larger
group. Size and power disparities might influence conflict-reduction
interventions in other ways. The smaller group, for example, is likely
to interact more frequently with the larger group than the larger group
interacts with the smaller group. Contact-based interventions may
therefore have little effect on members of the smaller group, who have
much personal experience with members of the larger group. Group
dynamics other than size and power disparities could similarly change
strategies to reduce group conflict, such as the homogeneity and
hierarchy of each group and the centralization or decentralization of
power within each group. Future research should investigate how these
group dynamics affect the types of institutions that groups develop to
alleviate conflict, and should determine which type of interventions are
most effective in each context.

\hypertarget{references}{%
\section*{References}\label{references}}
\addcontentsline{toc}{section}{References}

\hypertarget{refs}{}
\leavevmode\hypertarget{ref-abbass2012no}{}%
Abbass, Isah Mohammed. 2012. ``No Retreat No Surrender: Conflict for
Survival Between Fulani Pastoralists and Farmers in Northern Nigeria.''
\emph{European Scientific Journal} 8 (1): 331--46.

\leavevmode\hypertarget{ref-nyt2018nigeria}{}%
Akinwotu, Emmanuel. 2018. ``Nigeria's Farmers and Herders Fight a Deadly
Battle for Scarce Resources.'' \emph{New York Times}.
\url{https://www.nytimes.com/2018/06/25/world/africa/nigeria-herders-farmers.html}.

\leavevmode\hypertarget{ref-allport1954prejudice}{}%
Allport, Gordon. 1954. ``The Nature of Prejudice.'' \emph{Garden City,
NJ Anchor}.

\leavevmode\hypertarget{ref-arkes1985psychology}{}%
Arkes, Hal R, and Catherine Blumer. 1985. ``The Psychology of Sunk
Cost.'' \emph{Organizational Behavior and Human Decision Processes} 35
(1): 124--40.

\leavevmode\hypertarget{ref-axelrod1986evolutionary}{}%
Axelrod, Robert. 1986. ``An Evolutionary Approach to Norms.''
\emph{American Political Science Review} 80 (4): 1095--1111.

\leavevmode\hypertarget{ref-axelrod1981evolution}{}%
Axelrod, Robert, and William D Hamilton. 1981. ``The Evolution of
Cooperation.'' \emph{Science} 211 (4489): 1390--6.

\leavevmode\hypertarget{ref-bandura1999moral}{}%
Bandura, Albert. 1999. ``Moral Disengagement in the Perpetration of
Inhumanities.'' \emph{Personality and Social Psychology Review} 3 (3):
193--209.

\leavevmode\hypertarget{ref-bandura2014social}{}%
---------. 2014. ``Social Cognitive Theory of Moral Thought and
Action.'' In \emph{Handbook of Moral Behavior and Development}, 69--128.
Psychology Press.

\leavevmode\hypertarget{ref-bar2007sociopsychological}{}%
Bar-Tal, Daniel. 2007. ``Sociopsychological Foundations of Intractable
Conflicts.'' \emph{American Behavioral Scientist} 50 (11): 1430--53.

\leavevmode\hypertarget{ref-batson1997empathy}{}%
Batson, C Daniel, Marina P Polycarpou, Eddie Harmon-Jones, Heidi J
Imhoff, Erin C Mitchener, Lori L Bednar, Tricia R Klein, and Lori
Highberger. 1997. ``Empathy and Attitudes: Can Feeling for a Member of a
Stigmatized Group Improve Feelings Toward the Group?'' \emph{Journal of
Personality and Social Psychology} 72 (1): 105.

\leavevmode\hypertarget{ref-beardsley2008agreement}{}%
Beardsley, Kyle. 2008. ``Agreement Without Peace? International
Mediation and Time Inconsistency Problems.'' \emph{American Journal of
Political Science} 52 (4): 723--40.

\leavevmode\hypertarget{ref-beber2012international}{}%
Beber, Bernd. 2012. ``International Mediation, Selection Effects, and
the Question of Bias.'' \emph{Conflict Management and Peace Science} 29
(4): 397--424.

\leavevmode\hypertarget{ref-benartzi2013behavioral}{}%
Benartzi, Shlomo, and Richard H Thaler. 2013. ``Behavioral Economics and
the Retirement Savings Crisis.'' \emph{Science} 339 (6124): 1152--3.

\leavevmode\hypertarget{ref-bendor1991doubt}{}%
Bendor, Jonathan, Roderick M Kramer, and Suzanne Stout. 1991. ``When in
Doubt... Cooperation in a Noisy Prisoner's Dilemma.'' \emph{Journal of
Conflict Resolution} 35 (4): 691--719.

\leavevmode\hypertarget{ref-bornstein1992free}{}%
Bornstein, Gary. 1992. ``The Free-Rider Problem in Intergroup Conflicts
over Step-Level and Continuous Public Goods.'' \emph{Journal of
Personality and Social Psychology} 62 (4): 597.

\leavevmode\hypertarget{ref-bornstein2003intergroup}{}%
---------. 2003. ``Intergroup Conflict: Individual, Group, and
Collective Interests.'' \emph{Personality and Social Psychology Review}
7 (2): 129--45.

\leavevmode\hypertarget{ref-bornstein1989within}{}%
Bornstein, Gary, Amnon Rapoport, Lucia Kerpel, and Tani Katz. 1989.
``Within-and Between-Group Communication in Intergroup Competition for
Public Goods.'' \emph{Journal of Experimental Social Psychology} 25 (5):
422--36.

\leavevmode\hypertarget{ref-bohm2018psychology}{}%
Böhm, Robert, Hannes Rusch, and Jonathan Baron. 2018. ``The Psychology
of Intergroup Conflict: A Review of Theories and Measures.''
\emph{Journal of Economic Behavior \& Organization}.

\leavevmode\hypertarget{ref-bohm2016makes}{}%
Böhm, Robert, Hannes Rusch, and Özgür Gürerk. 2016. ``What Makes People
Go to War? Defensive Intentions Motivate Retaliatory and Preemptive
Intergroup Aggression.'' \emph{Evolution and Human Behavior} 37 (1):
29--34.

\leavevmode\hypertarget{ref-broockman2016durably}{}%
Broockman, David, and Joshua Kalla. 2016. ``Durably Reducing
Transphobia: A Field Experiment on Door-to-Door Canvassing.''
\emph{Science} 352 (6282): 220--24.

\leavevmode\hypertarget{ref-brown2019group}{}%
Brown, Rupert, and Samuel Pehrson. 2019. \emph{Group Processes: Dynamics
Within and Between Groups}. John Wiley \& Sons.

\leavevmode\hypertarget{ref-bryan2019values}{}%
Bryan, Christopher J, David S Yeager, and Cintia P Hinojosa. 2019. ``A
Values-Alignment Intervention Protects Adolescents from the Effects of
Food Marketing.'' \emph{Nature Human Behaviour} 3 (6): 596--603.

\leavevmode\hypertarget{ref-campbell1965ethno}{}%
Campbell, Donald T. 1965. ``Ethnocentric and Other Altruistic Motives.''
In \emph{Nebraska Symposium on Motivation}, 13:283--311.

\leavevmode\hypertarget{ref-chang2013war}{}%
Chang, Yang-Ming, and Zijun Luo. 2013. ``War or Settlement: An Economic
Analysis of Conflcit with Endogenous and Increasing Destruction.''
\emph{Defence and Peace Economics} 24 (1): 23--46.

\leavevmode\hypertarget{ref-chang2017endogenous}{}%
---------. 2017. ``Endogenous Destruction in Conflict: Theory and
Extensions.'' \emph{Economic Inquiry} 55 (1): 479--500.

\leavevmode\hypertarget{ref-cikara2014their}{}%
Cikara, Mina, Emile Bruneau, Jay J Van Bavel, and Rebecca Saxe. 2014.
``Their Pain Gives Us Pleasure: How Intergroup Dynamics Shape Empathic
Failures and Counter-Empathic Responses.'' \emph{Journal of Experimental
Social Psychology} 55: 110--25.

\leavevmode\hypertarget{ref-coe2012costly}{}%
Coe, Andrew J. 2012. ``Costly Peace: A New Rationalist Explanation for
War.'' \emph{Unpublished Manuscript, University of Southern California}.

\leavevmode\hypertarget{ref-converse1970attitudes}{}%
Converse, Philip E. 1970. ``Attitudes and Non-Attitudes: Continuation of
a Dialogue.'' \emph{The Quantitative Analysis of Social Problems} 168:
189.

\leavevmode\hypertarget{ref-cotula2004land}{}%
Cotula, Lorenzo, Camilla Toulmin, Ced Hesse, and others. 2004.
\emph{Land Tenure and Administration in Africa: Lessons of Experience
and Emerging Issues}. International Institute for Environment;
Development London.

\leavevmode\hypertarget{ref-crescenzi2007reputation}{}%
Crescenzi, Mark JC. 2007. ``Reputation and Interstate Conflict.''
\emph{American Journal of Political Science} 51 (2): 382--96.

\leavevmode\hypertarget{ref-crescenzi2011supply}{}%
Crescenzi, Mark JC, Kelly M Kadera, Sara McLaughlin Mitchell, and
Clayton L Thyne. 2011. ``A Supply Side Theory of Mediation 1.''
\emph{International Studies Quarterly} 55 (4): 1069--94.

\leavevmode\hypertarget{ref-daniel2018anti}{}%
Daniel, Soni. 2018. ``Anti-Open Grazing Law: Nass, Benue, Kwara, Taraba
Tackle Defence Minister.'' \emph{Vanguard}.
\url{https://www.vanguardngr.com/2018/06/anti-open-grazing-law-nass-benue-kwara-taraba-tackle-defence-minister/}.

\leavevmode\hypertarget{ref-de2008motivated}{}%
De Dreu, Carsten KW, Bernard A Nijstad, and Daan van Knippenberg. 2008.
``Motivated Information Processing in Group Judgment and Decision
Making.'' \emph{Personality and Social Psychology Review} 12 (1):
22--49.

\leavevmode\hypertarget{ref-de2009effects}{}%
De Sanctis, Luca, and Tobias Galla. 2009. ``Effects of Noise and
Confidence Thresholds in Nominal and Metric Axelrod Dynamics of Social
Influence.'' \emph{Physical Review E} 79 (4): 046108.

\leavevmode\hypertarget{ref-di2017effectiveness}{}%
Di Salvatore, Jessica, and Andrea Ruggeri. 2017. ``Effectiveness of
Peacekeeping Operations.'' \emph{Oxford Research Encyclopedia of
Politics}.

\leavevmode\hypertarget{ref-ditlmann2016can}{}%
Ditlmann, Ruth K, and Cyrus Samii. 2016. ``Can Intergroup Contact Affect
Ingroup Dynamics? Insights from a Field Study with Jewish and
Arab-Palestinian Youth in Israel.'' \emph{Peace and Conflict: Journal of
Peace Psychology} 22 (4): 380.

\leavevmode\hypertarget{ref-doyle2000international}{}%
Doyle, Michael W, and Nicholas Sambanis. 2000. ``International
Peacebuilding: A Theoretical and Quantitative Analysis.'' \emph{American
Political Science Review} 94 (4): 779--801.

\leavevmode\hypertarget{ref-duncan1976differential}{}%
Duncan, Birt L. 1976. ``Differential Social Perception and Attribution
of Intergroup Violence: Testing the Lower Limits of Stereotyping of
Blacks.'' \emph{Journal of Personality and Social Psychology} 34 (4):
590.

\leavevmode\hypertarget{ref-fearon1994ethnic}{}%
Fearon, James D. 1994. ``Ethnic War as a Commitment Problem.'' In
\emph{Annual Meetings of the American Political Science Association},
2--5.

\leavevmode\hypertarget{ref-fearon1995rationalist}{}%
---------. 1995. ``Rationalist Explanations for War.''
\emph{International Organization} 49 (3): 379--414.

\leavevmode\hypertarget{ref-fearon1998commitment}{}%
---------. 1998. ``Commitment Problems and the Spread of Ethnic
Conflict.'' \emph{The International Spread of Ethnic Conflict} 107.

\leavevmode\hypertarget{ref-fearon1996explaining}{}%
Fearon, James D, and David D Laitin. 1996. ``Explaining Interethnic
Cooperation.'' \emph{American Political Science Review} 90 (4): 715--35.

\leavevmode\hypertarget{ref-fein1997prejudice}{}%
Fein, Steven, and Steven J Spencer. 1997. ``Prejudice as Self-Image
Maintenance: Affirming the Self Through Derogating Others.''
\emph{Journal of Personality and Social Psychology} 73 (1): 31.

\leavevmode\hypertarget{ref-festinger1962cognitiveDissonance}{}%
Festinger, Leon. 1962. \emph{A Theory of Cognitive Dissonance}. Vol. 2.
Stanford university press.

\leavevmode\hypertarget{ref-fey2010shuttle}{}%
Fey, Mark, and Kristopher W Ramsay. 2010. ``When Is Shuttle Diplomacy
Worth the Commute? Information Sharing Through Mediation.'' \emph{World
Politics} 62 (4): 529--60.

\leavevmode\hypertarget{ref-fey2011uncertainty}{}%
---------. 2011. ``Uncertainty and Incentives in Crisis Bargaining:
Game-Free Analysis of International Conflict.'' \emph{American Journal
of Political Science} 55 (1): 149--69.

\leavevmode\hypertarget{ref-findley2011localized}{}%
Findley, Michael G, Josh Powell, Daniel Strandow, and Jeff Tanner. 2011.
``The Localized Geography of Foreign Aid: A New Dataset and Application
to Violent Armed Conflict.'' \emph{World Development} 39 (11):
1995--2009.

\leavevmode\hypertarget{ref-gaertner1993common}{}%
Gaertner, Samuel L, John F Dovidio, Phyllis A Anastasio, Betty A
Bachman, and Mary C Rust. 1993. ``The Common Ingroup Identity Model:
Recategorization and the Reduction of Intergroup Bias.'' \emph{European
Review of Social Psychology} 4 (1): 1--26.

\leavevmode\hypertarget{ref-gambetta_ch13}{}%
Gambetta, Diego, and others. 2000. ``Can We Trust Trust.'' \emph{Trust:
Making and Breaking Cooperative Relations} 13: 213--37.

\leavevmode\hypertarget{ref-gartner2011signs}{}%
Gartner, Scott Sigmund. 2011. ``Signs of Trouble: Regional Organization
Mediation and Civil War Agreement Durability.'' \emph{The Journal of
Politics} 73 (2): 380--90.

\leavevmode\hypertarget{ref-glowacki2017evolutionary}{}%
Glowacki, Luke, Michael L Wilson, and Richard W Wrangham. 2017. ``The
Evolutionary Anthropology of War.'' \emph{Journal of Economic Behavior
\& Organization}.

\leavevmode\hypertarget{ref-gneezy2012conflict}{}%
Gneezy, Ayelet, and Daniel MT Fessler. 2012. ``Conflict, Sticks and
Carrots: War Increases Prosocial Punishments and Rewards.''
\emph{Proceedings of the Royal Society B: Biological Sciences} 279
(1727): 219--23.

\leavevmode\hypertarget{ref-good2000individuals}{}%
Good, David. 2000. ``Individuals, Interpersonal Relations, and Trust.''
\emph{Trust: Making and Breaking Cooperative Relations}, 31--48.

\leavevmode\hypertarget{ref-grady2020lab}{}%
Grady, Christopher. 2020a. ``Contact Itself or Contact's Success? How
Intergroup Contact Improves Attitudes.'' PhD thesis, University of
Illinois.

\leavevmode\hypertarget{ref-grady2020farmer}{}%
---------. 2020b. ``Promoting Peace Amidst Group Conflict: An Intergroup
Contact Field Experiment in Nigeria.'' PhD thesis, University of
Illinois.

\leavevmode\hypertarget{ref-gubler2013humanizing}{}%
Gubler, Joshua R. 2013. ``When Humanizing the Enemy Fails: The Role of
Dissonance and Justification in Intergroup Conflict.'' In \emph{Annual
Meeting of the American Political Science Association}.

\leavevmode\hypertarget{ref-frontera2018nigeria}{}%
Hailemariam, Adium. 2018. ``Nigeria: Violence in the Middle Belt Becomes
Major Concern for President Buhari.'' \emph{Frontera}.
\url{https://frontera.net/news/africa/nigeria-violence-in-the-middle-belt-becomes-major-concern-for-president-buhari/}.

\leavevmode\hypertarget{ref-halevy2010relative}{}%
Halevy, Nir, Eileen Y Chou, Taya R Cohen, and Gary Bornstein. 2010.
``Relative Deprivation and Intergroup Competition.'' \emph{Group
Processes \& Intergroup Relations} 13 (6): 685--700.

\leavevmode\hypertarget{ref-hartzell2001stabilizing}{}%
Hartzell, Caroline, Matthew Hoddie, and Donald Rothchild. 2001.
``Stabilizing the Peace After Civil War: An Investigation of Some Key
Variables.'' \emph{International Organization} 55 (1): 183--208.

\leavevmode\hypertarget{ref-council2019nigeria}{}%
Harwood, Asch. 2019. ``Update: The Numbers Behind Sectarian Violence in
Nigeria.'' \emph{Council on Foreign Relations}.
\url{https://www.cfr.org/blog/update-numbers-behind-sectarian-violence-nigeria}.

\leavevmode\hypertarget{ref-icg2017nigeria}{}%
``Herders Against Farmers: Nigeria's Expanding Deadly Conflict.'' 2017.
International Crisis Group.

\leavevmode\hypertarget{ref-hewstone1990ultimate}{}%
Hewstone, Miles. 1990. ``The `Ultimate Attribution Error'? A Review of
the Literature on Intergroup Causal Attribution.'' \emph{European
Journal of Social Psychology} 20 (4): 311--35.

\leavevmode\hypertarget{ref-hewstone2006intergroup}{}%
Hewstone, Miles, Ed Cairns, Alberto Voci, Juergen Hamberger, and Ulrike
Niens. 2006. ``Intergroup Contact, Forgiveness, and Experience of `the
Troubles' in Northern Ireland.'' \emph{Journal of Social Issues} 62 (1):
99--120.

\leavevmode\hypertarget{ref-horowitz2001deadly}{}%
Horowitz, Donald L. 2001. \emph{The Deadly Ethnic Riot}. Univ of
California Press.

\leavevmode\hypertarget{ref-fa2019deadly}{}%
Ilo, Udo Jude, Jonathan-Ichavar Ier, and Yemi Adamolekun. 2019. ``The
Deadliest Conflict You've Never Heard of: Nigeria's Cattle Herders and
Farmers Wage a Resource War.'' \emph{Foreign Affairs}.
\url{https://www.foreignaffairs.com/articles/nigeria/2019-01-23/deadliest-conflict-youve-never-heard}.

\leavevmode\hypertarget{ref-jervis2017perception}{}%
Jervis, Robert. 2017. \emph{Perception and Misperception in
International Politics: New Edition}. Princeton University Press.

\leavevmode\hypertarget{ref-johnson2009overconfidence}{}%
Johnson, Dominic DP. 2009. \emph{Overconfidence and War}. Harvard
University Press.

\leavevmode\hypertarget{ref-kahneman2013prospect}{}%
Kahneman, Daniel, and Amos Tversky. 2013. ``Prospect Theory: An Analysis
of Decision Under Risk.'' In \emph{Handbook of the Fundamentals of
Financial Decision Making: Part I}, 99--127. World Scientific.

\leavevmode\hypertarget{ref-kaplan2010civilian}{}%
Kaplan, Oliver Ross. 2010. ``Civilian Autonomy in Civil War.''
PhD thesis, Stanford University.

\leavevmode\hypertarget{ref-kazdin1974covertModeling}{}%
Kazdin, Alan E. 1974. ``Covert Modeling, Model Similarity, and Reduction
of Avoidance Behavior.'' \emph{Behavior Therapy} 5 (3): 325--40.

\leavevmode\hypertarget{ref-keefer2012follow}{}%
Keefer, Philip. 2012. \emph{Why Follow the Leader? Collective Action,
Credible Commitment and Conflict}. The World Bank.

\leavevmode\hypertarget{ref-kertzer2018political}{}%
Kertzer, Joshua D, and Dustin Tingley. 2018. ``Political Psychology in
International Relations: Beyond the Paradigms.'' \emph{Annual Review of
Political Science} 21: 319--39.

\leavevmode\hypertarget{ref-kimbrough2017war}{}%
Kimbrough, Erik O, Kevin Laughren, and Roman Sheremeta. 2017. ``War and
Conflict in Economics: Theories, Applications, and Recent Trends.''
\emph{Journal of Economic Behavior \& Organization}.

\leavevmode\hypertarget{ref-kunda1990motivatedReasoning}{}%
Kunda, Ziva. 1990. ``The Case for Motivated Reasoning.''
\emph{Psychological Bulletin} 108 (3): 480.

\leavevmode\hypertarget{ref-kuusaana2015land}{}%
Kuusaana, Elias Danyi, and Kaderi Noagah Bukari. 2015. ``Land Conflicts
Between Smallholders and Fulani Pastoralists in Ghana: Evidence from the
Asante Akim North District (Aand).'' \emph{Journal of Rural Studies} 42:
52--62.

\leavevmode\hypertarget{ref-kydd2000trust}{}%
Kydd, Andrew. 2000. ``Trust, Reassurance, and Cooperation.''
\emph{International Organization} 54 (2): 325--57.

\leavevmode\hypertarget{ref-kydd2003side}{}%
---------. 2003. ``Which Side Are You on? Bias, Credibility, and
Mediation.'' \emph{American Journal of Political Science} 47 (4):
597--611.

\leavevmode\hypertarget{ref-kydd2006can}{}%
Kydd, Andrew H. 2006. ``When Can Mediators Build Trust?'' \emph{American
Political Science Review} 100 (3): 449--62.

\leavevmode\hypertarget{ref-lake2003international}{}%
Lake, David A. 2003. ``International Relations Theory and Internal
Conflict: Insights from the Interstices.'' \emph{International Studies
Review} 5 (4): 81--89.

\leavevmode\hypertarget{ref-little2017bargaining}{}%
Little, Andrew T, and Thomas Zeitzoff. 2017. ``A Bargaining Theory of
Conflict with Evolutionary Preferences.'' \emph{International
Organization} 71 (3): 523--57.

\leavevmode\hypertarget{ref-lopez2017determinants}{}%
Lopez, Anthony C, and Dominic DP Johnson. 2017. ``The Determinants of
War in International Relations.'' \emph{Journal of Economic Behavior \&
Organization}.

\leavevmode\hypertarget{ref-martiny2012shall}{}%
Martiny, Sarah E, Thomas Kessler, and Vivian L Vignoles. 2012. ``Shall I
Leave or Shall We Fight? Effects of Threatened Group-Based Self-Esteem
on Identity Management Strategies.'' \emph{Group Processes \& Intergroup
Relations} 15 (1): 39--55.

\leavevmode\hypertarget{ref-mazar2008dishonesty}{}%
Mazar, Nina, On Amir, and Dan Ariely. 2008. ``The Dishonesty of Honest
People: A Theory of Self-Concept Maintenance.'' \emph{Journal of
Marketing Research} 45 (6): 633--44.

\leavevmode\hypertarget{ref-mccarter2018models}{}%
McCarter, Matthew W, Kimberly A Wade-Benzoni, Darcy K Fudge Kamal, H Min
Bang, Steven J Hyde, and Reshma Maredia. 2018. ``Models of Intragroup
Conflict in Management: A Literature Review.'' \emph{Journal of Economic
Behavior \& Organization}.

\leavevmode\hypertarget{ref-mcdonnel2017graze}{}%
McDonnel, Tim. 2017. ``Why It's Now a Crime to Let Cattle Graze Freely
in 2 Nigerian States.'' \emph{National Public Radio (NPR)}, December.
\url{https://www.npr.org/sections/goatsandsoda/2017/12/12/569913821/why-its-now-a-crime-to-let-cattle-graze-freely-in-2-nigerian-states}.

\leavevmode\hypertarget{ref-mcdougal2015effect}{}%
McDougal, Topher L, Talia Hagerty, Lisa Inks, Claire-Lorentz Ugo-Ike,
Caitriona Dowd, Stone Conroy, Daniel Ogabiela, and others. 2015. ``The
Effect of Farmer-Pastoralist Violence on Income: New Survey Evidence
from Nigeria's Middle Belt States.'' \emph{Economics of Peace and
Security Journal} 10 (1): 54--65.

\leavevmode\hypertarget{ref-moon2016audience}{}%
Moon, Chungshik, and Mark Souva. 2016. ``Audience Costs, Information,
and Credible Commitment Problems.'' \emph{Journal of Conflict
Resolution} 60 (3): 434--58.

\leavevmode\hypertarget{ref-mousa2018overcome}{}%
Mousa, Salma. 2018. ``Overcoming the Trust Deficit: Intergroup Contact
and Associational Life in Post-Isis Iraq.''

\leavevmode\hypertarget{ref-nickerson1998confirmation}{}%
Nickerson, Raymond S. 1998. ``Confirmation Bias: A Ubiquitous Phenomenon
in Many Guises.'' \emph{Review of General Psychology} 2 (2): 175--220.

\leavevmode\hypertarget{ref-okpara2015conflicts}{}%
Okpara, Uche T, Lindsay C Stringer, Andrew J Dougill, and Mohammed D
Bila. 2015. ``Conflicts About Water in Lake Chad: Are Environmental,
Vulnerability and Security Issues Linked?'' \emph{Progress in
Development Studies} 15 (4): 308--25.

\leavevmode\hypertarget{ref-ostrom2006value}{}%
Ostrom, Elinor. 2006. ``The Value-Added of Laboratory Experiments for
the Study of Institutions and Common-Pool Resources.'' \emph{Journal of
Economic Behavior \& Organization} 61 (2): 149--63.

\leavevmode\hypertarget{ref-ostrom2003trust}{}%
Ostrom, Elinor, and James Walker. 2003. \emph{Trust and Reciprocity:
Interdisciplinary Lessons for Experimental Research}. Russell Sage
Foundation.

\leavevmode\hypertarget{ref-page2008little}{}%
Page-Gould, Elizabeth, Rodolfo Mendoza-Denton, and Linda R Tropp. 2008.
``With a Little Help from My Cross-Group Friend: Reducing Anxiety in
Intergroup Contexts Through Cross-Group Friendship.'' \emph{Journal of
Personality and Social Psychology} 95 (5): 1080.

\leavevmode\hypertarget{ref-paolini2010negative}{}%
Paolini, Stefania, Jake Harwood, and Mark Rubin. 2010. ``Negative
Intergroup Contact Makes Group Memberships Salient: Explaining Why
Intergroup Conflict Endures.'' \emph{Personality and Social Psychology
Bulletin} 36 (12): 1723--38.

\leavevmode\hypertarget{ref-petersen2002understanding}{}%
Petersen, Roger D. 2002. \emph{Understanding Ethnic Violence: Fear,
Hatred, and Resentment in Twentieth-Century Eastern Europe}. Cambridge
University Press.

\leavevmode\hypertarget{ref-pettigrew1998intergroup}{}%
Pettigrew, Thomas F. 1998. ``Intergroup Contact Theory.'' \emph{Annual
Review of Psychology} 49 (1): 65--85.

\leavevmode\hypertarget{ref-pettigrew2006meta}{}%
Pettigrew, Thomas F, and Linda R Tropp. 2006. ``A Meta-Analytic Test of
Intergroup Contact Theory.'' \emph{Journal of Personality and Social
Psychology} 90 (5): 751.

\leavevmode\hypertarget{ref-pettigrew2008does}{}%
---------. 2008. ``How Does Intergroup Contact Reduce Prejudice?
Meta-Analytic Tests of Three Mediators.'' \emph{European Journal of
Social Psychology} 38 (6): 922--34.

\leavevmode\hypertarget{ref-powell2006war}{}%
Powell, Robert. 2006. ``War as a Commitment Problem.''
\emph{International Organization} 60 (1): 169--203.

\leavevmode\hypertarget{ref-reed2016bargaining}{}%
Reed, William, David Clark, Timothy Nordstrom, and Daniel Siegel. 2016.
``Bargaining in the Shadow of a Commitment Problem.'' \emph{Research \&
Politics} 3 (3): 2053168016666848.

\leavevmode\hypertarget{ref-rohner2013war}{}%
Rohner, Dominic, Mathias Thoenig, and Fabrizio Zilibotti. 2013. ``War
Signals: A Theory of Trade, Trust, and Conflict.'' \emph{Review of
Economic Studies} 80 (3): 1114--47.

\leavevmode\hypertarget{ref-ross1991barriers}{}%
Ross, Lee, and Constance Stillinger. 1991. ``Barriers to Conflict
Resolution.'' \emph{Negotiation Journal} 7 (4): 389--404.

\leavevmode\hypertarget{ref-ross1995psychological}{}%
Ross, Lee, and Andrew Ward. 1995. ``Psychological Barriers to Dispute
Resolution.'' In \emph{Advances in Experimental Social Psychology},
27:255--304. Elsevier.

\leavevmode\hypertarget{ref-rubin2020rebel}{}%
Rubin, Michael A. 2020. ``Rebel Territorial Control and Civilian
Collective Action in Civil War: Evidence from the Communist Insurgency
in the Philippines.'' \emph{Journal of Conflict Resolution} 64 (2-3):
459--89.

\leavevmode\hypertarget{ref-runciman1966relative}{}%
Runciman, Walter Garrison, and Baron Runciman. 1966. \emph{Relative
Deprivation and Social Justice: A Study of Attitudes to Social
Inequality in Twentieth-Century England}. Vol. 13. University of
California Press Berkeley.

\leavevmode\hypertarget{ref-rusch2017logic}{}%
Rusch, Hannes, and Sergey Gavrilets. 2017. ``The Logic of Animal
Intergroup Conflict: A Review.'' \emph{Journal of Economic Behavior \&
Organization}.

\leavevmode\hypertarget{ref-sambanis2013social}{}%
Sambanis, Nicholas, and Moses Shayo. 2013. ``Social Identification and
Ethnic Conflict.'' \emph{American Political Science Review} 107 (2):
294--325.

\leavevmode\hypertarget{ref-schwartz1990toward}{}%
Schwartz, Shalom H, and Wolfgang Bilsky. 1990. ``Toward a Theory of the
Universal Content and Structure of Values: Extensions and Cross-Cultural
Replications.'' \emph{Journal of Personality and Social Psychology} 58
(5): 878.

\leavevmode\hypertarget{ref-sherif1988robbersCave}{}%
Sherif, Muzafer, O Harvey, B White, W Hood, and C Sherif. 1988.
``Intergroup Conflict and Cooperation: The Robbers Cave Experiment,
Norman: Institute of Group Relations, University of Oklahoma.''
reprinted by Wesleyan University Press.

\leavevmode\hypertarget{ref-slantchev2003power}{}%
Slantchev, Branislav L. 2003. ``The Power to Hurt: Costly Conflict with
Completely Informed States.'' \emph{American Political Science Review}
97 (1): 123--33.

\leavevmode\hypertarget{ref-slantchev2012borrowed}{}%
---------. 2012. ``Borrowed Power: Debt Finance and the Resort to
Arms.'' \emph{American Political Science Review} 106 (4): 787--809.

\leavevmode\hypertarget{ref-smith2003mediation}{}%
Smith, Alastair, and Allan Stam. 2003. ``Mediation and Peacekeeping in a
Random Walk Model of Civil and Interstate War.'' \emph{International
Studies Review} 5 (4): 115--35.

\leavevmode\hypertarget{ref-Stephan2000integratedThreat}{}%
Stephan, Walter G, Cookie White Stephan, and Stuart Oskamp. 2000. ``An
Integrated Threat Theory of Prejudice.'' \emph{Reducing Prejudice and
Discrimination}, 23--45.

\leavevmode\hypertarget{ref-tajfel1981groups}{}%
Tajfel, Henri. 1981. \emph{Human Groups and Social Categories: Studies
in Social Psychology}. CUP Archive.

\leavevmode\hypertarget{ref-tajfel1971social}{}%
Tajfel, Henri, Michael G Billig, Robert P Bundy, and Claude Flament.
1971. ``Social Categorization and Intergroup Behaviour.'' \emph{European
Journal of Social Psychology} 1 (2): 149--78.

\leavevmode\hypertarget{ref-tajfel1979integrative}{}%
Tajfel, Henri, and John C Turner. 1979. ``An Integrative Theory of
Intergroup Conflict.'' \emph{The Social Psychology of Intergroup
Relations} 33 (47): 74.

\leavevmode\hypertarget{ref-tavris2008mistakes}{}%
Tavris, Carol, and Elliot Aronson. 2008. \emph{Mistakes Were Made (but
Not by Me): Why We Justify Foolish Beliefs, Bad Decisions, and Hurtful
Acts}. Houghton Mifflin Harcourt.

\leavevmode\hypertarget{ref-thebaud2001sahel}{}%
Thébaud, Brigitte, and Simon Batterbury. 2001. ``Sahel Pastoralists:
Opportunism, Struggle, Conflict and Negotiation. A Case Study from
Eastern Niger.'' \emph{Global Environmental Change} 11 (1): 69--78.

\leavevmode\hypertarget{ref-thomas2018sahara}{}%
Thomas, Natalie, and Sumant Nigam. 2018. ``Twentieth-Century Climate
Change over Africa: Seasonal Hydroclimate Trends and Sahara Desert
Expansion.'' \emph{Journal of Climate} 31 (9): 3349--70.

\leavevmode\hypertarget{ref-tonah2002fulani}{}%
Tonah, Steve. 2002. ``Fulani Pastoralists, Indigenous Farmers and the
Contest for Land in Northern Ghana.'' \emph{Africa Spectrum}, 43--59.

\leavevmode\hypertarget{ref-turner1979social}{}%
Turner, John C, Rupert J Brown, and Henri Tajfel. 1979. ``Social
Comparison and Group Interest in Ingroup Favouritism.'' \emph{European
Journal of Social Psychology} 9 (2): 187--204.

\leavevmode\hypertarget{ref-unah2018nigeria}{}%
Unah, Linus. 2018. ``In Nigeria's Diverse Middle Belt, a Drying
Landscape Deepens Violent Divides.'' \emph{Christian Science Minitor}.

\leavevmode\hypertarget{ref-vallone1985hostileMedia}{}%
Vallone, Robert P, Lee Ross, and Mark R Lepper. 1985. ``The Hostile
Media Phenomenon: Biased Perception and Perceptions of Media Bias in
Coverage of the Beirut Massacre.'' \emph{Journal of Personality and
Social Psychology} 49 (3): 577.

\leavevmode\hypertarget{ref-wallensteen2014talking}{}%
Wallensteen, Peter, and Isak Svensson. 2014. ``Talking Peace:
International Mediation in Armed Conflicts.'' \emph{Journal of Peace
Research} 51 (2): 315--27.

\leavevmode\hypertarget{ref-walter2002committing}{}%
Walter, Barbara F. 2002. \emph{Committing to Peace: The Successful
Settlement of Civil Wars}. Princeton University Press.

\leavevmode\hypertarget{ref-waltz2010theory}{}%
Waltz, Kenneth N. 2010. \emph{Theory of International Politics}.
Waveland Press.

\leavevmode\hypertarget{ref-ward1997naive}{}%
Ward, Andrew, L Ross, E Reed, E Turiel, and T Brown. 1997. ``Naive
Realism in Everyday Life: Implications for Social Conflict and
Misunderstanding.'' \emph{Values and Knowledge}, 103--35.

\leavevmode\hypertarget{ref-warner2001personal}{}%
Warner, John T, and Saul Pleeter. 2001. ``The Personal Discount Rate:
Evidence from Military Downsizing Programs.'' \emph{American Economic
Review} 91 (1): 33--53.

\leavevmode\hypertarget{ref-weisel2015ingroup}{}%
Weisel, Ori, and Robert Böhm. 2015. ```Ingroup Love' and `Outgroup Hate'
in Intergroup Conflict Between Natural Groups.'' \emph{Journal of
Experimental Social Psychology} 60: 110--20.

\leavevmode\hypertarget{ref-wolford2011information}{}%
Wolford, Scott, Dan Reiter, and Clifford J Carrubba. 2011.
``Information, Commitment, and War.'' \emph{Journal of Conflict
Resolution} 55 (4): 556--79.

\leavevmode\hypertarget{ref-wood2000attitude}{}%
Wood, Wendy. 2000. ``Attitude Change: Persuasion and Social Influence.''
\emph{Annual Review of Psychology} 51 (1): 539--70.

\leavevmode\hypertarget{ref-wu1995cope}{}%
Wu, Jianzhong, and Robert Axelrod. 1995. ``How to Cope with Noise in the
Iterated Prisoner's Dilemma.'' \emph{Journal of Conflict Resolution} 39
(1): 183--89.

\leavevmode\hypertarget{ref-yang2013similarity}{}%
Yang, Nianhua. 2013. ``A Similarity Based Trust and Reputation
Management Framework for Vanets.'' \emph{International Journal of Future
Generation Communication and Networking} 6 (2): 25--34.

\end{document}
