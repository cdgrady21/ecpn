%\documentclass[]{article}
\documentclass[11pt]{article}
\usepackage[usenames,dvipsnames]{xcolor}

\usepackage[T1]{fontenc}
%\usepackage{lmodern}
\usepackage{tgtermes}
\usepackage{amssymb,amsmath}
%\usepackage[margin=1in]{geometry}
\usepackage[letterpaper,bottom=1in,top=1in,right=1.25in,left=1.25in,includemp=FALSE]{geometry}
\usepackage{pdfpages}
\usepackage[small]{caption}
\usepackage{subcaption}
\usepackage{graphicx}

\usepackage{ifxetex,ifluatex}
\usepackage{fixltx2e} % provides \textsubscript
% use microtype if available
\IfFileExists{microtype.sty}{\usepackage{microtype}}{}
\ifnum 0\ifxetex 1\fi\ifluatex 1\fi=0 % if pdftex
\usepackage[utf8]{inputenc}
\else % if luatex or xelatex
\usepackage{fontspec}
\ifxetex
\usepackage{xltxtra,xunicode}
\fi
\defaultfontfeatures{Mapping=tex-text,Scale=MatchLowercase}
\newcommand{\euro}{€}
\fi
%

\usepackage{fancyvrb}

\usepackage{ctable,longtable}

\usepackage[section]{placeins}
\usepackage{float} % provides the H option for float placement
\restylefloat{figure}
\usepackage{dcolumn} % allows for different column alignments
\newcolumntype{.}{D{.}{.}{1.2}}

\usepackage{booktabs} % nicer horizontal rules in tables

%Assume we want graphics always
%\usepackage{graphicx}
% We will generate all images so they have a width \maxwidth. This means
% that they will get their normal width if they fit onto the page, but
% are scaled down if they would overflow the margins.
%% \makeatletter
%% \def\maxwidth{\ifdim\Gin@nat@width>\linewidth\linewidth
%%   \else\Gin@nat@width\fi}
%% \makeatother
%% \let\Oldincludegraphics\includegraphics
%% \renewcommand{\includegraphics}[1]{\Oldincludegraphics[width=\maxwidth]{#1}}
%%\graphicspath{{.}{../Soccom_Code/socom_2013/}}


%% \ifxetex
%% \usepackage[pagebackref=true, setpagesize=false, % page size defined by xetex
%% unicode=false, % unicode breaks when used with xetex
%% xetex]{hyperref}
%% \else
\usepackage[pagebackref=true, unicode=true, bookmarks=true, pdftex]{hyperref}
% \fi


\hypersetup{breaklinks=true,
  bookmarks=true,
  pdfauthor={Christopher Grady},
  pdftitle={TBD, tentatively: Group Conflict, Collective Action Problems, and Intergroup Contact},
  colorlinks=true,
  linkcolor=BrickRed,
  citecolor=blue, %MidnightBlue,
  urlcolor=BrickRed,
  % urlcolor=blue,
  % linkcolor=magenta,
  pdfborder={0 0 0}}

%\setlength{\parindent}{0pt}
%\setlength{\parskip}{6pt plus 2pt minus 1pt}
\usepackage{parskip}
\setlength{\emergencystretch}{3em}  % prevent overfull lines
\providecommand{\tightlist}{%
  \setlength{\itemsep}{0pt}\setlength{\parskip}{0pt}}

%% Insist on this.
\setcounter{secnumdepth}{2}

\VerbatimFootnotes % allows verbatim text in footnotes

\title{TBD, tentatively: Group Conflict, Collective Action Problems, and
Intergroup Contact}

\author{
Christopher Grady
}


\date{June 13, 2020}


\usepackage{versions}
\makeatletter
\renewcommand*\versionmessage[2]{\typeout{*** `#1' #2. ***}}
\renewcommand*\beginmarkversion{\sffamily}
  \renewcommand*\endmarkversion{}
\makeatother

\excludeversion{comment}

%\usepackage[margins=1in]{geometry}

\usepackage[compact,bottomtitles]{titlesec}
%\titleformat{ ⟨command⟩}[⟨shape⟩]{⟨format⟩}{⟨label⟩}{⟨sep⟩}{⟨before⟩}[⟨after⟩]
\titleformat{\section}[hang]{\Large\bfseries}{\thesection}{.5em}{\hspace{0in}}[\vspace{-.2\baselineskip}]
\titleformat{\subsection}[hang]{\large\bfseries}{\thesubsection}{.5em}{\hspace{0in}}[\vspace{-.2\baselineskip}]
%\titleformat{\subsubsection}[hang]{\bfseries}{\thesubsubsection}{.5em}{\hspace{0in}}[\vspace{-.2\baselineskip}]
\titleformat{\subsubsection}[hang]{\bfseries}{\thesubsubsection}{1ex}{\hspace{0in}}[\vspace{-.2\baselineskip}]
\titleformat{\paragraph}[runin]{\bfseries\itshape}{\theparagraph}{1ex}{}{\vspace{-.2\baselineskip}}
%\titleformat{\paragraph}[runin]{\itshape}{\theparagraph}{1ex}{}{\vspace{-.2\baselineskip}}

%%\titleformat{\subsection}[hang]{\bfseries}{\thesubsection}{.5em}{\hspace{0in}}[\vspace{-.2\baselineskip}]
%%%\titleformat*{\subsection}{\bfseries\scshape}
%%%\titleformat{\subsubsection}[leftmargin]{\footnotesize\filleft}{\thesubsubsection}{.5em}{}{}
%%\titleformat{\subsubsection}[hang]{\small\bfseries}{\thesubsubsection}{.5em}{\hspace{0in}}[\vspace{-.2\baselineskip}]
%%\titleformat{\paragraph}[runin]{\itshape}{\theparagraph}{1ex}{}{\vspace{-.5\baselineskip}}

%\titlespacing*{ ⟨command⟩}{⟨left⟩}{⟨beforesep⟩}{⟨aftersep⟩}[⟨right⟩]
\titlespacing{\section}{0pc}{1.5ex plus .1ex minus .2ex}{.5ex plus .1ex minus .1ex}
\titlespacing{\subsection}{0pc}{1.5ex plus .1ex minus .2ex}{.5ex plus .1ex minus .1ex}
\titlespacing{\subsubsection}{0pc}{1.5ex plus .1ex minus .2ex}{.5ex plus .1ex minus .1ex}



%% These next lines tell latex that it is ok to have a single graphic
%% taking up most of a page, and they also decrease the space around
%% figures and tables.
\renewcommand\floatpagefraction{.9}
\renewcommand\topfraction{.9}
\renewcommand\bottomfraction{.9}
\renewcommand\textfraction{.1}
\setcounter{totalnumber}{50}
\setcounter{topnumber}{50}
\setcounter{bottomnumber}{50}
\setlength{\intextsep}{2ex}
\setlength{\floatsep}{2ex}
\setlength{\textfloatsep}{2ex}



\begin{document}
\VerbatimFootnotes

%\begin{titlepage}
%  \maketitle
%\vspace{2in}
%
%\begin{center}
%  \begin{large}
%    PROPOSAL WHITE PAPER
%
%BAA 14-013
%
%Can a Hausa Language Television Station Change Norms about Violence in Northern Nigeria? A Randomized Study of Media Effects on Violent Extremism
%
%Jake Bowers
%
%University of Illinois @ Urbana-Champaign (jwbowers@illinois.edu)
%
%\url{http://jakebowers.org}
%
%Phone: +12179792179
%
%Topic Number: 1
%
%Topic Title: Identity, Influence and Mobilization
%
%\end{large}
%\end{center}
%\end{titlepage}

\maketitle

\begin{abstract}

Group conflict of all types plagues humanity's dealings with itself.  Group conflict is caused by a commitment problem wherein neither side trusts the other to honor agreements.  Peace is a public good, so the commitment problem can be solved if groups have ways to compel radical group members to honor agreements + signal the ability and intent to compel radicals to other side.  Psychological barriers like prejudice may prevent groups from desiring peace/wanting to prevent radicals, and from sending/receiving signals about commitment to peace.  Intergroup contact that achieves a joint goal can remove prejudice, provide incentive for peace and cooperation, and allow signals to be more accurately perceived.  To send strong, unambiguous signals, however, both groups must be involved in punishing defectors: only ingroup punishing their own can be perceived as too lenient, only outgroup punishing other side can be perceived as too harsh.  I apply this perspective to the the cases of farmer-pastoralist conflict in Nigeria.  I close by discussing outstanding issues, such as initiating the onset of joint-punishment institutions, and implications for peacebuilding programs.

\end{abstract}

\hypertarget{introduction}{%
\section{Introduction}\label{introduction}}

Groups fight for the resources -- land, wealth, power, or otherwise --
that fighting destroys. As such, groups could almost always divide the
resources before war in a way that benefits them more than their postwar
settlement. This is the enigma of violent conflict: why do groups engage
in violent conflict when violent conflict often destroys the very thing
the groups are fighting for? The general answer is that groups suffer
from a commitment problem: both sides are better off negotiating a
peaceful agreement, but conflict occurs because neither group trusts the
other to abide by negotiated agreements (Fearon 1995; Powell 2006; Reed
et al. 2016). If groups trusted each other to abide by agreements, they
would not need to fight and each group would enjoy superior outcomes
relative to their outcomes after fighting.

One of the most common mechanisms for overcoming commitment problems and
resolving group conflict is third-party enforcement of agreements. Third
party enforcement reduces commitment problems by incentivizing each side
to honor agreements out of their own interest. Though each group may
have an incentive to defect on an agreement after it is made, the groups
have less incentive to defect if a strong third party is capable of and
willing to punish defection from bargained agreements (Fearon 1994).
Third parties to enforce agreements can often effectively reduce
conflict (Di Salvatore and Ruggeri 2017; Doyle and Sambanis 2000;
Gartner 2011; Hartzell, Hoddie, and Rothchild 2001; Wallensteen and
Svensson 2014; Walter 2002), but they also suffer from limitations.
First, third party intervention is costly. The cost results in
intervention only when the third party has an interest in one side
winning the dispute or when conflict has already escalated into
violence, so intervention is rarely used to reduce the persistent,
smaller-scale violence that plague many countries (Fey and Ramsay 2010;
Greig 2005; Kydd 2006). Second, third parties are relatively ineffective
at ending ongoing conflicts compared to maintaining peace in a
post-conflict setting (Bratt 1996; Doyle and Sambanis 2006; Gilligan,
Sergenti, and others 2008). And third, third party punishment is
exogenous to the conflicting groups and therefore does not create trust
between those groups that will last beyond the third party's presence
(Gambetta and others 2000; Rohner, Thoenig, and Zilibotti 2013).
Weinstein (2005) estimated that 75\% of civil wars resume within 10
years of UN intervention, and Beardsley (2008) showed that international
mediation reduces conflict in the short-term but not the long-term.

Commitment problems can be overcome and conflict resolved without
relying on third party intervention. Groups can build trust and resolve
conflicts by cultivating reputations for trustworthiness in repeated
interactions. Repeated interactions between groups, like third parties,
help overcome commitment problems by providing each group with the
incentive to honor agreements out of their own interest. Here the
incentive comes not from fear of punishment but the prospect of better
future outcomes from cooperation than defection. Though each group may
have an incentive to defect on an agreement today if the groups will not
interact tomorrow, the groups have an incentive to cooperate now if
their behavior today will be reciprocated by the other side in future
interactions (Axelrod and Hamilton 1981; Kydd 2000; Ostrom and Walker
2003).\footnote{I discuss repeated interactions and reputations
  together, but these mechanisms are subtly different. The repeated
  interaction mechanism is generally mobilized for contexts with just
  two groups. In those contexts, cooperation in previous interactions
  assists in obtaining cooperative behavior in the future from the same
  partner; other potential partners are unnecessary. The reputation
  mechanism is generally mobilized for contexts in which many groups
  observe the behavior of many other groups. In those contexts a good
  reputation assists in obtaining \emph{other} cooperative partners;
  repeated interactions with the same partner are unnecessary. I discuss
  these mechanisms together because both rely on creating cooperative
  expectations from bargaining partners in future interactions.} Both
groups stand to gain more from enduring cooperation than enduring
defection.

Solving commitment problems through repeated interactions also faces
limitations, however, especially for groups in or with a history of
conflict. Repeated interactions only solve commitment problems if both
groups prefer peace (cooperation) to fighting (defection). Though peace
is in each group's material interest, preferences often deviate from
material interest (Fehr and Fischbacher 2002). In the case of conflict,
groups may not want peace because they derive psychological benefit from
feelings of group superiority and from harm to a hated outgroup (Cikara
et al. 2014; Fein and Spencer 1997; Tajfel 1981; Weisel and Böhm 2015)
and when their animosity leads them to prefer relative gains over the
outgroup to absolute gains for themselves (Halevy et al. 2010; Turner,
Brown, and Tajfel 1979; Waltz 2010). Even if groups desire peace, they
maintain an incentive to act ``tough'' to get the best deal, and so may
not signal their desire for peace (Ross and Ward 1995). And even if
groups signal their desire for peace, phenomena like confirmation bias,
cognitive dissonance, and motivated reasoning can bias perceptions of
outgroup behavior and prevent groups from accurately perceiving peaceful
intentions (De Dreu, Nijstad, and Knippenberg 2008; Duncan 1976;
Festinger 1962; Kunda 1990; Nickerson 1998; Tajfel 1969; Vallone, Ross,
and Lepper 1985; Ward et al. 1997). These non-material factors can
prevent future interactions from solving groups' commitment problem.

An additional challenge with building trust between groups has not
received much attention in the literature: each group faces a
within-group collective action problem in addition to the between-group
commitment problem. The collective action problem occurs because each
group is a collection of individual group members with their own
behavioral incentives, not cohesive entities. Peace benefits all members
of a group, but contributing to peace is costly for group members
because they must refrain from engaging in behaviors that benefit
themselves at the expense of the outgroup. Achieving peace, as with any
public good, requires overcoming this collective action problem: the
group must compel group members to contribute to peace despite the
members' individual incentive to shirk and rely on others to bear its
cost. One group trusting the other to honor an agreement, then, means
not only trusting that a high proportion of outgroup members desire
peace, but also that those outgroup members can and will compel less
cooperative group members to honor a peace agreement (Fearon and Laitin
1996). Peace, after all, can be derailed by a few radicals (De Sanctis
and Galla 2009; Sambanis and Shayo 2013).

This is a bleak picture. Often, neither the presence of third parties
nor repeated interaction are sufficient to build trust between groups.
Third parties may be unwilling to intervene early enough to prevent
violence, rarely intervene with enough force to stop ongoing conflict,
and do not necessarily create the trust needed for long-term peace.
Repeated interactions can create the trust needed for long-term peace,
but getting cooperation started is difficult because of psychological
costs to cooperation, incentives to increase bargaining power by acting
uncooperative, and cognitive biases that prevent groups from receiving
each other's cooperative signals. And beyond signaling the intent of the
group's majority to abide by peace agreements, groups must go further
and signal their willingness and capacity to punish their own group
members if a group member defects on the agreement.

Despite the bleakness, insights from psychology and political science
provide means to overcome the issues that prevent groups from developing
trust in repeated interactions. When repeated interactions help groups
cooperate they do so because each group sees that peaceful cooperation
is in both their interests; repeated interactions fail to bring about
peace when groups prefer fighting or when groups think the other side
prefers fighting. To help groups overcome their commitment problems
through repeated interactions, an intervention must shift preferences
towards peace and provide an opportunity for each group to observe this
preference in the other. Structured intergroup contact -- cross-group
interactions in which groups cooperate to achieve shared goals -- can
show groups that peace is in both their interests, remove the
psychological barriers to each group identifying that the other wants
peace, and provide opportunities for each group to send costly signals
of their intent to cooperate and willingness to punish ingroup members
who do not. Achieving a joint goal demonstrates to each group that both
their interests are best served through cooperation. The prospect of
benefit through cross-group cooperation motivates cooperation and
attitude change (C. D. Grady 2020; Rohner, Thoenig, and Zilibotti 2013)
as well as the adoption of group norms to ensure cooperation, like
punishment of ingroup members whose behavior undermines peace (Axelrod
1986; Fearon and Laitin 1996). Believing that the other side wants peace
also creates the expectation that the other side will solve its
collective action problem and police their own because it is in their
interest to do so.

Intergroup contact is a good first step to show groups that both benefit
from cooperation and help groups send and receive signals of
cooperation. Even if contact reduces biases that cause misperceptions of
these signals, however, misperceptions are still possible and can derail
cooperative equilibria (Bendor, Kramer, and Stout 1991; Jervis 2017; Wu
and Axelrod 1995). Members of one group may accidentally engage in
action that harms the other or may perceive outgroup behavior as
aggressive when it is not. For long-term peace, institutions must
minimize and manage these misperceptions.

One common mechanism for reducing group conflict is ingroup policing in
which each side polices its own miscreants (Fearon and Laitin 1996).
Ingroup policing is generally an excellent institution for preventing
escalations of violence and for maintaining peace, but I suggest that
mistrust between groups in or with a recent history of conflict may make
ingroup policing unsustainable as a solution to prevent violence because
of the potential for one group to devalue the costs imposed by the
ingroup on their defector. Groups in conflict reactively devalue costly
behaviors done by the other side (Maoz et al. 2002; Millard and Porter
2018; Ross and Ward 1995; Ward et al. 1997) on the logic that ``\(X\)
behavior must not be very costly, since they were willing to do it''
(Kahneman (1992) calls a similar phenomenon ``concession aversion'').
This reactive devaluation causes problems for ingroup policing, since it
can create the perception that any ingroup punishment was too lenient.
This can cause the outgroup to exact their own punishment, which may be
viewed as too harsh, causing a spiral of violence. Instead of ingroup
policing, I propose that both groups should be involved in the
punishment of group members who harm the other side. The involvement of
both groups minimizes the possibility that either side believes the
punishment was too lenient or harsh.

I begin by describing how non-material factors can lead to group
conflict by preventing each group from trusting the other to cooperate
even when it is in each group's material interest to cooperate. Next I
propose intergroup contact and joint-punishment institutions as a way
that groups avoid falling into a spiral of violence. I then apply this
perspective to farmer-pastoralist conflict in Nigeria and describe the
form that joint-punishment institutions have taken there, as well as
consequences for conflict when no such institutions exist. I conclude by
describing implications for peacebuilding programs and considering
avenues for empirical research.

\hypertarget{non-material-factors-that-increase-likelihood-of-conflict}{%
\section{Non-material factors that increase likelihood of
conflict}\label{non-material-factors-that-increase-likelihood-of-conflict}}

Violent conflict is not materially rational because it destroys the
resources groups are fighting for. As such, several non-material factors
should contribute to conflict. By looking at groups as a collection of
individual group members, I identify two main ways that these
non-material factors contribute to conflict. First, they shrink the
range of peace agreements that each side prefers to peace by adding
costs to peace and benefits to fighting. Second, they bias perceptions
of the other side's preferences such that neither side thinks the other
prefers peace or can be trusted to abide by peace agreements. I also
identify an internal collective action problem that leads to conflict
and explain why that problem is exceptionally difficult for group's to
solve.

\hypertarget{costs-of-peace-and-benefits-of-fighting}{%
\subsection{Costs of Peace and Benefits of
Fighting}\label{costs-of-peace-and-benefits-of-fighting}}

Non-material factors add costs to peace in three main ways: (1) costs of
changing attitudes and institutions, (2) loss of self-esteem, and (3)
loss of sunk costs. First, to the extent that accepting peace requires
individuals in the group to change attitudes about the outgroup, group
members must accept the psychological costs of attitude change.
Individuals use many strategies to maintain existing attitudes --
searching for information that confirms pre-existing beliefs,
counterarguing information divergent with their beliefs -- suggesting
that costs to attitude change are not insubstantial (Kunda 1990;
Nickerson 1998). Changing negative attitudes towards an enemy may be
especially challenging because negative outgroup attitudes are supported
by a ``sociopsychological infrastructure'' that feeds conflict (Bar-Tal
2007). At the group-level, a peace agreement requires groups to
dismantle the sociopsychological infrastructure that fed conflict and
any social norms/institutions developed from it that encouraged violence
against the outgroup (Bornstein 2003). Those institutions must be
reformed to prevent, not encourage, intergroup violence.

Second, peace agreements may cause a loss of self-esteem for individual
group members. Group members derive self-esteem from positively
comparing their group to a rival group, and and any agreement in which
their side acknowledges the legitimacy of the other challenges this
group-based boost to self-esteem (Brown and Pehrson 2019; Fein and
Spencer 1997; Martiny, Kessler, and Vignoles 2012; Tajfel 1981; Tajfel
and Turner 1979; Wood 2000). Group members may also pay psychological
costs to their self-esteem from losing their rationalization for
engaging in discriminatory or aggressive behaviors towards outgroup
members. People rationalize their behavior to maintain a moral
self-image (Bandura 1999, 2014; Mazar, Amir, and Ariely 2008), and
people who harmed the outgroup must come to terms with past behaviors
that are now deemed immoral.

Third, individuals are affected by the sunk cost fallacy and want the
gain from a peace deal to make up for the cost of fighting even though
those costs were already born (Arkes and Blumer 1985). Accepting a peace
agreement after fighting that could have been achieved without fighting
is to admit that past behavior was a mistake, and people are loathe to
admit their mistakes (Tavris and Aronson 2008). It is ``dissonant for
the disputing parties to accept terms today that could have been
achieved, without the ensuing costs, at some earlier point in time''
(Ross and Ward 1995, 264).

{[}chris: I feel that sunk costs are important but I have not yet been
able to articulate it well. Is it a loss of self-esteem?{]}

To see how these psychological costs to peace might sabotage peace
efforts, imagine that groups fighting over land sign a peace agreement
that assigns some land to each group. This agreement may make material
sense in that each group can enjoy the benefits of some land instead of
fighting for all of it. But in addition to the material cost of losing
potential access to land granted to the other side, each group must
begin punishing outgroup violence rather than encouraging it.
Individuals in the group must adjust their behavior to conform to new
norms and rethink the status of their group relative to the outgroup;
they can no longer ``bask in the reflected glory'' of their group's
superiority (Brown and Pehrson 2019, 312). Individuals who felt
justified in overcharged outgroup members for services or physically
harming outgroup members are now told that such behaviors are wrong and
immoral. Individuals must also accept that their sacrifices to benefit
fighting -- dead friends and loved ones, material deprivation, time and
energy spent fighting -- were for naught: the sacrifices did not help
the group win. Sustaining conflict against the outgroup allows group
members to maintain existing attitudes and behaviors and preserves the
possibility that their group will be victorious.

Other than avoiding these costs, group members may prefer fighting due
to its non-material benefits. First, group members may feel pleasure in
response to outgroup pain (Cikara et al. 2014; Weisel and Böhm 2015).
This pleasure adds psychological utility to participating in conflict
for individual group members. Second, individuals may like gaining glory
and other social rewards for participating in violent conflict.
Societies use various social carrots and sticks to encourage
participation in conflict (Gneezy and Fessler 2012), and group members
give up those rewards when conflict ends. These social rewards add
social utility to participating in conflict. Third, in addition to
non-material benefits of fighting, each side may overestimate their
chances of victory, causing them to discount the costs of conflict
(Johnson 2009).

As a result of these costs to peace and benefits to conflict, there may
exist no peace agreement that both sides prefer to fighting. Each side
prefers mutual cooperation in repeated interactions only if their
utility from mutual cooperation exceeds their utility from mutual
defection. Due to non-material costs of peace and benefits of fighting,
sides may prefer fighting, to say nothing of each side's tendency to
overestimate their chances of victory (Johnson 2009) Even if side's do
not overestimate their chance of victory, each side may only accept a
peace agreement that materially favor their side because the material
gains from peace must overcome its non-material costs. Indeed, many
group members are willing to accept lower absolute gains to increase
relative gain over the outgroup {[}Turner, Brown, and Tajfel (1979);
Waltz (2010); Halevy et al. (2010);{]}. The preference for gains
relative to the outgroup makes any mutually beneficial peace agreement
impossible.

{[}chris: tikz graph showing payoffs? ``Conflict often discussed as an
iterative PD. In one-off, both sides prefer to defect, but in repeated
interactions each side cooperates if their cooperation incites
cooperation in the other side. But adding costs to cooperation and
benefits to fighting change that calculus.''{]}

\hypertarget{biased-perceptions}{%
\subsection{Biased Perceptions}\label{biased-perceptions}}

Though non-material factors can increase the costs of peace and increase
benefits of conflict, the high cost of conflict may be such that both
sides still prefer peace. Even when both sides prefer peace,
psychological biases can prevent each side from accurately perceiving
the other's desire for peace and negotiating a peace agreement. This
happens in two main ways: (1) cognitive dissonance, motivated reasoning,
and confirmation bias\footnote{\emph{Cognitive dissonance} is the mental
  discomfort that occurs when one individual holds two contradictory
  beliefs. Individuals resolve this dissonance by rejecting one of the
  contradictory beliefs (Festinger 1962; Tavris and Aronson 2008).
  Individuals tend to resolve this dissonance by rejecting the newer or
  less central belief because it is costly to reject older and more
  central beliefs (Bryan, Yeager, and Hinojosa 2019; Converse 1970;
  Schwartz and Bilsky 1990). \emph{Motivated reasoning} and
  \emph{confirmation bias} are related concepts. Motivated reasoning is
  the tendency for individuals to ``reason'' in whatever way allows them
  to reach their desired conclusion (Kunda 1990). Confirmation bias is
  the tendency for individuals to interpret new information and search
  memory for information to confirm existing beliefs (Nickerson 1998).}
cause each side to interpret the other's behavior in a way that supports
existing negative attitudes and (2) loss aversion and reactive
devaluation\footnote{\emph{Loss aversion} refers to individuals'
  tendency to avoid losses more than they seek equivalent gains
  (Kahneman and Tversky 2013). Reactive devaluation is the tendency for
  individuals to undervalue concessions and proposals from antagonists
  (Ross and Stillinger 1991; Ward et al. 1997).} cause the two sides to
overvalue concessions they give up and undervalue concessions they
receive. These phenomena introduce significant friction in the ability
of groups to build trusting relationships and avoid conflict.

Cognitive dissonance, motivated reasoning, and confirmation bias cause
group members to interpret and recall outgroup behavior in negative ways
that prevent building trust. Defensive action by outgroup members may be
interpreted as belligerent and threatening while belligerent action by
one's own group is seen as defensive and justified (Duncan 1976;
Vallone, Ross, and Lepper 1985; Ward et al. 1997); positive actions by
outgroup members may be re-interpreted as negative to avoid cognitive
dissonance (Good 2000; Gubler 2013; Paolini, Harwood, and Rubin 2010);
and unequivocally positive behavior may be dismissed as an exception
while negative attributes are believed to define all outgroup members
(Hewstone 1990). These phenomena can prevent costly signals of
cooperative intent, one of the main ways groups solve commitment
problems, from being interpreted accurately and reassuring each side of
the other's desire for peace (Gambetta and others 2000; Kydd 2000;
Rohner, Thoenig, and Zilibotti 2013).

If groups manage to signal their desire for peace despite these
psychological biases, loss aversion and reactive devaluation can
sabotage the negotiation process. Individuals tend to weigh losses more
heavily than equivalent gains, so anything the group gives up may be
magnified in importance (Kahneman and Tversky 2013). At the same time,
individuals tend to undervalue concessions from antagonists (Ross and
Stillinger 1991; Ward et al. 1997). These biased valuations of
concessions made and concessions given can prevent groups from reaching
a mutually beneficial agreement. They also interfere with group's
sending costly signals of cooperative intent by lowering the perceived
costliness of any cooperative signal.

Compounding these perceptual problems, intergroup trust is hampered by a
lack of opportunities to learn about the other side. Groups in conflict
tend to limit contact between the two sides (Bornstein 2003), so most
learning must come from observing the outgroup's interactions with other
groups. Even if those opportunities are available, few of the outgroup's
interactions will be with groups that are relevant for predicting the
outgroup's behavior towards the ingroup, reducing the usefulness of
observational learning (Kazdin 1974; Yang 2013). The main opportunity to
observe outgroup behavior and learn their reputation is the ingroup's
own interactions with the outgroup, which are limited and hampered by
perceptual biases.

\hypertarget{problem-of-collective-action}{%
\subsection{Problem of Collective
Action}\label{problem-of-collective-action}}

Implicit in the discussion of these non-material factors is that groups
are made up of individuals and that individual cognitive and
psychological biases influence conflict dynamics. Following that logic
-- that groups are collections of individuals with preferences that may
differ from the group as a whole -- leads to an additional challenge to
groups' negotiating peace agreements: peace is a public good. Group
members must refrain from behavior that benefits them individually at
the expense of the outgroup, but if peace is achieved group members
enjoy its benefits whether or not they bore its costs. Without the
ability to compel behavior for the public good, groups cannot credibly
commit to honor negotiated agreements.\footnote{Bornstein (Bornstein
  (1992); Bornstein (2003)) was among the first to consider how the
  preferences of individuals complicate discussions of group conflict.
  His focus, and the focus of most subsequent scholarship, was how
  groups encourage group members to participate in violence against the
  other side, since winning the fight benefits all group members but
  only group members who fight bear the costs.}

Collective action problems are notoriously difficult to solve. In the
case of intergroup conflict, compelling group members to sacrifice now
for the long-term good of the group is made even more difficult because
individuals may be killed before peace is achieved and thus never enjoy
the benefits of collective action. Individuals are notoriously bad even
at saving for retirement, where individuals can sacrifice now to benefit
themselves later (Benartzi and Thaler 2013; Warner and Pleeter 2001).
Individuals may rationally discount future payoffs quite significantly
in conflict settings where the benefits accrue to a group.

Fearon and Laitin (1996) showed that ingroup policing, an institution
wherein groups punish their own miscreants, can help groups avoid an
escalation of intergroup violence; they credit the relative paucity of
group conflict to such institutions. Ingroup policing is an excellent
institution for preventing violence from escalating and for maintaining
peace in many contexts, but I suggest that perceptual biases and
mistrust between groups in or with a recent history of conflict makes
ingroup policing unsustainable as a solution in those contexts. With
ingroup policing, groups are restrained from overly lenient punishment
of their own people by the threat of violence escalating if either group
suspects the other of acting in bad faith. Since ingroup policing
depends on the perceptions group members, if those perceptions are
biased such that members of each group suspect the other is trying to
take advantage of the ingroup policing institution, then the institution
breaks down as each group does not trust the other to allocate
sufficient punishment to their own side.

In the next section, I will discuss ways to reduce the biases that cause
misperceptions and a modification of ingroup policing that should be
less susceptible to misperceptions when they do occur.

\hypertarget{reducing-group-conflict}{%
\section{Reducing group conflict}\label{reducing-group-conflict}}

In the previous section, I discussed the non-material factors that
contribute to conflict by (1) adding costs to peace and benefits to
fighting, and (2) biasing perceptions of the other side such that
neither group can reassure the other side about peaceful intentions. I
also discussed how groups suffer from a collective action problem in
compelling group members to honor peace agreements and why ingroup
policing may not help groups in conflict reassure the other side that
they have solved their collective action problem. In this section, I
discuss intergroup contact as a method to reduce the costs of peace, the
benefits of fighting, and the biases that cause misperceptions. I then
propose a modification of ingroup policing that should be less
susceptible to misperceptions when they do occur.

\hypertarget{intergroup-contact}{%
\subsection{Intergroup contact}\label{intergroup-contact}}

Intergroup contact, interactions between group members in which members
of different groups work together to achieve common goals, could help
groups overcome problems that cause conflict. First, achieving a goal by
working with the outgroup could make groups prefer peace to fighting by
demonstrating the benefits of cross-group cooperation. Whereas some
psychological phenomena add costs to peace, intergroup contact could add
previously unforeseen benefits to peace. The prospect of benefits
through cooperation can motivate individuals to develop more positive
attitudes towards cooperation (C. D. Grady 2020; Rohner, Thoenig, and
Zilibotti 2013) and groups to develop norms that encourage cooperation
{[}Axelrod (1986); chris: cite?{]}. More positive individual attitudes
and social norms towards the outgroup should also add costs to fighting,
since harming the outgroup loses its luster.

Second, along with increasing the utility of cooperation so that each
group prefers cooperation to fighting, achieving a joint goal through
cooperation reassures each group that the other side also prefers
cooperation. If each side prefers cooperation but neither side is aware
of the other side's preferences, groups may not cooperate to avoid the
costs of being betrayed. Each side can trust the other to engage in
cooperative behavior when both side's know that it is in each side's
interest to do so (Gambetta and others 2000).

Third, intergroup contact can reduce perceptual biases that prevent
groups from accurately perceiving the other side's preferences and
building trust. Intergroup contact allows groups members to learn about
each other based on personal experience, which can dispel stereotypes,
reduce feeling of threat and anxiety, engender feelings of empathy, and
make group commonalities salient (Allport 1954; Batson et al. 1997;
Broockman and Kalla 2016; Gaertner et al. 1993; Page-Gould,
Mendoza-Denton, and Tropp 2008; Pettigrew and Tropp 2008). All of these
mechanisms increase the likelihood that group members perceive
cooperation to be in their interest. It is unlikely that one side will
expect cross-group interaction to be in the group's interest if that
group member fears outgroup members and holds negative stereotypes about
the outgroup's work ethic and honesty. Through these mechanisms, contact
builds trust even between conflicting groups (C. Grady 2020; Hewstone et
al. 2006).

Fourth, intergroup contact provides groups with opportunities to
interact and send direct signals of trustworthiness. Direct cross-group
interaction and communication reduces competition in behavioral games
(Bornstein et al. 1989; Ostrom 2006) and in formal models (Rohner,
Thoenig, and Zilibotti 2013). These signals can serve as
confidence-building measures and allow groups to start small and
low-risk and gradually increasing as groups build trust over time; trust
is one of the few resources that increases with use (Gambetta and others
2000). Importantly, intergroup contact gives each group the opportunity
to signal willingness to punish their own members if those members
jeopardize peace.

Intergroup contact can provide incentives for cooperation, reduce
perceptual biases, and provide opportunities for costly signals of
trustworthiness. It is not a panacea, however, and groups can still
misperceive the behavior and intentions of the other side. When both
groups desire cooperation, the risk of cooperative behavior being
misperceived as uncooperative behavior is the main threat to enduring
peace (Bendor, Kramer, and Stout 1991; Jervis 2017; Wu and Axelrod
1995). If groups trust each other to police their own, each side must
worry that the other group will not punish their own members with
sufficient force to deter their harmful or discriminatory actions.
Groups are, after all, biased in favor of their own side. Groups need a
collaborative institutional structure to minimize misperceptions and
their effect.

\hypertarget{misperceptions-and-joint-punishment-institutions}{%
\subsection{Misperceptions and joint-punishment
institutions}\label{misperceptions-and-joint-punishment-institutions}}

When groups are cooperating, misperceptions are a major danger to
continued cooperation. To see why, imagine that two groups are
cooperating over a joint feast. One group is charged with bringing meat
and milk, the other with bringing vegetables and grains. Imagine the
vegetable and grain side contributes less than their fair share due to a
bad harvest. What should the other side do? Since they do not know why
the other side has come without sufficient food, they cannot rule out
the possibility that the other side is being treacherous. Do they allow
the non-contributors to partake in the feast, or do they take their food
and go home? If they allow the non-contributors to join the feast, they
might be preserving cooperative relations and ensuring future feasts,
which is good for both sides. Or they might be allowing the other side
to get away with betrayal, which is bad for their side and good for the
side of the non-contributors.\footnote{This situation follows Axelrod
  and Hamilton (1981) and Fearon and Laitin (1996) in conceptualizing
  intergroup cooperation as an iterative prisoners dilemma. Two groups
  interact simultaneously for an unlimited number of rounds, either
  cooperating or defecting. Both sides most prefer to defect when the
  other cooperates and least prefer to cooperate when the other defects;
  mutual cooperation is the Pareto efficient outcome in that it
  maximizes overall value. When each side cooperates, there is a small
  chance that it accidentally defects or that it's cooperation gets
  interpreted as defection.}

In this scenario, both sides have intended to cooperate, but one side's
attempt to cooperate looks like defection. Since both group's prefer to
cooperate, mutual cooperation can only be interrupted by an accident
(one side tries to cooperate and accidentally defects) or
misinterpretation (one side cooperates but it gets interpreted as
defection). In this case, the continuation of cooperation depends on the
meat and milk side perceiving the grain and vegetable side's meager
contribution as cooperation and not defection. If their perceptions are
biased against the outgroup, they may attribute the lack of food to ill
intent.

The problem of misperceptions is exacerbated when considering the groups
as collections of individuals with different preferences. Rather than
the groups meeting to cooperate or defect, members of each group meet to
cooperate or defect. Imagine a member of Group A defects against a
member of Group B. If the groups are relying on each side to punish
their own, then Group A punishes their defector. Suppose he defects
against Group B again at his next opportunity. It may appear to Group B
that (1) Group A has not punished their own group member, so he
continues to defect, or (2) Group A's punishment against their own group
member was too lenient. In either scenario, Group B might punish him
more harshly to deter future defection.\footnote{Ingroup policing has
  many virtues over other means to punish defectors. Most important is
  that the ingroup is most likely to know who defected, whereas the
  outgroup is only likely to know that someone from the other group
  defected (Fearon and Laitin 1996).}

Since Group B does not have a say in Group A's punishment, they must
trust that Group A punishes their defectors and does so sufficiently to
deter future defection. But neither group can know with certainty how
much punishment is ``enough'' to deter future defection, so some
punishments are bound to fail. These failures can delegitimize the
institution of ingroup policing unless the groups share a substantial
amount of intergroup trust. This amount of intergroup trust is unlikely
to characterize the relationships of groups in or with a recent history
of violent conflict.

Rather than the institution of ingroup policing as the way for groups to
ensure cooperation, I suggest that many groups use collaborative
``joint-punishment'' institutions. These collaborative institutions
involve representatives from both sides who jointly decide on an
appropriate punishment for any behavior in which a member of one group
harms a member of the other. Having both groups involved vastly reduces
the possibility that either side believes a defector went unpunished or
misperceives a genuine attempt at punishment as overly lenient. It
maintains the main benefit of ingroup policing in that each group still
helps identify their defecting group members but does more to shield
both groups from the temptation to punish outgroup defection themselves.

In the next section, I apply this perspective to farmer-pastoralist
conflict in Nigeria and describe the form that institutions to improve
intergroup cooperation have taken in that context.

\hypertarget{application-farmer-pastoralist-conflict}{%
\section{Application: Farmer-Pastoralist
Conflict}\label{application-farmer-pastoralist-conflict}}

The conflict between farmers and pastoralists in Nigeria is a textbook
case of a commitment problem preventing peace. The groups maintain
complementary ways of life, making cooperation beneficial for both.
Pastoralists have an excess of protein in the form of meat, milk, and
other animal products, but they grow little in the way of grains, tubers
and vegetables; the farmers have an excess of grains, tubers, and
vegetables but they own few animals and have limited access to animal
products.\footnote{Farming villages typically stock chickens for eggs
  and meat as their main protein source, with a few goats and sheep.
  They do not have excess food year round to support large animals like
  cows. Pastoralists are semi-migratory so few stay in one place long
  enough to cultivate crops. Those who stay in a ``home base'' will set
  aside some land from rice or yams, but most of the land is left for
  cattle to graze.} Farmers also want animals to graze on their lands
after harvest season to replenish the soil with animal waste, and
pastoralists want to graze their animals on crop residue (stalks,
leaves, seed pods, and other inedible parts of the plant) that is left
on the fields after harvest.

By all accounts, farmers and pastoralists benefited from their
complementarity for several generations (Herders against farmers 2017;
Thébaud and Batterbury 2001; Tonah 2002). There were, of course,
disagreements between sedentary farmers and mobile pastoralists, but
their relationship was characterized more by harmony than conflict.
Recent decades, however, have brought farmers and pastoralists more into
conflict. Historically, farming was more common in southern Nigeria and
pastoralism more common in the north, but the two ways of life
increasingly overlap geographically. Farmers have moved north into
marginal agricultural lands by the increasing food needs of Nigeria's
booming population, which grew from 50 million at Independence in 1960
to 200 million today (Abbass 2012; Kuusaana and Bukari 2015). At the
same time, pastoralists have been pushed further south by the expansion
of the Sahara, which brought them to higher population density areas
(Okpara et al. 2015; Thomas and Nigam 2018). Less land and more people
who depend on the land is a recipe for conflict over land and resources.

Farmer-pastoralist conflict has exploded in recent years. The most
recent conflict escalation caused 7,000 deaths from 2014-2019 and
displaced hundreds of thousands of people from their homes (Akinwotu
2018; Daniel 2018; Harwood 2019; Ilo, Ier, and Adamolekun 2019). The
scale of economic damage is unknown, but farmer-pastoralist conflict
\emph{before} this escalation cost Nigeria \$13 billion annually in lost
economic productivity (McDougal et al. 2015). This violence has also
impeded food production, leading to an impending food crisis
(Hailemariam 2018; Ilo, Ier, and Adamolekun 2019; Unah 2018).

The proximate causes of violence are farmers sowing seeds on
pastoralists' grazing lands and pastoralists grazing their cattle on
farmers' crops. If either side retaliates -- a farmer by stealing cattle
from the pastoralists' herds, a pastoralist by grazing on more farmland
-- the scope of the conflict can rapidly expand. The farmer whose crops
were destroyed by cattle does not know which herd grazed on his land;
cattle he steals in revenge do not necessarily come from the
transgressing herd. Pastoralists, likewise, do not know which farmer
stole their cattle; the crops they destroy in revenge do not necessarily
come from the transgressing farm. From there, a little war often breaks
out. As one reporter noted, ``The countryside is littered with the
charred ruins of homes, schools, police stations, mosques and
churches.'' (McDonnel 2017). In one case I witnessed, a farmer took
revenge against cattle grazing on his farmland by poisoning the crop
residue left on his fields after harvest. After grazing on the residue,
the cattle of dozens of pastoralists became sick and died. More violence
followed.

The land conflict is exacerbated by ethnic and religious differences
between groups, which feeds mutual distrust. The pastoralists are almost
all from the Fulani ethnic group and practice Islam; the farmers are
from a non-Fulani ethnic groups and practice Islam and Christianity,
though the violence is worst where the farmers are homogeneously
Christian. Each group sees the other as biased towards their own side
for economic, cultural, and religious reasons. Each group also sees
their way of life as superior. Farmers see nomadic life as outdated,
backwards, and anti-progress; the pastoralists think that sedentary
farming makes one weak. One pastoralist commented to me that if he
dropped off a sedentary adult and his young child in the forest, the
sedentary adult would depend on his child to survive.

Despite their cultural differences and competition for scarce land,
mutual complementarity remains. Pastoralists still have animal products
-- though more farmers have bought animals in recent years, pastoralists
still control roughly 90\% of Nigeria's livestock (Herders against
farmers 2017) -- and farmers still have tubers, vegetables, grains, and
the resultant crop residue. The violence is extremely costly to both
sides, so both have an incentive to avoid conflict. Community leaders
recognize that peace is in the interest of their communities, but many
have been unable to prevent the violence. In interviews conducted in
2016 and 2019 by C. Grady (2020), community leaders from farming and
pastoral communities expressed their desire for peace between the two
groups and blamed deviants from the other side for ongoing violence.
Farmers argue that the local pastoralists do not prevent other
pastoralists who migrate through from destroying cropland, and
pastoralists argue that farmers on the outskirts of the farming village
encroach on grazing routes more each year.

Peace between many farming and pastoral groups is prevented by a lack of
trust. Both sides are better off cooperating peacefully than fighting,
but neither side trusts the other to honor agreements that would prevent
fighting. This is a classic commitment problem. The reasons each group
does not trust the other side, however, is not that they believe the
other side doesn't desire peace, but rather that the farming group
cannot credibly commit to preventing all farmers from expanding into
grazing lands, and the pastoral group cannot credibly commit to
preventing all herders from grazing cattle on croplands. Their
between-group commitment problem is driven by a within-group collective
action problem because neither group trusts the other to punish their
own.

\hypertarget{resolving-farmer-pastoralist-conflict}{%
\subsection{Resolving farmer-pastoralist
conflict}\label{resolving-farmer-pastoralist-conflict}}

Though conflict between farming and pastoral groups is common, not all
farming and pastoral groups are in conflict. Communities that have
successfully navigated farmer-pastoralist tension generally have a
noticeable commonality: an institution comprised of farmers and
pastoralists to handle cases that threaten intergroup relations. This
institution can tap into ingroup networks to identify transgressors from
each side, much like an ingroup policing institution. Since both groups
are represented, however, neither the plaintiff, the defendant, nor
their group can credibly claim that the case was decided unfairly due to
group bias.

These joint institutions function much like trial by jury.
Representatives from each group meet, hear from the case's ``plaintiff''
and ``defendant'', and decide an appropriate punishment. For common
occurrences -- crop damage, cattle rustling, and the like -- there may
also be a code of laws: predecided penalties, such as a set cost per
acre of cropland destroyed or head of cattle stolen. In past decades,
agreements between the traditional leaders of each community function
similarly, by setting compensation for specific actions. Conflict is
often blamed on the breakdown of such institutions (Cotula et al. 2004;
Kuusaana and Bukari 2015; Tonah 2002).

C. Grady (2020) evaluated a contact-based intervention that increased
trust between farmers and pastoralists. Though they do not measure the
extent to which the intervention affected institutional structures that
govern intergroup relations, they describe farmers and pastoralists who
had participated in the contact intervention meeting to jointly decide
an appropriate punishment for vigilante farmers who intended to harm
pastoralists. The would-be perpetrators, hailing from a nearby village,
were noticed and arrested by the hometown farmers while on their way to
steal the pastoralists' cattle. Rather than decide unilaterally how to
punish the vigilantes, leaders from the farming community called in
leaders from the pastoral community to jointly decide a punishment.
Together the groups decided to disarm the vigilantes and let them go
free without further punishment -- a solution proposed by the
\emph{pastoralists}. Had the farmers unilaterally decided to let the
vigilantes go free, the pastoralists may have interpreted the punishment
as too lenient and accused the farmers of bias. Since the pastoralists
had a say in the decision-making process, however, the group's were able
to build trust through cooperation. The pastoralists appreciated the
farmers calling a joint meeting, and the farmers appreciated the
pastoralists magnanimity in proposing a lenient punishment.

C. Grady (2020) also describe a contrasting situation in which a farmers
and pastoralists who had not participated in the contact intervention
failed due to the lack of any joint institutions. In that situation,
both sides were participating in a public goods game in which money
raised would be jointly administered by both groups. These groups had no
preexisting structure to handle situations that concerned both sides,
and neither side trusted the other to hold the money. The pastoralists
claimed that the money would be squandered by the corruption of the
farming community's leader if a farmer held it; the farmers claimed that
the pastoralists would migrate away with the money if a pastoralist held
it. In the end, the groups agreed that the NGO administering the public
goods game would hold the money and disburse it in chunks when the
pastoralist leader and the farming leader agreed on its use.

{[}paragraph saying something specific about contact interventions? Or
just short summary paragraph.{]}

\hypertarget{policy-implications-for-peacebuilding-programs}{%
\section{Policy Implications for Peacebuilding
Programs}\label{policy-implications-for-peacebuilding-programs}}

\begin{itemize}
\item
  achieving a joint goal \emph{that could not be achieved alone}.

  \begin{itemize}
  \tightlist
  \item
    comparative advantage of each group.
  \end{itemize}
\item
  Identify whether conflict is materially irrational? Think about
  conflict as deviation from rationality and how to demonstrate the
  rationality of peace to each group.
\item
  Think about incentives of individual group members, not just groups as
  a whole.
\item
  Do not try to solve groups conflict externally; help groups solve own
  problems.
\item
  Programs that demonstrate material rationality of peace. Contact, what
  about others?
\item
  Important for both groups to benefit equally. Otherwise creating a
  power disparity. Favoring one group probably makes things worse.
\item
  Reduce psych biases
\item
  Groups need strength to compel group members.
\item
  Opportunities to observe other side, signal C.
\item
  Contact programs \emph{can} do these things, but others may also. More
  contact, if mismanaged, runs risk of exacerbating conflict.
\item
  Joint punishment institutions: legitimacy of institutions paramount.
\end{itemize}

\hypertarget{avenues-for-future-research}{%
\section{Avenues for Future
Research}\label{avenues-for-future-research}}

\begin{itemize}
\item
  Test contact programs with groups in conflict and with history of
  conflict. Look for collaborative institutions.
\item
  Compare contact programs that do and do not (1) work to achieve goal,
  (2) actually achieve it.
\item
  Best means to demonstrate material rationality of peace. Trade,
  differentiated groups?
\item
  Best means to reduce psych/cognitive biases
\item
  Need for enforcement at onset to incentivize Cooperation?
\item
  Role of elites: contact says they must support
\item
  Difficulty with very decentralized groups
\item
  Power disparities between groups
\item
  Contact's differential effects with these power disparities.
\end{itemize}

\hypertarget{references}{%
\section*{References}\label{references}}
\addcontentsline{toc}{section}{References}

\hypertarget{refs}{}
\leavevmode\hypertarget{ref-abbass2012no}{}%
Abbass, Isah Mohammed. 2012. ``No Retreat No Surrender: Conflict for
Survival Between Fulani Pastoralists and Farmers in Northern Nigeria.''
\emph{European Scientific Journal} 8(1): 331--46.

\leavevmode\hypertarget{ref-nyt2018nigeria}{}%
Akinwotu, Emmanuel. 2018. ``Nigeria's Farmers and Herders Fight a Deadly
Battle for Scarce Resources.'' \emph{New York Times}.
\url{https://www.nytimes.com/2018/06/25/world/africa/nigeria-herders-farmers.html}.

\leavevmode\hypertarget{ref-allport1954prejudice}{}%
Allport, Gordon. 1954. ``The Nature of Prejudice.'' \emph{Garden City,
NJ Anchor}.

\leavevmode\hypertarget{ref-arkes1985psychology}{}%
Arkes, Hal R, and Catherine Blumer. 1985. ``The Psychology of Sunk
Cost.'' \emph{Organizational behavior and human decision processes}
35(1): 124--40.

\leavevmode\hypertarget{ref-axelrod1986evolutionary}{}%
Axelrod, Robert. 1986. ``An Evolutionary Approach to Norms.''
\emph{American political science review} 80(4): 1095--1111.

\leavevmode\hypertarget{ref-axelrod1981evolution}{}%
Axelrod, Robert, and William D Hamilton. 1981. ``The Evolution of
Cooperation.'' \emph{science} 211(4489): 1390--6.

\leavevmode\hypertarget{ref-bandura1999moral}{}%
Bandura, Albert. 1999. ``Moral Disengagement in the Perpetration of
Inhumanities.'' \emph{Personality and social psychology review} 3(3):
193--209.

\leavevmode\hypertarget{ref-bandura2014social}{}%
---------. 2014. ``Social Cognitive Theory of Moral Thought and
Action.'' In \emph{Handbook of Moral Behavior and Development},
Psychology Press, 69--128.

\leavevmode\hypertarget{ref-bar2007sociopsychological}{}%
Bar-Tal, Daniel. 2007. ``Sociopsychological Foundations of Intractable
Conflicts.'' \emph{American Behavioral Scientist} 50(11): 1430--53.

\leavevmode\hypertarget{ref-batson1997empathy}{}%
Batson, C Daniel et al. 1997. ``Empathy and Attitudes: Can Feeling for a
Member of a Stigmatized Group Improve Feelings Toward the Group?''
\emph{Journal of personality and social psychology} 72(1): 105.

\leavevmode\hypertarget{ref-beardsley2008agreement}{}%
Beardsley, Kyle. 2008. ``Agreement Without Peace? International
Mediation and Time Inconsistency Problems.'' \emph{American journal of
political science} 52(4): 723--40.

\leavevmode\hypertarget{ref-benartzi2013behavioral}{}%
Benartzi, Shlomo, and Richard H Thaler. 2013. ``Behavioral Economics and
the Retirement Savings Crisis.'' \emph{Science} 339(6124): 1152--3.

\leavevmode\hypertarget{ref-bendor1991doubt}{}%
Bendor, Jonathan, Roderick M Kramer, and Suzanne Stout. 1991. ``When in
Doubt... Cooperation in a Noisy Prisoner's Dilemma.'' \emph{Journal of
conflict resolution} 35(4): 691--719.

\leavevmode\hypertarget{ref-bornstein1992free}{}%
Bornstein, Gary. 1992. ``The Free-Rider Problem in Intergroup Conflicts
over Step-Level and Continuous Public Goods.'' \emph{Journal of
Personality and Social Psychology} 62(4): 597.

\leavevmode\hypertarget{ref-bornstein2003intergroup}{}%
---------. 2003. ``Intergroup Conflict: Individual, Group, and
Collective Interests.'' \emph{Personality and social psychology review}
7(2): 129--45.

\leavevmode\hypertarget{ref-bornstein1989within}{}%
Bornstein, Gary, Amnon Rapoport, Lucia Kerpel, and Tani Katz. 1989.
``Within-and Between-Group Communication in Intergroup Competition for
Public Goods.'' \emph{Journal of Experimental Social Psychology} 25(5):
422--36.

\leavevmode\hypertarget{ref-bratt1996assessing}{}%
Bratt, Duane. 1996. ``Assessing the Success of Un Peacekeeping
Operations.'' \emph{International Peacekeeping} 3(4): 64--81.

\leavevmode\hypertarget{ref-broockman2016durably}{}%
Broockman, David, and Joshua Kalla. 2016. ``Durably Reducing
Transphobia: A Field Experiment on Door-to-Door Canvassing.''
\emph{Science} 352(6282): 220--24.

\leavevmode\hypertarget{ref-brown2019group}{}%
Brown, Rupert, and Samuel Pehrson. 2019. \emph{Group Processes: Dynamics
Within and Between Groups}. John Wiley \& Sons.

\leavevmode\hypertarget{ref-bryan2019values}{}%
Bryan, Christopher J, David S Yeager, and Cintia P Hinojosa. 2019. ``A
Values-Alignment Intervention Protects Adolescents from the Effects of
Food Marketing.'' \emph{Nature human behaviour} 3(6): 596--603.

\leavevmode\hypertarget{ref-cikara2014their}{}%
Cikara, Mina, Emile Bruneau, Jay J Van Bavel, and Rebecca Saxe. 2014.
``Their Pain Gives Us Pleasure: How Intergroup Dynamics Shape Empathic
Failures and Counter-Empathic Responses.'' \emph{Journal of experimental
social psychology} 55: 110--25.

\leavevmode\hypertarget{ref-converse1970attitudes}{}%
Converse, Philip E. 1970. ``Attitudes and Non-Attitudes: Continuation of
a Dialogue.'' \emph{The quantitative analysis of social problems} 168:
189.

\leavevmode\hypertarget{ref-cotula2004land}{}%
Cotula, Lorenzo, Camilla Toulmin, Ced Hesse, and others. 2004.
\emph{Land Tenure and Administration in Africa: Lessons of Experience
and Emerging Issues}. International Institute for Environment;
Development London.

\leavevmode\hypertarget{ref-daniel2018anti}{}%
Daniel, Soni. 2018. ``Anti-Open Grazing Law: Nass, Benue, Kwara, Taraba
Tackle Defence Minister.'' \emph{Vanguard}.
\url{https://www.vanguardngr.com/2018/06/anti-open-grazing-law-nass-benue-kwara-taraba-tackle-defence-minister/}.

\leavevmode\hypertarget{ref-de2008motivated}{}%
De Dreu, Carsten KW, Bernard A Nijstad, and Daan van Knippenberg. 2008.
``Motivated Information Processing in Group Judgment and Decision
Making.'' \emph{Personality and Social Psychology Review} 12(1): 22--49.

\leavevmode\hypertarget{ref-de2009effects}{}%
De Sanctis, Luca, and Tobias Galla. 2009. ``Effects of Noise and
Confidence Thresholds in Nominal and Metric Axelrod Dynamics of Social
Influence.'' \emph{Physical Review E} 79(4): 046108.

\leavevmode\hypertarget{ref-di2017effectiveness}{}%
Di Salvatore, Jessica, and Andrea Ruggeri. 2017. ``Effectiveness of
Peacekeeping Operations.'' \emph{Oxford Research Encyclopedia of
Politics}.

\leavevmode\hypertarget{ref-doyle2000international}{}%
Doyle, Michael W, and Nicholas Sambanis. 2000. ``International
Peacebuilding: A Theoretical and Quantitative Analysis.'' \emph{American
political science review} 94(4): 779--801.

\leavevmode\hypertarget{ref-doyle2006making}{}%
---------. 2006. \emph{Making War and Building Peace: United Nations
Peace Operations}. Princeton University Press.

\leavevmode\hypertarget{ref-duncan1976differential}{}%
Duncan, Birt L. 1976. ``Differential Social Perception and Attribution
of Intergroup Violence: Testing the Lower Limits of Stereotyping of
Blacks.'' \emph{Journal of personality and social psychology} 34(4):
590.

\leavevmode\hypertarget{ref-fearon1994ethnic}{}%
Fearon, James D. 1994. ``Ethnic War as a Commitment Problem.'' In
\emph{Annual Meetings of the American Political Science Association},
2--5.

\leavevmode\hypertarget{ref-fearon1995rationalist}{}%
---------. 1995. ``Rationalist Explanations for War.''
\emph{International organization} 49(3): 379--414.

\leavevmode\hypertarget{ref-fearon1996explaining}{}%
Fearon, James D, and David D Laitin. 1996. ``Explaining Interethnic
Cooperation.'' \emph{American political science review} 90(4): 715--35.

\leavevmode\hypertarget{ref-fehr2002social}{}%
Fehr, Ernst, and Urs Fischbacher. 2002. ``Why Social Preferences
Matter--the Impact of Non-Selfish Motives on Competition, Cooperation
and Incentives.'' \emph{The economic journal} 112(478): C1--C33.

\leavevmode\hypertarget{ref-fein1997prejudice}{}%
Fein, Steven, and Steven J Spencer. 1997. ``Prejudice as Self-Image
Maintenance: Affirming the Self Through Derogating Others.''
\emph{Journal of personality and Social Psychology} 73(1): 31.

\leavevmode\hypertarget{ref-festinger1962cognitiveDissonance}{}%
Festinger, Leon. 1962. 2 \emph{A Theory of Cognitive Dissonance}.
Stanford university press.

\leavevmode\hypertarget{ref-fey2010shuttle}{}%
Fey, Mark, and Kristopher W Ramsay. 2010. ``When Is Shuttle Diplomacy
Worth the Commute? Information Sharing Through Mediation.'' \emph{World
Politics} 62(4): 529--60.

\leavevmode\hypertarget{ref-gaertner1993common}{}%
Gaertner, Samuel L et al. 1993. ``The Common Ingroup Identity Model:
Recategorization and the Reduction of Intergroup Bias.'' \emph{European
review of social psychology} 4(1): 1--26.

\leavevmode\hypertarget{ref-gambetta_ch13}{}%
Gambetta, Diego, and others. 2000. ``Can We Trust Trust.'' \emph{Trust:
Making and breaking cooperative relations} 13: 213--37.

\leavevmode\hypertarget{ref-gartner2011signs}{}%
Gartner, Scott Sigmund. 2011. ``Signs of Trouble: Regional Organization
Mediation and Civil War Agreement Durability.'' \emph{The Journal of
Politics} 73(2): 380--90.

\leavevmode\hypertarget{ref-gilligan2008interventions}{}%
Gilligan, Michael J, Ernest J Sergenti, and others. 2008. ``Do Un
Interventions Cause Peace? Using Matching to Improve Causal Inference.''
\emph{Quarterly Journal of Political Science} 3(2): 89--122.

\leavevmode\hypertarget{ref-gneezy2012conflict}{}%
Gneezy, Ayelet, and Daniel MT Fessler. 2012. ``Conflict, Sticks and
Carrots: War Increases Prosocial Punishments and Rewards.''
\emph{Proceedings of the Royal Society B: Biological Sciences}
279(1727): 219--23.

\leavevmode\hypertarget{ref-good2000individuals}{}%
Good, David. 2000. ``Individuals, Interpersonal Relations, and Trust.''
\emph{Trust: Making and breaking cooperative relations}: 31--48.

\leavevmode\hypertarget{ref-grady2020farmer}{}%
Grady, Christopher. 2020. ``Promoting Peace Amidst Group Conflict: An
Intergroup Contact Field Experiment in Nigeria.'' PhD thesis. University
of Illinois.

\leavevmode\hypertarget{ref-grady2020lab}{}%
Grady, Christopher D. 2020. ``Contact Itself or Contact's Success? How
Intergroup Contact Improves Attitudes.'' PhD thesis. University of
Illinois.

\leavevmode\hypertarget{ref-greig2005stepping}{}%
Greig, J Michael. 2005. ``Stepping into the Fray: When Do Mediators
Mediate?'' \emph{American Journal of Political Science} 49(2): 249--66.

\leavevmode\hypertarget{ref-gubler2013humanizing}{}%
Gubler, Joshua R. 2013. ``When Humanizing the Enemy Fails: The Role of
Dissonance and Justification in Intergroup Conflict.'' In \emph{Annual
Meeting of the American Political Science Association},

\leavevmode\hypertarget{ref-frontera2018nigeria}{}%
Hailemariam, Adium. 2018. ``Nigeria: Violence in the Middle Belt Becomes
Major Concern for President Buhari.'' \emph{Frontera}.
\url{https://frontera.net/news/africa/nigeria-violence-in-the-middle-belt-becomes-major-concern-for-president-buhari/}.

\leavevmode\hypertarget{ref-halevy2010relative}{}%
Halevy, Nir, Eileen Y Chou, Taya R Cohen, and Gary Bornstein. 2010.
``Relative Deprivation and Intergroup Competition.'' \emph{Group
Processes \& Intergroup Relations} 13(6): 685--700.

\leavevmode\hypertarget{ref-hartzell2001stabilizing}{}%
Hartzell, Caroline, Matthew Hoddie, and Donald Rothchild. 2001.
``Stabilizing the Peace After Civil War: An Investigation of Some Key
Variables.'' \emph{International organization} 55(1): 183--208.

\leavevmode\hypertarget{ref-council2019nigeria}{}%
Harwood, Asch. 2019. ``Update: The Numbers Behind Sectarian Violence in
Nigeria.'' \emph{Council on Foreign Relations}.
\url{https://www.cfr.org/blog/update-numbers-behind-sectarian-violence-nigeria}.

\leavevmode\hypertarget{ref-icg2017nigeria}{}%
\emph{Herders Against Farmers: Nigeria's Expanding Deadly Conflict}.
2017. International Crisis Group.

\leavevmode\hypertarget{ref-hewstone1990ultimate}{}%
Hewstone, Miles. 1990. ``The `Ultimate Attribution Error'? A Review of
the Literature on Intergroup Causal Attribution.'' \emph{European
Journal of Social Psychology} 20(4): 311--35.

\leavevmode\hypertarget{ref-hewstone2006intergroup}{}%
Hewstone, Miles et al. 2006. ``Intergroup Contact, Forgiveness, and
Experience of `the Troubles' in Northern Ireland.'' \emph{Journal of
Social Issues} 62(1): 99--120.

\leavevmode\hypertarget{ref-fa2019deadly}{}%
Ilo, Udo Jude, Jonathan-Ichavar Ier, and Yemi Adamolekun. 2019. ``The
Deadliest Conflict You've Never Heard of: Nigeria's Cattle Herders and
Farmers Wage a Resource War.'' \emph{Foreign Affairs}.
\url{https://www.foreignaffairs.com/articles/nigeria/2019-01-23/deadliest-conflict-youve-never-heard}.

\leavevmode\hypertarget{ref-jervis2017perception}{}%
Jervis, Robert. 2017. \emph{Perception and Misperception in
International Politics: New Edition}. Princeton University Press.

\leavevmode\hypertarget{ref-johnson2009overconfidence}{}%
Johnson, Dominic DP. 2009. \emph{Overconfidence and War}. Harvard
University Press.

\leavevmode\hypertarget{ref-kahneman1992reference}{}%
Kahneman, Daniel. 1992. ``Reference Points, Anchors, Norms, and Mixed
Feelings.'' \emph{Organizational behavior and human decision processes}
51(2): 296--312.

\leavevmode\hypertarget{ref-kahneman2013prospect}{}%
Kahneman, Daniel, and Amos Tversky. 2013. ``Prospect Theory: An Analysis
of Decision Under Risk.'' In \emph{Handbook of the Fundamentals of
Financial Decision Making: Part I}, World Scientific, 99--127.

\leavevmode\hypertarget{ref-kazdin1974covertModeling}{}%
Kazdin, Alan E. 1974. ``Covert Modeling, Model Similarity, and Reduction
of Avoidance Behavior.'' \emph{Behavior Therapy} 5(3): 325--40.

\leavevmode\hypertarget{ref-kunda1990motivatedReasoning}{}%
Kunda, Ziva. 1990. ``The Case for Motivated Reasoning.''
\emph{Psychological bulletin} 108(3): 480.

\leavevmode\hypertarget{ref-kuusaana2015land}{}%
Kuusaana, Elias Danyi, and Kaderi Noagah Bukari. 2015. ``Land Conflicts
Between Smallholders and Fulani Pastoralists in Ghana: Evidence from the
Asante Akim North District (Aand).'' \emph{Journal of rural studies} 42:
52--62.

\leavevmode\hypertarget{ref-kydd2000trust}{}%
Kydd, Andrew. 2000. ``Trust, Reassurance, and Cooperation.''
\emph{International Organization} 54(2): 325--57.

\leavevmode\hypertarget{ref-kydd2006can}{}%
Kydd, Andrew H. 2006. ``When Can Mediators Build Trust?'' \emph{American
Political Science Review} 100(3): 449--62.

\leavevmode\hypertarget{ref-maoz2002reactive}{}%
Maoz, Ifat, Andrew Ward, Michael Katz, and Lee Ross. 2002. ``Reactive
Devaluation of an `Israeli' Vs.`Palestinian' Peace Proposal.''
\emph{Journal of Conflict Resolution} 46(4): 515--46.

\leavevmode\hypertarget{ref-martiny2012shall}{}%
Martiny, Sarah E, Thomas Kessler, and Vivian L Vignoles. 2012. ``Shall I
Leave or Shall We Fight? Effects of Threatened Group-Based Self-Esteem
on Identity Management Strategies.'' \emph{Group Processes \& Intergroup
Relations} 15(1): 39--55.

\leavevmode\hypertarget{ref-mazar2008dishonesty}{}%
Mazar, Nina, On Amir, and Dan Ariely. 2008. ``The Dishonesty of Honest
People: A Theory of Self-Concept Maintenance.'' \emph{Journal of
marketing research} 45(6): 633--44.

\leavevmode\hypertarget{ref-mcdonnel2017graze}{}%
McDonnel, Tim. 2017. ``Why It's Now a Crime to Let Cattle Graze Freely
in 2 Nigerian States.'' \emph{National Public Radio (NPR)}.
\url{https://www.npr.org/sections/goatsandsoda/2017/12/12/569913821/why-its-now-a-crime-to-let-cattle-graze-freely-in-2-nigerian-states}.

\leavevmode\hypertarget{ref-mcdougal2015effect}{}%
McDougal, Topher L et al. 2015. ``The Effect of Farmer-Pastoralist
Violence on Income: New Survey Evidence from Nigeria's Middle Belt
States.'' \emph{Economics of Peace and Security Journal} 10(1): 54--65.

\leavevmode\hypertarget{ref-millard2018testing}{}%
Millard, Matt, and Chase Porter. 2018. ``Testing the Hard Case: The
Psychological Roots of Reactive Devaluation and the Iranian Nuclear
Deal.'' \emph{Journal of Political Science} 46.

\leavevmode\hypertarget{ref-nickerson1998confirmation}{}%
Nickerson, Raymond S. 1998. ``Confirmation Bias: A Ubiquitous Phenomenon
in Many Guises.'' \emph{Review of general psychology} 2(2): 175--220.

\leavevmode\hypertarget{ref-okpara2015conflicts}{}%
Okpara, Uche T, Lindsay C Stringer, Andrew J Dougill, and Mohammed D
Bila. 2015. ``Conflicts About Water in Lake Chad: Are Environmental,
Vulnerability and Security Issues Linked?'' \emph{Progress in
Development Studies} 15(4): 308--25.

\leavevmode\hypertarget{ref-ostrom2006value}{}%
Ostrom, Elinor. 2006. ``The Value-Added of Laboratory Experiments for
the Study of Institutions and Common-Pool Resources.'' \emph{Journal of
Economic Behavior \& Organization} 61(2): 149--63.

\leavevmode\hypertarget{ref-ostrom2003trust}{}%
Ostrom, Elinor, and James Walker. 2003. \emph{Trust and Reciprocity:
Interdisciplinary Lessons for Experimental Research}. Russell Sage
Foundation.

\leavevmode\hypertarget{ref-page2008little}{}%
Page-Gould, Elizabeth, Rodolfo Mendoza-Denton, and Linda R Tropp. 2008.
``With a Little Help from My Cross-Group Friend: Reducing Anxiety in
Intergroup Contexts Through Cross-Group Friendship.'' \emph{Journal of
personality and social psychology} 95(5): 1080.

\leavevmode\hypertarget{ref-paolini2010negative}{}%
Paolini, Stefania, Jake Harwood, and Mark Rubin. 2010. ``Negative
Intergroup Contact Makes Group Memberships Salient: Explaining Why
Intergroup Conflict Endures.'' \emph{Personality and Social Psychology
Bulletin} 36(12): 1723--38.

\leavevmode\hypertarget{ref-pettigrew2008does}{}%
Pettigrew, Thomas F, and Linda R Tropp. 2008. ``How Does Intergroup
Contact Reduce Prejudice? Meta-Analytic Tests of Three Mediators.''
\emph{European Journal of Social Psychology} 38(6): 922--34.

\leavevmode\hypertarget{ref-powell2006war}{}%
Powell, Robert. 2006. ``War as a Commitment Problem.''
\emph{International organization} 60(1): 169--203.

\leavevmode\hypertarget{ref-reed2016bargaining}{}%
Reed, William, David Clark, Timothy Nordstrom, and Daniel Siegel. 2016.
``Bargaining in the Shadow of a Commitment Problem.'' \emph{Research \&
Politics} 3(3): 2053168016666848.

\leavevmode\hypertarget{ref-rohner2013war}{}%
Rohner, Dominic, Mathias Thoenig, and Fabrizio Zilibotti. 2013. ``War
Signals: A Theory of Trade, Trust, and Conflict.'' \emph{Review of
Economic Studies} 80(3): 1114--47.

\leavevmode\hypertarget{ref-ross1991barriers}{}%
Ross, Lee, and Constance Stillinger. 1991. ``Barriers to Conflict
Resolution.'' \emph{Negotiation journal} 7(4): 389--404.

\leavevmode\hypertarget{ref-ross1995psychological}{}%
Ross, Lee, and Andrew Ward. 1995. ``Psychological Barriers to Dispute
Resolution.'' In \emph{Advances in Experimental Social Psychology},
Elsevier, 255--304.

\leavevmode\hypertarget{ref-sambanis2013social}{}%
Sambanis, Nicholas, and Moses Shayo. 2013. ``Social Identification and
Ethnic Conflict.'' \emph{American Political Science Review} 107(2):
294--325.

\leavevmode\hypertarget{ref-schwartz1990toward}{}%
Schwartz, Shalom H, and Wolfgang Bilsky. 1990. ``Toward a Theory of the
Universal Content and Structure of Values: Extensions and Cross-Cultural
Replications.'' \emph{Journal of personality and social psychology}
58(5): 878.

\leavevmode\hypertarget{ref-tajfel1969cognitive}{}%
Tajfel, Henri. 1969. ``Cognitive Aspects of Prejudice.'' \emph{Journal
of biosocial science} 1(S1): 173--91.

\leavevmode\hypertarget{ref-tajfel1981groups}{}%
---------. 1981. \emph{Human Groups and Social Categories: Studies in
Social Psychology}. CUP Archive.

\leavevmode\hypertarget{ref-tajfel1979integrative}{}%
Tajfel, Henri, and John C Turner. 1979. ``An Integrative Theory of
Intergroup Conflict.'' \emph{The social psychology of intergroup
relations} 33(47): 74.

\leavevmode\hypertarget{ref-tavris2008mistakes}{}%
Tavris, Carol, and Elliot Aronson. 2008. \emph{Mistakes Were Made (but
Not by Me): Why We Justify Foolish Beliefs, Bad Decisions, and Hurtful
Acts}. Houghton Mifflin Harcourt.

\leavevmode\hypertarget{ref-thebaud2001sahel}{}%
Thébaud, Brigitte, and Simon Batterbury. 2001. ``Sahel Pastoralists:
Opportunism, Struggle, Conflict and Negotiation. A Case Study from
Eastern Niger.'' \emph{Global environmental change} 11(1): 69--78.

\leavevmode\hypertarget{ref-thomas2018sahara}{}%
Thomas, Natalie, and Sumant Nigam. 2018. ``Twentieth-Century Climate
Change over Africa: Seasonal Hydroclimate Trends and Sahara Desert
Expansion.'' \emph{Journal of Climate} 31(9): 3349--70.

\leavevmode\hypertarget{ref-tonah2002fulani}{}%
Tonah, Steve. 2002. ``Fulani Pastoralists, Indigenous Farmers and the
Contest for Land in Northern Ghana.'' \emph{Africa Spectrum}: 43--59.

\leavevmode\hypertarget{ref-turner1979social}{}%
Turner, John C, Rupert J Brown, and Henri Tajfel. 1979. ``Social
Comparison and Group Interest in Ingroup Favouritism.'' \emph{European
journal of social psychology} 9(2): 187--204.

\leavevmode\hypertarget{ref-unah2018nigeria}{}%
Unah, Linus. 2018. ``In Nigeria's Diverse Middle Belt, a Drying
Landscape Deepens Violent Divides.'' \emph{Christian Science Minitor}.

\leavevmode\hypertarget{ref-vallone1985hostileMedia}{}%
Vallone, Robert P, Lee Ross, and Mark R Lepper. 1985. ``The Hostile
Media Phenomenon: Biased Perception and Perceptions of Media Bias in
Coverage of the Beirut Massacre.'' \emph{Journal of personality and
social psychology} 49(3): 577.

\leavevmode\hypertarget{ref-wallensteen2014talking}{}%
Wallensteen, Peter, and Isak Svensson. 2014. ``Talking Peace:
International Mediation in Armed Conflicts.'' \emph{Journal of Peace
Research} 51(2): 315--27.

\leavevmode\hypertarget{ref-walter2002committing}{}%
Walter, Barbara F. 2002. \emph{Committing to Peace: The Successful
Settlement of Civil Wars}. Princeton University Press.

\leavevmode\hypertarget{ref-waltz2010theory}{}%
Waltz, Kenneth N. 2010. \emph{Theory of International Politics}.
Waveland Press.

\leavevmode\hypertarget{ref-ward1997naive}{}%
Ward, Andrew et al. 1997. ``Naive Realism in Everyday Life: Implications
for Social Conflict and Misunderstanding.'' \emph{Values and knowledge}:
103--35.

\leavevmode\hypertarget{ref-warner2001personal}{}%
Warner, John T, and Saul Pleeter. 2001. ``The Personal Discount Rate:
Evidence from Military Downsizing Programs.'' \emph{American Economic
Review} 91(1): 33--53.

\leavevmode\hypertarget{ref-weinstein2005autonomous}{}%
Weinstein, Jeremy M. 2005. ``Autonomous Recovery and International
Intervention in Comparative Perspective.'' \emph{Available at SSRN
1114117}.

\leavevmode\hypertarget{ref-weisel2015ingroup}{}%
Weisel, Ori, and Robert Böhm. 2015. ```Ingroup Love' and `Outgroup Hate'
in Intergroup Conflict Between Natural Groups.'' \emph{Journal of
experimental social psychology} 60: 110--20.

\leavevmode\hypertarget{ref-wood2000attitude}{}%
Wood, Wendy. 2000. ``Attitude Change: Persuasion and Social Influence.''
\emph{Annual review of psychology} 51(1): 539--70.

\leavevmode\hypertarget{ref-wu1995cope}{}%
Wu, Jianzhong, and Robert Axelrod. 1995. ``How to Cope with Noise in the
Iterated Prisoner's Dilemma.'' \emph{Journal of Conflict resolution}
39(1): 183--89.

\leavevmode\hypertarget{ref-yang2013similarity}{}%
Yang, Nianhua. 2013. ``A Similarity Based Trust and Reputation
Management Framework for Vanets.'' \emph{International Journal of Future
Generation Communication and Networking} 6(2): 25--34.

\end{document}
