%\documentclass[]{article}
\documentclass[11pt]{article}
\usepackage[usenames,dvipsnames]{xcolor}

\usepackage[T1]{fontenc}
%\usepackage{lmodern}
\usepackage{tgtermes}
\usepackage{amssymb,amsmath}
%\usepackage[margin=1in]{geometry}
\usepackage[letterpaper,bottom=1in,top=1in,right=1.25in,left=1.25in,includemp=FALSE]{geometry}
\usepackage{pdfpages}
\usepackage[small]{caption}
\usepackage{subcaption}
\usepackage{graphicx}
\usepackage{multirow}

\usepackage{ifxetex,ifluatex}
\usepackage{fixltx2e} % provides \textsubscript
% use microtype if available
\IfFileExists{microtype.sty}{\usepackage{microtype}}{}
\ifnum 0\ifxetex 1\fi\ifluatex 1\fi=0 % if pdftex
\usepackage[utf8]{inputenc}
\else % if luatex or xelatex
\usepackage{fontspec}
\ifxetex
\usepackage{xltxtra,xunicode}
\fi
\defaultfontfeatures{Mapping=tex-text,Scale=MatchLowercase}
\newcommand{\euro}{€}
\fi
%

\usepackage{fancyvrb}

\usepackage{ctable,longtable}

\usepackage[section]{placeins}
\usepackage{float} % provides the H option for float placement
\restylefloat{figure}
\usepackage{dcolumn} % allows for different column alignments
\newcolumntype{.}{D{.}{.}{1.2}}

\usepackage{booktabs} % nicer horizontal rules in tables

%Assume we want graphics always
%\usepackage{graphicx}
% We will generate all images so they have a width \maxwidth. This means
% that they will get their normal width if they fit onto the page, but
% are scaled down if they would overflow the margins.
%% \makeatletter
%% \def\maxwidth{\ifdim\Gin@nat@width>\linewidth\linewidth
%%   \else\Gin@nat@width\fi}
%% \makeatother
%% \let\Oldincludegraphics\includegraphics
%% \renewcommand{\includegraphics}[1]{\Oldincludegraphics[width=\maxwidth]{#1}}
%%\graphicspath{{.}{../Soccom_Code/socom_2013/}}


%% \ifxetex
%% \usepackage[pagebackref=true, setpagesize=false, % page size defined by xetex
%% unicode=false, % unicode breaks when used with xetex
%% xetex]{hyperref}
%% \else
\usepackage[pagebackref=true, unicode=true, bookmarks=true, pdftex]{hyperref}
% \fi


\hypersetup{breaklinks=true,
  bookmarks=true,
  pdfauthor={Christopher Grady},
  pdftitle={Group Conflict and Collective Action: Poli Econ Review},
  colorlinks=true,
  linkcolor=BrickRed,
  citecolor=blue, %MidnightBlue,
  urlcolor=BrickRed,
  % urlcolor=blue,
  % linkcolor=magenta,
  pdfborder={0 0 0}}

%\setlength{\parindent}{0pt}
%\setlength{\parskip}{6pt plus 2pt minus 1pt}
\usepackage{parskip}
\setlength{\emergencystretch}{3em}  % prevent overfull lines
\providecommand{\tightlist}{%
  \setlength{\itemsep}{0pt}\setlength{\parskip}{0pt}}

%% Insist on this.
\setcounter{secnumdepth}{2}

\VerbatimFootnotes % allows verbatim text in footnotes

\title{Group Conflict and Collective Action: Poli Econ Review}

\author{
Christopher Grady
}


\date{June 15, 2020}


\usepackage{versions}
\makeatletter
\renewcommand*\versionmessage[2]{\typeout{*** `#1' #2. ***}}
\renewcommand*\beginmarkversion{\sffamily}
  \renewcommand*\endmarkversion{}
\makeatother

\excludeversion{comment}

%\usepackage[margins=1in]{geometry}

\usepackage[compact,bottomtitles]{titlesec}
%\titleformat{ ⟨command⟩}[⟨shape⟩]{⟨format⟩}{⟨label⟩}{⟨sep⟩}{⟨before⟩}[⟨after⟩]
\titleformat{\section}[hang]{\Large\bfseries}{\thesection}{.5em}{\hspace{0in}}[\vspace{-.2\baselineskip}]
\titleformat{\subsection}[hang]{\large\bfseries}{\thesubsection}{.5em}{\hspace{0in}}[\vspace{-.2\baselineskip}]
%\titleformat{\subsubsection}[hang]{\bfseries}{\thesubsubsection}{.5em}{\hspace{0in}}[\vspace{-.2\baselineskip}]
\titleformat{\subsubsection}[hang]{\bfseries}{\thesubsubsection}{1ex}{\hspace{0in}}[\vspace{-.2\baselineskip}]
\titleformat{\paragraph}[runin]{\bfseries\itshape}{\theparagraph}{1ex}{}{\vspace{-.2\baselineskip}}
%\titleformat{\paragraph}[runin]{\itshape}{\theparagraph}{1ex}{}{\vspace{-.2\baselineskip}}

%%\titleformat{\subsection}[hang]{\bfseries}{\thesubsection}{.5em}{\hspace{0in}}[\vspace{-.2\baselineskip}]
%%%\titleformat*{\subsection}{\bfseries\scshape}
%%%\titleformat{\subsubsection}[leftmargin]{\footnotesize\filleft}{\thesubsubsection}{.5em}{}{}
%%\titleformat{\subsubsection}[hang]{\small\bfseries}{\thesubsubsection}{.5em}{\hspace{0in}}[\vspace{-.2\baselineskip}]
%%\titleformat{\paragraph}[runin]{\itshape}{\theparagraph}{1ex}{}{\vspace{-.5\baselineskip}}

%\titlespacing*{ ⟨command⟩}{⟨left⟩}{⟨beforesep⟩}{⟨aftersep⟩}[⟨right⟩]
\titlespacing{\section}{0pc}{1.5ex plus .1ex minus .2ex}{.5ex plus .1ex minus .1ex}
\titlespacing{\subsection}{0pc}{1.5ex plus .1ex minus .2ex}{.5ex plus .1ex minus .1ex}
\titlespacing{\subsubsection}{0pc}{1.5ex plus .1ex minus .2ex}{.5ex plus .1ex minus .1ex}



%% These next lines tell latex that it is ok to have a single graphic
%% taking up most of a page, and they also decrease the space around
%% figures and tables.
\renewcommand\floatpagefraction{.9}
\renewcommand\topfraction{.9}
\renewcommand\bottomfraction{.9}
\renewcommand\textfraction{.1}
\setcounter{totalnumber}{50}
\setcounter{topnumber}{50}
\setcounter{bottomnumber}{50}
\setlength{\intextsep}{2ex}
\setlength{\floatsep}{2ex}
\setlength{\textfloatsep}{2ex}



\begin{document}
\VerbatimFootnotes

%\begin{titlepage}
%  \maketitle
%\vspace{2in}
%
%\begin{center}
%  \begin{large}
%    PROPOSAL WHITE PAPER
%
%BAA 14-013
%
%Can a Hausa Language Television Station Change Norms about Violence in Northern Nigeria? A Randomized Study of Media Effects on Violent Extremism
%
%Jake Bowers
%
%University of Illinois @ Urbana-Champaign (jwbowers@illinois.edu)
%
%\url{http://jakebowers.org}
%
%Phone: +12179792179
%
%Topic Number: 1
%
%Topic Title: Identity, Influence and Mobilization
%
%\end{large}
%\end{center}
%\end{titlepage}

\maketitle

\hypertarget{commitment-problem}{%
\subsection{Commitment Problem}\label{commitment-problem}}

An example of groups having a Commitment problem is formalized in Table
{[}chris: tab name{]}.

\begin{table}[h!]
\begin{center}
\setlength{\extrarowheight}{2pt}
\begin{tabular}{cc|c|c|}
    & \multicolumn{1}{c}{} & \multicolumn{2}{c}{Player $2$}\\
    & \multicolumn{1}{c}{} & \multicolumn{1}{c}{$Cooperate$}  & \multicolumn{1}{c}{$Defect$} \\\cline{3-4}
    \multirow{2}*{Player $1$}  & $Cooperate$ & $(2,2)$ & $(0,3)$ \\\cline{3-4}
      & $Defect$ & $(3,0)$ & $(1,1)$ \\\cline{3-4}
\end{tabular}
\caption{\label{tab:comProb}\textbf{Example of Commitment Problem.} Numbers represent payoffs to players.  The first number in each cell represents Player 1's payoffs, the second number represents Player 2's payoffs.  Player's want the highest payoff.}
\end{center}
\end{table}

Sides would commit to mutual cooperation if they could to avoid dual
defection, but they cannot credibly commit to cooperating because they
both prefer to defect regardless of what the other does.

Solution is some way for each side to commit to cooperating. Typically
done by adding costs to defection or through repeated interactions.

\hypertarget{information-problem}{%
\subsection{Information Problem}\label{information-problem}}

Imagine a situation in which both groups prefer to cooperate if the
other side cooperates, but groups do not know the preferences of the
other side. They will fail to cooperate if neither side knows the
preferences of the other.

An example of groups having an information problem is Formalized in
Table {[}chris: tab name{]}.

\begin{table}[h!]
\begin{center}
\setlength{\extrarowheight}{2pt}
\begin{subtable}{0.48\linewidth}
\caption{\label{tab:info_a}\textbf{World A} \\Both sides prefer to cooperate if the other side cooperates but defect if the other side defects.}
\begin{tabular}{cc|c|c|}
    & \multicolumn{1}{c}{} & \multicolumn{2}{c}{Player $2$}\\
    & \multicolumn{1}{c}{} & \multicolumn{1}{c}{$Cooperate$}  & \multicolumn{1}{c}{$Defect$} \\\cline{3-4}
    \multirow{2}*{Player $1$}  & $Cooperate$ & $(4,4)$ & $(0,3)$ \\\cline{3-4}
      & $Defect$ & $(3,0)$ & $(1,1)$ \\\cline{3-4}
\end{tabular}
\end{subtable}%
\hfill
\begin{subtable}{0.48\linewidth}
\caption{\label{tab:info_b}\textbf{World B} \\Player 1 prefers mutual cooperation, but Player 2 prefers to defect regardless of Player 1's Behavior.}
\begin{tabular}{cc|c|c|}
    & \multicolumn{1}{c}{} & \multicolumn{2}{c}{Player $2$}\\
    & \multicolumn{1}{c}{} & \multicolumn{1}{c}{$Cooperate$}  & \multicolumn{1}{c}{$Defect$} \\\cline{3-4}
    \multirow{2}*{Player $1$}  & $Cooperate$ & $(4,2)$ & $(0,3)$ \\\cline{3-4}
      & $Defect$ & $(3,0)$ & $(1,1)$ \\\cline{3-4}
\end{tabular}
\end{subtable}
\caption{\label{tab:infoProb}\textbf{Example of Information Problem.} Numbers represent payoffs to players.  The first number in each cell represents Player 1's payoffs, the second number represents Player 2's payoffs.  Player's want the highest payoff.}
\end{center}
\end{table}

\hypertarget{preferences-problem}{%
\subsection{Preferences Problem}\label{preferences-problem}}

Not a bargaining issue if one or both sides prefer fighting. I will
refer to this as a ``Preferences Problem'' causing conflict. Mutual
defection maximizes overall utility; the preferences problem is only a
problem normatively.

\begin{table}[h!]
\begin{center}
\setlength{\extrarowheight}{2pt}
\begin{tabular}{cc|c|c|}
    & \multicolumn{1}{c}{} & \multicolumn{2}{c}{Player $2$}\\
    & \multicolumn{1}{c}{} & \multicolumn{1}{c}{$Cooperate$}  & \multicolumn{1}{c}{$Defect$} \\\cline{3-4}
    \multirow{2}*{Player $1$}  & $Cooperate$ & $(1,1)$ & $(0,3)$ \\\cline{3-4}
      & $Defect$ & $(3,0)$ & $(2,2)$ \\\cline{3-4}
\end{tabular}
\caption{\label{tab:prefProb}\textbf{Example of Preferences Problem.} Numbers represent payoffs to players.  The first number in each cell represents Player 1's payoffs, the second number represents Player 2's payoffs.  Player's want the highest payoff.}
\end{center}
\end{table}

Sides would not commit to cooperation even if they could; knowledge
about each other's preferences does not avoid defection.

\end{document}
