%\documentclass[]{article}
\documentclass[11pt]{article}
\usepackage[usenames,dvipsnames]{xcolor}

\usepackage[T1]{fontenc}
%\usepackage{lmodern}
\usepackage{tgtermes}
\usepackage{amssymb,amsmath}
%\usepackage[margin=1in]{geometry}
\usepackage[letterpaper,bottom=1in,top=1in,right=1.25in,left=1.25in,includemp=FALSE]{geometry}
\usepackage{pdfpages}
\usepackage[small]{caption}

\usepackage{ifxetex,ifluatex}
\usepackage{fixltx2e} % provides \textsubscript
% use microtype if available
\IfFileExists{microtype.sty}{\usepackage{microtype}}{}
\ifnum 0\ifxetex 1\fi\ifluatex 1\fi=0 % if pdftex
\usepackage[utf8]{inputenc}
\else % if luatex or xelatex
\usepackage{fontspec}
\ifxetex
\usepackage{xltxtra,xunicode}
\fi
\defaultfontfeatures{Mapping=tex-text,Scale=MatchLowercase}
\newcommand{\euro}{€}
\fi
%

\usepackage{fancyvrb}

\usepackage{ctable,longtable}

\usepackage[section]{placeins}
\usepackage{float} % provides the H option for float placement
\restylefloat{figure}
\usepackage{dcolumn} % allows for different column alignments
\newcolumntype{.}{D{.}{.}{1.2}}

\usepackage{booktabs} % nicer horizontal rules in tables

%Assume we want graphics always
\usepackage{graphicx}
% We will generate all images so they have a width \maxwidth. This means
% that they will get their normal width if they fit onto the page, but
% are scaled down if they would overflow the margins.
%% \makeatletter
%% \def\maxwidth{\ifdim\Gin@nat@width>\linewidth\linewidth
%%   \else\Gin@nat@width\fi}
%% \makeatother
%% \let\Oldincludegraphics\includegraphics
%% \renewcommand{\includegraphics}[1]{\Oldincludegraphics[width=\maxwidth]{#1}}
\graphicspath{{.}{../Soccom_Code/socom_2013/}}


%% \ifxetex
%% \usepackage[pagebackref=true, setpagesize=false, % page size defined by xetex
%% unicode=false, % unicode breaks when used with xetex
%% xetex]{hyperref}
%% \else
\usepackage[pagebackref=true, unicode=true, bookmarks=true, pdftex]{hyperref}
% \fi


\hypersetup{breaklinks=true,
  bookmarks=true,
  pdfauthor={Christopher Grady, Rebecca Wolfe, Danjuma Dawop, and Lisa Inks},
  pdftitle={Improving Intergroup Relations Amid Group Conflict: An Intergroup Contact Field Experiment in Nigeria - Introduction},
  colorlinks=true,
  linkcolor=BrickRed,
  citecolor=blue, %MidnightBlue,
  urlcolor=BrickRed,
  % urlcolor=blue,
  % linkcolor=magenta,
  pdfborder={0 0 0}}

%\setlength{\parindent}{0pt}
%\setlength{\parskip}{6pt plus 2pt minus 1pt}
\usepackage{parskip}
\setlength{\emergencystretch}{3em}  % prevent overfull lines
\providecommand{\tightlist}{%
  \setlength{\itemsep}{0pt}\setlength{\parskip}{0pt}}

%% Insist on this.
\setcounter{secnumdepth}{2}

\VerbatimFootnotes % allows verbatim text in footnotes

\title{Improving Intergroup Relations Amid Group Conflict: An Intergroup
Contact Field Experiment in Nigeria - Introduction}

\author{
Christopher Grady, Rebecca Wolfe, Danjuma Dawop, and Lisa Inks
}


\date{November 13, 2019}


\usepackage{versions}
\makeatletter
\renewcommand*\versionmessage[2]{\typeout{*** `#1' #2. ***}}
\renewcommand*\beginmarkversion{\sffamily}
  \renewcommand*\endmarkversion{}
\makeatother

\excludeversion{comment}

%\usepackage[margins=1in]{geometry}

\usepackage[compact,bottomtitles]{titlesec}
%\titleformat{ ⟨command⟩}[⟨shape⟩]{⟨format⟩}{⟨label⟩}{⟨sep⟩}{⟨before⟩}[⟨after⟩]
\titleformat{\section}[hang]{\Large\bfseries}{\thesection}{.5em}{\hspace{0in}}[\vspace{-.2\baselineskip}]
\titleformat{\subsection}[hang]{\large\bfseries}{\thesubsection}{.5em}{\hspace{0in}}[\vspace{-.2\baselineskip}]
%\titleformat{\subsubsection}[hang]{\bfseries}{\thesubsubsection}{.5em}{\hspace{0in}}[\vspace{-.2\baselineskip}]
\titleformat{\subsubsection}[hang]{\bfseries}{\thesubsubsection}{1ex}{\hspace{0in}}[\vspace{-.2\baselineskip}]
\titleformat{\paragraph}[runin]{\bfseries\itshape}{\theparagraph}{1ex}{}{\vspace{-.2\baselineskip}}
%\titleformat{\paragraph}[runin]{\itshape}{\theparagraph}{1ex}{}{\vspace{-.2\baselineskip}}

%%\titleformat{\subsection}[hang]{\bfseries}{\thesubsection}{.5em}{\hspace{0in}}[\vspace{-.2\baselineskip}]
%%%\titleformat*{\subsection}{\bfseries\scshape}
%%%\titleformat{\subsubsection}[leftmargin]{\footnotesize\filleft}{\thesubsubsection}{.5em}{}{}
%%\titleformat{\subsubsection}[hang]{\small\bfseries}{\thesubsubsection}{.5em}{\hspace{0in}}[\vspace{-.2\baselineskip}]
%%\titleformat{\paragraph}[runin]{\itshape}{\theparagraph}{1ex}{}{\vspace{-.5\baselineskip}}

%\titlespacing*{ ⟨command⟩}{⟨left⟩}{⟨beforesep⟩}{⟨aftersep⟩}[⟨right⟩]
\titlespacing{\section}{0pc}{1.5ex plus .1ex minus .2ex}{.5ex plus .1ex minus .1ex}
\titlespacing{\subsection}{0pc}{1.5ex plus .1ex minus .2ex}{.5ex plus .1ex minus .1ex}
\titlespacing{\subsubsection}{0pc}{1.5ex plus .1ex minus .2ex}{.5ex plus .1ex minus .1ex}



%% These next lines tell latex that it is ok to have a single graphic
%% taking up most of a page, and they also decrease the space around
%% figures and tables.
\renewcommand\floatpagefraction{.9}
\renewcommand\topfraction{.9}
\renewcommand\bottomfraction{.9}
\renewcommand\textfraction{.1}
\setcounter{totalnumber}{50}
\setcounter{topnumber}{50}
\setcounter{bottomnumber}{50}
\setlength{\intextsep}{2ex}
\setlength{\floatsep}{2ex}
\setlength{\textfloatsep}{2ex}



\begin{document}
\VerbatimFootnotes

%\begin{titlepage}
%  \maketitle
%\vspace{2in}
%
%\begin{center}
%  \begin{large}
%    PROPOSAL WHITE PAPER
%
%BAA 14-013
%
%Can a Hausa Language Television Station Change Norms about Violence in Northern Nigeria? A Randomized Study of Media Effects on Violent Extremism
%
%Jake Bowers
%
%University of Illinois @ Urbana-Champaign (jwbowers@illinois.edu)
%
%\url{http://jakebowers.org}
%
%Phone: +12179792179
%
%Topic Number: 1
%
%Topic Title: Identity, Influence and Mobilization
%
%\end{large}
%\end{center}
%\end{titlepage}

\maketitle

\hypertarget{introduction}{%
\section{Introduction}\label{introduction}}

How can groups in conflict improve intergroup relations? Conflict
between groups has caused 2 million deaths since the year 2000 (Sundberg
and Melander 2013), forcibly displaced over 70 million people from their
homes in 2018 (UNHCR 2019), threatens food supplies in numerous
countries (Verwimp and others 2012), and extracts a psychological toll
on participants and victims (Schomerus and Rigterink 2018). Intergroup
animosity perpetuates conflict long after the original grievance is
immaterial or forgotten (Deutsch 1973; McDonnel 2017; Tajfel and Turner
1979), so improving intergroup relations is vital to stem the human,
economic, social, and psychological costs of violent group conflict.

Scholars and policymakers/practicioners consider cooperative intergroup
contact one of the most effective tools for improving intergroup
relations. The hypothesis that cooperative contact improves intergroup
relations -- known as the contact hypothesis (Allport 1954) -- motivates
many interventions, from integrated sports teams and public housing to
roommate assignment in college dorms and location settlement of
immigrants (chris: add cites) to peacebuilding programs (Ditlmann,
Samii, and Zeitzoff 2017). Through these types of interventions,
intergroup contact has improved relations between white people and black
people in the U.S. South Africa, and Norway (Burns, Corno, and La
Ferrara 2015; Carrell, Hoekstra, and West 2015; Finseraas and Kotsadam
2017; Marmaros and Sacerdote 2006), Christians and Muslims in Iraq and
Nigeria ({\textbf{???}}; Scacco and Warren 2018), Jews and Arabs in
Israel/Palestine (Ditlmann and Samii 2016; Weiss 2019; Yablon 2012), and
Hindus and Muslims in India (Barnhardt 2009).

Despite these successes, scholars know little about the effects of
intergroup contact for groups engaged in a violent conflict. Cooperative
intergroup contact has only recently been tested in the field and never
with groups engaged in violent conflict (Maoz 2011; Paluck, Green, and
Green 2017). If one of the goals of cooperative contact is to mitigate
violent conflict, interventions based on cooperative contact must be
tested between groups in a violent conflict.

Many of the conditions present during conflict could interfere with the
mechanisms through which contact improves relations. Scholars theorize
that contact improves relations mainly through providing information
that dispels stereotypes, increasing empathy and perspective-taking,
reducing anxiety about interacting with outgroup members, and making
salient a superordinate identity that includes both groups (Broockman
and Kalla 2016; Dovidio et al. 2017; Gaertner and Dovidio 2014;
Page-Gould, Mendoza-Denton, and Tropp 2008; Pettigrew and Tropp 2008).
These mechanisms assume that negative attitudes result from
unfamiliarity, and that ``familiarity breed{[}s{]} liking'' (Pettigrew
and Tropp 2006, 766). But the animosity of groups in conflict is driven
more by opposing interests and negative experiences than inexperience.
By preventing these mechanisms, violent conflict could dull, prevent, or
even reverse the predicted positive effects of contact.

Moreover, contact can only meaningfully improve group relations -- as
opposed to interpersonal relations between members of opposing groups --
if several conditions are met. First, cooperative contact with
individual outgroup members must cause ingroup members to update about
the entire outgroup. Second, the newly positive attitudes of ingroup
members who interacted with outgroup members must diffuse to other
ingroup members. And third, intergroup attitudes must improve for both
groups. Due to opposing group interests, negative experiences, existing
grievances, and power differentials, groups involved in violent conflict
may not meet these conditions. For these reasons, some scholars believe
group reconciliation cannot begin until their conlict is resolved
(Bar-Tal 2000).

To learn about whether cooperative contact can improve intergroup
relations amidst violent group conflict, we conducted a field experiment
with conflicting farmer and pastoralist communities in Nigeria. More
than an occupational difference, farmers who cultivate crops and
pastoralists who graze cattle define a major social cleavage in many
parts of the world. These groups conflict over land rights, which define
both of their livelihoods. Farmer-pastoralist conflict has escalated
throughout the Sahel in recent years, and nowhere more than in Nigeria.
The most recent conflict escalation has caused 7,000 deaths from
2014-2019, displaced hundreds of thousands of people from their homes,
and costs \$13 billion annually in lost economic productivity (Akinwotu
2018; Daniel 2018; Harwood 2019; McDougal et al. 2015).

We randomly assigned communities with ongoing farmer-pastoralist
violence to receive a contact-based intervention or serve as a control
group. The intervention formed mixed-group committees and provided them
with funds to build infrastructure that would benefit both communities;
committees then collaboratively chose and constructed infrastructure
projects.\footnote{The communities built boreholes, market stalls,
  primary health care facilities, etc.} The program also provided
mediation training to each community's leaders and held forums where the
groups discussed the underlying drivers of conflict. To measure the
effects of the intervention, we conducted pre- and post-intervention
surveys, a post-intervention natural public goods behavioral
game,\footnote{In a public goods game (PGG), research subjects are given
  money and told they can keep the money or donate it to a public fund.
  Money donated to the public fund is multiplied by some amount and then
  shared with all subjects. Our PGG is \emph{natural} because it was
  conducted in a natural setting, rather than a lab. The funding for the
  PGG came from the National Science Foundation under Grant No.~1656871.}
and twelve months of systematic observations in markets and social
events during the intervention.

We find that the program increased intergroup contact, intergroup trust,
and perceptions of physical security. We see signs of the positive
effects in fieldwork as well as in data -- in one of the treatment
sites, farmers defended pastoralists from a group of anti-pastoralist
vigilantes, rather than assist the vigilantes in removing the
pastoralists and claiming their land. Our results also show that the
intervention affected communities as a whole, not just community members
directly involved in the intergroup contact. Individuals who directly
engaged in intergroup contact changed the most positively from baseline
to endline, but we also observe positive spillovers of trust to group
members for whom we did not exogenously increase intergroup contact.

This study expands our knowledge about group conflict in several ways.
First, this study teaches us about the capacity of intergroup contact to
improve intergroup relations and reduce conflict. Peacebuilding
organizations implement numerous contact-based interventions in violent
contexts each year (Ditlmann, Samii, and Zeitzoff 2017), but its
efficacy to improve intergroup attitudes amid real-world conflict is an
open question (Ditlmann, Samii, and Zeitzoff 2017; Paluck, Green, and
Green 2017). To our knowledge this is the first field experimental test
of a contact-based intervention implemented during an active conflict.
The results suggest that contact-based peacebuilding programs can
effectively improve relationships between conflicting groups and is
especially relevant to conflict resolution in the cases of intergroup
and intercommunal conflicts.

Second, we contribute to the literature about informal structures, such
as social norms, in solving collective action problems. Conflict between
farmers and pastoralists is a collective action problem in that both
groups would be materially better off avoiding violence through
compromise and cooperation, but each has the incentive to take advantage
of the other. Individuals within each group face the same dilemma: they
prefer a compromise, but their incentives are to free-ride and allow
others to bear the cost of compromise. In rural Nigeria, as with many
contexts, no formal institutions exist to encourage cooperation and so
groups must develop informal structures to achieve collective action
(Ostrom 2000). This intervention showed how informal structures to solve
collection action problems can naturally develop through repeated
intergroup interactions. Our intervention only engaged a small
percentage of each community, yet its effects diffused to other
community members. Creating informal structures that diffuse the
attitudinal effects of cooperative contact are a way of scaling up
contact-based interventions.\footnote{chris: this paragraph is too long
  and not focused enough. Would appreciate comments.}

Third, this paper teaches us about settling disputes between sedentary
peoples and nomadic peoples. Violent conflict between settled peoples
and nomadic peoples is on the rise throughout Africa (Kuusaana and
Bukari 2015; Mwamfupe 2015; Nnoko-Mewanu 2018). This study focuses on
the Fulani, the largest semi-nomadic people on Earth (Encyclopedia
2017). Their way of life makes them targets for violence throughout
Africa. Along with this conflict in Nigeria, Fulani in Mali have been
the targets of violence so severe that researchers at Armed Conflict
Location \& Event Data Project called it ``ethnic cleansing'' (Economist
2019). Understanding how to prevent violent conflict between Fulani and
settled peoples can help prevent violence that targets other nomadic and
semi-nomadic peoples, such as the Tuaregs in West Africa, Uyghurs in
Central Asia, Kochi in Afghanistan, and Khoisan of Southern Africa.
Preventing such violence could help preserve a dying way of life.

\begin{center}\rule{0.5\linewidth}{\linethickness}\end{center}

\hypertarget{references}{%
\section*{References}\label{references}}
\addcontentsline{toc}{section}{References}

\hypertarget{refs}{}
\leavevmode\hypertarget{ref-nyt2018nigeria}{}%
Akinwotu, Emmanuel. 2018. ``Nigeria's Farmers and Herders Fight a Deadly
Battle for Scarce Resources.'' \emph{New York Times}.
\url{https://www.nytimes.com/2018/06/25/world/africa/nigeria-herders-farmers.html}.

\leavevmode\hypertarget{ref-allport1954prejudice}{}%
Allport, Gordon. 1954. ``The Nature of Prejudice.'' \emph{Garden City,
NJ Anchor}.

\leavevmode\hypertarget{ref-barnhardt2009near}{}%
Barnhardt, Sharon. 2009. ``Near and Dear? Evaluating the Impact of
Neighbor Diversity on Inter-Religious Attitudes.'' \emph{Unpublished
working paper}.

\leavevmode\hypertarget{ref-bar2000intractable}{}%
Bar-Tal, Daniel. 2000. ``From Intractable Conflict Through Conflict
Resolution to Reconciliation: Psychological Analysis.'' \emph{Political
Psychology} 21(2): 351--65.

\leavevmode\hypertarget{ref-broockman2016durably}{}%
Broockman, David, and Joshua Kalla. 2016. ``Durably Reducing
Transphobia: A Field Experiment on Door-to-Door Canvassing.''
\emph{Science} 352(6282): 220--24.

\leavevmode\hypertarget{ref-burns2015interaction}{}%
Burns, Justine, Lucia Corno, and Eliana La Ferrara. 2015.
\emph{Interaction, Prejudice and Performance. Evidence from South
Africa}. Working paper.

\leavevmode\hypertarget{ref-carrell2015impact}{}%
Carrell, Scott E, Mark Hoekstra, and James E West. 2015. \emph{The
Impact of Intergroup Contact on Racial Attitudes and Revealed
Preferences}. National Bureau of Economic Research.

\leavevmode\hypertarget{ref-daniel2018anti}{}%
Daniel, Soni. 2018. ``Anti-Open Grazing Law: Nass, Benue, Kwara, Taraba
Tackle Defence Minister.'' \emph{Vanguard}.
\url{https://www.vanguardngr.com/2018/06/anti-open-grazing-law-nass-benue-kwara-taraba-tackle-defence-minister/}.

\leavevmode\hypertarget{ref-deutsch1973resolution}{}%
Deutsch, Morton. 1973. \emph{The Resolution of Conflict: Constructive
and Destructive Processes}. Yale University Press.

\leavevmode\hypertarget{ref-ditlmann2016can}{}%
Ditlmann, Ruth K, and Cyrus Samii. 2016. ``Can Intergroup Contact Affect
Ingroup Dynamics? Insights from a Field Study with Jewish and
Arab-Palestinian Youth in Israel.'' \emph{Peace and Conflict: Journal of
Peace Psychology} 22(4): 380.

\leavevmode\hypertarget{ref-ditlmann2017addressing}{}%
Ditlmann, Ruth K, Cyrus Samii, and Thomas Zeitzoff. 2017. ``Addressing
Violent Intergroup Conflict from the Bottom up?'' \emph{Social Issues
and Policy Review} 11(1): 38--77.

\leavevmode\hypertarget{ref-dovidio2017reducing}{}%
Dovidio, John F, Angelika Love, Fabian MH Schellhaas, and Miles
Hewstone. 2017. ``Reducing Intergroup Bias Through Intergroup Contact:
Twenty Years of Progress and Future Directions.'' \emph{Group Processes
\& Intergroup Relations} 20(5): 606--20.

\leavevmode\hypertarget{ref-economist2019militias}{}%
Economist, The. 2019. ``Malicious Malitias: States in the Sahel Have
Unleashed Ethnic Gangs with Guns.'' \emph{The Economist}.
\url{https://www.economist.com/middle-east-and-africa/2019/05/04/states-in-the-sahel-have-unleashed-ethnic-gangs-with-guns}.

\leavevmode\hypertarget{ref-fulanisize2017}{}%
Encyclopedia, New World. 2017. ``Fulani --- New World Encyclopedia,''
\url{//www.newworldencyclopedia.org/p/index.php?title=Fulani\&oldid=1004777}.

\leavevmode\hypertarget{ref-finseraas2017does}{}%
Finseraas, Henning, and Andreas Kotsadam. 2017. ``Does Personal Contact
with Ethnic Minorities Affect Anti-Immigrant Sentiments? Evidence from a
Field Experiment.'' \emph{European Journal of Political Research} 56(3):
703--22.

\leavevmode\hypertarget{ref-gaertner2014reducing}{}%
Gaertner, Samuel L, and John F Dovidio. 2014. \emph{Reducing Intergroup
Bias: The Common Ingroup Identity Model}. Psychology Press.

\leavevmode\hypertarget{ref-council2019nigeria}{}%
Harwood, Asch. 2019. ``Update: The Numbers Behind Sectarian Violence in
Nigeria.'' \emph{Council on Foreign Relations}.
\url{https://www.cfr.org/blog/update-numbers-behind-sectarian-violence-nigeria}.

\leavevmode\hypertarget{ref-kuusaana2015land}{}%
Kuusaana, Elias Danyi, and Kaderi Noagah Bukari. 2015. ``Land Conflicts
Between Smallholders and Fulani Pastoralists in Ghana: Evidence from the
Asante Akim North District (Aand).'' \emph{Journal of rural studies} 42:
52--62.

\leavevmode\hypertarget{ref-maoz2011protracted}{}%
Maoz, Ifat. 2011. ``Does Contact Work in Protracted Asymmetrical
Conflict? Appraising 20 Years of Reconciliation-Aimed Encounters Between
Israeli Jews and Palestinians.'' \emph{Journal of Peace Research} 48(1):
115--25.

\leavevmode\hypertarget{ref-marmaros2006friendships}{}%
Marmaros, David, and Bruce Sacerdote. 2006. ``How Do Friendships Form?''
\emph{The Quarterly Journal of Economics} 121(1): 79--119.

\leavevmode\hypertarget{ref-mcdonnel2017graze}{}%
McDonnel, Tim. 2017. ``Why It's Now a Crime to Let Cattle Graze Freely
in 2 Nigerian States.'' \emph{National Public Radio (NPR)}.
\url{https://www.npr.org/sections/goatsandsoda/2017/12/12/569913821/why-its-now-a-crime-to-let-cattle-graze-freely-in-2-nigerian-states}.

\leavevmode\hypertarget{ref-mcdougal2015effect}{}%
McDougal, Topher L et al. 2015. ``The Effect of Farmer-Pastoralist
Violence on Income: New Survey Evidence from Nigeria's Middle Belt
States.'' \emph{Economics of Peace and Security Journal} 10(1): 54--65.

\leavevmode\hypertarget{ref-mwamfupe2015persistence}{}%
Mwamfupe, Davis. 2015. ``Persistence of Farmer-Herder Conflicts in
Tanzania.'' \emph{International Journal of Scientific and Research
Publications} 5(2): 1--8.

\leavevmode\hypertarget{ref-hrc2018farmer}{}%
Nnoko-Mewanu, Juliana. 2018. ``Farmer-Herder Conflicts on the Rise in
Africa.'' \emph{Human Rights Watch}.

\leavevmode\hypertarget{ref-ostrom2000collective}{}%
Ostrom, Elinor. 2000. ``Collective Action and the Evolution of Social
Norms.'' \emph{Journal of economic perspectives} 14(3): 137--58.

\leavevmode\hypertarget{ref-page2008little}{}%
Page-Gould, Elizabeth, Rodolfo Mendoza-Denton, and Linda R Tropp. 2008.
``With a Little Help from My Cross-Group Friend: Reducing Anxiety in
Intergroup Contexts Through Cross-Group Friendship.'' \emph{Journal of
personality and social psychology} 95(5): 1080.

\leavevmode\hypertarget{ref-paluck2017contact}{}%
Paluck, Elizabeth Levy, Seth Green, and Donald P Green. 2017. ``The
Contact Hypothesis Revisited.''

\leavevmode\hypertarget{ref-pettigrew2006meta}{}%
Pettigrew, Thomas F, and Linda R Tropp. 2006. ``A Meta-Analytic Test of
Intergroup Contact Theory.'' \emph{Journal of personality and social
psychology} 90(5): 751.

\leavevmode\hypertarget{ref-pettigrew2008does}{}%
---------. 2008. ``How Does Intergroup Contact Reduce Prejudice?
Meta-Analytic Tests of Three Mediators.'' \emph{European Journal of
Social Psychology} 38(6): 922--34.

\leavevmode\hypertarget{ref-scacco2018nigeria}{}%
Scacco, Alexandra, and Shana S Warren. 2018. ``Can Social Contact Reduce
Prejudice and Discrimination? Evidence from a Field Experiment in
Nigeria.'' \emph{American Political Science Review} 112(3): 654--77.

\leavevmode\hypertarget{ref-ucdp}{}%
Sundberg, Ralph, and Erik Melander. 2013. ``Introducing the Ucdp
Georeferenced Event Dataset.'' \emph{Journal of Peace Research} 50(4):
523--32.

\leavevmode\hypertarget{ref-tajfel1979integrative}{}%
Tajfel, Henri, and John C Turner. 1979. ``An Integrative Theory of
Intergroup Conflict.'' \emph{The social psychology of intergroup
relations} 33(47): 74.

\leavevmode\hypertarget{ref-unhcr2019}{}%
UNHCR. 2019. \emph{UNHCR Statistical Yearbook}.
https://www.unhcr.org/en-us/figures-at-a-glance.html: United Nations
High Commission for Refugees.

\leavevmode\hypertarget{ref-verwimp2012food}{}%
Verwimp, Philip, and others. 2012. ``Food Security, Violent Conflict and
Human Development: Causes and Consequences.'' \emph{United Nations
Development Programme Working Paper}: 1--13.

\leavevmode\hypertarget{ref-weiss2019curing}{}%
Weiss, Chagai M. 2019. ``Curing Prejudice Through Representative
Bureaucracies: Evidence from a Natural Experiment in Israeli Medical
Clinics.''

\leavevmode\hypertarget{ref-yablon2012we}{}%
Yablon, Yaacov B. 2012. ``Are We Preaching to the Converted? The Role of
Motivation in Understanding the Contribution of Intergroup Encounters.''
\emph{Journal of Peace Education} 9(3): 249--63.

\end{document}
