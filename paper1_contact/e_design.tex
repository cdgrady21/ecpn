\documentclass[]{article}
\usepackage{lmodern}
\usepackage{amssymb,amsmath}
\usepackage{ifxetex,ifluatex}
\usepackage{fixltx2e} % provides \textsubscript
\ifnum 0\ifxetex 1\fi\ifluatex 1\fi=0 % if pdftex
  \usepackage[T1]{fontenc}
  \usepackage[utf8]{inputenc}
\else % if luatex or xelatex
  \ifxetex
    \usepackage{mathspec}
  \else
    \usepackage{fontspec}
  \fi
  \defaultfontfeatures{Ligatures=TeX,Scale=MatchLowercase}
\fi
% use upquote if available, for straight quotes in verbatim environments
\IfFileExists{upquote.sty}{\usepackage{upquote}}{}
% use microtype if available
\IfFileExists{microtype.sty}{%
\usepackage{microtype}
\UseMicrotypeSet[protrusion]{basicmath} % disable protrusion for tt fonts
}{}
\usepackage[margin=1in]{geometry}
\usepackage{hyperref}
\hypersetup{unicode=true,
            pdftitle={Improving Intergroup Relations Amid Group Conflict - Design},
            pdfauthor={Christopher Grady},
            pdfborder={0 0 0},
            breaklinks=true}
\urlstyle{same}  % don't use monospace font for urls
\usepackage{graphicx,grffile}
\makeatletter
\def\maxwidth{\ifdim\Gin@nat@width>\linewidth\linewidth\else\Gin@nat@width\fi}
\def\maxheight{\ifdim\Gin@nat@height>\textheight\textheight\else\Gin@nat@height\fi}
\makeatother
% Scale images if necessary, so that they will not overflow the page
% margins by default, and it is still possible to overwrite the defaults
% using explicit options in \includegraphics[width, height, ...]{}
\setkeys{Gin}{width=\maxwidth,height=\maxheight,keepaspectratio}
\IfFileExists{parskip.sty}{%
\usepackage{parskip}
}{% else
\setlength{\parindent}{0pt}
\setlength{\parskip}{6pt plus 2pt minus 1pt}
}
\setlength{\emergencystretch}{3em}  % prevent overfull lines
\providecommand{\tightlist}{%
  \setlength{\itemsep}{0pt}\setlength{\parskip}{0pt}}
\setcounter{secnumdepth}{5}
% Redefines (sub)paragraphs to behave more like sections
\ifx\paragraph\undefined\else
\let\oldparagraph\paragraph
\renewcommand{\paragraph}[1]{\oldparagraph{#1}\mbox{}}
\fi
\ifx\subparagraph\undefined\else
\let\oldsubparagraph\subparagraph
\renewcommand{\subparagraph}[1]{\oldsubparagraph{#1}\mbox{}}
\fi

%%% Use protect on footnotes to avoid problems with footnotes in titles
\let\rmarkdownfootnote\footnote%
\def\footnote{\protect\rmarkdownfootnote}

%%% Change title format to be more compact
\usepackage{titling}

% Create subtitle command for use in maketitle
\providecommand{\subtitle}[1]{
  \posttitle{
    \begin{center}\large#1\end{center}
    }
}

\setlength{\droptitle}{-2em}

  \title{Improving Intergroup Relations Amid Group Conflict - Design}
    \pretitle{\vspace{\droptitle}\centering\huge}
  \posttitle{\par}
    \author{Christopher Grady}
    \preauthor{\centering\large\emph}
  \postauthor{\par}
      \predate{\centering\large\emph}
  \postdate{\par}
    \date{December 10, 2019}


\begin{document}
\maketitle

\hypertarget{research-design}{%
\section{Research Design}\label{research-design}}

We evaluate the effects of Engaging Communities for Peace in Nigeria
(ECPN) with a site-level field experiment. Each site contains two
communities, one of farmers and one of pastoralists. The communities
within a site engaged in deadly clashes within one year of our scoping
exercise.\footnote{To identify eligible sites, we undertook a scoping
  exercise to determine if the two communities in an implementation site
  had a demonstrated need for a peacebuilding program and were willing
  to participate in one. We defined ``demonstrated need'' as the
  communities engaging in violent clashes within one year of the scoping
  exercise. Willingness to participate in the program was obtained
  through conversations with community leaders, none of whom refused the
  program.} We identified fifteen sites eligible for the study and
surveyed \textasciitilde{}50 randomly selected respondents per
community. We then randomly selected the communities in ten of fifteen
sites to receive the ECPN program, blocking by state so that an equal
proportion of sites in Benue and Nassarawa received the program. After
18 months, we surveyed another \textasciitilde{}50 randomly selected
respondents and \textasciitilde{}10 respondents from the baseline survey
per community. In between the surveys, we monitored farmer-pastoralist
interactions in markets and at social events.

This designs gives us two datasets to analyze. First, we aggregate the
randomly-sampled individuals to compare communities before and after
ECPN. Communities were randomly assigned to receive ECPN or function as
a control group, which allows us to determine the causal effect of ECPN
at a community-level. This comparison between communities that received
or did not receive ECPN is our main analysis.

Second, we supplement the community-level analysis by creating a dataset
of \textasciitilde{}10 respondents per community before and after ECPN.
From our baseline random sample, we identified and resurveyed (1) ECPN
committee participants, (2) respondents who lived in intervention sites
but did not participate in ECPN committees, and (3) respondents from the
control group, who neither participated in ECPN committees nor lived in
communities where ECPN was implemented. We then compare the change of
participants and nonparticipants in intervention sites to the change in
control respondents. The main goal of this analysis is to learn about
the effect of participating directly in ECPN committees, and thus
directly experiencing intergroup contact, relative to the effect of
living in communities where ECPN was implemented but not participating
in committees, and thus only experiencing indirect intergroup contact.
Our ability to make generalizable causal claims about participation is
limited, though, because individuals in intervention sites were not
randomized into participation or nonparticipation with ECPN
committees.\footnote{We initially randomly assigned baseline survey
  respondents to be part of ECPN committees, but random assignment
  proved difficult. Many people who were not selected wanted to be on
  the committees, and some people who were selected were not able to
  participate or could not be located when the committees were launched.
  As a result, people self-selected into committees.}

{[}chris: figure showing sampling strategy, numbers per group, and
timeline.{]}

\hypertarget{estimation}{%
\subsection{Estimation}\label{estimation}}

Here we describe our estimation procedure for the community-level
analysis and the individual-level analysis. For both analyses we
estimate one-tailed ``greater than'' tests because our hypotheses are
that the change in outcomes for treatment units will be \emph{greater
than} control, not that the change in outcomes for treatment units will
be \emph{different} than control. Both analyses also use randomization
inference for \(p\)-values and bootstrapping for standard errors. The
specifics of each procedure are described in Appendix A.

We use the difference-in-differences framework to estimate the effect of
ECPN. We have two observations per community: a baseline outcome and an
endline outcome. To maximize precision, we generally predict endline
outcomes while controlling for baseline outcomes.

\(Y_{i,j} = \beta_0 + \beta_1Z_{i,j} + X_{i,j} + \delta_j + \epsilon_{i,j}\)

Where \(i\) is the community in state \(j\), \(Z\) is the treatment
indicator, \(X\) is the outcome at baseline, and \(Y\) is the outcome at
endline. \(\delta\) is a fixed effect for the state \(j\) in which the
community belongs.

However, the ``controlling-for'' method is biased when treatment
assignment correlates with baseline outcomes.\footnote{For a comparison
  between the controlling-for method and the differencing method, see
  \url{https://declaredesign.org/blog/2019-01-15-change-scores.html}.}
Therefore, when the baseline difference between treatment and control
groups is greater than 0.20 standard deviations, we use the
``differencing'' method typical of difference-in-differences estimation.
This method sacrifices power to ensure an unbiased estimate of the
average treatment effect.

\(Y_{i,j} = \beta_0 + \beta_1Z_{i,j} + \delta_j + \epsilon_{i,j}\)

Where \(i\) is the community in state \(j\), \(Z\) is the treatment
indicator, and \(Y\) is the change in outcome from baseline to endline.
\(\delta\) is a fixed effect for the state \(j\) in which the community
belongs.

We use randomization inference for \(p\)-values and bootstrapping for
standard errors because our units of analysis, communities and
individuals, are clustered in sites and we have only fifteen sites.
Analytic standard errors may underestimate the uncertainty of our causal
estimate {[}cite{]}. With randomization inference, we first shuffle the
treatment variable to break the relationship between treatment and
outcomes. Next we regress outcomes on treatment using the equations
specified above. We then store the resulting coefficient. Lastly, we
repeat that process 10,000 times to create the distribution of
coefficients we would observe if treatment had no effect on outcomes --
the null hypothesis. Our \(p\)-value is the proportion of the null
distribution that is greater than or equal to our observed coefficient.
Bootstrapping for standard errors is similar, but instead of shuffling
the treatment indicator we resample units with replacement.
Bootstrapping yields a distribution of possible treatment effects given
the observed data, and the 95\% confidence interval is between the
coefficients at the 2.5th percentile and the 97.5th percentile. More
details can be found in Appendix A.\footnote{Randomization inference: We
  mimic our randomization process by randomizing the intervention to
  communities in site-level clusters and within state blocks. This means
  that both communities in an implementation site (farmers and
  pastoralists) will always be treated together and that assignment to
  the intervention is conducted separately in Nasarawa and Benue, just
  as the intervention was assigned in this study. This procedure ensures
  that our null distribution is created by randomizing the intervention
  between exchangeable units.}

\hypertarget{outcomes}{%
\subsection{Outcomes}\label{outcomes}}

We measured three outcomes to estimate the effect of ECPN: (1)
intergroup attitudes, (2) intergroup contact, and (3) insecurity. If
ECPN improved intergroup relations, we would expect respondents to
report better attitudes towards the outgroup, more intergroup contact
and willingness to engage in intergroup contact, and reduced insecurity
due to violence. We also measured three mechanisms from the contact
literature through which contact could affect outcomes: (1)
empathy/perspective-taking, (2) perceived threat, and (3) ingroup
expansion. Lastly, we measured a placebo outcome that may be affected by
social desirability: attitudes about violence. We measured these
outcomes with survey self-reports, survey experiments, a natural-field
behavioral game, and monitoring of farmer-pastoralist interaction in
markets and social events.

For most survey self-reports, we combine together several survey
questions to create an index. We create both additive indices and
inverse-covariance weighted indices. Inverse-covariance weighting
constructs an index by down-weighting index questions that are
correlated with other index questions and up-weighting those that are
uncorrelated with other questions. This approach maximizes the amount of
unique information the index takes from each question and prevents
``double counting'' when two questions measure the same thing. We report
results using inverse-covariance weighted indices, but results hold with
additive indices. Results with additive indices are included in Appendix
2.

\hypertarget{primary-outcomes}{%
\subsubsection{Primary outcomes}\label{primary-outcomes}}

\textbf{Intergroup affect}: Our first outcome is affect towards the
other side. A primary goal of our contact intervention, and of much
previous contact research, was for individual's attitudes to improve,
Changing attitudes towards the other side is one pathway towards
improving intergroup relations and changing behavior, though recent
research shows that attitude change is not necessary for behavioral
change {[}@paluck2009jsp; @scacco2018nigeria{]}.

We measure intergroup affect with survey self-reports and an endorsement
experiment. The survey questions include two measures of intergroup
trust and a five item social distance scale created for the
farmer-pastoralist context. We create an index measuring intergroup
affect with these seven questions; the index alpha is {[}chris:
alpha{]}.

In an endorsement experiment, respondents are asked how much they
support a hypothetical policy. In the treatment condition, the policy is
`endorsed' by a group that the respondent has a positive or negative
opinion about. In the control condition, the policy is not endorsed by
any group. The average difference in support between the endorsed and
unendorsed policy represents the change in support for the policy
because of the group's endorsement. In our case, we asked respondents
how much they would support a water policy if it was endorsed by a
farmer organization (asked of pastoralists), if it was endorsed by a
pastoralist organization (asked of farmers), or if no endorsement was
mentioned (the control condition posed to both pastoralists and
farmers). Support was measured on a 5-point scale, where high values
indicated support and low values indicated opposition.

\textbf{Intergroup contact}: Our second outcome is intergroup contact
that occurs outside of the intervention. Natural, voluntary intergroup
contact provides behavioral evidence that farmer-pastoralist relations
are improving. We measure intergroup contact with survey self-reports,
monitoring of farmer-pastoralists interactions in markets and social
events, and a survey experiment.\footnote{Much of the self-reports and
  the observations are overdispersed count data. We recode all count
  data as rank.}\footnote{We also attempted to measure willingness to
  engage in contact with a second survey experiment, a list experiment.
  List experiments are used to provide anonymity to respondents and
  encourage them to give honest answers to sensitive questions. In a
  list experiment, the researcher randomly assigns respondents to one of
  two (or more) conditions. Individuals in the control condition are
  presented with a list of three items; individuals in the treatment
  condition see the same list plus an additional item, which is the item
  of interest and the one on which the experimenter wants to ensure the
  respondent of anonymity. Subjects are asked how many items apply to
  them. The average difference between the treatment and control
  conditions represents the percentage of respondents who responded to
  the sensitive item. In our case, the sensitive item read ``When you
  have to interact with a member of {[}the other group{]} in the
  market.'' Our list experiment failed. The endline difference between
  the 3-item list and the 4-item list is negative in almost half of all
  communities, implying that the presence of the 4th item slightly
  decreased the number of reported items. This decrease violates the
  assumptions of the list experiment. The average difference between the
  3-item list and the 4-item list should not be negative; the addition
  of a 4th item should not decrease the propensity of subjects to
  respond to other items. We therefore exclude the list experiment as a
  measure.}

The self-reports and behavioral observations tell us the real,
descriptive change in intergroup contact. The survey self-reports ask if
and how often the respondent interacted with the other group in the past
month. The respondents are asked about interaction in markets, at public
social events, in the respondent's own home, at the home of a member of
the other group, and in any other way. The responses are then ranked,
scaled from 0-1, and combined into an index. The behavioral observations
provide a measure of contact independent of response biases.

In the markets, we measured interactions related to buying and selling
market goods, such as the number of farmer and pastoralist sellers
present and the number of farmer and pastoralist buyers. We then create
a farmers index and a pastoralist index to measure the presence of
farmers and pastoralists in the market. At social events, we measured
the number of members of the other group in attendance and the number
who ate or drank anything\footnote{Taking food or beverages at a social
  event is a sign of closeness and intimacy in these contexts. Casual
  attendees would not take food or beverages}, both in absolute numbers
and as a percentage of total attendees. We then create measures for the
number of farmers and pastoralists attending social events and the
number of farmers and pastoralists eating at social events.\footnote{Observations
  were made in two periods: July 2016 -- February 2017, immediately
  after the project commenced but before joint project committees
  convened, and September 2017 -- December 2017, after project
  committees convened but before the endline survey began. Events that
  occurred February 2017 or earlier are baseline measurements; events
  occurring September 2017 or later are endline measurements.}

A survey experiment, which we are calling the \emph{percent experiment},
tells us about respondents' willingness to engage in contact. It asks
respondents two questions about their willingness to interact with
members of the other side. We asked respondents if they would (1) join a
group and (2) live in a community with some percentage of the other
group. The percentage is randomized between 5\%, 25\%, 50\%, and 75\%;
the percentage is the same for those two questions but varies across
individuals. We take the mean response so that a respondent saying yes
to both is assigned a 1, a respondent saying yes to one is assigned a
0.5, and a respondent saying no to both is assigned a 0. These questions
allow us to determine if treatment communities become more willing to
interact with outgroup members and if treatment communities become less
sensitive to higher proportions of the outgroup.\footnote{This
  experiment was based on a question from the GSS asking respondents if
  they would favor or oppose living in a neighborhood that was half
  white/black.}

\textbf{Insecurity}: Our third outcome is feelings of insecurity due to
conflict. The end goal of ECPN is to reduce conflict between farmers and
pastoralist. The disaggregated and diffuse nature of the conflict makes
obtaining an accurate measure of violent conflict extremely
difficult.\footnote{Asking respondents to recount the number of violent
  events does not accurately measure the scale of the conflict because
  those answers are determined by the awareness and memory of the
  community members. Awareness of individual violent events is low
  because many of the violent events occur in fields and grazing routes
  far from the town center and residential areas. In addition, ECPN
  sought to increase awareness of violent events through its conflict
  forums. The type of event that all community members are aware of --
  large massacres, burning of homes, etc\ldots{} -- generally lead to
  the disintegration of both communities as community members flee the
  area fearing further violence or reprisals. These large-scale events
  are rare and none occurred in intervention or control communities
  during the study.} Instead, we measured the effect that violent
conflict has on individuals. We ask respondents if they avoid any areas
during the day or night due to insecurity and if insecurity restricted
them from engaging in various activities, such as grazing their animals,
working on their farms, fetching water for their families, and working
for wages. We combined these ten insecurity questions into an index,
with high values indicating low perceptions of insecurity and low values
indicating high perceptions of insecurity. The index alpha is {[}chris:
alpha{]}.

\textbf{Violence Placebo}: Several of our outcomes are survey
self-reports, and all self-reports could be affected by social
desirability bias. Our survey results are suspect if respondents in
treatment communities learned the ``correct'' answers better than
respondents in control communities. If social desirability accounts for
the effect in survey self-reports, we would also expect differences
between treatment and control for other normatively desirable attitudes.
To test social desirability effects, we conduct a placebo analysis using
attitudes about violence as a placebo. Attitudes about violence are a
good candidate for a placebo because intergroup contact should not
affect attitudes about violence, but respondents may feel social
pressure to answer violence questions in a desirable way. We measure
attitudes about violence with a six question index asking respondents if
it is always, sometimes, rarely, or never justified to use violence in
certain situations, such as retaliating against violence or bringing
criminals to justice.

\hypertarget{mechanisms}{%
\subsubsection{Mechanisms}\label{mechanisms}}

The primary outcomes of intergroup affect, intergroup contact, and
insecurity tell us if ECPN worked but provide no evidence for how the
program worked. Previous work on contact specified three mechanisms
through which contact affects attitudes: empathy/perspective-taking,
threat/anxiety, and ingroup expansion {[}@pettigrew2008does;
@al2013intergroup; @dovidio2017reducing{]}. We do not manipulate these
mechanisms directly, and so cannot make causal claims about the
mediating role of these variables for ECPN. But we can provide
exploratory evidence that these mechanisms played a role if (1) ECPN
affects these mechanisms and (2) these mechanisms affect intergroup
affect, intergroup contact, and insecurity.

\textbf{Threat}: We use three self-report survey questions to measure
threat felt by the outgroup. These questions ask if the outgroup is a
threat to the respondent's community, believe in different morals than
the respondent's community, and overly influence the respondent's
community.\footnote{These threat questions are based on questions from
  @van2007testing}.

\textbf{Empathy/Perspective-taking}: We measure empathy with two
questions and perspective-taking with one question. For empathy, one
question asks if the respondent's group would help a member of the other
side if something unfortunate happened to that person, like a serious
illness or the death of a parent. The second questions is the same but
asks if someone from the other group would help someone from the
respondent's group. For perspective-taking, the question asks who the
respondent believes is responsible for the violence between their
community and the other community: the other group or both
groups.\footnote{We planned to create an index with these three
  questions, but their alpha was below 0.70 and improved to 0.90 without
  the perspective-taking question.}

\textbf{Ingroup expansion}: We measured respondents' recategorization of
their ingroup to include outgroup members with eight survey self-reports
and a public goods game. Five survey questions ask respondents to answer
questions about ``people in this area, including people from the other
group'', such as if the groups share the same morals and if the groups
work together to achieve common goals. Three more questions ask the
respondent about the groups working together on specific goals, such as
repairing a road or solving a water supply problem.

We also used a natural-field public goods game to measure the ability of
the groups to cooperate to achieve a common goal. If ECPN causes
respondents to incorporate the former outgroup into their ingroup, then
we expect those communities to better cooperate in a public goods game.
Compared with lab-based behavioral games, whose choice-making situations
are necessarily artificial, the choice-making situation of a
natural-field game is akin to the choices people make in their lives
(Harrison and List 2004; Winking and Mizer 2013). Because these
communities often decide how to contribute to some public good, such as
repairing a borehole or a market, we chose to use a natural-field public
goods game (PGG) as a realistic behavioral measure of
cooperation.\footnote{This game is similar to the one implemented by
  Fearon, Humphreys, and Weinstein (2009) as part of a similar study on
  community-driven development in Liberia.}

These designs and measurements put us in a strong position to identify
effects if effects exist. First, we have data at the community-level and
individual-level. If the two analyses show similar relationships, we can
be more sure that those relationships are not spurious. Second, both
community and individual-level analyses use a baseline/endline + control
group design to differentiate a secular trend from a treatment effect.
Many things change in the social environment between the beginning and
the end of ECPN that could deteriorate intergroup relations, especially
an economic downturn in Nigeria and the anti-grazing law in Benue. By
comparing the \emph{change} in the treatment group to the \emph{change}
in the control, we are more certain that differences are due to ECPN and
not other factors. Third, outcomes are measured using survey
self-reports, survey experiments, a behavioral game, and monitoring of
social behavior. If we observe similar relationships across multiple
modes we can be more certain that the relationship is not spurious.


\end{document}
