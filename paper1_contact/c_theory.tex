%\documentclass[]{article}
\documentclass[11pt]{article}
\usepackage[usenames,dvipsnames]{xcolor}

\usepackage[T1]{fontenc}
%\usepackage{lmodern}
\usepackage{tgtermes}
\usepackage{amssymb,amsmath}
%\usepackage[margin=1in]{geometry}
\usepackage[letterpaper,bottom=1in,top=1in,right=1.25in,left=1.25in,includemp=FALSE]{geometry}
\usepackage{pdfpages}
\usepackage[small]{caption}

\usepackage{ifxetex,ifluatex}
\usepackage{fixltx2e} % provides \textsubscript
% use microtype if available
\IfFileExists{microtype.sty}{\usepackage{microtype}}{}
\ifnum 0\ifxetex 1\fi\ifluatex 1\fi=0 % if pdftex
\usepackage[utf8]{inputenc}
\else % if luatex or xelatex
\usepackage{fontspec}
\ifxetex
\usepackage{xltxtra,xunicode}
\fi
\defaultfontfeatures{Mapping=tex-text,Scale=MatchLowercase}
\newcommand{\euro}{€}
\fi
%

\usepackage{fancyvrb}

\usepackage{ctable,longtable}

\usepackage[section]{placeins}
\usepackage{float} % provides the H option for float placement
\restylefloat{figure}
\usepackage{dcolumn} % allows for different column alignments
\newcolumntype{.}{D{.}{.}{1.2}}

\usepackage{booktabs} % nicer horizontal rules in tables

%Assume we want graphics always
\usepackage{graphicx}
% We will generate all images so they have a width \maxwidth. This means
% that they will get their normal width if they fit onto the page, but
% are scaled down if they would overflow the margins.
%% \makeatletter
%% \def\maxwidth{\ifdim\Gin@nat@width>\linewidth\linewidth
%%   \else\Gin@nat@width\fi}
%% \makeatother
%% \let\Oldincludegraphics\includegraphics
%% \renewcommand{\includegraphics}[1]{\Oldincludegraphics[width=\maxwidth]{#1}}
\graphicspath{{.}{../Soccom_Code/socom_2013/}}


%% \ifxetex
%% \usepackage[pagebackref=true, setpagesize=false, % page size defined by xetex
%% unicode=false, % unicode breaks when used with xetex
%% xetex]{hyperref}
%% \else
\usepackage[pagebackref=true, unicode=true, bookmarks=true, pdftex]{hyperref}
% \fi


\hypersetup{breaklinks=true,
  bookmarks=true,
  pdfauthor={Christopher Grady, Rebecca Wolfe, Danjuma Dawop, and Lisa Inks},
  pdftitle={Improving Intergroup Relations Amid Group Conflict: An Intergroup Contact Field Experiment in Nigeria - Theory},
  colorlinks=true,
  linkcolor=BrickRed,
  citecolor=blue, %MidnightBlue,
  urlcolor=BrickRed,
  % urlcolor=blue,
  % linkcolor=magenta,
  pdfborder={0 0 0}}

%\setlength{\parindent}{0pt}
%\setlength{\parskip}{6pt plus 2pt minus 1pt}
\usepackage{parskip}
\setlength{\emergencystretch}{3em}  % prevent overfull lines
\providecommand{\tightlist}{%
  \setlength{\itemsep}{0pt}\setlength{\parskip}{0pt}}

%% Insist on this.
\setcounter{secnumdepth}{2}

\VerbatimFootnotes % allows verbatim text in footnotes

\title{Improving Intergroup Relations Amid Group Conflict: An Intergroup
Contact Field Experiment in Nigeria - Theory}

\author{
Christopher Grady, Rebecca Wolfe, Danjuma Dawop, and Lisa Inks
}


\date{November 12, 2019}


\usepackage{versions}
\makeatletter
\renewcommand*\versionmessage[2]{\typeout{*** `#1' #2. ***}}
\renewcommand*\beginmarkversion{\sffamily}
  \renewcommand*\endmarkversion{}
\makeatother

\excludeversion{comment}

%\usepackage[margins=1in]{geometry}

\usepackage[compact,bottomtitles]{titlesec}
%\titleformat{ ⟨command⟩}[⟨shape⟩]{⟨format⟩}{⟨label⟩}{⟨sep⟩}{⟨before⟩}[⟨after⟩]
\titleformat{\section}[hang]{\Large\bfseries}{\thesection}{.5em}{\hspace{0in}}[\vspace{-.2\baselineskip}]
\titleformat{\subsection}[hang]{\large\bfseries}{\thesubsection}{.5em}{\hspace{0in}}[\vspace{-.2\baselineskip}]
%\titleformat{\subsubsection}[hang]{\bfseries}{\thesubsubsection}{.5em}{\hspace{0in}}[\vspace{-.2\baselineskip}]
\titleformat{\subsubsection}[hang]{\bfseries}{\thesubsubsection}{1ex}{\hspace{0in}}[\vspace{-.2\baselineskip}]
\titleformat{\paragraph}[runin]{\bfseries\itshape}{\theparagraph}{1ex}{}{\vspace{-.2\baselineskip}}
%\titleformat{\paragraph}[runin]{\itshape}{\theparagraph}{1ex}{}{\vspace{-.2\baselineskip}}

%%\titleformat{\subsection}[hang]{\bfseries}{\thesubsection}{.5em}{\hspace{0in}}[\vspace{-.2\baselineskip}]
%%%\titleformat*{\subsection}{\bfseries\scshape}
%%%\titleformat{\subsubsection}[leftmargin]{\footnotesize\filleft}{\thesubsubsection}{.5em}{}{}
%%\titleformat{\subsubsection}[hang]{\small\bfseries}{\thesubsubsection}{.5em}{\hspace{0in}}[\vspace{-.2\baselineskip}]
%%\titleformat{\paragraph}[runin]{\itshape}{\theparagraph}{1ex}{}{\vspace{-.5\baselineskip}}

%\titlespacing*{ ⟨command⟩}{⟨left⟩}{⟨beforesep⟩}{⟨aftersep⟩}[⟨right⟩]
\titlespacing{\section}{0pc}{1.5ex plus .1ex minus .2ex}{.5ex plus .1ex minus .1ex}
\titlespacing{\subsection}{0pc}{1.5ex plus .1ex minus .2ex}{.5ex plus .1ex minus .1ex}
\titlespacing{\subsubsection}{0pc}{1.5ex plus .1ex minus .2ex}{.5ex plus .1ex minus .1ex}



%% These next lines tell latex that it is ok to have a single graphic
%% taking up most of a page, and they also decrease the space around
%% figures and tables.
\renewcommand\floatpagefraction{.9}
\renewcommand\topfraction{.9}
\renewcommand\bottomfraction{.9}
\renewcommand\textfraction{.1}
\setcounter{totalnumber}{50}
\setcounter{topnumber}{50}
\setcounter{bottomnumber}{50}
\setlength{\intextsep}{2ex}
\setlength{\floatsep}{2ex}
\setlength{\textfloatsep}{2ex}



\begin{document}
\VerbatimFootnotes

%\begin{titlepage}
%  \maketitle
%\vspace{2in}
%
%\begin{center}
%  \begin{large}
%    PROPOSAL WHITE PAPER
%
%BAA 14-013
%
%Can a Hausa Language Television Station Change Norms about Violence in Northern Nigeria? A Randomized Study of Media Effects on Violent Extremism
%
%Jake Bowers
%
%University of Illinois @ Urbana-Champaign (jwbowers@illinois.edu)
%
%\url{http://jakebowers.org}
%
%Phone: +12179792179
%
%Topic Number: 1
%
%Topic Title: Identity, Influence and Mobilization
%
%\end{large}
%\end{center}
%\end{titlepage}

\maketitle

\hypertarget{theory}{%
\section{Theory}\label{theory}}

\begin{itemize}
\item
  Summary of typical contact theory stuff.
\item
  To understand why cooperative contact might not work for groups in
  conflict, we must specify the mechanisms through which contact works
  and then specify why contextual factors/conditions somehow
  block/prevent that mechanism.
\item
  How intergroup contact works:

  \begin{itemize}
  \tightlist
  \item
    Knowledge effect: experience with outgroup replaces misperceptions
    and stereotypes. Hear their perspective and understand point of view
    == less prejudice \& no longer attribute negative motivations to the
    outgroup. (Allport 1954)
  \item
    Expand ingroup to include the former outgroup (Gaertner and Dovidio
    2014)
  \item
    Reduce anxiety, uncertainty, and threat (Lee 2001; Page-Gould,
    Mendoza-Denton, and Tropp 2008; Paolini et al. 2004; Pettigrew and
    Tropp 2008)
  \item
    Increase perspective taking/empathy for outgroup (Broockman and
    Kalla 2016; Pettigrew and Tropp 2008)
  \end{itemize}
\item
  Diffuse to others through extended contact or social norms.
\end{itemize}

\hypertarget{how-conflict-could-prevent-relations-from-improving}{%
\subsection{How conflict could prevent relations from
improving}\label{how-conflict-could-prevent-relations-from-improving}}

\textbf{Predicted mechanisms assume negative stereotypes, arising from
lack of knowledge/experience, cause dislike}

Contact theory assumes that negative stereotypes cause intergroup
animosity. Stereotypes, natural mental shortcuts that help an individual
understand his/her experiences, are especially likely to go awry and
create animosity when an individual has little or no experience with
members of another group. Without intergroup experience, stereotypes
will misrepresent groups and create imagined differences between
in-group and outgroup members. To remove these negative stereotypes new
experiences must override them, allowing an individual to re-
conceptualize the outgroup.

The mechanisms through which contact improves an individual's negative
outgroup attitudes assume that the negative attitudes are caused by a
lack of experience with the other side.

The information mechanism assumes minimal previous experience; these
groups have many negative experiences, and many group members have been
directly harmed. Real information about the other side will not decrease
animosity when it is probably true that some members of the other side
want to cause physical harm to your side.

Empathy seems to apply to prejudice that advantaged groups feel towards
disadvantaged, but empathy/perspective-taking should not improve
relations if the other side is expected to be belligerent (Kertzer,
Brutger, and Quek 2018). Anxiety based on unfamiliarity with outgroup is
different than anxiety based on anxiety-inducing previous experiences.

Superordinate identities unlikely for groups in violent conflict, who
will have few shared identities and whose material interests are
opposed.

Negative contact experiences worsen intergroup relations (Paolini,
Harwood, and Rubin 2010). Individuals with the worst attitudes may
experience backlash effects (Gubler 2011).

\textbf{Contact could do bad things}

Not all intergroup contact decreases animosity. Negative intergroup
interactions increase prejudice and lead to more negative attitudes
towards outgroup members (Paolini et al 2010; Enos 2014; sands 2017;
condra and linardi 2019). Groups in conflict more likely than others to
have negative contact experiences.

Even well-structured positive interactions can increase negative
attitudes by causing cognitive dissonance in prejudiced individuals,
which causes them to cling more strongly to their prejudicial beliefs
(Gubler 2011). Groups in conflict more likely than others to have
strong, stable, negative opinions towards the outgroup.

\textbf{Conditions under which contact works are not present}

Conditions must be met for intergroup contact to improve intergroup
relations and reduce violent conflict. First, ingroup members must
generalize their contact with a few outgroup members to the entire
outgroup. Second, the positive effects of contact must diffuse
throughout the ingroup. Third, attitudes of both groups must improve.
Even if contact effectively changes the attitudes of individual group
members, contact cannot have meaningful effects on reducing violent
conflict unless these conditions are met.

These conditions may not be met by groups engaged in violent conflict.
First, ingroup members who cooperate with outgroup members may not
perceive those outgroup members as typical of the outgroup due to
previous negative experiences with outgroup members; consequently,
ingroup members may not use interactions with those outgroup members to
update their attitudes about the outgroup as a whole. Second, even if
contact changes the attitudes of ingroup members who experience contact,
norms that affect other ingroup members may not develop. Liking other
group hard to link to a positive ingroup trait, like morality. Third,
cooperative contact may only improve relations of the more powerful
group towards the less powerful group; the less powerful group may
perceive themselves as victims of the other group's injustice.

\textbf{Contact does not change cause of conflict}

Real world violence caused by real world problems. These groups have
misaligned incentives. Literatures in psych, IR, and econ that predict
no change after intergroup contact because groups incentives still
fundamentally misaligned. Pastoralists want to graze, farmers do not
want them to graze. Contact not change the fundamental causes of
conflict; things that cause conflict also cause negative intergroup
attitudes. Could get peace through negotiation, but no bargaining range.
Could get peace through negotiation, but cannot trust all group members
to abide by deal.

\textbf{Summary}

The contact hypothesis was devised to explain racial animus in the
United States. It's mechanisms address problems that arise due to lack
of experience: replacement of incorrect stereotypes, reduced anxiety due
to uncertainty, ability to see things from outgroup's perspective, and
seeing a shared identity with the outgroup. The contact hypothesis is
almost always tested on groups who are spatially segregated and have
limited opportunities for interaction -- white people and black people,
Muslims and Christians, Jews and Arabs\footnote{chris: do Jews and Arabs
  fit into this?}. These mechanisms may not function for groups with a
recent history of negative and violent interaction. How would contact
improve attitudes when negative attitudes are based on negative
experience, rather than inexperience?

\hypertarget{how-contact-can-improve-relations-even-amid-conflict}{%
\subsection{How contact can improve relations even amid
conflict}\label{how-contact-can-improve-relations-even-amid-conflict}}

\begin{itemize}
\item
  Contact as opportunity to show that intergroup cooperation is
  materially good for both groups.

  \begin{itemize}
  \tightlist
  \item
    Related: contact to foster trade? (Rohner, Thoenig, and Zilibotti
    2013) (Jha and Shayo 2019 - Valuing Peace).
  \end{itemize}
\item
  Contact as opportunity to send costly signal of type: trustworthy,
  fair, hard-working.
\item
  Contact to create norm of ingroup policing.
\item
  Contact increases likelihood of peace through negotiation:

  \begin{itemize}
  \tightlist
  \item
    opportunity to find bargaining range
  \item
    opportunity ot build trust that both groups will abide by deal,
    sanction ingroup members who defect.
  \end{itemize}
\end{itemize}

Chris: Remind self not to bring up things I provide no evidence for.

\hypertarget{summary}{%
\subsection{Summary}\label{summary}}

{[}If needed, summary of argument.{]}

\hypertarget{hypotheses}{%
\subsection{Hypotheses}\label{hypotheses}}

Separate families of Competing hypotheses.

Hyp 1: Contact good.

Hyp 2: Contact bad/neutral.

\begin{center}\rule{0.5\linewidth}{\linethickness}\end{center}

\hypertarget{references}{%
\section{References}\label{references}}

\hypertarget{refs}{}
\leavevmode\hypertarget{ref-allport1954prejudice}{}%
Allport, Gordon. 1954. ``The Nature of Prejudice.'' \emph{Garden City,
NJ Anchor}.

\leavevmode\hypertarget{ref-broockman2016durably}{}%
Broockman, David, and Joshua Kalla. 2016. ``Durably Reducing
Transphobia: A Field Experiment on Door-to-Door Canvassing.''
\emph{Science} 352(6282): 220--24.

\leavevmode\hypertarget{ref-gaertner2014reducing}{}%
Gaertner, Samuel L, and John F Dovidio. 2014. \emph{Reducing Intergroup
Bias: The Common Ingroup Identity Model}. Psychology Press.

\leavevmode\hypertarget{ref-gubler2011diss}{}%
Gubler, Joshua R. 2011. ``The Micro-Motives of Intergroup Aggression: A
Case Study in Israel.'' PhD thesis. University of Michigan.

\leavevmode\hypertarget{ref-kertzer2018empathy}{}%
Kertzer, Joshua D, Ryan Brutger, and Kai Quek. 2018. ``Strategic Empathy
and the Security Dilemma: Cross-National Experimental Evidence from
China and the United States.''

\leavevmode\hypertarget{ref-lee2001mere}{}%
Lee, Angela Y. 2001. ``The Mere Exposure Effect: An Uncertainty
Reduction Explanation Revisited.'' \emph{Personality and Social
Psychology Bulletin} 27(10): 1255--66.

\leavevmode\hypertarget{ref-page2008little}{}%
Page-Gould, Elizabeth, Rodolfo Mendoza-Denton, and Linda R Tropp. 2008.
``With a Little Help from My Cross-Group Friend: Reducing Anxiety in
Intergroup Contexts Through Cross-Group Friendship.'' \emph{Journal of
personality and social psychology} 95(5): 1080.

\leavevmode\hypertarget{ref-paolini2010negative}{}%
Paolini, Stefania, Jake Harwood, and Mark Rubin. 2010. ``Negative
Intergroup Contact Makes Group Memberships Salient: Explaining Why
Intergroup Conflict Endures.'' \emph{Personality and Social Psychology
Bulletin} 36(12): 1723--38.

\leavevmode\hypertarget{ref-paolini2004effects}{}%
Paolini, Stefania, Miles Hewstone, Ed Cairns, and Alberto Voci. 2004.
``Effects of Direct and Indirect Cross-Group Friendships on Judgments of
Catholics and Protestants in Northern Ireland: The Mediating Role of an
Anxiety-Reduction Mechanism.'' \emph{Personality and Social Psychology
Bulletin} 30(6): 770--86.

\leavevmode\hypertarget{ref-pettigrew2008does}{}%
Pettigrew, Thomas F, and Linda R Tropp. 2008. ``How Does Intergroup
Contact Reduce Prejudice? Meta-Analytic Tests of Three Mediators.''
\emph{European Journal of Social Psychology} 38(6): 922--34.

\leavevmode\hypertarget{ref-rohner2013war}{}%
Rohner, Dominic, Mathias Thoenig, and Fabrizio Zilibotti. 2013. ``War
Signals: A Theory of Trade, Trust, and Conflict.'' \emph{Review of
Economic Studies} 80(3): 1114--47.

\end{document}
