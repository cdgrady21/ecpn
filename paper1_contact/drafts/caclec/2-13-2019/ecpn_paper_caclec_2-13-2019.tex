%\documentclass[]{article}
\documentclass[11pt]{article}
\usepackage[usenames,dvipsnames]{xcolor}

\usepackage[T1]{fontenc}
%\usepackage{lmodern}
\usepackage{tgtermes}
\usepackage{amssymb,amsmath}
%\usepackage[margin=1in]{geometry}
\usepackage[letterpaper,bottom=1in,top=1in,right=1.25in,left=1.25in,includemp=FALSE]{geometry}
\usepackage{pdfpages}
\usepackage[small]{caption}

\usepackage{ifxetex,ifluatex}
\usepackage{fixltx2e} % provides \textsubscript
% use microtype if available
\IfFileExists{microtype.sty}{\usepackage{microtype}}{}
\ifnum 0\ifxetex 1\fi\ifluatex 1\fi=0 % if pdftex
\usepackage[utf8]{inputenc}
\else % if luatex or xelatex
\usepackage{fontspec}
\ifxetex
\usepackage{xltxtra,xunicode}
\fi
\defaultfontfeatures{Mapping=tex-text,Scale=MatchLowercase}
\newcommand{\euro}{€}
\fi
%

\usepackage{fancyvrb}

\usepackage{ctable,longtable}

\usepackage[section]{placeins}
\usepackage{float} % provides the H option for float placement
\restylefloat{figure}
\usepackage{dcolumn} % allows for different column alignments
\newcolumntype{.}{D{.}{.}{1.2}}

\usepackage{booktabs} % nicer horizontal rules in tables

%Assume we want graphics always
\usepackage{graphicx}
% We will generate all images so they have a width \maxwidth. This means
% that they will get their normal width if they fit onto the page, but
% are scaled down if they would overflow the margins.
%% \makeatletter
%% \def\maxwidth{\ifdim\Gin@nat@width>\linewidth\linewidth
%%   \else\Gin@nat@width\fi}
%% \makeatother
%% \let\Oldincludegraphics\includegraphics
%% \renewcommand{\includegraphics}[1]{\Oldincludegraphics[width=\maxwidth]{#1}}
\graphicspath{{.}{../Soccom_Code/socom_2013/}}


%% \ifxetex
%% \usepackage[pagebackref=true, setpagesize=false, % page size defined by xetex
%% unicode=false, % unicode breaks when used with xetex
%% xetex]{hyperref}
%% \else
\usepackage[pagebackref=true, unicode=true, bookmarks=true, pdftex]{hyperref}
% \fi


\hypersetup{breaklinks=true,
  bookmarks=true,
  pdfauthor={Christopher Grady},
  pdftitle={Intergroup Contact Amidst Escalating Conflict - Introduction},
  colorlinks=true,
  linkcolor=BrickRed,
  citecolor=blue, %MidnightBlue,
  urlcolor=BrickRed,
  % urlcolor=blue,
  % linkcolor=magenta,
  pdfborder={0 0 0}}

%\setlength{\parindent}{0pt}
%\setlength{\parskip}{6pt plus 2pt minus 1pt}
\usepackage{parskip}
\setlength{\emergencystretch}{3em}  % prevent overfull lines
\providecommand{\tightlist}{%
  \setlength{\itemsep}{0pt}\setlength{\parskip}{0pt}}

%% Insist on this.
\setcounter{secnumdepth}{2}

\VerbatimFootnotes % allows verbatim text in footnotes

\title{Intergroup Contact Amidst Escalating Conflict - Introduction}

\author{
Christopher Grady
}


\date{February 11, 2019}


\usepackage{versions}
\makeatletter
\renewcommand*\versionmessage[2]{\typeout{*** `#1' #2. ***}}
\renewcommand*\beginmarkversion{\sffamily}
  \renewcommand*\endmarkversion{}
\makeatother

\excludeversion{comment}

%\usepackage[margins=1in]{geometry}

\usepackage[compact,bottomtitles]{titlesec}
%\titleformat{ ⟨command⟩}[⟨shape⟩]{⟨format⟩}{⟨label⟩}{⟨sep⟩}{⟨before⟩}[⟨after⟩]
\titleformat{\section}[hang]{\Large\bfseries}{\thesection}{.5em}{\hspace{0in}}[\vspace{-.2\baselineskip}]
\titleformat{\subsection}[hang]{\large\bfseries}{\thesubsection}{.5em}{\hspace{0in}}[\vspace{-.2\baselineskip}]
%\titleformat{\subsubsection}[hang]{\bfseries}{\thesubsubsection}{.5em}{\hspace{0in}}[\vspace{-.2\baselineskip}]
\titleformat{\subsubsection}[hang]{\bfseries}{\thesubsubsection}{1ex}{\hspace{0in}}[\vspace{-.2\baselineskip}]
\titleformat{\paragraph}[runin]{\bfseries\itshape}{\theparagraph}{1ex}{}{\vspace{-.2\baselineskip}}
%\titleformat{\paragraph}[runin]{\itshape}{\theparagraph}{1ex}{}{\vspace{-.2\baselineskip}}

%%\titleformat{\subsection}[hang]{\bfseries}{\thesubsection}{.5em}{\hspace{0in}}[\vspace{-.2\baselineskip}]
%%%\titleformat*{\subsection}{\bfseries\scshape}
%%%\titleformat{\subsubsection}[leftmargin]{\footnotesize\filleft}{\thesubsubsection}{.5em}{}{}
%%\titleformat{\subsubsection}[hang]{\small\bfseries}{\thesubsubsection}{.5em}{\hspace{0in}}[\vspace{-.2\baselineskip}]
%%\titleformat{\paragraph}[runin]{\itshape}{\theparagraph}{1ex}{}{\vspace{-.5\baselineskip}}

%\titlespacing*{ ⟨command⟩}{⟨left⟩}{⟨beforesep⟩}{⟨aftersep⟩}[⟨right⟩]
\titlespacing{\section}{0pc}{1.5ex plus .1ex minus .2ex}{.5ex plus .1ex minus .1ex}
\titlespacing{\subsection}{0pc}{1.5ex plus .1ex minus .2ex}{.5ex plus .1ex minus .1ex}
\titlespacing{\subsubsection}{0pc}{1.5ex plus .1ex minus .2ex}{.5ex plus .1ex minus .1ex}



%% These next lines tell latex that it is ok to have a single graphic
%% taking up most of a page, and they also decrease the space around
%% figures and tables.
\renewcommand\floatpagefraction{.9}
\renewcommand\topfraction{.9}
\renewcommand\bottomfraction{.9}
\renewcommand\textfraction{.1}
\setcounter{totalnumber}{50}
\setcounter{topnumber}{50}
\setcounter{bottomnumber}{50}
\setlength{\intextsep}{2ex}
\setlength{\floatsep}{2ex}
\setlength{\textfloatsep}{2ex}



\begin{document}
\VerbatimFootnotes

%\begin{titlepage}
%  \maketitle
%\vspace{2in}
%
%\begin{center}
%  \begin{large}
%    PROPOSAL WHITE PAPER
%
%BAA 14-013
%
%Can a Hausa Language Television Station Change Norms about Violence in Northern Nigeria? A Randomized Study of Media Effects on Violent Extremism
%
%Jake Bowers
%
%University of Illinois @ Urbana-Champaign (jwbowers@illinois.edu)
%
%\url{http://jakebowers.org}
%
%Phone: +12179792179
%
%Topic Number: 1
%
%Topic Title: Identity, Influence and Mobilization
%
%\end{large}
%\end{center}
%\end{titlepage}

\maketitle

\begin{abstract}

Can intergroup contact reduce group prejudice even when economic incentives are opposed and even amidst an escalating conflict?  Do these effects diffuse to ingroup members with no direct contact?  A voluminous literature on intergroup prejudice fails to answer these questions; some theories suggest yes while other theories suggest no.  Few if any studies test these theories in the context of opposing economic incentives and active violence, yet opposing economic incentives and active violence are the hallmark of many world conflicts where social science theories are put into practice.  We answer these questions by conducting a field experiment with 30 communities in Nigeria, where farmers and pastoralists are embroiled in a deadly conflict over land use.  We find that a yearlong intergroup contact intervention for a small subset of community members reduces prejudice and increases perceptions of physical security, even for community members not directly involved in the intervention.  This experiment demonstrates the efficacy of intergroup contact, even in the context of opposed economic incentives and escalating conflict. 

\end{abstract}

\section{Introduction}\label{introduction}

Does intergroup contact reduce prejudice and conflict in contexts of
intergroup violence? Direct intergroup contact interventions are often
implemented to reduce real-world conflict between groups, and these
real-world conflicts are defined by active violence and resource
competition. Despite a plethora of research about intergroup contact and
widespread use of intergroup contact programs to reduce violent
conflict, we have almost no evidence about the effects of intergroup
contact in violent contexts or where economic imperatives push groups
apart. Can intergroup contact overcome prejudice when the groups are
engaged in violent conflict and compete over resources?

Decades of research demonstrate that intergroup contact can reduce
group-level prejudice in a variety of contexts and for a variety of
groups. Intergroup contact reduces prejudice towards different racial
and ethnic groups (Burns, Corno, and La Ferrara 2015; Katz and Zalk
1978; Marmaros and Sacerdote 2006; Yablon 2012), different religious
groups (Barnhardt 2009; Scacco and Warren 2016; Yablon 2012), women
(Finseraas et al. 2016), people with physical disabilities (Krahé and
Altwasser 2006), gay people (Grutzeck and Gidycz 1997), and immigrants
(Finseraas and Kotsadam 2017). The effects of intergroup contact have
been demonstrated in a variety of countries and using a variety of
methodological approaches (Paluck, Green, and Green 2017; Pettigrew and
Tropp 2006). The efficacy of intergroup contact to reduce prejudice
amidst violent conflict, however, is still an open question. None of
these studies involve groups in active conflict or groups competing for
resources; few of these studies even involve adults beyond college age.

Active conflict and economic competition could prevent contact's
positive effects, and may even cause contact to increase prejudice,
because active conflict and economic competition give group members a
material motivation for prejudice and hate. In an active conflict, the
groups have materially damaged each other; in economic competition, one
group's gain is the other group's loss. Both situations are common in
conflicts worldwide, but scholars have yet to grapple with how these
conditions affect intergroup contact or to test intergroup contact in
these conditions. Contact theory suggests that contact should work in
these contexts, provided the contact itself is conducted under proper
conditions. If the contact allows group members to (i) cooperate towards
(ii) common goals in (iii) an equal status context with (iv) the support
of authorities, contact should reduce prejudice. Intergroup contact,
even when the groups are in conflict and competition, should provide
experiences that reduce stereotypes about outgroup members (Allport
1954; Gaertner and Dovidio 2014), reduce anxiety and uncertainty towards
the outgroup (Lee 2001; Page-Gould, Mendoza-Denton, and Tropp 2008;
Paolini et al. 2004), and increase empathy towards the outgroup
(Broockman and Kalla 2016; Pettigrew and Tropp 2008).

But will contact effectively reduce prejudice when the wider social
context promotes it? Other perspectives on intergroup prejudice, like
realistic group conflict theory and psychological theories like
motivated reasoning and cognitive dissonance, would not predict improved
relations from intergroup contact in this context. Intergroup contact
does not change the underlying causes of prejudice -- competition over
indivisible resources -- and so will not reduce prejudice (Campbell
1965). Nor does intergroup contact change the history of violence that
results from outgroup prejudice, feeds outgroup prejudice, and whose
justification requires outgroup prejudice. In violent contexts, any
intergroup contact could increase prejudice because group members are
motivated to interpret intergroup interactions negatively (Klein and
Kunda 1992; Paolini, Harwood, and Rubin 2010), and the cognitive
dissonance generated from even positive intergroup contact may cause a
backlash effect that increases prejudice (Gubler 2011).

Farmer-pastoralist conflict in Nigeria's Middlebelt is an ideal context
to learn about the effect of intergroup contact on intergroup attitudes
in a conflict environment. As Nigeria's population expands and arable
land recedes, economic imperatives have pushed subsistence farmers and
pastoralists into a deadly conflict. Though far more attention is paid
towards Boko Haram in Nigeria's northeast, repercussions of
farmer-pastoralist conflict in Nigeria's Middlebelt are just as
significant. The conflict has ravaged Nigeria's Middlebelt, a mostly
rural region considered Nigeria's breadbasket and home to nearly 100
vibrant peoples and cultures. Farmer-pastoralist violence in the
Middlebelt has caused an estimated 60,000 deaths (Obaji 2016), hundreds
of thousands of internally displaced peoples (Daniel 2018; Shand 2017),
and \$13 billion of lost economic productivity annually (McDougal et al.
2015), greatly stressing Nigeria's economic and social infrastructure.
Beyond Nigeria, farmer-pastoralist conflict plagues numerous sub-Saharan
African countries, where 60\% of the world's estimated 50-100 million
pastoralists live (Omar 1992; Sheik-Mohamed and Velema 1999). It's
likely that farmer-pastoralist conflict will become an increasingly
large problem as demographic, economic, and climatic trends continue to
decrease land availability.

We conduct a field experiment with farmer and pastoralist communities in
two Nigerian states to determine if intergroup contact effectively
reduces prejudice between groups in conflict. We identified fifteen
sites where farmer and pastoral groups had engaged in violent conflict
within the previous twelve months. We then randomly assigned ten of
fifteen conflict sites to receive a yearlong peacebuilding program based
around intergroup contact called \emph{Engaging Communities for Peace in
Nigeria} (ECPN), with the other five sites serving as the control
group.\footnote{Based on the success of ECPN, Mercy Corps received
  further funding to implement the peacebuilding program in the control
  sites \emph{after} the final evaluation of ECPN was completed. The
  control sites were not informed that they would receive a
  peacebuilding program during the evaluation.} The program formed
committees with equal numbers of farmers and pastoralists, including
community leaders from both groups, and tasked them with constructing
two infrastructure projects that would benefit both communities: (1) a
borehole to increase access to potable water and (2) a project of their
choosing, such as a primary health center, a school, or an expanded
market building. The intergroup contact via committees was structured to
meet the four conditions that Allport (1954) theorized were necessary
for intergroup contact to reduce prejudice: two groups (i) cooperating
to achieve (ii) joint goals in (iii) an equal status context with (iv)
the support of authorities.

The results of this field experiment demonstrate that intergroup contact
can effectively reduce prejudice even in contexts of escalating
intergroup violence and even when the groups compete for scarce
resources. Communities that received the ECPN program increased their
intergroup contact, trust in the outgroup, and perceptions of physical
security relative to the control group that received no program. The
results also show that contact for a relatively small percentage of a
group can affect attitudes of group members with no exogenous increase
in contact with the outgroup. We observe the most positive changes from
individuals directly involved in the intergroup committees, but we also
observe smaller diffusion effects to group members who were not involved
in the intergroup contact intervention.

In this article, we begin by reviewing the literature on intergroup
prejudice, focusing on the theory of intergroup contact and highlighting
conditions under which contact may be ineffective at reducing prejudice.
We then discuss what farmer-pastoralist conflict can teach us about
intergroup contact, describe our experimental intervention and two
designs to evaluate the effect of the intervention, and present the
results of the study. We conclude by connecting these findings to
theories of group prejudice.

\begin{center}\rule{0.5\linewidth}{\linethickness}\end{center}

\subsection{My Thoughts}\label{my-thoughts}

\smallskip

\begin{itemize}
\item
  Does pitch of why this topic is important sell you?
\item
  How are paragraph and sentence structures?
\item
  Better to be able to say a potential mechanism? ``Results suggest that
  the contact worked by decreasing threat/increasing perceptions of
  benefit from cooperation/something''
\item
  Frame with both violent context + conflict over resources? Does that
  help by making this broad (union) or hurt by making it too narrow
  (intersection)
\item
  Should I/How to better distinguish the cause of conflict (scarce
  resources) and the consequences of the conflict (history of violence)?
\item
  Should I discuss rational choice theories here or at all? What are
  some rational choice theories about conflict?
\item
  Should I bring up diffusion of contact interventions the fact that we
  want peacebuilding interventions to affect people who don't directly
  participate?
\item
  Fast and loose with ``reduce conflict and prejudice''? Just make it
  prejudice?
\item
  The 4 conditions and the (i), (ii), (iii) format I use to list
  them\ldots{}.
\end{itemize}

\section*{References}\label{references}
\addcontentsline{toc}{section}{References}

\hypertarget{refs}{}
\hypertarget{ref-allport1954prejudice}{}
Allport, Gordon. 1954. ``The Nature of Prejudice.'' \emph{Garden City,
NJ Anchor}.

\hypertarget{ref-barnhardt2009near}{}
Barnhardt, Sharon. 2009. ``Near and Dear? Evaluating the Impact of
Neighbor Diversity on Inter-Religious Attitudes.'' \emph{Unpublished
working paper}.

\hypertarget{ref-broockman2016durably}{}
Broockman, David, and Joshua Kalla. 2016. ``Durably Reducing
Transphobia: A Field Experiment on Door-to-Door Canvassing.''
\emph{Science} 352(6282): 220--24.

\hypertarget{ref-burns2015interaction}{}
Burns, Justine, Lucia Corno, and Eliana La Ferrara. 2015.
\emph{Interaction, Prejudice and Performance. Evidence from South
Africa}. Working paper.

\hypertarget{ref-campbell1965ethno}{}
Campbell, Donald T. 1965. ``Ethnocentric and Other Altruistic Motives.''
In \emph{Nebraska Symposium on Motivation}, 283--311.

\hypertarget{ref-daniel2018anti}{}
Daniel, Soni. 2018. ``Anti-Open Grazing Law: Nass, Benue, Kwara, Taraba
Tackle Defence Minister.'' \emph{Vanguard}.

\hypertarget{ref-finseraas2017does}{}
Finseraas, Henning, and Andreas Kotsadam. 2017. ``Does Personal Contact
with Ethnic Minorities Affect Anti-Immigrant Sentiments? Evidence from a
Field Experiment.'' \emph{European Journal of Political Research} 56(3):
703--22.

\hypertarget{ref-finseraas2016women}{}
Finseraas, Henning, Åshild A Johnsen, Andreas Kotsadam, and Gaute
Torsvik. 2016. ``Exposure to Female Colleagues Breaks the Glass
Ceiling---Evidence from a Combined Vignette and Field Experiment.''
\emph{European Economic Review} 90: 363--74.

\hypertarget{ref-gaertner2014reducing}{}
Gaertner, Samuel L, and John F Dovidio. 2014. \emph{Reducing Intergroup
Bias: The Common Ingroup Identity Model}. Psychology Press.

\hypertarget{ref-grutzeck1997effects}{}
Grutzeck, Sasha, and Christine A Gidycz. 1997. ``The Effects of a Gay
and Lesbian Speaker Panel on College Students Attitudes and Behaviors:
The Importance of Context Effects.'' \emph{Imagination Cognition and
Personality} 17(1): 65--81.

\hypertarget{ref-gubler2011diss}{}
Gubler, Joshua R. 2011. ``The Micro-Motives of Intergroup Aggression: A
Case Study in Israel.'' PhD thesis.

\hypertarget{ref-katz1978race}{}
Katz, Phyllis A, and Sue R Zalk. 1978. ``Modification of Children's
Racial Attitudes.'' \emph{Developmental Psychology} 14(5): 447.

\hypertarget{ref-klein1992motivated}{}
Klein, William M, and Ziva Kunda. 1992. ``Motivated Person Perception:
Constructing Justifications for Desired Beliefs.'' \emph{Journal of
experimental social psychology} 28(2): 145--68.

\hypertarget{ref-krahe2006disabled}{}
Krahé, Barbara, and Colette Altwasser. 2006. ``Changing Negative
Attitudes Towards Persons with Physical Disabilities: An Experimental
Intervention.'' \emph{Journal of Community \& Applied Social Psychology}
16(1): 59--69.

\hypertarget{ref-lee2001mere}{}
Lee, Angela Y. 2001. ``The Mere Exposure Effect: An Uncertainty
Reduction Explanation Revisited.'' \emph{Personality and Social
Psychology Bulletin} 27(10): 1255--66.

\hypertarget{ref-marmaros2006friendships}{}
Marmaros, David, and Bruce Sacerdote. 2006. ``How Do Friendships Form?''
\emph{The Quarterly Journal of Economics} 121(1): 79--119.

\hypertarget{ref-mcdougal2015effect}{}
McDougal, Topher L et al. 2015. ``The Effect of Farmer-Pastoralist
Violence on Income: New Survey Evidence from Nigeria's Middle Belt
States.'' \emph{Economics of Peace and Security Journal} 10(1): 54--65.

\hypertarget{ref-obaji2016war}{}
Obaji, Philip Jr. 2016. ``The Nigerian War That Has Slaughtered More
People Than Boko Haram.'' \emph{The Daily Beast}.

\hypertarget{ref-omar1992health}{}
Omar, Mayeh A. 1992. ``Health Care for Nomads Too, Please.'' In
\emph{World Health Forum (Who)}, World Health Organization, 307--10.

\hypertarget{ref-page2008little}{}
Page-Gould, Elizabeth, Rodolfo Mendoza-Denton, and Linda R Tropp. 2008.
``With a Little Help from My Cross-Group Friend: Reducing Anxiety in
Intergroup Contexts Through Cross-Group Friendship.'' \emph{Journal of
personality and social psychology} 95(5): 1080.

\hypertarget{ref-paluck2017contact}{}
Paluck, Elizabeth Levy, Seth Green, and Donald P Green. 2017. ``The
Contact Hypothesis Revisited.''

\hypertarget{ref-paolini2010negative}{}
Paolini, Stefania, Jake Harwood, and Mark Rubin. 2010. ``Negative
Intergroup Contact Makes Group Memberships Salient: Explaining Why
Intergroup Conflict Endures.'' \emph{Personality and Social Psychology
Bulletin} 36(12): 1723--38.

\hypertarget{ref-paolini2004effects}{}
Paolini, Stefania, Miles Hewstone, Ed Cairns, and Alberto Voci. 2004.
``Effects of Direct and Indirect Cross-Group Friendships on Judgments of
Catholics and Protestants in Northern Ireland: The Mediating Role of an
Anxiety-Reduction Mechanism.'' \emph{Personality and Social Psychology
Bulletin} 30(6): 770--86.

\hypertarget{ref-pettigrew2006meta}{}
Pettigrew, Thomas F, and Linda R Tropp. 2006. ``A Meta-Analytic Test of
Intergroup Contact Theory.'' \emph{Journal of personality and social
psychology} 90(5): 751.

\hypertarget{ref-pettigrew2008does}{}
---------. 2008. ``How Does Intergroup Contact Reduce Prejudice?
Meta-Analytic Tests of Three Mediators.'' \emph{European Journal of
Social Psychology} 38(6): 922--34.

\hypertarget{ref-Scacco2016}{}
Scacco, Alexandra, and Shana S Warren. 2016. ``Youth Vocational Training
and Conflict Mitigation: An Experimental Test of Social Contact Theory
in Nigeria.''

\hypertarget{ref-icg2017herder}{}
Shand, Mike. 2017. ``Herders Against Farmers: Nigeria's Expanding Deadly
Conflict.'' \emph{International Crisis Group Reports}.

\hypertarget{ref-sheik1999health}{}
Sheik-Mohamed, Abdikarim, and Johan P Velema. 1999. ``Where Health Care
Has No Access: The Nomadic Populations of Sub-Saharan Africa.''
\emph{Tropical medicine \& international health} 4(10): 695--707.

\hypertarget{ref-yablon2012we}{}
Yablon, Yaacov B. 2012. ``Are We Preaching to the Converted? The Role of
Motivation in Understanding the Contribution of Intergroup Encounters.''
\emph{Journal of Peace Education} 9(3): 249--63.

\end{document}
